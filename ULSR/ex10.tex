Cet exercice traite de la construction et de l'analyse d'un arbre
al{\'e}atoire croissant. Il teste la capacit{\'e} des candidats {\`a}
mod{\'e}liser un processus stochastique, {\`a} calculer des probabilit{\'e}s
conditionnelles et {\`a} d{\'e}terminer les caract{\'e}ristiques statistiques
(esp{\'e}rance, variance) de certaines propri{\'e}t{\'e}s de l'arbre.

\begin{exercise}[]
On construit un arbre al{\'e}atoire de la mani{\`e}re suivante: on commence
par une racine $S_1$ ; on lui ajoute un premier descendant direct $S_2$. Puis,
{\`a} l'{\'e}tape $N + 1$, on choisit un sommet $S_i$ uniform{\'e}ment (avec
$i \in \llbracket 1, N \rrbracket$) et on lui ajoute un descendant direct
$S_{N + 1}$.

1. Calculer l'esp{\'e}rance et la variance du nombre de descendants directs
de $S_1$ {\`a} l'{\'e}tape $N$.

2. Calculer la probabilit{\'e} que $S_2$ ait $k$ descendants (directs ou non)
{\`a} l'issue de l'{\'e}tape $N$.

3. Calculer l'eps{\'e}rance et la variance du nombre de feuilles (sommets sans
descendants) de l'arbre {\`a} l'{\'e}tape $N$.

\end{exercise}

\subsection*{Solution. (SABIR Ilyass, ZINE Akram)}
\addcontentsline{toc}{subsection}{Solution. (SABIR Ilyass - ZINE Akram)}

1. {\`A} chaque {\'e}tape $k$ (de $2$ {\`a} $N$), le sommet $S_1$ a une
probabilit{\'e} de $\frac{1}{k - 1}$ d'{\^e}tre choisi pour recevoir un
nouveau descendant direct. On d{\'e}finit pour tout $k = 2, \ldots, N$, la
variable al{\'e}atoire $X_k$ par :
\[ X_k = \left\{\begin{array}{l}
     1 \tmop{si} S_1 \tmop{est} \tmop{choisi} {\`a} l' {\'e} \tmop{tape} k\\
     0 \tmop{sinon}
   \end{array}\right. \]


L'esp{\'e}rance du nombre de descendants directs de $S_1$ {\`a} l'{\'e}tape N
:
\begin{eqnarray*}
  \mathbb{E} \left[ \underset{k = 2}{\overset{N}{\sum}} X_k \right] & = &
  \sum_{k = 2}^N \mathbb{E}[X_k]\\
  & = & \sum_{k = 2}^N \frac{1}{k - 1}\\
  & = & H_{N - 1}
\end{eqnarray*}


o{\`u} $H_{N - 1}$ est le $(N - 1)$-i{\`e}me nombre harmonique.

\

La variance du nombre de descendants directs de $S_1$ {\`a} l'{\'e}tape N :

On a par ind{\'e}pendance entre $X_2, X_3, \ldots, X_N$
\begin{eqnarray*}
  \tmop{Var} \left[ \underset{k = 2}{\overset{N}{\sum}} X_k \right] & = &
  \sum_{k = 2}^N \text{Var} [X_k]\\
  & = &  \sum_{k = 2}^N \left( \frac{1}{k - 1}  \left( 1 - \frac{1}{k - 1}
  \right) \right)\\
  & = & H_{N - 1} - \sum_{k = 2}^N \frac{1}{(k - 1)^2}
\end{eqnarray*}


2. Le nombre total de sommets est $N$. On a la racine $S_1$, qui n'est pas un
descendant de $S_2$. $S_2$ lui-m{\^e}me ne doit pas {\^e}tre compt{\'e} comme
son propre descendant.

Les sommets $S_3, S_4, \ldots, S_N$ sont les $N - 2$ sommets restants qui
peuvent potentiellement {\^e}tre des descendants de $S_2$.

Le nombre total d'arbres que l'on peut construire est : $(N - 1) !$

\

Le nombre de fa{\c c}ons de choisir les $k$ descendants de $S_2$ parmi les $N
- 2$ sommets est : $\binom{N - 2}{k}$

\

Le nombre de fa{\c c}ons de construire un arbre r{\'e}cursif avec $k + 1$
sommets (incluant $S_2$) est :
\[ T_{\text{sous-arbre}} = k! \]


puisque $S_2$ est la racine fixe du sous-arbre.

Les $N - k - 2$ sommets restants (qui ne sont pas descendants de $S_2$)
forment un autre arbre r{\'e}cursif enracin{\'e} en $S_1$. Le nombre de fa{\c
c}ons de construire cet arbre est :
\[ T_{\text{compl{\'e}ment}} = (N - k - 2) ! \]


Le nombre total d'arbres favorables est :
\[ \text{Nombre d'arbres favorables} = \binom{N - 2}{k} \times
   T_{\text{sous-arbre}} \times T_{\text{compl{\'e}ment}} \]


D'o{\`u}, la probabilit{\'e} que $S_2$ ait exactement $k$ descendants (directs
ou indirects), est :
\begin{eqnarray*}
  P & \assign & \frac{\binom{N - 2}{k} \times k! \times (N - k - 2) !}{(N - 1)
  !}\\
  & = & \frac{(N - 2) !}{(N - 1) !}\\
  & = & \frac{1}{N - 1}
\end{eqnarray*}


3. Notons $L_N$ le nombre de feuilles dans un arbre al{\'e}atoire de $N$
sommets.

Soit $\sigma = (\sigma_1, \ldots, \sigma_{N - 1})$ une permutation sur $\{2,
\ldots, N\}$. On peut construire un arbre r{\'e}cursif avec les n{\oe}uds $1,
2, \ldots, N$ en prenant 1 comme racine, et en attachant le n{\oe}ud $i \geq
2$ au n{\oe}ud le plus {\`a} droite $j$ de $\sigma$ qui pr{\'e}c{\`e}de $i$ et
qui est inf{\'e}rieur {\`a} $i$. S'il n'existe pas un tel {\'e}l{\'e}ment $j$,
alors on d{\'e}finit la racine 1 comme le parent de $i$.

De la construction de l'arbre ci-dessus, $L_N$ peut {\^e}tre d{\'e}fini par :
\begin{equation}
  L_N = \sum_{i = 1}^{N - 2} I\{\sigma_i > \sigma_{i + 1} \}+ 1.
\end{equation}


Ceci est obtenu en observant que chaque apparition de descentes dans $\sigma$
signifie qu'une feuille sera ajout{\'e}e {\`a} l'arbre $T_n$. De plus, le
dernier {\'e}l{\'e}ment $\sigma_{n - 1}$ de $\sigma$ est toujours une feuille.

$\mathbb{E}[L_N] = N / 2$ d{\'e}coule du fait que $\mathbb{P}(I\{\sigma_i >
\sigma_{i + 1} \}= 1) = 1 / 2$.

De plus,
\begin{eqnarray*}
  \mathbb{E}[L_N^2] & = & \mathbb{E} \left[ \left( \sum_{i = 1}^{N - 2}
  I\{\sigma_i > \sigma_{i + 1} \}+ 1 \right)^2 \right]\\
  & = & \mathbb{E} \left[ 3 \sum_{i = 1}^{N - 2} I\{\sigma_i > \sigma_{i + 1}
  \}+ \sum_{1 \leqslant i \neq j \leqslant N - 2} I\{\sigma_i > \sigma_{i + 1}
  \}I\{\sigma_j > \sigma_{j + 1} \}+ 1 \right]
\end{eqnarray*}


Or, pour tout $i, j \in \llbracket 1, N - 2 \rrbracket$, on a :
\[ \mathbb{P}(I\{\sigma_i > \sigma_{i + 1} \}|I\{\sigma_j > \sigma_{j + 1} \}=
   1) = \left\{\begin{array}{ll}
     1 / 6 & \text{si } |i - j| = 1,\\
     1 / 4 & \text{si } |i - j| > 1
   \end{array}\right. \]


Ainsi, on obtient :

\[  \]
\begin{eqnarray*}
  \mathbb{E}[L_N^2] & = & \frac{3 (N - 2)}{2} + \frac{(N - 3) (N - 4)}{4} +
  \frac{2 (N - 3)}{6} + 1\\
  & = & \frac{N^2}{4} + \frac{N}{12} 
\end{eqnarray*}


Par cons{\'e}quent,
\[ \tmop{Var} [L^2_N] = \frac{N}{12} \]

\[ \maltese \maltese \maltese \maltese \maltese \maltese \maltese \]
