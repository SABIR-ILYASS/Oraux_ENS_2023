Ce probl{\`e}me explore diff{\'e}rentes versions du th{\'e}or{\`e}me du point
fixe pour des applications contractantes et non expansives dans
$\mathbb{R}^n$.

\begin{exercise}[]
On consid{\`e}re l'espace vectoriel $\mathbb{R}^n$ {\'e}quip{\'e} de la norme
euclidienne $\| . \|$.

Une application $A : \mathbb{R}^n \rightarrow \mathbb{R}^n$ est dite
contractante s'il existe $k < 1$ tel que, pour tous $x, y \in \mathbb{R}^n$,
$\| A (x) - A (y) \| \leqslant k \| x - y \|$, et non expansive si $\| A (x) -
A (y) \| \leqslant \| x - y \|$ pour tous $x, y \in \mathbb{R}^n$.

1. Montrer qu'une application contractante admet un unique point fixe.

2. Soient $A : \mathbb{R}^n \rightarrow \mathbb{R}^n$ une application non
expansive et $K \subset \mathbb{R}^n$ un sous-ensemble convexe, ferm{\'e} et
born{\'e} tel que $A (K) \subseteq K$. Montrer que $A$ admet un point fixe.

3. Soit $A : \mathbb{R}^n \rightarrow \mathbb{R}^n$ une application non
expansive. Montrer que pour tout $R > 0$, l'application
\[ \tilde{A} (x) = \min (1, R \parallel A (x) \parallel^{- 1}) A (x) \]
est non expansive et admet un point fixe.

4. Soit $A : \mathbb{R}^n \rightarrow \mathbb{R}^n$ une application non
expansive. On suppose que l'ensemble

$S : =\{x \in \mathbb{R}^n : \exists \lambda \in [0, 1], x = \lambda A (x)\}$
est born{\'e}. Montrer que $A$ admet un point fixe.

\end{exercise}

\subsection*{Solution. (SABIR Ilyass)}
\addcontentsline{toc}{subsection}{Solution. (SABIR Ilyass)}


1. Soit $x_0 \in \mathbb{R}^n$ un point arbitraire, et d{\'e}finissons la
suite $(x_n)$ par r{\'e}currence :
\[ x_{n + 1} = A (x_n) . \]


Montrons que la suite $(x_n)_{n \in \mathbb{N}}$ converge.

On a pour tout $n \geqslant 1$,
\begin{eqnarray*}
  \|x_{n + 1} - x_n \| & = & \|A (x_n) - A (x_{n - 1})\|\\
  & \leqslant & k\|x_n - x_{n - 1} \|
\end{eqnarray*}


Par r{\'e}currence, on obtient pour tout $n \in \mathbb{N}$:
\[ \|x_{n + 1} - x_n \| \leq k^n \|x_1 - x_0 \| \]


Pour tout $p > n$, consid{\'e}rons $\|x_p - x_n \|$ :
\begin{eqnarray*}
  \|x_p - x_n \| & \leqslant &  \sum_{i = n}^{p - 1} \|x_{i + 1} - x_i \|\\
  & \leqslant & \|x_1 - x_0 \| \sum_{i = n}^{p - 1} k^i \\
  & = & \|x_1 - x_0 \| \frac{k^n - k^p}{1 - k}\\
  & \leqslant & \frac{k^n}{1 - k} \|x_1 - x_0 \|
\end{eqnarray*}


Comme $k \in [0, 1 [$, $k^n  \underset{n \to \infty}{\to} 0$. Donc, la suite
$(x_n)_{n \in \mathbb{N}}$ est de Cauchy. Puisque $\mathbb{R}^n$ est complet,
la suite $(x_n)_{n \in \mathbb{N}}$ converge vers un point $x^{\ast} \in
\mathbb{R}^n$.

Par continuit{\'e} de $A$ (car elle est lipschitzienne), on a :
\[ x^{\ast} = \underset{n \to \infty}{\lim}  x_n = \underset{n \to
   \infty}{\lim} A (x_{n - 1}) = A \left( \underset{n \to \infty}{\lim} x_{n -
   1} \right) = A (x^{\ast}) . \]


Donc, $x^{\ast}$ est un point fixe de $A$.

\tmtextbf{Unicit{\'e} du point fixe :}

Supposons qu'il existe un autre point fixe $y^{\ast}$ tel que $A (y^{\ast}) =
y^{\ast}$. Alors :
\[ \|x^{\ast} - y^{\ast} \|=\|A (x^{\ast}) - A (y^{\ast})\| \leq k\|x^{\ast} -
   y^{\ast} \|. \]


Ainsi,
\[ (1 - k)\|x^{\ast} - y^{\ast} \| \leq 0. \]


Comme $k < 1$, on a $\|x^{\ast} - y^{\ast} \|= 0$, donc $x^{\ast} = y^{\ast}$.

En conclusion, l'application $A$ admet un unique point fixe dans
$\mathbb{R}^n$.

\

2. Soient $A : \mathbb{R}^n \rightarrow \mathbb{R}^n$ une application non
expansive et $K \subset \mathbb{R}^n$ un sous-ensemble convexe, ferm{\'e} et
born{\'e} tel que $A (K) \subset K$. Montrer que $A$ admet un point fixe.

Comme $K$ est un sous-ensemble ferm{\'e} et born{\'e} de $\mathbb{R}^n$. Il
est donc compact d'apr{\`e}s le th{\'e}or{\`e}me de Heine-Borel.

Or, $A$ est une application non expansive, c'est-{\`a}-dire que pour tous $x,
y \in \mathbb{R}^n$, on a :
\[ \|A (x) - A (y)\| \leq \|x - y\|. \]


Cela implique que $A$ est lipschitzienne. Par cons{\'e}quent, $A$ est continue
sur $\mathbb{R}^n$, et en particulier sur $K$.

Le th{\'e}or{\`e}me du point fixe de Brouwer (voir le lemme 1) affirme que
toute application continue d'un compact convexe non vide de $\mathbb{R}^n$
dans lui-m{\^e}me poss{\`e}de au moins un point fixe.

Comme $K$ est compact, convexe et non vide (puisque tout compact dans
$\mathbb{R}^n$ est non vide), et que $A (K) \subset K$, on peut appliquer le
th{\'e}or{\`e}me de Brouwer {\`a} l'application $A$ restreinte {\`a} $K$.

\tmtextbf{Lemme 1. (Th{\'e}or{\`e}me du point fixe de Brouwer)}

Soit $K \subset \mathbb{R}^n$ un compact convexe non vide. Toute application
continue $f : K \rightarrow K$ admet au moins un point fixe.

\tmtextbf{Preuve du lemme 1.}

Supposons par l'absurde que $f$ n'a pas de point fixe sur $K$. Alors, pour
tout $x \in K$, $f (x) \neq x$.

D{\'e}finissons l'application $g : K \rightarrow \partial K$ (o{\`u} $\partial
K$ d{\'e}signe le bord de $K$) par :
\[ g (x) = x + \lambda (x) [f (x) - x], \]


o{\`u} $\lambda (x) > 0$ est choisi de sorte que $g (x) \in \partial K$.

Puisque $K$ est convexe, le segment $[x, f (x)]$ est contenu dans $K$. Comme
$f (x) \neq x$, le vecteur $f (x) - x$ est non nul. En prolongeant ce segment
au-del{\`a} de $f (x)$, nous sortons de $K$ car $K$ est compact. Donc, il
existe un scalaire $\lambda (x) > 0$ tel que $g (x) = x + \lambda (x) [f (x) -
x]$ appartient {\`a} $\partial K$.

L'application $g$ est continue car elle compos{\'e}e de fonctions continues.
Elle envoie $K$ dans $\partial K$.

Il est connu qu'il n'existe pas de r{\'e}traction continue d'un compact
convexe $K$ de $\mathbb{R}^n$ sur son bord $\partial K$ (c'est une
propri{\'e}t{\'e} topologique des espaces contractiles). En effet, cela
violerait les propri{\'e}t{\'e}s homologiques ou homotopiques de $K$.

Donc, l'application $f$ admet au moins un point fixe dans $K$.

\

3. Soit $R > 0$, montrons que
\[ \tilde{A} (x) = \min \left( 1, \dfrac{R}{\|A (x)\|} \right) A (x) \]


est non expansive

Soient $x, y \in \mathbb{R}^n$,

\tmtextbf{Cas 1 : }Si $\|A (x)\|< R$ et $\|A (y)\|< R$, on a alors
\begin{eqnarray*}
  \| \tilde{A} (x) - \tilde{A} (y)\| & = & \|A (x) - A (y) \|\\
  & = & \|x - y\|
\end{eqnarray*}


\tmtextbf{Cas 2 :} Si $\|A (x)\| \geq R$ et $\|A (y)\| \geq R$, on a alors
\begin{eqnarray*}
  \frac{1}{R^2} \| \tilde{A} (x) - \tilde{A} (y)\|^2 & = & \| \dfrac{1}{\|A
  (x)\|} A (x) - \dfrac{1}{\|A (y)\|} A (y) \|^2\\
  & = & 2 - \frac{2}{\|A (x)\|\|A (y)\|} \langle A (x), A (y) \rangle \qquad
\end{eqnarray*}


Pour conclure, il suffit de montrer que :
\[ 2 - \frac{2}{\|A (x)\|\|A (y)\|} \langle A (x), A (y) \rangle \leqslant
   \frac{1}{R^2} \|A (x) - A (y)\|^2 \quad (1) \]


En posant $\alpha =\|A (x)\| \geq R$, $\beta =\|A (y)\| \geq R$ et $\cos
(\theta) = \frac{\langle A (x), A (y) \rangle}{\|A (x)\|\|A (y)\|}$, pour
montrer $(1)$, cela revient {\`a} montrer que
\[ \frac{\alpha^2}{R^2} + \frac{\beta^2}{R^2} + 2 \cos (\theta) \left( 1 -
   \frac{\alpha \beta}{R^2} \right) \geqslant 2 \]


Puisque $1 - \frac{\alpha \beta}{R^2} < 0$, il suffit de montrer que
\[ \frac{\alpha^2}{R^2} + \frac{\beta^2}{R^2} - 2 \left( 1 - \frac{\alpha
   \beta}{R^2} \right) \geqslant 2 \]


Ce qui est {\'e}quivalent {\`a} $(\alpha + \beta)^2 \geqslant 4 R^2$ et ceci
est vrai car $\alpha, \beta \geqslant R$.

\tmtextbf{Cas 3 :} Si $\|A (x)\| \geq R$ et $\|A (y)\|< R$ (respectivement si
$\|A (x)\|< R$ et $\|A (y)\| \geqslant R$), par sym{\'e}trie, on peut supposer
sans perte de g{\'e}n{\'e}ralit{\'e} que $\|A (x)\| \geq R$ et $\|A (y)\|< R$,
on a alors
\begin{eqnarray*}
  \| \tilde{A} (x) - \tilde{A} (y)\|^2 & = & \| \dfrac{R}{\|A (x)\|} A (x) - A
  (y)\|^2\\
  & = & R^2 +\|A (y)\|^2 - 2 \dfrac{R}{\|A (x)\|}  \langle A (x), A (y)
  \rangle
\end{eqnarray*}


En posant $\alpha =\|A (x)\| \geq R$, $\beta =\|A (y)\|< R$ et $\cos (\theta)
= \frac{\langle A (x), A (y) \rangle}{\|A (x)\|\|A (y)\|}$, pour montrer que
\[ R^2 +\|A (y)\|^2 - 2 \dfrac{R}{\|A (x)\|}  \langle A (x), A (y) \rangle
   \leqslant \|A (x) - A (y) \|^2 \]


Il suffit de montrer que
\[ R^2 + 2 \beta (\alpha - R) \cos (\theta) \leqslant \alpha^2 \]


Puisque $\alpha - R > 0$ et $\beta < R$, il suffit de montrer que
\[ R^2 + 2 R (\alpha - R) \leqslant \alpha^2 \]


Cette in{\'e}galit{\'e} est {\'e}quivalente {\`a} $(\alpha - R)^2 \geqslant
0$.

\

D'o{\`u} pour tout $x, y \in \mathbb{R}^n$, on a
\[ \| \tilde{A} (x) - \tilde{A} (y)\| \leqslant \|x - y\| \]


Ainsi $\tilde{A}$ est non expansive.

L'application $\tilde{A}$ est continue (car combinaison de fonctions
continues) et {\`a} valeurs dans la boule ferm{\'e}e $B (0, R$) de
$\mathbb{R}^n$, c'est-{\`a}-dire $\| \tilde{A} (x)\| \leq R$ pour tout $x \in
\mathbb{R}^n$.

Or, $B (0, R$)est sous-ensemble de $\mathbb{R}^n$ convexe, ferm{\'e} et
born{\'e}, donc d'apr{\`e}s la question pr{\'e}c{\'e}dente, $\tilde{A}$ admet
un point fixe.

\

4. Pour chaque $\lambda \in [0, 1 [\nobracket$, d{\'e}finissons l'application
$T_{\lambda} : \mathbb{R}^n \rightarrow \mathbb{R}^n$ par :
\[ T_{\lambda} (x) = \lambda A (x) . \]


Pour tout $x, y \in \mathbb{R}^n$,
\[ \begin{aligned}
     \|T_{\lambda} (x) - T_{\lambda} (y)\| & =\| \lambda A (x) - \lambda A
     (y)\|\\
     & = \lambda \|A (x) - A (y)\|\\
     & \leq \lambda \|x - y\|.
   \end{aligned} \]


Puisque $\lambda \in [0, 1 [\nobracket$, alors $T_{\lambda}$ est une
application contractante avec une constante de contraction $\lambda < 1$.

Ainsi, d'apr{\`e}s la question 1, $T_{\lambda}$ admet un unique point fixe
$x_{\lambda}$ tel que
\[ x_{\lambda} = T_{\lambda} (x_{\lambda}) = \lambda A (x_{\lambda}) \]


Ainsi, pout tout $\lambda \in [0, 1 [\nobracket,$ il existe $x_{\lambda} \in
\mathbb{R}^n$ tel que $x_{\lambda} = \lambda A (x_{\lambda})$.

En particulier, pour tout $n \in \mathbb{N}^{\ast}$, il existe $x_n \in
\mathbb{R}^n$, tel que $x_n = \left( 1 - \frac{1}{n} \right) A (x_n)$.

Puisque pour tout $n \in \mathbb{N}$, $x_n \in S$ et que $S$ est born{\'e},
alors $(x_n)_{n \in \mathbb{N}}$ est born{\'e}e,

D'apr{\`e}s le th{\'e}or{\`e}me de Bolzano-Weierstrass, la suite born{\'e}e
$(x_n)$ poss{\`e}de une sous-suite convergente $(x_{\varphi (n)})_{n \in
\mathbb{N}}$. Notons $x$ sa limite. (o{\`u} $\varphi : \mathbb{N} \rightarrow
\mathbb{N}$ strictement croissante).

Ainsi, pour tout $n \in \mathbb{N}$
\[ x_{\varphi (n)} = \left( 1 - \frac{1}{\varphi (n)} \right) A (x_{\varphi
   (n)}) \]


Puisque $A$ est non expensive, elle est lipschitizienne, en particulier elle
esr continue sur $\mathbb{R}^n$. Par passage {\`a} la limite $n \rightarrow +
\infty$, on trouve
\[ x = A (x) \]


D'o{\`u} le r{\'e}sultat.
\[ \maltese \maltese \maltese \maltese \maltese \maltese \maltese \]
