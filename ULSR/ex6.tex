Cet exercice traite d'une in{\'e}galit{\'e} matricielle classique, connue sous
le nom de th{\'e}or{\`e}me de Schur-Horn. Il teste la compr{\'e}hension des
candidats sur les propri{\'e}t{\'e}s des matrices sym{\'e}triques et leur
capacit{\'e} {\`a} manipuler des in{\'e}galit{\'e}s impliquant des valeurs
propres.

\begin{exercise}[]
Soit $A = (a_{i, j})_{1 \leqslant i, j \leqslant n} \in S_n (\mathbb{R})$ une
matrice r{\'e}elle sym{\'e}trique. On note $\lambda_1 (A) \leqslant \cdots
\leqslant \lambda_n (A)$ les valeurs propres de $A$. Montrer pour tout $k \in
\{ 1, \ldots, n \}$ l'in{\'e}galit{\'e}
\[ \lambda_1 (A) + \cdots + \lambda_k (A) \leqslant a_{1, 1} + \cdots + a_{k,
   k} \leqslant \lambda_{n - k + 1} (A) + \cdots + \lambda_n (A) \]
\end{exercise}

\subsection*{Solution. (SABIR Ilyass)}
\addcontentsline{toc}{subsection}{Solution. (SABIR Ilyass)}


Pour tout $A \in S_n (\mathbb{R})$. Remarquons d'abord que si $\lambda_1 (A)
\leq \ldots \leq \lambda_n (A)$ sont les valeurs propres de $A$, alors les
valeurs propres de $- A \in S_n (\mathbb{R})$ sont $- \lambda_n (A) \leq
\ldots \leq - \lambda_1 (A)$.

Donc, si
\[ \lambda_1 (A) + \ldots + \lambda_k (A) \leq a_{1, 1} + \ldots + a_{k, k}
\]
pour tout $k \in \{1, \ldots, n\}$ et pour toute matrice $A \in S_n
(\mathbb{R})$, alors, puisque $- A \in S_n (\mathbb{R})$, on a pour tout $k
\in \{1, \ldots, n\}$ :
\[ - \lambda_{n - k + 1} (A) - \ldots - \lambda_n (A) \leq - a_{1, 1} - \ldots
   - a_{k, k} . \]


Donc, il suffit de montrer que $\lambda_1 (A) + \ldots + \lambda_k (A) \leq
a_{1, 1} + \ldots + a_{k, k}$ pour tout $k \in \llbracket 1, n \rrbracket$

\tmtextbf{D{\'e}finition 1.}

Soit $m \in \mathbb{N}^{\ast}$ tel que $m \leq n$, et soit $V$ un sous-espace
vectoriel de $\mathbb{R}^n$ de dimension $m$. On d{\'e}finit Tr$(A|_V)$ comme
suit :
\[ \text{Tr} (A|_V) \assign \sum_{i = 1}^m v_i^T Av_i \]


o{\`u} $v_1, \ldots, v_m$ constituent une base orthonorm{\'e}e quelconque de
$V$. Il est facile de voir que cette expression est ind{\'e}pendante du choix
de la base orthonorm{\'e}e, et donc Tr$(A|_V)$ est bien d{\'e}finie.

\tmtextbf{Lemme 1.}

Pour tout $1 \leq k \leq n$, on a :
\[ \lambda_1 (A) + \ldots + \lambda_k (A) =   \underset{\dim (V) =
   k}{\underset{V \subset \mathbb{R}^n}{\sup}} \text{Tr} (A|_V) \]


\tmtextbf{Preuve du lemme 1.}

Soient $e_1, \ldots, e_n$ une base orthogonale form{\'e}e par les vecteurs
propres associ{\'e}s {\`a} $\lambda_1 (A), \ldots, \lambda_n (A)$
respectivement (en appliquant le th{\'e}or{\`e}me spectral). On a :
\[ \lambda_1 (A) + \ldots + \lambda_k (A) = \text{Tr} (A|_{\text{vect} (e_1,
   \ldots, e_k)}) \]


et
\[ \lambda_1 (A) + \ldots + \lambda_k (A) \leq \underset{\dim (V) =
   k}{\underset{V \subset \mathbb{R}^n}{\sup}} \text{Tr} (A|_V) . \]


Il reste {\`a} montrer que $\lambda_1 (A) + \ldots + \lambda_k (A) \geq
\underset{\dim (V) = k}{\underset{V \subset \mathbb{R}^n}{\sup}}  \text{Tr}
(A|_V)$.

Montrons par r{\'e}currence sur $n \in \mathbb{N}^{\ast}$ que :
\[ (\mathcal{P}_n) : \forall A \in S_n (\mathbb{R}), \forall k \in \{1,
   \ldots, n\}, \lambda_1 (A) + \ldots + \lambda_k (A) \geq \underset{\dim (V)
   = k}{\underset{V \subset \mathbb{R}^n}{\sup}}  \text{Tr} (A|_V) . \]


C'est trivial pour $n = 1$. Supposons que $\mathcal{P}_n$ soit vraie et
montrons $\mathcal{P}_{n + 1}$.

Soit $A \in S_{n + 1} (\mathbb{R})$ et $k \in \{1, \ldots, n + 1\}$. Notons
$(e_1, \ldots, e_{n + 1})$ une base orthonormale form{\'e}e par les vecteurs
propres de $A$, associ{\'e}e aux valeurs propres $\lambda_1 (A), \ldots,
\lambda_{n + 1} (A)$ respectivement.

\tmtextbf{Si $k = 1$}, on a pour tous $\beta_1, \ldots, \beta_{n + 1}$ non
tous nuls, et pour $v = \frac{1}{\sqrt{\beta_1^2 + \ldots + \beta_{n + 1}^2}} 
\sum_{k = 1}^{n + 1} \beta_k e_k$, on a :
\[ \text{Tr} (A|_{\text{vect} (v)}) = v^T Av = \frac{1}{\beta_1^2 + \ldots +
   \beta_{n + 1}^2}  \sum_{k = 1}^{n + 1} \beta_k^2 \lambda_k (A) \]


et
\[ \text{Tr} (A|_{\text{vect} (v)}) \leq \frac{1}{\beta_1^2 + \ldots +
   \beta_{n + 1}^2}  \sum_{k = 1}^{n + 1} \beta_k^2 \lambda_1 (A) = \lambda_1
   (A) . \]


Donc,
\[ \lambda_1 (A) \geq \sup_{\tmscript{\begin{array}{c}
     V \subset \mathbb{R}^{n + 1}\\
     \dim (V) = 1
   \end{array}}}  \text{Tr} (A|_V) . \]


\tmtextbf{Si $k > 1$}, soit $V$ un sous-espace vectoriel de $\mathbb{R}^{n +
1}$ de dimension $k$. Alors $V$ contient un sous-espace $V'$ de dimension $k -
1$ inclus dans vect$(e_2, \ldots, e_{n + 1})$. En appliquant l'hypoth{\`e}se
de r{\'e}currence {\`a} la restriction de $A$ {\`a} vect$(e_2, \ldots, e_{n +
1})$, qui a pour valeurs propres $\lambda_2 (A), \ldots, \lambda_{n + 1} (A)$,
on obtient :
\[ \lambda_2 (A) + \ldots + \lambda_k (A) \geq \text{Tr} (A|_{V'}) . \]


D'autre part, si $v$ est un vecteur unitaire dans le compl{\'e}ment orthogonal
de $V'$ dans $V$, on voit {\`a} partir du cas $k = 1$ que :
\[ \lambda_1 (A) \geq v^T Av. \]


Par suite, on a :
\[ \lambda_1 (A) + \ldots + \lambda_k (A) \geq \text{Tr} (A|_V) . \]


D'o{\`u} :
\[ \lambda_1 (A) + \ldots + \lambda_k (A) \geq \underset{\dim (V) =
   k}{\underset{V \subset \mathbb{R}^n}{\sup}}  \text{Tr} (A|_V) . \]


Cela conclut la preuve par r{\'e}currence sur $n \in \mathbb{N}^{\ast}$.

\

En appliquant ce lemme {\`a} $e_i = {(0, \ldots, 0, 1, 0, \ldots, 0 )^T}
,$o{\`u} $1$est {\`a} la i-{\`e}me position pour tout $i = 1, \ldots, n$, on
obtient :
\begin{eqnarray*}
  \lambda_1 (A) + \ldots + \lambda_k (A) & = & \underset{\dim (V) =
  k}{\underset{V \subset \mathbb{R}^n}{\sup}}  \text{Tr} (A|_V)\\
  & \geqslant & \text{Tr} (A|_{\{ e_i \}_{1 \leqslant i \leqslant n}})\\
  & = & \sum_{i = 1}^k e_i^T Ae_i\\
  & = & a_{1, 1} + \cdots + a_{k, k}
\end{eqnarray*}


D'o{\`u} le r{\'e}sultat.
\[ \maltese \maltese \maltese \maltese \maltese \maltese \maltese \]
