Ce probl{\`e}me examine les conditions sous lesquelles on peut d{\'e}duire
l'existence d'une limite pour une fonction {\`a} partir d'informations sur ses
d{\'e}riv{\'e}es successives.
\begin{exercise}[]
On consid{\`e}re un entier $k \geqslant 1$ et une fonction $f \in
\mathcal{C}^k (\mathbb{R}, \mathbb{R})$ telle que $\underset{j =
0}{\overset{k}{\sum}} f^{(j)} (x)$ admet une limite pour $x \rightarrow +
\infty$. Peut-on en d{\'e}duire que $f$ admet une limite en $x \rightarrow +
\infty$, selon le valeur de $k$?
\end{exercise}

\subsection*{Solution. (SABIR Ilyass)}
\addcontentsline{toc}{subsection}{Solution. (SABIR Ilyass)}


Soient $k \geq 1$ et $f \in C^k (\mathbb{R}, \mathbb{R})$ telle que
\[ S (x) = \sum_{j = 0}^k f^{(j)} (x) \]
admet une limite finie $L$ lorsque $x \to + \infty$.

\tmtextbf{Cas $k = 1$ :}

On a
\[ S (x) = f (x) + f' (x) \underset{x \to + \infty}{\to} L \]


Consid{\'e}rons l'{\'e}quation diff{\'e}rentielle :
\[ f' (x) + f (x) = S (x) . \]


C'est une {\'e}quation diff{\'e}rentielle lin{\'e}aire du premier ordre. Son
facteur int{\'e}grant est $\mu (x) = e^x$. En multipliant les deux membres par
$e^x$, on obtient :
\[ e^x f' (x) + e^x f (x) = e^x S (x), \]


ce qui {\'e}quivaut {\`a} :
\[ \frac{d}{dx} (e^x f (x)) = e^x S (x) . \]


En int{\'e}grant entre un point $x_0$ et $x$, on obtient :
\[ e^x f (x) = \int_{x_0}^x e^t S (t) \hspace{0.17em} dt + C. \]


O{\`u} $C$ est une constante.

Comme $S (x) \to L$ lorsque $x \to + \infty$, nous pouvons {\'e}crire $S (x) =
L + \varepsilon (x)$, avec $\varepsilon (x) \underset{x \rightarrow +
\infty}{\to} 0$.

Ainsi,
\[ \int_{x_0}^x e^t S (t) \hspace{0.17em} dt = Le^x - Le^{x_0} + \int_{x_0}^x
   e^t \varepsilon (t) \hspace{0.17em} dt. \]


Donc,
\[ e^x f (x) = Le^x - Le^{x_0} + \int_{x_0}^x e^t \varepsilon (t)
   \hspace{0.17em} dt + C, \]


D'o{\`u}
\[ f (x) = L + e^{- x}  (- Le^{x_0} + C) + e^{- x}  \int_{x_0}^x e^t
   \varepsilon (t) \hspace{0.17em} dt \]


Or, $e^{- x}  (- Le^{x_0} + C)  \underset{x \rightarrow + \infty}{\to} 0$, et
puisque $\varepsilon (x) \underset{x \rightarrow + \infty}{\to} 0$, alors pour
tout $\kappa > 0$, il existe $M \in \mathbb{R}$ tel que pour tout $x \geqslant
M$, on a
\[ | \varepsilon (x) | \leqslant \frac{\kappa}{2} \]


Ainsi pour tout $x \geqslant \max (M, x_0)$, on a
\begin{eqnarray*}
  \left| e^{- x}  \int_{x_0}^x e^t \varepsilon (t) \hspace{0.17em} dt \right|
  & \leqslant & e^{- x}  \int_{x_0}^x e^t  | \varepsilon (t) | \hspace{0.17em}
  dt\\
  & \leqslant & e^{- x}  \int_{x_0}^M e^t  | \varepsilon (t) |
  \hspace{0.17em} dt + \frac{\kappa}{2} e^{- x} \int_M^x e^t  \hspace{0.17em}
  dt\\
  & = & \frac{\kappa}{2} + e^{- x} \left( \int_{x_0}^M e^t  | \varepsilon (t)
  | \hspace{0.17em} dt - \frac{\kappa}{2} e^M \right)
\end{eqnarray*}


Avec, $e^{- x} \left( \int_{x_0}^M e^t  | \varepsilon (t) | \hspace{0.17em} dt
- \frac{\kappa}{2} e^M \right) \underset{x \rightarrow + \infty}{\to} 0$,
alors il existe $M' \in \mathbb{R}$ tel que pour tout $x \geqslant M'$
\[ e^{- x} \left( \int_{x_0}^M e^t  | \varepsilon (t) | \hspace{0.17em} dt -
   \frac{\kappa}{2} e^M \right) \leqslant \frac{\kappa}{2} \]


Ainsi, pour tout $x \geqslant \max (M, M', x_0)$, on a
\[ \left| e^{- x}  \int_{x_0}^x e^t \varepsilon (t) \hspace{0.17em} dt \right|
   \leqslant \kappa \]


D'o{\`u} $e^{- x}  \int_{x_0}^x e^t \varepsilon (t) \hspace{0.17em} dt
\underset{x \rightarrow + \infty}{\to} 0$ et par cons{\'e}quent $f (x)
\underset{x \rightarrow + \infty}{\to} L$.

\

D'o{\`u} $f$ admet une limite finie lorsque $x \to + \infty$.

\

\tmtextbf{Cas $k = 2$ :}

On a
\[ S (x) = f (x) + f' (x) + f'' (x) \underset{x \rightarrow + \infty}{\to} L
\]


Notons $Y = \left( \begin{array}{c}
  f\\
  f'
\end{array} \right)$, on a alors
\[ Y' + \left( \begin{array}{cc}
     0 & - 1\\
     1 & 1
   \end{array} \right) Y = \left( \begin{array}{c}
     0\\
     S
   \end{array} \right) \]


Posons $A = \left( \begin{array}{cc}
  0 & - 1\\
  1 & 1
\end{array} \right)$, on a alors
\[ \frac{d}{d x} (e^{A x} Y (x)) = e^{A x} \left( \begin{array}{c}
     0\\
     S (x)
   \end{array} \right) \]


Par suite
\[ Y (x) = e^{- A x} \int e^{A x} \left( \begin{array}{c}
     0\\
     S (x)
   \end{array} \right) d x + C \]
\[ \  \]


Par une approche similaire au cas $k = 1$, on peut montrer facilement que $Y
(x) \underset{x \rightarrow + \infty}{\rightarrow} \left( \begin{array}{c}
  L\\
  0
\end{array} \right)$

En particulier que $f (x) \underset{x \rightarrow + \infty}{\rightarrow} L$.

\

\tmtextbf{Cas $k \geq 3$ :}

Consid{\'e}rons l'{\'e}quation diff{\'e}rentielle
\[ \sum_{j = 0}^k f^{(j)} (x) = 0 \quad  (\maltese) \]


Il s'agit d'une {\'e}quation diff{\'e}rentielle lin{\'e}aire d'{\'e}quation
caract{\'e}ristique
\[ \sum_{j = 0}^k r^j = 0 \]


L'ensemble des solutions de l'{\'e}quation caract{\'e}ristique est
\[ \left\{ \exp \left( 2 i \pi \frac{l}{k + 1} \right) | l = 1, 2, \ldots, k +
   1 \nobracket \right\} \]


En prenant $r = \exp \left( \frac{2 i \pi}{k + 1} \right)$, on a alors
\[ h : x \longmapsto \mathcal{R}e (\exp (r x)) \]


est une solution de $(\maltese) .$

Or, pour tout $x \in \mathbb{R},$ on a
\begin{eqnarray*}
  h (x) & = & \mathcal{R}e \left( \exp \left( \exp \left( \frac{2 i \pi}{k +
  1} \right) x \right) \right)\\
  & = & \mathcal{R}e \left( \exp \left( \cos \left( \frac{2 \pi}{k + 1}
  \right) x + i \sin \left( \frac{2 \pi}{k + 1} \right) x \right) \right)\\
  & = & \exp \left( \cos \left( \frac{2 \pi}{k + 1} \right) x \right) \cos
  \left( \sin \left( \frac{2 \pi}{k + 1} \right) x \right)
\end{eqnarray*}


Or, $k \geqslant 3$, alors $\cos \left( \frac{2 \pi}{k + 1} \right) > 0$ et
$\sin \left( \frac{2 \pi}{k + 1} \right) > 0$, donc $h$ n'admet pas de limite
en $+ \infty$.


\[ \maltese \maltese \maltese \maltese \maltese \maltese \maltese \]
