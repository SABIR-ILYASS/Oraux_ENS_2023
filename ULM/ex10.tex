Cet exercice porte sur le r{\'e}sultant de deux polyn{\^o}mes. Il s'agit de
prouver certaines propri{\'e}t{\'e}s du r{\'e}sultant, notamment sa
sym{\'e}trie et son lien avec un d{\'e}terminant sp{\'e}cifique. L'exercice
fait appel {\`a} des notions d'alg{\`e}bre lin{\'e}aire et de th{\'e}orie des
polyn{\^o}mes.

\begin{exercise}[(R{\'e}sultant)]
Soient $A, B \in \mathbb{C} [X]$ deux polyn{\^o}mes unitaires, de degr{\'e}s
respectifs $a$ et $b$. Soit $M_{A, B}$ l'endomorphisme de~$\mathbb{C} [X] /
(A)$ d{\'e}fini par~$M_{A, B} ([P]) = [BP]$.

Soit $\mu_{A, B} = \det M_{A, B}$. Montrer que $\mu_{A, B} = (- 1)^{ab}
\mu_{B, A}$.

Soit $F_{A, B}$ l'unique endomorphisme de~$\mathbb{C} [X]_{a + b - 1}$ tel que
pour tout $U \in \mathbb{C} [X]_{a - 1}$ et tout $V \in \mathbb{C} [X]_{b -
1}$, $F_{A, B}  (U + X^a V) = BU + AV$.

Montrer que $\det (F_{A, B}) = \mu_{A, B}$

\textit{Le nombre $\mu_{A, B}$ est le r{\'e}sultant de $A$ et de $B$.}
\end{exercise}

\subsection*{Solution. (SABIR Ilyass)}
\addcontentsline{toc}{subsection}{Solution. (SABIR Ilyass)}

Soit $A, B \in \mathbb{C} [X]$, deux polyn{\^o}mes unitaires de degr{\'e}s
respectifs $a$ et $b$.

Soit $M_{A, B}$ l'endomorphisme de $\mathbb{C} [X] / (A)$ d{\'e}fini par
$M_{A, B} ([P]) = [BP]$ et soit $\mu_{A, B} = \det M_{A, B}$.

Montrons que
\[ \mu_{A, B} = (- 1)^{ab} \mu_{B, A} \]


Pour cela, on va montrer que :
\[ \mu_{A, B} = \underset{i = 1}{\overset{a}{\prod}} \underset{j =
   1}{\overset{b}{\prod}} (\alpha_i - \beta_j)  (\maltese) \]


L'identit{\'e} $(\maltese)$ nous permettra {\'e}galement de traiter la seconde
partie de l'{\'e}nonc{\'e} de l'exercice.

Notons $A = \underset{k = 1}{\overset{a}{\prod}} (X - \alpha_k)$ et $B =
\underset{k = 1}{\overset{a}{\prod}} (X - \beta_k)$ avec $\alpha_1, \ldots,
\alpha_a, \beta_1, \ldots, \beta_b \in \mathbb{C}$.

Supposons dans un premier temps que $\alpha_1, \alpha_2, \ldots, \alpha_a$
(respectivement $\beta_1, \beta_2, \ldots, \beta_b$) sont deux {\`a} deux
distincts.

Pour tout $j \in \llbracket 1, a \rrbracket$, on d{\'e}finit le polyn{\^o}me
$P_j \assign \underset{k \not{=} j}{\underset{k = 1}{\overset{a}{\prod}}}
\frac{X - \alpha_k}{\alpha_j - \alpha_k}$. On a $(P_j)_{1 \leqslant j
\leqslant a}$ est une base de $\mathbb{C} [X] / (A)$. De plus, pour tout $j
\in \llbracket 1, a \rrbracket$, il existe $Q_j \in \mathbb{R} [X]$ tel que
\begin{eqnarray*}
  B.P_j & = & Q_j A + M_{A, B} (P_j)
\end{eqnarray*}


Pour tout $i \in \llbracket 1, a \rrbracket \backslash \{ j \} \nobracket$, on
a $M_{A, B} (P_j) (\alpha_i) = 0$ et $M_{A, B} (P_j) (\alpha_j) = B
(\alpha_j)$

On en d{\'e}duit, par interpolation de Lagrange, que
\[ M_{A, B} (P_j) = B (\alpha_j) P_j \]


Ainsi,
\begin{eqnarray*}
  \mu_{A, B} & = & \det (M_{A, B})\\
  & = & \underset{}{\overset{}{}} \underset{j = 1}{\overset{a}{\prod}} B
  (\alpha_j)\\
  & = & \underset{i = 1}{\overset{a}{\prod}} \underset{j =
  1}{\overset{b}{\prod}} (\alpha_i - \beta_j)
\end{eqnarray*}


Puisque l'ensemble des polyn{\^o}mes sind{\'e}s {\`a} racines simples dense
dans l'ensemble des polyn{\^o}mes scind{\'e}s, et que l'application $(A, B)
\in \mathbb{C} [X]^2 \longmapsto \mu_{A, B}$ est continue (car elle est une
composition d'applications continues).

On a pour $A : = \underset{k = 1}{\overset{a}{\prod}} (X - \alpha_k)$ et $B :
= \underset{k = 1}{\overset{a}{\prod}} (X - \beta_k)$, avec $\alpha_1, \ldots,
\alpha_a, \beta_1, \ldots, \beta_b \in \mathbb{C}$.

On a
\[ \begin{array}{lll}
     \mu_{B, A} & = & \underset{j = 1}{\overset{b}{\prod}} \underset{i =
     1}{\overset{a}{\prod}} (\beta_j - \alpha_i)
   \end{array} \]


En particulier, par sym{\'e}trie :
\begin{eqnarray*}
  \mu_{B, A} & = & \underset{j = 1}{\overset{b}{\prod}} \underset{i =
  1}{\overset{a}{\prod}} (\beta_j - \alpha_i)\\
  & = & \underset{i = 1}{\overset{a}{\prod}} \underset{j =
  1}{\overset{b}{\prod}} (- 1) (\alpha_i - \beta_j)\\
  & = & (- 1)^{a b} \mu_{A, B}
\end{eqnarray*}


Montrons maintenant que $\det (F_{A, B}) = \mu_{A, B}$.

Notons $A = \underset{k = 0}{\overset{a}{\sum}} \lambda_k X^k$ et $B =
\underset{k = 0}{\overset{b}{\sum}} \gamma_k X^k$, o{\`u} $\lambda_0, \ldots,
\lambda_a, \gamma_0, \ldots, \gamma_b \in \mathbb{C}$.

On a alors pour tout $n \in \llbracket 0, a - 1 \rrbracket$,
\begin{eqnarray*}
  F_{A, B} (X^n) & = & B.X^n\\
  & = & \underset{k = 0}{\overset{b}{\sum}} \gamma_k X^{n + k}
\end{eqnarray*}


Et pour tout $n \in \llbracket a, a + b - 2 \rrbracket$
\begin{eqnarray*}
  F_{A, B} (X^n) & = & A.X^{n - a}\\
  & = & \underset{k = 0}{\overset{a}{\sum}} \lambda_k X^{n - a + k}
\end{eqnarray*}


La matrice de $F_{A, B} $ dans la base $\mathcal{B}= (X^n)_{0 \leqslant n
\leqslant a + b - 2}$ est donc :
\[ \tmop{mat}_{\mathcal{B}} (F_{A, B}) = \left( \begin{array}{ccccccccc}
     \lambda_0 & \lambda_1 & \ldots & \lambda_a & 0 & 0 & 0 & \ldots & 0\\
     0 & \lambda_0 & \lambda_1 & \ldots & \lambda_a & 0 & 0 & \ldots & 0\\
     &  &  &  &  &  &  &  & \\
     \vdots &  &  & \ddots &  &  &  &  & \vdots\\
     &  &  &  &  &  &  &  & \\
     0 & 0 & \ldots & 0 & 0 & \lambda_0 & \lambda_1 & \ldots & \lambda_a\\
     \beta_0 & \beta_1 & \ldots & \beta_b & 0 & 0 & 0 & \ldots & 0\\
     0 & \beta_0 & \beta_1 & \ldots & \beta_b & 0 & 0 & \ldots & 0\\
     &  &  &  &  &  &  &  & \\
     \vdots &  &  & \ddots &  &  &  &  & \vdots\\
     &  &  &  &  &  &  &  & \\
     0 & 0 & \ldots & 0 & 0 & \beta_0 & \beta_1 & \ldots & \beta_b
   \end{array} \right)  (\tmop{Matrice} \tmop{de} \tmop{Sylvester}) \]


Pour conclure, il suffit de montrer que $\det (F_{A, B}) = \underset{j =
1}{\overset{b}{\prod}} \underset{i = 1}{\overset{a}{\prod}} (\beta_j -
\alpha_i)$.

Soient $k, l \in \mathbb{C}$. Si l'on remplace $A$ par $kA$ et $B$ par $lB$,
on a :
\[ \det (F_{kA, lB}) = k^b l^a \det (F_{A, B}) \]


En effet, les $b$ premi{\`e}res lignes de la matrice sont multipli{\'e}es par
$k$ (contribution de $k^b$), les $a$ derni{\`e}res lignes sont multipli{\'e}es
par $l$ (contribution de $l^a$).

\

Par homog{\'e}n{\'e}it{\'e}, on peut se ramener au cas o{\`u} $A$ et $B$ sont
unitaires. Il suffit donc de d{\'e}montrer l'{\'e}galit{\'e} pour :
\[ A (X) = \prod_{i = 1}^a (X - \alpha_i) \quad \text{et} \quad B (X) =
   \prod_{j = 1}^b (X - \beta_j) \]


o{\`u} $\alpha_1, \ldots, \alpha_a$ et $\beta_1, \ldots, \beta_b$ sont les
racines respectives de $A$ et $B$.

On a alors $\det (F_{A, B}) \in \mathbb{C}[\alpha_1, \ldots, \alpha_a,
\beta_1, \ldots, \beta_b]$. Pour tous $i \in \llbracket 1, a \rrbracket$ et $j
\in \llbracket 1, b \rrbracket$, $(\beta_j - \alpha_i)$ divise $\det (F_{A,
B})$.

Si $\alpha_i = \beta_j$, alors : $A (X) = (X - \alpha_i) A_1 (X)$ et $B (X) =
(X - \beta_j) B_1 (X)$

On peut construire un vecteur non nul $v$ dans le noyau de $F_{A, B}$ :
\[ v = X^b A_1 - X^a B_1 \]


Ce vecteur est non nul car les degr{\'e}s des polyn{\^o}mes sont
diff{\'e}rents. Donc $F_{A, B}$ n'est pas inversible et $\det (F_{A, B}) = 0$.

Par les points pr{\'e}c{\'e}dents, on peut {\'e}crire :
\[ \det (F_{A, B}) = \lambda \prod_{j = 1}^b \prod_{i = 1}^a (\beta_j -
   \alpha_i) \]


o{\`u} $\lambda$ est une constante dans $\mathbb{C}$.

Pour calculer $\lambda$, consid{\'e}rons le cas particulier o{\`u} :
\[ A (X) = X^a  \quad \text{et} \quad B (X) = 1 \]


Dans ce cas, la matrice $F_{A, B}$ devient triangulaire, et tous les
{\'e}l{\'e}ments diagonaux sont {\'e}gaux {\`a} 1.

Donc $\det (F_{A, B}) = 1$.

Par cons{\'e}quent, $\lambda = 1$.

On a donc d{\'e}montr{\'e} que :
\[ \det (F_{A, B}) = \prod_{j = 1}^b \prod_{i = 1}^a (\beta_j - \alpha_i) =
   \mu_{A, B} \]
\[ \maltese \maltese \maltese \maltese \maltese \maltese \maltese \]
