L'exercice 27 porte sur le th{\'e}or{\`e}me d'Hermite-Sylvester. Il demande
d'{\'e}tudier les propri{\'e}t{\'e}s d'une matrice construite {\`a} partir des
racines d'un polyn{\^o}me, notamment son rang et sa positivit{\'e}. L'exercice
fait appel {\`a} des notions d'alg{\`e}bre lin{\'e}aire et de th{\'e}orie des
polyn{\^o}mes.
\begin{exercise}[(Th{\'e}or{\`e}me d'Hermite-Sylvester)]
Soit~$P \in \mathbb{R} [X]$ un polyn{\^o}me de degr{\'e}~$n \geq 1$.
Soit~$\lambda_1, \ldots, \lambda_n \in \mathbb{C}$ ses racines, avec
multiplicit{\'e}. Soit~$H \in \mathbb{C}^{n \times n}$ la matrice d{\'e}finie
par
\[ H_{i, j} = \sum_{k = 1}^n \lambda_k^{i + j} . \]
Montrer que~$H$ est une matrice sym{\'e}trique r{\'e}elle. Montrer que le rang
de~$H$ est {\'e}gal au nombre de racines distincte, et que~$H$ est positive si
et seulement si toutes les racines sont r{\'e}elles.

\end{exercise}

\subsection*{Solution. (SABIR Ilyass)}
\addcontentsline{toc}{subsection}{Solution. (SABIR Ilyass)}

Soit $P \in \mathbb{R} [X]$ de degr{\'e} $n \geqslant 1$. Soient $\lambda_1,
\ldots, \lambda_n \in \mathbb{C}$ ses racines, avec multiplicit{\'e}.

Montrons que la matrice $H$ est une matrice r{\'e}elle sym{\'e}trique,

Pour tout $i, j \in \llbracket 1, n \rrbracket$, il est clair que \ $H_{i, j}
= H_{j, i}$. Donc $H$ est une matrice sym{\'e}trique.

\

Notons $\alpha_1, \ldots, \alpha_r \in \mathbb{R}$ et $\beta_1,
\overline{\beta_1}, \ldots, \beta_s, \overline{\beta_s} \in \mathbb{C}
\backslash \mathbb{R}$ les racines de $P$, et notons pour tout $i \in
\llbracket 1, r \rrbracket$ et pour tout $j \in \llbracket 1, s \rrbracket$
$\zeta_i$ l'ordre de multiplicit{\'e} de $\alpha_i$ et $\rho_j$ l'ordre de
multiplicit{\'e} de $\beta_j$ et de $\overline{{\beta_j} }$.

On a alors, pour tout $i, j \in \llbracket 1, n \rrbracket$
\begin{eqnarray*}
  H_{i, j} & = & \sum_{k = 1}^r {\zeta_k}  \lambda_k^{i + j} + \sum_{k = 1}^s
  \rho_k (\beta_k^{i + j} + \overline{{\beta_k}  }^{i + j})\\
  & = & \sum_{k = 1}^r {\zeta_k}  \lambda_k^{i + j} + 2 \sum_{k = 1}^s \rho_k
  \tmop{Re} (\beta_k^{i + j})\\
  & \in & \mathbb{R}
\end{eqnarray*}


Donc $H$ est une matrice r{\'e}elle. Montrons que le rang de~$H$ est {\'e}gal
au nombre de racines distincts.

\

Tout d'abord, observons que chaque racine $\lambda_k$ (compt{\'e}e avec
multiplicit{\'e}) d{\'e}finit un vecteur $v_k \in \mathbb{C}^n$ dont les
composantes sont :
\[ v_k = \left(\begin{array}{c}
     \lambda_k^1\\
     \lambda_k^2\\
     \vdots\\
     \lambda_k^n
   \end{array}\right) \]
Alors, $H$ peut s'exprimer comme :
\[ H = \sum_{k = 1}^n v_k v_k^T \]


\

Comme $P$ peut avoir des racines multiples, on peut regrouper les vecteurs
correspondant aux racines identiques.

\

Soient $\lambda^{(1)}, \ldots, \lambda^{(m)}$ les racines distinctes de $P$,
avec des multiplicit{\'e}s $n_1, \ldots, n_m$ (donc $\sum_{i = 1}^m n_i = n$).
D{\'e}finissons :
\[ w_i = \left(\begin{array}{c}
     (\lambda^{(i)})^1\\
     (\lambda^{(i)})^2\\
     \vdots\\
     (\lambda^{(i)})^n
   \end{array}\right) \]


Alors, $H$ devient :
\[ H = \sum_{i = 1}^m n_i w_i w_i^T \]


Comme chaque $w_i w_i^T$ est une matrice de rang $1$, et que les $w_i$
correspondant aux racines distinctes sont lin{\'e}airement ind{\'e}pendants
(car des racines distinctes donnent des vecteurs de puissances
lin{\'e}airement ind{\'e}pendants), le rang de $H$ est {\'e}gal au nombre de
racines distinctes :
\[ \mathrm{rang} (H) = m \]


o{\`u} $m$ est le nombre de racines distinctes de $P$.

\

Pour tout vecteur $x \in \mathbb{R}^n$, on a :
\[ x^T Hx = \sum_{k = 1}^n (x^T v_k)^2 \geq 0 \]


Cela montre que $H$ est semi-d{\'e}finie positive.
\begin{itemize}
  \item \tmtextbf{Si toutes les racines sont r{\'e}elles :}
  
  Les vecteurs $w_i$ sont r{\'e}els et lin{\'e}airement ind{\'e}pendants, donc
  $H$ est d{\'e}finie positive :
  \[ x^T Hx > 0 \quad \text{pour tout } x \in \mathbb{R}^n \setminus \{0\} \]
  \item \tmtextbf{S'il y a des racines complexes :}
  
  Les racines complexes apparaissent en paires conjugu{\'e}es $\lambda,
  \bar{\lambda}$. Les vecteurs correspondants $v_{\lambda}$ et
  $v_{\bar{\lambda}}$ v{\'e}rifient :
  \[ v_{\bar{\lambda}} = \overline{v_{\lambda}} \]
  La contribution {\`a} $H$ d'une paire de conjugu{\'e}s complexes est :
  \[ v_{\lambda} v_{\lambda}^T + v_{\bar{\lambda}} v_{\bar{\lambda}}^T \]
  Cette somme est r{\'e}elle et sym{\'e}trique, mais introduit une
  d{\'e}pendance lin{\'e}aire, ce qui fait que $H$ est seulement
  semi-d{\'e}finie positive. Sp{\'e}cifiquement, il existe des vecteurs non
  nuls $x$ tels que $x^T Hx = 0$, indiquant que $H$ n'est pas d{\'e}finie
  positive.
\end{itemize}


\tmtextbf{Commentaire.}

Pour montrer que $H$ est r{\'e}elle, on peut utiliser aussi le r{\'e}sultat
classique des formules de Newton :

\

\tmtextbf{Lemme 1. (Formules de Newton)}

Soit $N \geqslant 2$ un entier, et $K$ un corps commutatif, $x_1, \ldots, x_N$
des {\'e}l{\'e}ments de $K$. On consid{\`e}re, pour tout $p \in \mathbb{N}$ :
\[ S_p (x_1, \ldots, x_N) \assign \underset{i = 1}{\overset{N}{\sum}} x^p_i \]


\

On note $\sigma_1, \ldots, \sigma_N$ les fonctions sym{\'e}triques
{\'e}l{\'e}mentaires de $x_1, \ldots, x_N$ d{\'e}finies par :
\[ \sigma_k (x_1, \ldots, x_N) = \overset{}{\underset{1 \leqslant j_1 < \cdots
   < j_k \leqslant N}{\sum}}  \underset{l = 1}{\overset{k}{\prod}} x_{j_l},
   \forall k = 1, \ldots, n \]


Pour tout $p \geqslant N$, on a :
\[ S_p (x_1, \ldots, x_N) + \underset{i = 1}{\overset{N}{\sum}} (- 1)^i
   \sigma_i (x_1, \ldots, x_N) S_{p - i} (x_1, \ldots, x_N) = 0 \]


Et pour tout $p \in \llbracket 1, N - 1 \rrbracket$, on a :
\[ S_p (x_1, \ldots, x_N) + \underset{i = 1}{\overset{p - 1}{\sum}} (- 1)^i
   \sigma_i (x_1, \ldots, x_N) S_{p - i} (x_1, \ldots, x_N) + (- 1)^p p
   \sigma_p = 0 \]


\tmtextbf{Preuve du lemme 1.}

Soit $N \geqslant 2$ un entier, $K$ un corps commutatif, $x_1, \ldots, x_N$
des {\'e}l{\'e}ments de $K$, et soit $p \in \mathbb{N}$.

Pour simplifier les notations, on note simplement $S_k $au lieu de $S_k (x_1,
\ldots, x_N)$ (respectivement $\sigma_k$ au lieu de $\sigma_k (x_1, \ldots,
x_N)$) pour tout $k = 1, \ldots, N$.

Si $n \geqslant p$, on a :
\[ \underset{i = 1}{\overset{N}{\prod}} (X - x_i) = X^N + \underset{i =
   1}{\overset{N}{\sum}} (- 1)^i \sigma_i X^{N - i} \]


Donc, pour tout $j \in \llbracket 1, N \rrbracket$, on a :
\[ x_j^N + \underset{i = 1}{\overset{N}{\sum}} (- 1)^i \sigma_i x_j^{N - i} =
   \underset{i = 1}{\overset{N}{\prod}} (x_j - x_i) = 0 \]


Par suite, en mutlipliant par $x^{p - N}$, on obtient :
\[ x_j^p + \underset{i = 1}{\overset{N}{\sum}} (- 1)^i \sigma_i x_j^{p - i} =
   0 \]


Ainsi :
\begin{eqnarray*}
  \underset{j = 1}{\overset{N}{\sum}} \left( x_j^p + \underset{i =
  1}{\overset{N}{\sum}} (- 1)^i \sigma_i x_j^{p - i} \right) & = & \underset{j
  = 1}{\overset{N}{\sum}} x_j^p + \underset{i = 1}{\overset{N}{\sum}} (- 1)^i
  \sigma_i \underset{j = 1}{\overset{N}{\sum}} x_j^{p - i}\\
  & = & S_p + \underset{i = 1}{\overset{N}{\sum}} (- 1)^i \sigma_i S_{p - i}
\end{eqnarray*}


D'o{\`u} :
\[ S_p + \underset{i = 1}{\overset{N}{\sum}} (- 1)^i \sigma_i S_{p - i} = 0 \]


Si $p \in \llbracket 1, N - 1 \rrbracket$, on a, pour tout $k \in \llbracket
1, p - 1 \rrbracket$ :
\begin{eqnarray*}
  \sigma_k S_{p - k} & = & \left( \overset{}{\underset{1 \leqslant j_1 <
  \cdots < j_k \leqslant N}{\sum}}  \underset{l = 1}{\overset{k}{\prod}}
  x_{j_l} \right) S_{p - k}\\
  & = & \overset{}{\underset{1 \leqslant j_1 < \cdots < j_k \leqslant
  N}{\sum}} \underset{m = 1}{\overset{N}{\sum}}  \underset{l =
  1}{\overset{k}{\prod}} x_{j_l} x^{p - k}_m\\
  & = & \overset{}{\underset{1 \leqslant m \leqslant N, m \neq j_1, \ldots,
  j_k}{\underset{1 \leqslant j_1 < \cdots < j_k \leqslant N}{\sum}}} 
  \underset{l = 1}{\overset{k}{\prod}} x_{j_l} x^{p - k}_m + \underset{i =
  1}{\overset{k}{\sum}} \overset{}{\underset{m = j_i}{\underset{1 \leqslant
  j_1 < \cdots < j_k \leqslant N}{\sum}}}  \underset{l =
  1}{\overset{k}{\prod}} x_{j_l} x^{p - k}_m\\
  & = & \overset{}{\underset{1 \leqslant m \leqslant N, m \neq j_1, \ldots,
  j_k}{\underset{1 \leqslant j_1 < \cdots < j_k \leqslant N}{\sum}}} 
  \underset{l = 1}{\overset{k}{\prod}} x_{j_l} x^{p - k}_m + \underset{1
  \leqslant j_1 < \cdots < j_k \leqslant N}{\sum} \underset{i =
  1}{\overset{k}{\sum}} \left( \underset{l \not{=} i}{\underset{l =
  1}{\overset{k}{\prod}}} x_{j_l} \right) x^{p - k + 1}_{j_i}\\
  & = & \overset{}{\underset{1 \leqslant m \leqslant N, m \neq j_1, \ldots,
  j_k}{\underset{1 \leqslant j_1 < \cdots < j_k \leqslant N}{\sum}}} 
  \underset{l = 1}{\overset{k}{\prod}} x_{j_l} x^{p - k}_m +
  \overset{}{\underset{1 \leqslant m \leqslant N, m \neq j_1, \ldots, j_{k -
  1}}{\underset{1 \leqslant j_1 < \cdots < j_{k - 1} \leqslant N}{\sum}}} 
  \underset{l = 1}{\overset{k - 1}{\prod}} x_{j_l} x^{p - k + 1}_m
\end{eqnarray*}


Par suite :
\[ (- 1)^k \sigma_k S_{p - k} = (- 1)^k \overset{}{\underset{1 \leqslant m
   \leqslant N, m \neq j_1, \ldots, j_k}{\underset{1 \leqslant j_1 < \cdots <
   j_k \leqslant N}{\sum}}}  \underset{l = 1}{\overset{k}{\prod}} x_{j_l} x^{p
   - k}_m - (- 1)^{k - 1} \overset{}{\underset{1 \leqslant m \leqslant N, m
   \neq j_1, \ldots, j_{k - 1}}{\underset{1 \leqslant j_1 < \cdots < j_{k - 1}
   \leqslant N}{\sum}}}  \underset{l = 1}{\overset{k - 1}{\prod}} x_{j_l} x^{p
   - k + 1}_m \]


Ainsi, par t{\'e}l{\'e}scopage, on a :
\begin{eqnarray*}
  \underset{i = 1}{\overset{p - 1}{\sum}} (- 1)^i \sigma_i S_{p - i} & = & (-
  1)^{p - 1} \overset{}{\underset{1 \leqslant m \leqslant N, m \neq j_1,
  \ldots, j_{p - 1}}{\underset{1 \leqslant j_1 < \cdots < j_{p - 1} \leqslant
  N}{\sum}}}  \underset{l = 1}{\overset{k}{\prod}} x_{j_l} x_m - S_p\\
  & = & (- 1)^{p - 1} p \sigma_p - S_p
\end{eqnarray*}


D'o{\`u}
\[ S_p + \underset{i = 1}{\overset{p - 1}{\sum}} (- 1)^i \sigma_i S_{p - i} +
   (- 1)^p p \sigma_p = 0 \]


\

Montrons par r{\'e}currance sur $p \in \mathbb{N}$ que $S_p (\lambda_1,
\ldots, \lambda_n) \in \mathbb{R}$.

On a $P = \underset{i = 1}{\overset{n}{\prod}} (X - \lambda_i) \in \mathbb{R}
[X] \underset{}{\overset{}{}},$ donc pour tout $k \in \llbracket 1, n
\rrbracket$, $\sigma_i (\lambda_1, \ldots, \lambda_n) \in \mathbb{R}$.

On a pour $p = 0$, $S_0 (\lambda_1, \ldots, \lambda_n) = n \in \mathbb{R}$.

Soit $p \in \mathbb{N}$, supposons que $S_0, \ldots, S_p \in \mathbb{R}$, et
montrons que $S_{p + 1} \in \mathbb{R}$.

Si $p + 1 \geqslant n$, on a
\[ S_{p + 1} = \underset{i = 1}{\overset{N}{\sum}} (- 1)^{i - 1} \sigma_i
   (\lambda_1, \ldots, \lambda_n) S_{p - i} (\lambda_1, \ldots, \lambda_n) \in
   \mathbb{R} \]


Si $p + 1 < n$, on a alors
\[ S_p (\lambda_1, \ldots, \lambda_n) = \underset{i = 1}{\overset{p -
   1}{\sum}} (- 1)^{i - 1} \sigma_i (\lambda_1, \ldots, \lambda_n) S_{p - i}
   (\lambda_1, \ldots, \lambda_n) + (- 1)^{p - 1} p \sigma_p \in \mathbb{R} \]


D'o{\`u} $S_p (\lambda_1, \ldots, \lambda_n)$ pour tout $p \in \mathbb{N}$.

Ainsi, pour tout $i, j \in \llbracket 1, n \rrbracket$ on a
\[ H_{i, j} = S_{i + j} \in \mathbb{R} \]


Revenons {\`a} d{\'e}montrer que $H$ est r{\'e}elle, On garde les m{\^e}mes
notations que dans le lemme 1.

D'apr{\`e}s le lemme, pour tout $p \in \mathbb{N}$ :
\[ \left\{\begin{array}{ll}
     S_p (\lambda_1, \ldots, \lambda_n) + \underset{i = 1}{\overset{N}{\sum}}
     (- 1)^i \sigma_i (\lambda_1, \ldots, \lambda_n) S_{p - i} (\lambda_1,
     \ldots, \lambda_n) = 0 & \tmop{si} p \geqslant n\\
     S_p (\lambda_1, \ldots, \lambda_n) + \underset{i = 1}{\overset{p -
     1}{\sum}} (- 1)^i \sigma_i (\lambda_1, \ldots, \lambda_n) S_{p - i}
     (\lambda_1, \ldots, \lambda_n) + (- 1)^p p \sigma_p = 0 & \tmop{si} p < n
   \end{array}\right. \]


\tmtextbf{Remarque.}

Pour plus de d{\'e}tails, vous pouvez consulter l'{\'e}preuve Math2 du
concours Mines-Ponts, Fili{\`e}re PC, 2021.

\[ \maltese \maltese \maltese \maltese \maltese \maltese \maltese \]