Cet exercice s'int{\'e}resse aux g{\'e}n{\'e}rateurs d'un groupe de matrices.
Il demande de prouver qu'un certain sous-groupe de $\tmop{GL}_2 (\mathbb{Z})$
est engendr{\'e} par deux matrices sp{\'e}cifiques. L'exercice fait appel
{\`a} des notions de th{\'e}orie des groupes et d'alg{\`e}bre lin{\'e}aire.

\begin{exercise}[(G{\'e}n{\'e}rateurs d'un groupe de matrices)]
Soit $G$ le sous-ensemble de $\mathrm{GL}_2 (\mathbb{Z})$ des matrices {\`a}
coefficients entiers $\left( \begin{array}{ll}
  a & b\\
  c & d
\end{array} \right)$ telles que $a \equiv d \equiv 1 - c \equiv 1 \pmod{3}$
et~$ad - bc = 1$.

Montrer que $G$ est un sous-groupe engendr{\'e} par $A = \left(
\begin{array}{ll}
  1 & 1\\
  0 & 1
\end{array} \right)$ et $B = \left( \begin{array}{ll}
  1 & 0\\
  3 & 1
\end{array} \right)$.

\textit{ C'est une variante, {\`a} peine plus subtile, d'un r{\'e}sultat
analogue sur $\mathrm{SL}_2 (\mathbb{Z})$.}
\end{exercise}

\subsection*{Solution. (SABIR Ilyass)}
\addcontentsline{toc}{subsection}{Solution. (SABIR Ilyass)}

Soit $H$ le sous-groupe de $G$ engendr{\'e} par $A = \left(\begin{array}{cc}
  1 & 1\\
  0 & 1
\end{array}\right)$ et $B = \left(\begin{array}{cc}
  1 & 0\\
  3 & 1
\end{array}\right)$.

Montrons que $H = G$.

On a pour tout $n, m \in \mathbb{Z}$
\[ A^n = \left(\begin{array}{cc}
     1 & n\\
     0 & 1
   \end{array}\right) \]


Et
\[ B^m = \left(\begin{array}{cc}
     1 & 0\\
     3 m & 1
   \end{array}\right) \]


Ces matrices sont dans $G$, et comme $G$ est stable par produit, donc $H
\subseteq G$.

Soit $M = \left(\begin{array}{cc}
  a & b\\
  c & d
\end{array}\right) \in G$.

Consid{\'e}rons l'ensemble $E$ des matrices de la forme $MQ$ avec $Q \in H$.
Toutes ces matrices sont dans $G$, car $G$ est stable par produit. Parmi ces
matrices, choisissons $A' = \left(\begin{array}{cc}
  a' & b'\\
  c' & d'
\end{array}\right)$ qui minimise $|a' |$.

Pour tous $m, n \in \mathbb{Z}$, d{\'e}finissons :
\begin{eqnarray*}
  A'' & \assign & A' A^{- n} B^{- m}\\
  & = & \left(\begin{array}{cc}
    a' & b'\\
    c' & d'
  \end{array}\right) \left(\begin{array}{cc}
    1 & - n\\
    0 & 1
  \end{array}\right) \left(\begin{array}{cc}
    1 & 0\\
    - 3 m & 1
  \end{array}\right)\\
  & = & \left(\begin{array}{cc}
    a' - 3 m (b' - na') & \ast\\
    \ast & \ast
  \end{array}\right)
\end{eqnarray*}


Par minimalit{\'e} de $|a' |$, on a pour tout $m, n \in \mathbb{Z}$:
\[ |a' | \leq |a' - 3 m (b' - na') | \]


Supposons que pour tout $n \in \mathbb{Z}$, $b' - na' \neq 0$. Choisissons
$n_0 \in \mathbb{Z}$ l'entier le plus proche de $\frac{b'}{a'}$. Alors :
\begin{eqnarray*}
  0 & < & |b' - n_0 a' |\\
  & = & |a' | \cdot | \frac{b'}{a'} - n_0 |\\
  & \leqslant & | a'  | \cdot \frac{1}{2}
\end{eqnarray*}


Choisissons ensuite $m_0 \in \mathbb{Z}$ l'entier le plus proche de
$\frac{a'}{3 (b' - n_0 a')}$. On obtient :
\begin{eqnarray*}
  |a' | & \leqslant & |a' - 3 m_0 (b' - n_0 a') |\\
  & = & 3| b' - n_0 a' | \cdot | \frac{a'}{3 (b' - n_0 a')} - m_0 |\\
  & < & 3 \cdot \frac{1}{2} |a' | \cdot \frac{1}{2}\\
  & = & \frac{3}{4} |a' |
\end{eqnarray*}


Ceci est absurde, car $a' \neq 0$ (car $a' \equiv 1 \pmod{3}$).

Donc, il existe $n \in \mathbb{Z}$ tel que $b' = na'$. En prenant $m = 0$, on
a : $A'' = A' A^{- n} = \left(\begin{array}{cc}
  a' & 0\\
  c' & d'
\end{array}\right) \in G$

Comme $A'' \in G$, on a $a' d' = 1$ et $a' \equiv d' \equiv 1 \pmod{3}$. Donc
$a' = d' = 1$.

Posons $c' = 3 q$ avec $q \in \mathbb{Z}$. Alors :
\[ A'' = \left(\begin{array}{cc}
     1 & 0\\
     3 q & 1
   \end{array}\right) = B^q \]


On a donc : $A' = B^q A^n$, et comme $A' = MQ$ avec $Q \in H$, on obtient : $M
= B^q A^n Q^{- 1} \in H$

D'o{\`u} $G = H$

Ainsi, $G$ est bien le sous-groupe de $GL_2 (\mathbb{Z})$ engendr{\'e} par $A$
et $B$.


\[ \maltese \maltese \maltese \maltese \maltese \maltese \maltese \]
