Cet exercice, intitul{\'e} ``Certification de racines", propose un crit{\`e}re
pour garantir l'existence et l'unicit{\'e} d'un z{\'e}ro d'une fonction
diff{\'e}rentiable dans une boule donn{\'e}e. L'exercice fait appel {\`a} des
notions d'analyse et de topologie.
\begin{exercise}[(Certification de racines)]
Soit $f : \mathbb{R}^n \to \mathbb{R}^n$ une application de classe
$\mathcal{C}^1$. Soit $x \in \mathbb{R}^n$, soit $B$ la boule unit{\'e}
ferm{\'e}e. On suppose que pour tout $u, v \in B$,
\[ - f (x) + v - \mathrm{d} f (x + u) \cdot v \in \tfrac{1}{2} B. \]
Montrer que $f$ admet un unique z{\'e}ro dans la boule $x + B$.

\end{exercise}

\subsection*{Solution. (ZINE Akram)}
\addcontentsline{toc}{subsection}{Solution. (ZINE Akram)}


Soit $x \in B$, alors $\|f (x) - x\| \leq \frac{1}{2}$. En effet,
\[ f (x) = f (0) + \int_0^1 df_{tx} (x) \hspace{0.17em} dt = f (0) + \int_0^1
   (x - f (0) + h (t))  \hspace{0.17em} dt, \]


avec $\|h (t)\| \leq \frac{1}{2}$ pour tout $t \in [0, 1]$. Donc,
\[ \|f (x) - x\|= \left| \int_0^1 h (t) \hspace{0.17em} dt \right| \leq
   \int_0^1 \|h (t)\| \hspace{0.17em} dt \leq \frac{1}{2} . \]


Soit $g (x) =\|f (x)\|^2$ sur $B$. Comme $g$ est continue sur le compact $B$,
elle admet un minimum.

Avec $v = 0$, on a $\|f (0)\| \leq \frac{1}{2}$. Si $x \in \partial B$, alors
$\|f (x)\| \geq \frac{1}{2}$. Donc, le minimum de $g$ sur $B$ est au plus
$\frac{1}{4}$. Si le minimum est atteint en $0$, c'est fait, sinon il est
atteint {\`a} l'int{\'e}rieur de $B$.

Soit $a \in \overset{\circ}{B}$ un point o{\`u} $g$ atteint son minimum. Alors
$dg_a = 0$, c'est-{\`a}-dire pour tout \ $h \in \mathbb{R}^n, \langle df_a
(h), f (a) \rangle = 0$. Si $df_a$ est inversible, on en d{\'e}duit $f (a) =
0$.

Pour prouver que $df_a$ est injective, supposons le contraire. Alors, il
existe $w \in \ker (df_a)$, $\|w\|= 1$. On a aussi $df_a (- w) = 0$. Donc,
$\|w - f (0)\| \leq \frac{1}{2}$ et $\|- w - f (0)\| \leq \frac{1}{2}$, ce qui
est impossible, car $\|w - (- w)\|= 2$ alors que la sph{\`e}re $S (f (0),
\frac{1}{2})$ a un diam{\`e}tre de $1$.

Enfin, supposons qu'il existe $a, b \in B$ tels que $f (a) = f (b) = 0$ et $a
\neq b$. On a :
\[ 0 = \int_0^1 df_{(1 - t) a + tb} (b - a) \hspace{0.17em} dt =\|a - b\|
   \int_0^1 df_{(1 - t) a + tb} \left( \frac{b - a}{\|b - a\|} \right) 
   \hspace{0.17em} dt. \]


Or,
\[ df_{(1 - t) a + tb} \left( \frac{b - a}{\|b - a\|} \right) = \frac{b -
   a}{\|b - a\|} - f (0) + h (t), \]


avec $\|h (t)\| \leq \frac{1}{2}$. On obtient alors
\[ \frac{b - a}{\|b - a\|} = f (0) - \int_0^1 h (t) d t. \]


Puisque $\|f (0)\| \leq \frac{1}{2}$ et $\|h\| \leq \frac{1}{2}$, on a
{\'e}galit{\'e} dans l'in{\'e}galit{\'e} triangulaire :
\[ f (0) = - \int_0^1 h (t) d t = \frac{b - a}{2\|b - a\|} . \]


En proc{\'e}dant de mani{\`e}re similaire, on trouve $f (0) = \frac{a -
b}{2\|b - a\|}$, ce qui implique $a = b$, ce qui est une contradiction.
\[ \maltese \maltese \maltese \maltese \maltese \maltese \maltese \]
