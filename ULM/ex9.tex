L'exercice 9 {\'e}tudie les sous-groupes des isom{\'e}tries affines du plan.
Il demande de prouver l'existence de certaines translations dans un
sous-groupe d'isom{\'e}tries v{\'e}rifiant des conditions sp{\'e}cifiques.
L'exercice fait appel {\`a} des notions de g{\'e}om{\'e}trie affine et de
th{\'e}orie des groupes.
\begin{exercise}[({\'E}tude des sous-groupes des isom{\'e}tries
affines)]
Soit $G$ un sous-groupe du groupe des isom{\'e}tries du plan affine
$\mathbb{R}^2$. On suppose que pour tout $x \in \mathbb{R}^2$, il existe~$g
\in G$ tel que~$g (x) \neq x$. Montrer que $G$ contient une translation non
triviale.

Si de plus~$G$ ne stabilise aucune droite, montrer que $G$ contient une
deuxi{\`e}me translation non parall{\`e}le {\`a} la premi{\`e}re.

\end{exercise}

\subsection*{Solution. (ZINE Akram)}
\addcontentsline{toc}{subsection}{Solution. (ZINE Akram)}

Soit $G$ un sous-groupe du groupe des isom{\'e}tries du plan affine
$\mathbb{R}^2$. On suppose que pour tout $x \in \mathbb{R}^2$, il existe $g
\in G$ tel que $g (x) \neq x$. Montrons que $G$ contient une translation non
triviale.

On utilisera pour les deux questions le fait que les 4 isom{\'e}tries
possibles du plan sont :
\begin{enumeratenumeric}
  \item Une transaltion
  
  \item Une rotation
  
  \item Une r{\'e}flexion
  
  \item Une r{\'e}flexion glissante
\end{enumeratenumeric}


Pour s'en convaincre il suffit d'utiliser la classification des isom{\'e}tries
vectorielles du plan en rotation et r{\'e}flexion, puis translater par un
vecteur $v$.

\tmtextbf{Lemme 1.}

La composition d'une rotation et d'une r{\'e}flexion dans le plan donne une
r{\'e}flexion glissante si le centre de rotation n'est pas situ{\'e} sur la
ligne de r{\'e}flexion.

\tmtextbf{Preuve du lemme 1.}

Alignons l'axe des $x$ avec la ligne de r{\'e}flexion $l$. Ainsi, la
r{\'e}flexion par rapport {\`a} $l$ est donn{\'e}e par :
\[ M (x, y) = (x, - y) \]


Pla{\c c}ons le centre de rotation $O$ en $(0, d)$ o{\`u} $d \neq 0$, puisque
$l$ ne passe pas par $O$.

Consid{\'e}rons une rotation autour du point $O$ d'angle $\theta$. La matrice
de rotation est :
\[ R_{\theta} = \left[ \begin{array}{cc}
     \cos \theta & - \sin \theta\\
     \sin \theta & \cos \theta
   \end{array} \right] . \]


Pour effectuer une rotation autour de $O$, nous devons translater les
coordonn{\'e}es de mani{\`e}re {\`a} ce que $O$ soit {\`a} l'origine :
\[ \tilde{P} = P - O. \]


Apr{\`e}s la rotation, nous obtenons :
\[ \tilde{P}' = R_{\theta}  \tilde{P} . \]


Ensuite, nous revenons aux coordonn{\'e}es d'origine :
\[ P' = \tilde{P}' + O. \]


Ainsi , les coordonn{\'e}es apr{\`e}s translation vers $O$ sont :
\[ \tilde{P} = \left[ \begin{array}{c}
     x\\
     y - d
   \end{array} \right] . \]


Apr{\`e}s rotation de $\tilde{P}$ :
\[ \tilde{P}' = R_{\theta}  \tilde{P} = \left[ \begin{array}{cc}
     \cos \theta & - \sin \theta\\
     \sin \theta & \cos \theta
   \end{array} \right] \left[ \begin{array}{c}
     x\\
     y - d
   \end{array} \right] . \]


En combinant les composantes, nous obtenons :
\[ \tilde{x}' = \cos \theta \cdot x - \sin \theta \cdot (y - d), \]
\[ \tilde{y}' = \sin \theta \cdot x + \cos \theta \cdot (y - d) . \]


Par suite,
\[ x' = \tilde{x}', \]
\[ y' = \tilde{y}' + d = \sin \theta \cdot x + \cos \theta \cdot (y - d) + d.
\]


Ainsi,
\[ y' = \sin \theta \cdot x + \cos \theta \cdot y - \cos \theta \cdot d + d.
\]


Par suite,
\[ y'' = - y' = - (\sin \theta \cdot x + \cos \theta \cdot y - \cos \theta
   \cdot d + d) . \]


En combinant les composantes $x''$ et $y''$, nous obtenons :
\[ x'' = \cos \theta \cdot x - \sin \theta \cdot (y - d), \]
\[ y'' = - \sin \theta \cdot x - \cos \theta \cdot y + \cos \theta \cdot d -
   d. \]


En forme matricielle, cela donne :
\[ \left[ \begin{array}{c}
     x''\\
     y''
   \end{array} \right] = \left[ \begin{array}{cc}
     \cos \theta & - \sin \theta\\
     - \sin \theta & - \cos \theta
   \end{array} \right] \left[ \begin{array}{c}
     x\\
     y
   \end{array} \right] + \left[ \begin{array}{c}
     \sin \theta \cdot d\\
     \cos \theta \cdot d - d
   \end{array} \right] . \]


Le terme $\left[ \begin{array}{c}
  \sin \theta \cdot d\\
  \cos \theta \cdot d - d
\end{array} \right]$ repr{\'e}sente une translation, et la matrice
\[ A = \left[ \begin{array}{cc}
     \cos \theta & - \sin \theta\\
     - \sin \theta & - \cos \theta
   \end{array} \right] \]


indique qu'il s'agit d'une r{\'e}flexion suivie d'une translation le long de
l'axe de r{\'e}flexion, c'est-{\`a}-dire une r{\'e}flexion glissante.

\tmtextbf{Lemme 2.}

La composition de deux rotations de centres diff{\'e}rents dans le plan affine
donne une rotation ou une translation. Plus pr{\'e}cis{\'e}ment, si les angles
de rotation sont oppos{\'e}s, la composition donne une translation. Sinon, la
composition reste une rotation. Dans les deux cas, on peut toujours construire
une translation.

\tmtextbf{Preuve du lemme 2.}

D{\'e}finissons les deux rotations :

- Premi{\`e}re rotation $R_1$ : de centre $O_1$ {\`a} la position $\vec{o}_1$
et d'angle $\theta_1$.

- Deuxi{\`e}me rotation $R_2$ : de centre : $O_2$ {\`a} la position
$\vec{o}_2$, et d'angle $\theta_2$.

Nous allons analyser la composition $T = R_2 \circ R_1$. Une rotation dans le
plan peut {\^e}tre repr{\'e}sent{\'e}e de la mani{\`e}re suivante :
\begin{enumerate}
  \item Translater le point de sorte que le centre de rotation soit {\`a}
  l'origine.
  
  \item Appliquer la matrice de rotation.
  
  \item Translater le point de retour {\`a} sa position d'origine.
\end{enumerate}


La matrice de rotation pour un angle $\theta$ est :
\[ R (\theta) = \left[ \begin{array}{cc}
     \cos \theta & - \sin \theta\\
     \sin \theta & \cos \theta
   \end{array} \right] . \]


La fonction de transformation pour une rotation autour d'un point $O$ est
donn{\'e}e par :
\[ R (P) = R (\theta) \cdot (P - \vec{o}) + \vec{o} . \]


Appliquons d'abord $R_1$ {\`a} un point $P$, puis $R_2$ au r{\'e}sultat :
\[ P' = R_1 (P) = R (\theta_1) \cdot (P - \vec{o}_1) + \vec{o}_1 . \]


Ensuite,
\[ P'' = R_2 (P') = R (\theta_2) \cdot (P' - \vec{o}_2) + \vec{o}_2 . \]


La composition $T (P)$ s'exprime alors comme :
\[ T (P) = R_2 (R_1 (P)) = R (\theta_2) \cdot [R (\theta_1) \cdot (P -
   \vec{o}_1) + \vec{o}_1 - \vec{o}_2] + \vec{o}_2 . \]


D{\'e}veloppons l'expression :
\[ T (P) = R (\theta_2) \cdot R (\theta_1) \cdot (P - \vec{o}_1) + R
   (\theta_2) \cdot (\vec{o}_1 - \vec{o}_2) + \vec{o}_2 . \]


On a :
\[ R (\theta_2) \cdot R (\theta_1) = R (\theta_1 + \theta_2) . \]


Ainsi, l'expression devient :
\[ T (P) = R (\theta_1 + \theta_2) \cdot (P - \vec{o}_1) + R (\theta_2) \cdot
   (\vec{o}_1 - \vec{o}_2) + \vec{o}_2 . \]


D{\'e}finissons :
\begin{itemize}
  \item \tmtextbf{Angle total de rotation} : $\theta = \theta_1 + \theta_2$.
  
  \item \tmtextbf{Vecteur de translation} : $\vec{t} = R (\theta_2) \cdot
  (\vec{o}_1 - \vec{o}_2) + \vec{o}_2$.
\end{itemize}


La transformation devient :
\[ T (P) = R (\theta) \cdot (P - \vec{o}_1) + \vec{t} . \]
\begin{itemize}
  \item \tmtextbf{Cas 1 : La somme des angles est nulle} ($\theta = 0$)
  
  Si $\theta_1 + \theta_2 = 0$, alors $R (\theta$) est la matrice
  identit{\'e}. La transformation $T (P$) se simplifie en :
  \[ T (P) = (P - \vec{o}_1) + \vec{t} = P + (\vec{t} - \vec{o}_1) . \]
  C'est une translation par le vecteur $\vec{v} = \vec{t} - \vec{o}_1$.
  
  \item \tmtextbf{Cas 2 : La somme des angles n'est pas nulle} ($\theta \neq
  0$)
  
  La transformation $T (P$) est une rotation d'angle $\theta$ autour d'un
  nouveau point. Il existe un composant de translation suppl{\'e}mentaire. Par
  cons{\'e}quent, $T$ est une rotation autour d'un nouveau centre.
\end{itemize}


\

La composition de deux rotations $R_1$ et $R_2$ de centres diff{\'e}rents est
{\'e}quivalente {\`a} :
\begin{itemize}
  \item Une rotation d'angle $\theta_1 + \theta_2$ autour d'un nouveau point,
  sauf si les angles sont oppos{\'e}s. On peut aussi construire dans ce cas
  une translation, en combinant cette rotation avec la rotation d'angle
  -($\theta_1 + \theta_2$) qui est de centre diff{\'e}rent d'apr{\`e}s la
  m{\^e}me construction,(puisque le centre d{\'e}pend de l'orientation).
  
  \item Dans le cas o{\`u} $\theta_1 + \theta_2 = 0$ et $\vec{o}_1 \neq
  \vec{o}_2$, la composition est une translation.
\end{itemize}


\tmtextbf{Lemme 3.}

La composition de deux r{\'e}flexions aux axes parall{\`e}les donne une
translation.

\

\tmtextbf{Preuve du lemme 3.}

Soient $L_1$ et $L_2$ deux droites parall{\`e}les distinctes dans le plan
affine $\mathbb{R}^2$, faisant un angle $\theta$ avec l'axe des abscisses.
Nous allons d{\'e}montrer que la composition des r{\'e}flexions par rapport
{\`a} $L_1$ et $L_2$ est une translation.
\begin{itemize}
  \item Supposons que les droites $L_1$ et $L_2$ soient parall{\`e}les et
  orient{\'e}es selon un angle $\theta$ avec l'axe des abscisses. Ainsi, un
  vecteur directeur commun {\`a} ces droites est donn{\'e} par :
  \[ \mathbf{u} = \left(\begin{array}{c}
       \cos \theta\\
       \sin \theta
     \end{array}\right) \]
  \item Un vecteur normal unitaire aux droites est :
  \[ \mathbf{n} = \left(\begin{array}{c}
       - \sin \theta\\
       \cos \theta
     \end{array}\right) \]
  \item Soit $d$ la distance entre les droites $L_1$ et $L_2$.
\end{itemize}


La r{\'e}flexion par rapport {\`a} une droite $L$ orient{\'e}e selon $\theta$
et situ{\'e}e {\`a} une distance $c$ de l'origine peut {\^e}tre exprim{\'e}e
comme une transformation affine :
\[ R_L (\mathbf{x}) = A \mathbf{x} + \mathbf{t} \]


o{\`u} $A$ est la matrice de r{\'e}flexion lin{\'e}aire et $\mathbf{t}$ est le
vecteur de translation.
\begin{itemize}
  \item La matrice de r{\'e}flexion par rapport {\`a} une droite orient{\'e}e
  selon $\theta$ est :
  \[ A = \left(\begin{array}{cc}
       \cos 2 \theta & \sin 2 \theta\\
       \sin 2 \theta & - \cos 2 \theta
     \end{array}\right) \]
  \item Le vecteur de translation d{\'e}pend de la position de la droite. Pour
  une droite $L$ situ{\'e}e {\`a} une distance $c$ le long du vecteur normal
  $\mathbf{n}$, le vecteur de translation est :
  \[ \mathbf{t} = 2 c \mathbf{n} = 2 c \left(\begin{array}{c}
       - \sin \theta\\
       \cos \theta
     \end{array}\right) \]
\end{itemize}


Ainsi, les r{\'e}flexions par rapport {\`a} $L_1$ et $L_2$ sont :
\[ R_{L_1} (\mathbf{x}) = A \mathbf{x} + 2 c_1 \mathbf{n} \]
\[ R_{L_2} (\mathbf{x}) = A \mathbf{x} + 2 c_2 \mathbf{n} \]


o{\`u} $c_1$ et $c_2$ sont les distances de $L_1$ et $L_2$ par rapport {\`a}
l'origine, respectivement.

Nous calculons la composition $R = R_{L_2} \circ R_{L_1}$.
\[ R (\mathbf{x}) = R_{L_2} (R_{L_1} (\mathbf{x})) = R_{L_2} (A \mathbf{x} + 2
   c_1 \mathbf{n}) = A (A \mathbf{x} + 2 c_1 \mathbf{n}) + 2 c_2 \mathbf{n} \]


Calculons chaque terme s{\'e}par{\'e}ment :

\begin{align*}
  A \cdot A \mathbf{x} & = A^2 \mathbf{x}\\
  A \cdot 2 c_1 \mathbf{n} & = 2 c_1 A \mathbf{n}
\end{align*}

\

\

On a :
\[ A^2 = \left(\begin{array}{cc}
     \cos 2 \theta & \sin 2 \theta\\
     \sin 2 \theta & - \cos 2 \theta
   \end{array}\right) \left(\begin{array}{cc}
     \cos 2 \theta & \sin 2 \theta\\
     \sin 2 \theta & - \cos 2 \theta
   \end{array}\right) = I \]
\[ A \mathbf{n} = \left(\begin{array}{cc}
     \cos 2 \theta & \sin 2 \theta\\
     \sin 2 \theta & - \cos 2 \theta
   \end{array}\right) \left(\begin{array}{c}
     - \sin \theta\\
     \cos \theta
   \end{array}\right) = \left(\begin{array}{c}
     - \cos 2 \theta \sin \theta + \sin 2 \theta \cos \theta\\
     - \sin 2 \theta \sin \theta - \cos 2 \theta \cos \theta
   \end{array}\right) \]


Donc,
\[ A \mathbf{n} = \left(\begin{array}{c}
     \sin \theta\\
     - \cos \theta
   \end{array}\right) = \mathbf{n} \]


Maintenant, revenons {\`a} la composition $R (\mathbf{x})$ :
\[ R (\mathbf{x}) = A^2 \mathbf{x} + 2 c_1 A \mathbf{n} + 2 c_2 \mathbf{n} = I
   \mathbf{x} + 2 c_1 \mathbf{n} + 2 c_2 \mathbf{n} = \mathbf{x} + 2 (c_1 +
   c_2) \mathbf{n} \]


La transformation $R$ est donc donn{\'e}e par :
\[ R (\mathbf{x}) = \mathbf{x} + 2 (c_2 - c_1) \mathbf{n} \]


o{\`u} $2 (c_2 - c_1) \mathbf{n}$ est le vecteur de translation.

Ainsi, la composition des r{\'e}flexions $R_{L_2} \circ R_{L_1}$ est une
translation de vecteur $\mathbf{t} = 2 (c_2 - c_1) \mathbf{n}$,
c'est-{\`a}-dire une translation parall{\`e}lement aux droites $L_1$ et $L_2$
et de longueur {\'e}gale au double de la distance entre elles.

1- Examinons cas par cas:
\begin{itemize}
  \item Si G contient une r{\'e}flexion glissante alors on obtient une
  translation non triviale en la composant par elle-m{\^e}me.
  
  \item Si G ne contient que des rotations, alors il contient deux rotations
  de centre diff{\'e}rents d'apr{\`e}s l'hypoth{\`e}se de l'{\'e}nonc{\'e}, et
  donc d'apr{\`e}s le \tmtextbf{Lemme 2} on peut construire une translation.
  
  \item Si G contient une rotation et une r{\'e}flexion dont la droite ne
  passe pas par le centre de rotation, et dans ce cas on peut construit une
  translation d'apr{\`e}s le \tmtextbf{Lemme 1}.
  
  \item Si G ne contient que des r{\'e}flexions: Si au moins deux
  pr{\'e}sentent des axes qui sont parall{\`e}les alors la composition des
  deux r{\'e}flexions r{\'e}sulte en une translation d'apr{\`e}s le
  \tmtextbf{Lemme 3}. Sinon, On a aux moins trois r{\'e}flexions qui
  pr{\'e}sentent des axes s{\'e}cants. Et leur composition donne deux matrices
  de rotations diff{\'e}rentes (Il suffit de multiplier les matrices de
  r{\'e}flexion).
\end{itemize}


2- Supposons que $G$ contienne une rotation $R$. Alors, pour toute translation
$T_{\mathbf{v}} \in G$, nous avons
\[ R^{- 1} \circ T_{\mathbf{v}} \circ R = T_{R^{- 1} (\mathbf{v})} . \]


\

Ainsi, nous obtenons une nouvelle translation $T_{R^{- 1} (\mathbf{v})}$.
Comme $R$ ne stabilise que son centre de rotation, cette nouvelle translation
n'est pas triviale et peut avoir une direction diff{\'e}rente de la
translation initiale.

Ensuite, si $G$ ne contient que des translations, et que ces translations sont
parall{\`e}les {\`a} une direction commune. Cela implique que les droites
parall{\`e}les {\`a} cette direction sont stabilis{\'e}es par $G$, ce qui
contredit l'hypoth{\`e}se selon laquelle $G$ ne stabilise aucune droite.

Consid{\'e}rons maintenant le cas o{\`u} $G$ contient des r{\'e}flexions mais
pas de rotations. Soit $M$ une r{\'e}flexion dans $G$. Alors, pour toute
translation $T_{\mathbf{v}} \in G$, nous avons
\[ M \circ T_{\mathbf{v}} \circ M = T_{M (\mathbf{v})} . \]


Si pour toutes les r{\'e}flexions $M \in G$, on a $M (\mathbf{v}) =
\mathbf{v}$, alors $G$ stabilise une droite parall{\`e}le {\`a} $\mathbf{v}$,
ce qui est une contradiction avec l'hypoth{\`e}se que $G$ ne stabilise aucune
droite.

Ainsi, $G$ doit contenir une seconde translation dont la direction est
diff{\'e}rente de la premi{\`e}re.
\[ \maltese \maltese \maltese \maltese \maltese \maltese \maltese \]
