L'exercice 16 s'int{\'e}resse aux valeurs rationnelles du cosinus. Il demande
de d{\'e}crire l'ensemble des nombres rationnels $r$ tels que c$\tmop{os} (r
\pi)$ est rationnel. Cet exercice combine des aspects de trigonom{\'e}trie et
de th{\'e}orie des nombres alg{\'e}briques.

\begin{exercise}[(Valeurs rationnelles du cosinus)]
D{\'e}crire l'ensemble des nombres rationels $r$ tels que $\cos (r \pi)$ soit
rationel.
\end{exercise}

\subsection*{Solution. (SABIR Ilyass)}
\addcontentsline{toc}{subsection}{Solution. (SABIR Ilyass)}

Notons $S = \{ r \in \mathbb{Q} | \nobracket \cos (r \pi) \in \mathbb{Q} \}$.

Puisque pour tout $n \in \mathbb{Z}$, on a $\cos ((n + r) \pi) = (- 1)^n \cos
(r \pi)$, alors
\[ S = (S \cap [0, 1 [) +\mathbb{Z} \]


On peut donc se concentrer seulement sur les nombres rationnels $r \in [0, 1
[$ tels que $\cos (r \pi) \in \mathbb{Q}$.

Soit $r \in \mathbb{Q} \cap [0, 1 [$ tel que $\cos (r \pi) \in \mathbb{Q}$.

Si $r = 0$, on a $\cos (r \pi) = 1 \in \mathbb{Q}$. Dans toute la suite, on
suppose que $r > 0$.

Notons $\frac{r}{2} = \frac{p}{q}$ avec $p, q \in \mathbb{N}$ tels que $q
\geqslant 2$, $p < \frac{q}{2}$ et $p \wedge q = 1$.

\

On a alors, $e^{i \pi r} = e^{2 i \pi \frac{p}{q}}$, qui est une racine
primitive $q$-{\`e}me de l'unit{\'e}. Donc $e^{i \pi r}$ annule :
\[ \Phi_q = \underset{k \wedge q = 1}{\underset{k \in \llbracket 1, q
   \rrbracket}{\prod}} \left( X - e^{2 i \pi \frac{k}{q}} \right) \]


\tmtextbf{Lemme 1. (Classique)}

Pour tout $n \in \mathbb{N}^{\ast}$, $\Phi_n \assign \underset{k \wedge q =
1}{\underset{k \in \llbracket 1, q \rrbracket}{\prod}} \left( X - e^{2 i \pi
\frac{k}{n}} \right)$ est irr{\'e}ductible dans $\mathbb{Q} [X]$.

\tmtextbf{Preuve du lemme 1.}

Nous allons d{\'e}montrer ce lemme en utilisant le crit{\`e}re d'Eisenstein
apr{\`e}s un changement de variable appropri{\'e}.

D{\'e}finissons le polyn{\^o}me
\[ \Psi_n (X) = \Phi_n (X + 1) \]


Les racines de $\Psi_n (X)$ sont les nombres
\[ \alpha_k = e^{2 i \pi k / n} - 1, o{\`u} \quad k \wedge n = 1 \]


On peut montrer facilement que les polyn{\^o}mes cyclotomiques ont des
coefficients entiers, par suite $\Psi_n (X) \in \mathbb{Z}[X]$.

Soit $p$ un nombre premier tel que $p$ divise $n$ mais $p^2$ ne divise pas
$n$. Cela signifie que $p$ est un facteur premier de $n$ apparaissant avec
multiplicit{\'e} 1 dans la d{\'e}composition en facteurs premiers de $n$.

Nous allons montrer que $\Psi_n (X)$ satisfait le crit{\`e}re d'Eisenstein
pour le nombre premier $p$ :
\begin{enumerate}
  \item Tous les coefficients $a_i$ pour $i \geq 1$ sont divisibles par $p$.
  
  \item Le coefficient dominant $a_0$ n'est pas divisible par $p$.
  
  \item Le terme constant $a_n$ n'est pas divisible par $p^2$.
\end{enumerate}


Les coefficients de $\Psi_n (X)$ sont des sommes sym{\'e}triques des racines
$\alpha_k$. Pour $i \geq 1$, les coefficients sont donn{\'e}s par :
\[ a_i = (- 1)^i  \sum_{1 \leq k_1 < \ldots < k_i \leq \varphi (n)}
   \alpha_{k_1} \cdots \alpha_{k_i} . \]


On va montrer que $p$ divise chaque $a_i$ pour $i \geq 1$.

Les racines $\alpha_k$ peuvent {\^e}tre exprim{\'e}es en termes de sommes de
racines de l'unit{\'e}, et en exploitant les propri{\'e}t{\'e}s des sommes
cyclotomiques modulo $p$, on montre que les sommes sym{\'e}triques sont
divisibles par $p$.

Le coefficient dominant $a_0$ est {\'e}gal {\`a} 1, car $\Psi_n (X)$ est un
polyn{\^o}me unitaire. Donc, $p$ ne divise pas $a_0$.

Le terme constant $a_n$ est donn{\'e} par
\[ a_n = (- 1)^n  \prod_{k = 1}^{\varphi (n)} \alpha_k . \]


on doit montrer que $p^2$ ne divise pas $a_n$.

Le terme constant est le produit des $\alpha_k = e^{2 i \pi k / n} - 1$. On
peut montrer que modulo $p$, ce produit est congru {\`a} $p$ ou $- p$, mais
pas divisible par $p^2$.

En effet, \ $k \wedge n = 1$, les nombres $k$ parcourent un syst{\`e}me
complet de repr{\'e}sentants des unit{\'e}s modulo $n$. Quand on r{\'e}duit
modulo $p$, les racines $e^{2 i \pi k / n}$ deviennent des racines primitives
$p$-i{\`e}mes de l'unit{\'e}, et on peut utiliser des identit{\'e}s classiques
des racines de l'unit{\'e} pour {\'e}tablir la non-divisibilit{\'e} par $p^2$.

{\'E}tant donn{\'e} que toutes les conditions du crit{\`e}re d'Eisenstein sont
satisfaites pour $\Psi_n (X)$, on conclut que $\Psi_n (X)$ est
irr{\'e}ductible dans $\mathbb{Q}[X]$.

Comme $\Psi_n (X)$ est irr{\'e}ductible dans $\mathbb{Q}[X]$ et que $\Phi_n
(X) = \Psi_n (X - 1)$, il en r{\'e}sulte que $\Phi_n (X)$ est {\'e}galement
irr{\'e}ductible dans $\mathbb{Q}[X]$.

D'o{\`u} le lemme.

\

On a pour tout $k \in \llbracket 1, q - 1 \rrbracket$ si $k \wedge q = 1$,
alors $(q - k) \wedge q = 1$.
\begin{eqnarray*}
  \Phi_q & = & \underset{k \wedge q = 1}{\underset{k \in \llbracket 1, q
  \rrbracket}{\prod}} \left( X - e^{2 i \pi \frac{k}{q}} \right)\\
  & = & \underset{k \wedge q = 1}{\underset{k \in \left\llbracket 1,
  \left\lfloor \frac{q}{2} \right\rfloor \right\rrbracket}{\prod}} \left( X -
  e^{2 i \pi \frac{k}{q}} \right) \left( X - e^{2 i \pi \frac{(q - k)}{q}}
  \right)\\
  & = & \underset{k \wedge q = 1}{\underset{k \in \left\llbracket 1,
  \left\lfloor \frac{q}{2} \right\rfloor \right\rrbracket}{\prod}} \left( X^2
  - 2 \cos \left( 2 \pi \frac{k}{q} \right) X + 1 \right)
\end{eqnarray*}


Puisque $\cos (\pi r) = \cos \left( 2 \pi \frac{p}{q} \right) \in \mathbb{Q}$,
alors $X^2 - 2 \cos (\pi r) X + 1 \in \mathbb{Q} [X]$, alors par
irr{\'e}ductibilit{\'e} de $\Phi_q$, on a
\[ \Phi_q = X^2 - 2 \cos (\pi r) X + 1 \]


En particulier $\deg (\Phi_q) = 2$.

D'autre part,
\[ \deg (\Phi_q) = \underset{}{\overset{}{}} \underset{k \wedge q =
   1}{\underset{k \in \llbracket 1, q \rrbracket}{\sum}} 1 = \varphi (q) \]


Avec $\varphi$ d{\'e}signe l'indicatrice d'Euler.

Ainsi, $q \in \mathbb{N}$, v{\'e}rifie l'{\'e}quation, $\varphi (q) = 2$.

D'apr{\`e}s le th{\'e}or{\`e}me fondamental de l'arithm{\'e}tique, on a
l'existence de $\alpha, \beta \in \mathbb{N}$ et $a_1, \ldots, a_r \in
\mathbb{N}$ et $p_1, \ldots, p_r > 3$ des nombres premiers tels que
\[ q = 2^{\alpha} 3^{\beta} \underset{}{\overset{}{}} \underset{i =
   1}{\overset{r}{\prod}} p^{a_i}_i \]


On a, alors
\[ \varphi (q) = 2^{\alpha} 3^{\beta - 1} \underset{i = 1}{\overset{r}{\prod}}
   (p_i - 1) p^{a_i - 1}_i \]


Or, $2^{\alpha} 3^{\beta - 1} \underset{i = 1}{\overset{r}{\prod}} (p_i - 1)
p^{a_i - 1}_i = 2$ si et seulement si $r = 0$ et $\alpha = 1$ et $\beta = 1$,
ainsi
\[ q = 6 \]


Par suite $\frac{r}{2} = \frac{1}{6}$. (Car $p \in \{ 1, 2, 3 \} \tmop{avec} p
\wedge q = 1$, donc $p = 1$)

\

Ainsi $r = \frac{1}{3}$, r{\'e}ciproquement $\cos \left( \frac{\pi}{3}
\right) = \frac{1}{2}$.

Par suite
\[ S = \left\{ 0, \frac{1}{3} \right\} +\mathbb{Z} \]


D'o{\`u} les seuls rationnels $r \in \mathbb{Q}$ tel que $\cos (r \pi) \in
\mathbb{Q}$ sont d{\'e}crits par l'ensemble $\left\{ n, n + \frac{1}{3} | n
\in \mathbb{Z} \nobracket \right\}$.

\

\tmtextbf{Commentaire.}

On a utilis{\'e} des r{\'e}sultats tr{\`e}s classiques de la th{\'e}orie des
nombres alg{\'e}briques. Pour plus de d{\'e}tails sur les m{\'e}thodes
utilis{\'e}es dans cet exercice, vous pouvez consulter l'{\'e}nonc{\'e} de
X/ENS, {\'e}preuve Math A, MP, 2019.
\[ \maltese \maltese \maltese \maltese \maltese \maltese \maltese \]
