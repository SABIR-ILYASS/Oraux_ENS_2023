Cet exercice traite des unions d{\'e}nombrables d'ensembles ferm{\'e}s. Il
demande de prouver que l'intervalle ouvert$] 0, 1 [$ne peut pas {\^e}tre
{\'e}crit comme union d{\'e}nombrable d'intervalles ferm{\'e}s disjoints
d'int{\'e}rieur non vide, puis d'{\'e}tendre ce r{\'e}sultat au carr{\'e}
ouvert ]$0, 1 [^2$ pour des disques ferm{\'e}s. L'exercice fait appel {\`a}
des notions de topologie et de th{\'e}orie de la mesure.

\begin{exercise}[(Unions de ferm{\'e}s)]
Montrer que $] 0, 1 [$ n'est pas l'union d'un nombre d{\'e}nombrable
d'intervalles ferm{\'e}s disjoints d'int{\'e}rieur non vide.

Montrer que le carr{\'e} ouvert $] 0, 1 [^2$ n'est pas l'union d'un nombre
d{\'e}nombrable de disques ferm{\'e}s.
\end{exercise}
\subsection*{Solution. (SABIR Ilyass)}
\addcontentsline{toc}{subsection}{Solution. (SABIR Ilyass)}

Pour r{\'e}pondre aux deux parties de l'exercice, on va montrer la
g{\'e}n{\'e}ralisation suivante :

\tmtextbf{G{\'e}n{\'e}ralisation de l'exercice :}

Pour tout $n \in \mathbb{N}^{\ast}$, on a $] 0, 1 [^n$ n'est pas l'union d'une
suite d{\'e}nombrable de ferm{\'e}es non vide deux {\`a} deux disjoints.

Soit $n \in \mathbb{N}^{\ast}$, Raisonnons par l'absurde en supposant que $]
0, 1 [^n$ peut {\^e}tre d{\'e}compos{\'e} en une union d{\'e}nombrable de
ferm{\'e}s non vides deux {\`a} deux disjoints, et montrons que cela conduit
{\`a} une contradiction avec la propri{\'e}t{\'e} de connexit{\'e} de $] 0, 1
[^n$.

Supposons qu'il existe une famille d{\'e}nombrable $(F_k)_{k \in \mathbb{N}}$
de sous-ensembles tels que :
\begin{itemize}
  \item Pour tout $k \in \mathbb{N}$, $F_k \subset] 0, 1 [^n$ est ferm{\'e}
  dans $] 0, 1 [^n$ et non vide.
  
  \item Pour tout $k, l \in \mathbb{N}$ avec $k \neq l$, $F_k \cap F_l =
  \emptyset$ (ils sont deux {\`a} deux disjoints).
  
  \item $] 0, 1 [^n = \underset{k = 1}{\overset{\infty}{\bigcup}} F_k$ (leur
  union est $] 0, 1 [^n$).
\end{itemize}


Pour aboutir {\`a} une contradiction, on va utiliser le lemme suivant, qui
pr{\'e}sente une propri{\'e}t{\'e} fondamentale des espaces connexes.

\tmtextbf{Lemme 1.}

Dans un espace connexe, la seule fa{\c c}on de le partitionner en ferm{\'e}s
disjoints est que l'un des ferm{\'e}s soit l'espace entier et les autres
soient vides. En d'autres termes, un espace connexe ne peut pas {\^e}tre
d{\'e}compos{\'e} en une union de plusieurs ferm{\'e}s non vides deux {\`a}
deux disjoints.

\tmtextbf{Preuve du lemme 1.}

Supposons que $X$ est un espace topologique connexe, et qu'il existe une
famille $(F_i)_{i \in I}$ de ferm{\'e}s de $X$ tels que :
\begin{itemize}
  \item Pour tout $i \in I$, $F_i$ est ferm{\'e} dans $X$ et non vide.
  
  \item Les $F_i$ sont deux {\`a} deux disjoints : $F_i \cap F_j = \emptyset$
  pour $i \neq j$.
  
  \item Leur union est l'espace : $X = \underset{i \in I}{\bigcup} F_i$.
\end{itemize}


On va montrer que n{\'e}cessairement un seul des $F_i$ est {\'e}gal {\`a} $X$
et que les autres sont vides.

Supposons par l'absurde qu'il existe au moins deux ferm{\'e}s non vides
disjoints $F_1$ et $F_2$ dans $X$.

Consid{\'e}rons les ensembles $A = F_1$ et $B = X \setminus F_1 = \underset{i
\in I}{\bigcup} F_i$.
\begin{itemize}
  \item $A$ est ferm{\'e} dans $X$ (par hypoth{\`e}se).
  
  \item $B$ est l'union de ferm{\'e}s (les $F_i$ pour $i \neq 1$), donc
  ferm{\'e} dans $X$.
  
  \item $A$ et $B$ sont disjoints (puisque les $F_i$ sont deux {\`a} deux
  disjoints).
  
  \item De plus, $A \cup B = X$.
\end{itemize}


Ainsi, nous avons partitionn{\'e} $X$ en deux ferm{\'e}s disjoints non vides
$A$ et $B$.

Selon la d{\'e}finition de la connexit{\'e}, un espace connexe ne peut pas
{\^e}tre partitionn{\'e} en deux ferm{\'e}s disjoints non vides.

Cette situation contredit donc la connexit{\'e} de $X$.

Donc, il n'est pas possible qu'il y ait au moins deux ferm{\'e}s non vides
disjoints dans une telle partition de $X$.

Par suite, un seul des $F_i$ est {\'e}gal {\`a} $X$, et les autres $F_i$ sont
vides.

\

Revenons {\`a} notre hypoth{\`e}se initiale sur $] 0, 1 [^n$.

On a suppos{\'e} que $] 0, 1 [^n$ est d{\'e}compos{\'e} en une union
d{\'e}nombrable de ferm{\'e}s non vides deux {\`a} deux disjoints ($F_k)_{k
\in \mathbb{N}}$.

Puisque $] 0, 1 [^n$ est un espace connexe ({\'e}tant un ouvert connexe de
$\mathbb{R}^n$), la propri{\'e}t{\'e} d{\'e}montr{\'e}e s'applique.

Selon cette propri{\'e}t{\'e}, il devrait y avoir un unique $F_k$ {\'e}gal
{\`a} $] 0, 1 [^n$ et les autres $F_k$ seraient vides.

Cependant, par hypoth{\`e}se, tous les $F_k$ sont non vides, ce qui est en
contradiction avec la conclusion de la propri{\'e}t{\'e}.

Ainsi, il est donc impossible que $] 0, 1 [^n$ soit l'union d{\'e}nombrable de
ferm{\'e}s non vides deux {\`a} deux disjoints.
\[ \maltese \maltese \maltese \maltese \maltese \maltese \maltese \]
