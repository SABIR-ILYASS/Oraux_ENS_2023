L'exercice 18 introduit une distance sur les matrices sym{\'e}triques
d{\'e}finies positives. Il demande de prouver certaines propri{\'e}t{\'e}s de
cette distance, notamment son invariance par conjugaison. L'exercice fait
appel {\`a} des notions d'alg{\`e}bre lin{\'e}aire et de g{\'e}om{\'e}trie des
espaces de matrices.
\begin{exercise}[(Une distance sur les matrices sym{\'e}triques)]
On note $\mathcal{S}^{+ +}_n$ l'ensemble des matrices r{\'e}elles
sym{\'e}triques de taille $n$ d{\'e}finies positives. Montrer que pour toute
paire $A, B \in \mathcal{S}^{+ +}_n$, il existe $G \in \mathrm{GL}_n
(\mathbb{R})$ tel que $B = GAG^t$.

Pour toute fonction $f : \mathbb{R}_+^{\ast} \to \mathbb{R}$ et $A \in
\mathcal{S}^{+ +}_n$, donner un sens {\`a} $f (A)$. {\`A} l'aide de cette
d{\'e}finition, on pose
\[ d (A, B) = \| \log (A^{- 1 / 2} BA^{- 1 / 2})\|, \]


o{\`u} $\|.\|$ est la norme d'op{\'e}rateur relative {\`a} la norme
euclidienne sur~$\mathbb{R}^n$. Montrer que
\[ d (GAG^t, GBG^t) = d (A, B) \]


pour tout $G \in \mathrm{GL}_n (\mathbb{R})$, puis que~$d$ d{\'e}finit une
distance sur $\mathcal{S}^{+ +}_n$.
\end{exercise}

\subsection*{Solution. (ZINE Akram)}
\addcontentsline{toc}{subsection}{Solution. (ZINE Akram)}

Soit $R \in \mathcal{S}^{+ +}_n$ tel que $A = R^T R$ et soit $S \in
\mathcal{S}^{+ +}_n$ tel que $B = S^2$. On pose $G = R^{- 1} S \in \text{GL}_n
(\mathbb{R})$. Alors, nous avons :
\[ G^T AG = S (R^{- 1} RR^{- 1}) S = S^2 = B. \]


Ainsi, pour toute paire $A, B \in \mathcal{S}^{+ +}_n$, il existe $G \in
\text{GL}_n (\mathbb{R})$ tel que $B = G^T AG$.

\

Soit $G \in O_n (\mathbb{R})$ et $\lambda_1, \ldots, \lambda_n > 0$ tels que
$A = G \text{Diag} (\lambda_1, \ldots, \lambda_n) G^{- 1}$. On d{\'e}finit :
\[ f (A) = G \text{Diag} (f (\lambda_1), \ldots, f (\lambda_n)) G^{- 1} . \]


Pour montrer que cette d{\'e}finition ne d{\'e}pend pas de la
d{\'e}composition, soit $\Sigma \subseteq \mathbb{R}_+$ le spectre de $A$. Il
existe un polyn{\^o}me interpolateur de Lagrange $Q \in \mathbb{R}[X]$ tel que
:
\[ \forall \lambda \in \Sigma, \quad f (\lambda) = Q (\lambda) . \]


Ainsi,
\[ f (A) = G \text{Diag} (Q (\lambda_1), \ldots, Q (\lambda_n)) G^{- 1} . \]


Soit $R \in \mathcal{S}^{+ +}_n$ tel que $R^2 = A$. Alors, $R^{- 1} BR^{- 1}
\in \mathcal{S}^{+ +}_n$, et on note ses valeurs propres par $0 < \lambda_1
\leq \ldots \leq \lambda_n$. On a :
\[ \lambda_n = \underset{y \neq 0}{\sup}   \frac{\langle By, y
   \rangle}{\langle Ay, y \rangle}, \quad \lambda_1 = \underset{y \neq 0}{\inf
   }  \frac{\langle By, y \rangle}{\langle Ay, y \rangle} . \]


La matrice $S = \ln (R^{- 1} BR^{- 1})$ a pour valeurs propres $\ln
(\lambda_1), \ldots, \ln (\lambda_n)$. La distance est alors d{\'e}finie par :
\[ d (A, B) = \max (| \ln (\lambda_1) |, | \ln (\lambda_n) |) = \underset{y
   \neq 0}{\sup} \left| \ln \left( \frac{\langle By, y \rangle}{\langle Ay, y
   \rangle} \right) \right| . \]


Montrons que $d (G^T AG, G^T BG) = d (A, B)$ pour tout $G \in \text{GL}_n
(\mathbb{R})$. On a :
\begin{eqnarray*}
  d (G^T AG, G^T BG) & = & \underset{Y \neq 0}{\sup} \left| \ln \left(
  \frac{\langle G^T BGY, Y \rangle}{\langle G^T AGY, Y \rangle} \right)
  \right|\\
  & = &  \underset{Y \neq 0}{\sup} \left| \ln \left( \frac{\langle BGY, GY
  \rangle}{\langle AGY, GY \rangle} \right) \right|\\
  & = & \underset{Z \neq 0}{\sup} \left| \ln \left( \frac{\langle BZ, Z
  \rangle}{\langle AZ, Z \rangle} \right) \right|\\
  & = & d (A, B)
\end{eqnarray*}
Ainsi, la distance est invariante par changement de base.

\tmtextbf{V{\'e}rification des propri{\'e}t{\'e}s de distance}

- \tmtextbf{Sym{\'e}trie} : Il est clair que $d (A, B) = d (B, A) \geq 0$.

- \tmtextbf{S{\'e}paration} : Si $d (A, B) = 0$, alors $\ln (R^{- 1} BR^{-
1}) = 0$, donc $R^{- 1} BR^{- 1} = I_n$, ce qui implique $A = B$.

Pour l'in{\'e}galit{\'e} triangulaire, soit $C \in \mathcal{S}^{+ +}_n$. Pour
tout $Y \in M_{n, 1} (\mathbb{R}) \setminus \{0\}$ :
\[ \frac{\langle BY, Y \rangle}{\langle AY, Y \rangle} = \frac{\langle BY, Y
   \rangle}{\langle CY, Y \rangle} \times \frac{\langle CY, Y \rangle}{\langle
   AY, Y \rangle} . \]


En appliquant le logarithme, on obtient :
\[ \left| \ln \left( \frac{\langle BY, Y \rangle}{\langle AY, Y \rangle}
   \right) \right| \leq \left| \ln \left( \frac{\langle BY, Y \rangle}{\langle
   CY, Y \rangle} \right) \right| + \left| \ln \left( \frac{\langle CY, Y
   \rangle}{\langle AY, Y \rangle} \right) \right| . \]


En prenant la borne sup{\'e}rieure, on obtient :
\[ d (A, B) \leq d (A, C) + d (C, B) . \]


Cela prouve que $d (A, B)$ d{\'e}finit bien une distance sur $S_{+ +}^n
(\mathbb{R})$.
\[ \maltese \maltese \maltese \maltese \maltese \maltese \maltese \]
