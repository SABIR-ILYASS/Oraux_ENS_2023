L'exercice 2 traite d'une propri{\'e}t{\'e} de divisibilit{\'e} concernant le
cardinal du groupe des matrices inversibles modulo un nombre premier. Il
demande de d{\'e}montrer que le cardinal de ${GL}_{n - 1} (\mathbb{Z}/
p\mathbb{Z})$ est divisible par $n$, pour $n \geq 3$ et $p$ premier impair.
Cet exercice combine des {\'e}l{\'e}ments d'alg{\`e}bre lin{\'e}aire et de
th{\'e}orie des groupes, avec une touche d'arithm{\'e}tique modulaire.

\begin{exercise}[(Une propri{\'e}t{\'e} de divisibilit{\'e} du cardinal
des matrices inversibles modulo $p$)]
Soit $p$ un entier premier impair, et $n \geq 3$ un entier. Montrer que $n$
divise le cardinal du groupe $\mathrm{GL}_{n - 1} (\mathbb{Z}/ p\mathbb{Z})$
des matrices inversibles de taille~$n - 1$ {\`a} coefficients
dans~$\mathbb{Z}/ p\mathbb{Z}$.
\end{exercise}
\subsection*{Solution. (SABIR Ilyass)}
\addcontentsline{toc}{subsection}{Solution. (SABIR Ilyass)}

Soient $p$ un nombre premier impair, et $n \geqslant 3$ un entier.

Une matrice $M \in \mathcal{M}_{n - 1} (\mathbb{Z}/ p\mathbb{Z})$ est
inversible si et seulement si les colonnes de $M$ forment une famille libre.

On a $p^{n - 1} - 1$ possibilit{\'e}s de choisir la premi{\`e}re colonne, pour
tout $k \in \llbracket 1, n - 2 \rrbracket$, si on choisit les $k$ premiers
colonnes, alors la $(k + 1)$-{\`e}me colonne ne doit pas {\^e}tre une
combinaison lin{\'e}aire des $k$ premiers colonnes. Donc on a $p^{n - 1} -
p^k$ possibilit{\'e}s pour choisir la $(k + 1) - {\`e} \tmop{me}$ colonne.

D'o{\`u}
\[ \tmop{Card} (\tmop{GL}_{n - 1} (\mathbb{Z}/ p\mathbb{Z})) = \underset{k =
   0}{\overset{n - 2}{\prod}} (p^{n - 1} - p^k) \]


Donc, il suffit de montrer que $n$ divise le produit $\underset{k =
0}{\overset{n - 2}{\prod}} (p^{n - 1} - p^k)$.

On a
\begin{eqnarray*}
  \underset{k = 0}{\overset{n - 2}{\prod}} (p^{n - 1} - p^k) & = & \underset{k
  = 0}{\overset{n - 2}{\prod}}  p^k (p^{n - k - 1} - 1)\\
  & = & p^{\frac{(n - 2) (n - 1)}{2}} \underset{k = 0}{\overset{n -
  2}{\prod}} (p^{n - k - 1} - 1)
\end{eqnarray*}


Or, le th{\'e}or{\`e}me fondamental de l'arithm{\'e}tique assure l'existence
de $k, q \in \mathbb{N}$ tels que
\[ n = p^k q \infixand p \wedge q = 1 \]


Puisque $k < \frac{(n - 2) (n - 1)}{2}$, alors $p^k$ divise $\underset{k =
0}{\overset{n - 2}{\prod}} (p^{n - 1} - p^k)$.

Montrons maintenant que $q$ divise $\underset{k = 0}{\overset{n - 2}{\prod}}
(p^{n - 1} - p^k)$ pour conclure.

Sans perte de g{\'e}n{\'e}ralit{\'e}, on peut supposer que $q \geqslant 2$.

On a d'apr{\`e}s le th{\'e}or{\`e}me de Fermat-Euler,
\[ p^{\varphi (q)} = 1 (\tmop{mod} q) \]


Avec $\varphi (q) \leqslant q \leqslant n - 2$, alors
\[ p^{n - 1} = p^{\varphi (q)} p^{n - 1 - \varphi (q)} = p^{n - 1 - \varphi
   (q)}  (\tmop{mod} q) \]


Ainsi $q$ divise $p^{n - 1} - p^{n - 1 - \varphi (q)}$, et donc $q$ divise
$\underset{}{\overset{}{}} \underset{k = 0}{\overset{n - 2}{\prod}} (p^{n - 1}
- p^k)$.

Par suite $q$ divise $\underset{k = 0}{\overset{n - 2}{\prod}} (p^{n - 1} -
p^k)$.

Or $p \wedge q = 1$, alors $p^k \wedge q = 1$, donc d'apr{\`e}s le lemme
d'Euclide $n = p^k q$ divise $\underset{k = 0}{\overset{n - 2}{\prod}} (p^{n -
1} - p^k)$.

D'o{\`u} le r{\'e}sultat.
\[ \maltese \maltese \maltese \maltese \maltese \maltese \maltese \]
