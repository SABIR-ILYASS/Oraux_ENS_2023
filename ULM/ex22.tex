L'exercice 22 porte sur la m{\'e}diane de moyennes de variables
al{\'e}atoires. Il demande de prouver une borne de probabilit{\'e} sur
l'{\'e}cart entre cette m{\'e}diane et la moyenne th{\'e}orique. L'exercice
fait appel {\`a} des notions de probabilit{\'e}s et de statistiques.

\begin{exercise}[(M{\'e}diane de moyennes)]
Soient $n, m \geq 1$ des entiers et soient $X_{i, j}$, pour $1 \leq i \leq n$
et $1 \leq j \leq m$, des variables al{\'e}atoires discr{\`e}tes i.i.d. de
variance $\sigma^2$ et de moyenne~$\mu$. Pour~$1 \leq i \leq n$, soit $Y_i =
\frac{1}{m}  \sum_{j = 1}^m X_{i, j}$. Soit $Z$ une m{\'e}diane de l'ensemble
$\{ Y_1, \ldots, Y_n \}$.

Montrer que
\[ \mathbb{P} \left[ |Z - \mu | \leq \frac{2 \sigma}{\sqrt{m}} \right] \geq 1
   - \left( \frac{3}{4} \right)^{\frac{n}{2}} . \]

\end{exercise}

\subsection*{Solution. (ZINE Akram, SABIR Ilyass)}
\addcontentsline{toc}{subsection}{Solution. (ZINE Akram, SABIR Ilyass)}

Esp{\'e}rance et Variance :\tmtextbf{}
\[ E [Y_i] = \mu \quad \text{et} \quad \text{Var} (Y_i) = \frac{\sigma^2}{m} .
\]


Pour chaque $Y_i$, appliquons l'in{\'e}galit{\'e} de Chebyshev pour estimer la
probabilit{\'e} que $Y_i$ s'{\'e}carte de $\mu$ de plus de $\frac{2
\sigma}{\sqrt{m}}$ :
\[ P \left( |Y_i - \mu | > \frac{2 \sigma}{\sqrt{m}} \right) \leq
   \frac{\text{Var} (Y_i)}{\left( \frac{2 \sigma}{\sqrt{m}} \right)^2} =
   \frac{\sigma^2 / m}{4 \sigma^2 / m} = \frac{1}{4} . \]


Ainsi,
\[ P \left( |Y_i - \mu | \leq \frac{2 \sigma}{\sqrt{m}} \right) \geq
   \frac{3}{4} . \]


La m{\'e}diane $Z$ de l'ensemble $\{Y_1, Y_2, \ldots, Y_n \}$ sera dans
l'intervalle $\left[ \mu - \frac{2 \sigma}{\sqrt{m}}, \mu + \frac{2
\sigma}{\sqrt{m}} \right]$ si au moins la moiti{\'e} des $Y_i$ se trouvent
dans cet intervalle.

D{\'e}finissons $S$ comme le nombre de $Y_i$ satisfaisant $|Y_i - \mu | \leq
\frac{2 \sigma}{\sqrt{m}}$. Chaque $Y_i$ satisfait cette condition avec une
probabilit{\'e} $p \geq \frac{3}{4}$, ind{\'e}pendamment des autres. Ainsi,
$S$ suit une loi binomiale $\mathcal{B}(n, p)$ avec $p \geq \frac{3}{4}$.

Pour obtenir une borne sur $P \left( S > \frac{n}{2} \right)$, nous allons
consid{\'e}rer la probabilit{\'e} compl{\'e}mentaire $P \left( S \leq
\frac{n}{2} \right)$ et la majorer.

Pour tout $t > 0$, l'in{\'e}galit{\'e} de Markov donne :


\[ P (S \leq k) = P (e^{- tS} \geq e^{- tk}) \leq \frac{E [e^{- tS}]}{e^{-
   tk}} . \]


Ici, nous choisissons $k = \frac{n}{2}$.

Comme $S = \sum_{i = 1}^n I_i$, o{\`u} $I_i$ est l'indicateur que $Y_i$ est
dans l'intervalle, et les $I_i$ sont ind{\'e}pendants, nous avons :
\[ E [e^{- tS}] = \prod_{i = 1}^n E [e^{- tI_i}] = (pe^{- t} + (1 - p))^n . \]


Nous devons choisir $t$ pour minimiser la borne. Pour cela, posons :
\[ f (t) = (pe^{- t} + 1 - p) e^{t / 2} . \]


Calculons la d{\'e}riv{\'e}e de $f (t)$ :
\[ f' (t) = \frac{d}{dt}  [(pe^{- t} + 1 - p) e^{t / 2}] = e^{t / 2}  \left( -
   pe^{- t} + \frac{1}{2} (pe^{- t} + 1 - p) \right) . \]


Pour trouver le minimum, r{\'e}solvons $f' (t) = 0$, cela revient {\`a}
r{\'e}soudre l'{\'e}quation
\[ - pe^{- t} + \frac{1}{2} (pe^{- t} + 1 - p) = 0 \]


Par suite,


\[ - 2 pe^{- t} + pe^{- t} + 1 - p = 0 \]


Ainsi,
\[ pe^{- t} = 1 - p \]


Donc,
\[ t = - \ln \left( \frac{1 - p}{p} \right) \]


Substituons $t = - \ln \left( \frac{1 - p}{p} \right)$ dans $f (t)$ :
\begin{eqnarray*}
  f (t) & = & (pe^{- t} + 1 - p) e^{t / 2} \\
  & = & \left( p \cdot \frac{1 - p}{p} + 1 - p \right) \left( \frac{p}{1 - p}
  \right)^{1 / 2}\\
  & = & 2 (1 - p) \cdot \frac{p}{1 - p} = 2 (p (1 - p))^{\frac{1}{2}}
\end{eqnarray*}


Ainsi, on a :
\[ P \left( S \leq \frac{n}{2} \right) \leq (4 p (1 - p))^{n / 2} . \]


$p (1 - p)$ d{\'e}croit pour $p \geq \frac{1}{2}$

Pour $p = \frac{3}{4}$, calculons cette expression :
\[ 4 p (1 - p) = 4 \cdot \frac{3}{4} \cdot \frac{1}{4} = \frac{3}{4} . \]


Donc,
\[ P \left( S \leq \frac{n}{2} \right) \leq \left( \frac{3}{4} \right)^{n / 2}
   . \]


Par cons{\'e}quent,
\[ P \left( S \geq \frac{n}{2} \right) \geq P \left( S > \frac{n}{2} \right)
   \geq 1 - \left( \frac{3}{4} \right)^{n / 2} . \]


Nous obtenons la borne souhait{\'e}e :
\[ P \left( |Z - \mu | \leq \frac{2 \sigma}{\sqrt{m}} \right) \geq 1 - \left(
   \frac{3}{4} \right)^{n / 2} . \]
\[ \maltese \maltese \maltese \maltese \maltese \maltese \maltese \]
