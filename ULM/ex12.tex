Cet exercice traite de l'impossibilit{\'e} de la densit{\'e} d'un certain
espace de translations. Il demande d'{\'e}tudier l'ensemble des translations
qui laissent invariant un espace engendr{\'e} par les translat{\'e}s entiers
d'une fonction {\`a} support compact. L'exercice fait appel {\`a} des notions
d'analyse fonctionnelle.

\begin{exercise}[(Impossibilit{\'e} de la densit{\'e} d'un certain
espace de translations)]
Soit $B (\mathbb{R})$ l'espace vectoriel des fonctions born{\'e}es sur
$\mathbb{R}$ muni de la norme uniforme. Soit $g : \mathbb{R} \to \mathbb{R}$
une fonction {\`a} support compact. On note $W (g) \subseteq B (\mathbb{R})$
l'espace engendr{\'e} par les translat{\'e}s $x \mapsto g (x - n)$ pour $n \in
\mathbb{Z}$.

Etudier l'ensemble $t \in \mathbb{R}$ tels que $\overline{W (g)}$ est
invariant par translation par $t$.
\end{exercise}

\subsection*{Solution. (ZINE Akram)}
\addcontentsline{toc}{subsection}{Solution. (ZINE Akram)}

On suppose que $g$ est non nulle, sans quoi le probl{\`e}me est trivial et $G
=\mathbb{R}$. Raisonnons par l'absurde en supposant que l'ensemble des $t \in
\mathbb{R}$ pour lesquels l'adh{\'e}rence de $W (g)$, not{\'e}e $\overline{W
(g)}$, est invariante par translation, est dense dans $\mathbb{R}$. Autrement
dit, supposons que ce sous-groupe $G \subset \mathbb{R}$ soit dense.

Puisque $\overline{W (g)}$ est invariant par $t$ pour tout $t \in G$, prenons
un {\'e}l{\'e}ment $t \in G$. Par d{\'e}finition de l'invariance par
translation, cela signifie que $g (x - t)$ peut {\^e}tre exprim{\'e}e comme
une combinaison lin{\'e}aire des translat{\'e}s de $g$ par des entiers, soit :
\[ g (x - t) = \sum_{n \in \mathbb{Z}} \alpha_n g (x - n) . \]


Notre objectif est maintenant de faire intervenir la transform{\'e}e de
Fourier pour mieux comprendre cette expression. D{\'e}finissons une suite de
sommes partielles
\[ g_N (x - t) = \sum_{|n| \leq N} \alpha_n g (x - n) \]


qui converge uniform{\'e}ment vers $g (x - t)$, car $g$ est born{\'e}e et de
support compact. Cela signifie que $g_N$ est major{\'e}e par une fonction de
support compact $f$ pour tout $N$ et tout $x$.

Ainsi, nous pouvons appliquer le th{\'e}or{\`e}me de convergence domin{\'e}e
pour justifier l'interversion entre la transform{\'e}e de Fourier et la
limite, en concluant que la transform{\'e}e de Fourier de $g_N$ converge vers
la transform{\'e}e de Fourier de $g$ lorsque $N \to \infty$.

\tmtextbf{Lemme 1. }

La transform{\'e}e de Fourier d'une fonction $g$ {\`a} support compact est
analytique.

\

\tmtextbf{Preuve du lemme 1. }

La transform{\'e}e de Fourier d'une fonction $g$ {\`a} support compact est
donn{\'e}e par :
\[ \hat{g} (\xi) = \int_{- a}^a g (t) e^{- 2 i \pi \xi t}  \hspace{0.17em} dt,
\]


o{\`u} $g$ est {\`a} support dans $[- a, a]$.

Nous pouvons d{\'e}velopper l'exponentielle complexe en s{\'e}rie de Taylor :
\[ e^{- 2 i \pi \xi t} = \sum_{k \geq 0} \frac{(- 2 i \pi t \xi)^k}{k!} \]


Ainsi,
\begin{eqnarray*}
  \hat{g} (\xi) & = &  \int_{- a}^a g (t) \left( \sum_{k \geq 0} \frac{(- 2 i
  \pi t \xi)^k}{k!} \right) dt\\
  & = & \sum_{k \geq 0} \xi^k  \int_{- a}^a g (t) \frac{(- 2 i \pi t)^k}{k!} 
  \hspace{0.17em} dt
\end{eqnarray*}


o{\`u} l'interversion de la somme et de l'int{\'e}grale est permise par la
convergence de
\[ \int_{- a}^a \sum_{k \geq 0} \left| \frac{(- 2 i \pi t \xi)^k}{k!} \right|
   |g (t) | \hspace{0.17em} dt \]


Cela montre que $\hat{g} (\xi)$ est exprim{\'e}e comme une s{\'e}rie
enti{\`e}re en $\xi$, ce qui signifie que $\hat{g} (\xi)$ est analytique en
tant que fonction de la variable complexe $\xi$ dans le plan complexe.

Utilisons le lemme suivant pour montrer qu'il existe $\xi$ et $\xi + 1$,
telles que $\hat{g} (\xi)$ et $\hat{g} (\xi + 1)$ soient non nulles.

\

D'apr{\`e}s le lemme pr{\'e}c{\'e}dent, si la transform{\'e}e de Fourier est
non nulle, alors elle est analytique, et donc ses z{\'e}ros sont isol{\'e}s.
Si $\hat{g}$ ne s'annule pas, la d{\'e}monstration est termin{\'e}e.

Sinon, il existe un $\xi$ tel que $\hat{g} (\xi) = 0$. Comme les z{\'e}ros
sont isol{\'e}s, il existe $\epsilon' > 0$ tel que pour tout $\epsilon \in] 0,
\epsilon']$, on ait $\hat{g} (\xi + \epsilon) \neq 0$.

\

D'autre part, pour la m{\^e}me raison, il existe $t \in] 0, \epsilon']$ tel
que $\hat{g} (\xi + 1 + t) \neq 0$.

Finalement, il existe $\xi$ et $\xi + 1$, telles que $\hat{g} (\xi)$ et
$\hat{g} (\xi + 1)$ soient non nulles.

La transform{\'e} de Fourier donne :
\[ \hat{g} (\xi) \left( e^{- 2 i \pi t \xi} - \sum_{n \in \mathbb{Z}} \alpha_n
   e^{- 2 i \pi n \xi} \right) = 0. \]
.

\

En prenant les valeur pour $\hat{g} (\xi)$ et $\hat{g} (\xi + 1)$ et on
observant que $e^{- 2 i \pi n \xi}$ est p{\'e}riodique de p{\'e}riode 1. On
obtient que $e^{- 2 i \pi t} = 1$, et donc $t \in \mathbb{Z}$.

Reprenons le cas o{\`u} $\hat{g} = 0$. Ceci implique, par injectivit{\'e} de
la transform{\'e}e de Fourier, que $g = 0$ presque partout au sens de la
mesure de Lebesgue.

Soit $S$ le support de $g$. Montrons qu'il existe $x \in S$, et $t \in G$ $n
\in \mathbb{Z}$, $x - n - t \nin S$.

Raisonnons par l'absurde et supposons que
\[ \forall x \in S, \forall t \in G, \exists n \in \mathbb{Z}, x - n - t \in S
\]


Ainsi, $G + S \subset \mathbb{Z}+ S$. Posons $A =\mathbb{Z}+ S$ est
ferm{\'e}. En effet, soit ($x_n) \in A^{\mathbb{N}}$ tel que ($x_n$) converge
vers $x$.

Ainsi, il existe ($z_n) \in \mathbb{Z}^{\mathbb{N}}$ et il existe ($s_n) \in
S^{\mathbb{N}}$ telles que $x_n = z_n + s_n$.

La suite ($z_n$) est born{\'e}e car S est compact et ($x_n$) converge. Il
existe donc une extractrice $\phi$ telle que ($z_{\phi (n)}$) converge vers $z
\in \mathbb{Z}$ . ($s_{\phi (n)}$) converge vers $s \in S$. Ainsi, ($x_{\phi
(n)}$) converge vers $z + s$.

Donc, ($x_n$) converge vers $z + s$.

Ce qui montre que $A$ est ferm{\'e} et dense, car il contient $S + G$ et donc
$A =\mathbb{R}$.

\

Or, puisque la mesure de Lebesgue de $S$ est nulle(comme on a suppos{\'e} que
$\hat{g}$ est nulle), alors la mesure de Lebesgue de $A$ l'est aussi.
Contradiction.

Ainsi, il existe $x \in S$ et $t \in G$, tels que pour tout $n \in \mathbb{Z},
\quad x - n - t \nin S$

{\'E}crivons
\[ g (x) = \sum_{n \in \mathbb{Z}} \alpha_n g (x - n - t) \]


On trouve que $g (x) = 0$, ce qui est absurde car $x \in S$. Ainsi, $G$ est
discret. Comme il contient $\mathbb{Z}$, on en conclut que $G = \frac{1}{n}
\mathbb{Z}$, avec $n \in \mathbb{N}^{\ast}$.
\[ \maltese \maltese \maltese \maltese \maltese \maltese \maltese \]
