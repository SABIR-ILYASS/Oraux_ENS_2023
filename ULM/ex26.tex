Cet exercice comporte deux parties distinctes. La premi{\`e}re traite des
matrices de trace nulle dans $\mathrm{SL}_2 (\mathbb{Z})$, demandant de
prouver une propri{\'e}t{\'e} de conjugaison. La seconde partie concerne les
sommes de deux carr{\'e}s, demandant de prouver une condition suffisante pour
qu'un entier soit la somme de deux carr{\'e}s. L'exercice fait appel {\`a} des
notions d'alg{\`e}bre lin{\'e}aire et de th{\'e}orie des nombres.

\begin{exercise}[(Matrices de traces nulles et sommes de deux
carr{\'e}s)]
Soient~$A, B \in \mathrm{SL}_2 (\mathbb{Z})$ (le groupe des matrices $2 \times
2$ {\`a} coefficients entiers et d{\'e}terminant~1).

1. Montrer que si~$\mathrm{Tr} (A) = \mathrm{Tr} (B) = 0$, alors~$A$ est
conjugu{\'e}e {\`a}~$B$ ou~$- B$.

2. Montrer que si~$n > 1$ et~$x$ sont des entiers tels que~$x^2 \equiv - 1
\pmod{n}$, alors il existe des entiers~$a$ et~$b$ tels que~$n = a^2 + b^2$.

\end{exercise}

\subsection*{Solution. (ETTOUSY Badr - ZINE Akram)}
\addcontentsline{toc}{subsection}{Solution. (ETTOUSY Badr - ZINE Akram)}


\tmtextbf{1.} Consid{\'e}rons $x, y, a, b, c, d \in \mathbb{Z}$, \ et
d{\'e}finissons $\alpha_{x, y} \assign \left(\begin{array}{c}
  x\\
  y
\end{array}\right) \in \mathbb{Z}^2$, et $A \assign \left(\begin{array}{cc}
  a & c\\
  b & d
\end{array}\right) \in \tmop{SL}_2 (\mathbb{Z})$.

Supposons que $b > 0$. On consid{\`e}re $M_{x, y} = (\alpha, A \alpha)$ et on
souhaite d{\'e}montrer que :
\[ \det M_{x, y} bx^2 + (d - a) xy - cy^2 = 1 \]


En utilisant la compl{\'e}tion du carr{\'e} et par le fait que $a + d = 0$ et
$a^2 + bc = - 1$, que :
\[ \det M_{x, y} = \frac{(bx - ay)^2 + y^2}{b} \]


Ainsi, $\det M_{x, y} > 0$ pour tout $(x, y) \in \mathbb{Z}^2 \setminus \{(0,
0)\}$.

\tmtextbf{D{\'e}termination de $e$}

D{\'e}finissons $e = \underset{(x, y) \in \mathbb{Z}^2 \setminus \{(0,
0)\}}{\inf} \det M_{x, y}$. Nous allons prouver le lemme suivant :

\

\tmtextbf{Lemme 1.}

$e = \underset{(x, y) \in \mathbb{Z}^2 \setminus \{(0, 0)\}}{\inf} \det M_{x,
y}$ est atteint et on a $3 e^2 \leq - 4 (bc + a^2)$.

\tmtextbf{Preuve du lemme 1.}

\

Notons $m = \underset{(x, y) \in \mathbb{R}^2 \setminus \{(0, 0)\}}{\inf}
\det M_{x, y}$. Il est atteint car le cercle unit{\'e} de $\mathbb{R}^2$ est
compact. Par homog{\'e}n{\'e}it{\'e} de la forme quadratique, on a
\[ \det M_{x, y} \geq m (x^2 + y^2) \]


Soit $N$ un entier naturel tel que $N \geq \sqrt{\frac{e + 1}{m}}$.
D'apr{\`e}s ce qui pr{\'e}c{\`e}de, si $(x, y) \in \mathbb{Z}^2$ et $\lvert x
\rvert > N$ ou $\lvert y \rvert > N$, alors $\det M_{x, y} \geq e + 1$.

Donc, $e = \underset{\lvert x \rvert \leq N, \lvert y \rvert \leq
N}{\underset{(x, y) \in \mathbb{Z}^2 \setminus \{(0, 0)\}}{\inf}}  \det M_{x,
y}$ est atteint car l'ensemble est fini.

Soit $(\alpha, \beta) \in \mathbb{Z}^2$ tel que $\tmop{detM}_{\alpha, \beta} =
e$. Notons $d = \tmop{pgcd} (\alpha, \beta)$ et $\alpha'$, $\beta'$ les
entiers tels que $\alpha = d \alpha'$ et $\beta = d \beta'$. On obtient
\[ e = d^2 q (\alpha', \beta') \geq d^2 e \]


Ainsi, $d = 1$. D'apr{\`e}s le th{\'e}or{\`e}me de B{\'e}zout, il existe
$(\gamma, \delta)$ tel que $\alpha \delta + \beta \delta = 1$. Posons $v = (-
\delta, \gamma)$.

$P = \left(\begin{array}{cc}
  \alpha & - \delta\\
  \beta & \gamma
\end{array}\right)$ et $P^{- 1}$ est la matrice de passage de $(u, v)$ {\`a}
$(e_1, e_2)$ (base canonique de $\mathbb{R}^2$). Donc, $\mathbb{Z}^2
=\mathbb{Z}u +\mathbb{Z}v$.

Il existe $(a', c')$ tels que, dans la base $(u, v)$ on ait $q (x, y) = ex^2 +
2 a' xy + c' y^2$. On a :
\[ \left(\begin{array}{cc}
     e & a'\\
     a' & c'
   \end{array}\right) = P^{\top} \left(\begin{array}{cc}
     b & - a\\
     - a & - c
   \end{array}\right) P \]


et donc $c' {e - a'}^2 = (\det P)^2  (- bc - a^2) = - bc - a^2$.

Pour tout $(x, y) \in \mathbb{Z}^2$, on obtient
\[ q (xu + yv) = e \left( x + \frac{a'}{e} y \right)^2 + \left( c' -
   \frac{(a')^2}{e} \right) y^2 \]


En prenant $y = 1$ et en choisissant $n \in \mathbb{Z}$ tel que $\lvert n +
\frac{a'}{e} \rvert \leq \frac{1}{2}$, on a
\[ e \leq q (nu + v) \leq \frac{e}{4} + c' - \frac{(a')^2}{e} \]


et donc $3 e^2 \leq - 4 (bc + a^2)$.

\tmtextbf{Conclusion.}

Ainsi, comme $e > 0$, on a $e = 1$. D'o{\`u} il existe $\alpha \in \mathbb{R}$
tel que $\det M = 1$.

La matrice $A$ dans la base $(\alpha, A \alpha)$ est sous la forme :
\[ \left(\begin{array}{cc}
     0 & m\\
     1 & n
   \end{array}\right) \]
avec $m, n \in \mathbb{Z}$. Le d{\'e}terminant et la trace sont conserv{\'e}s
par conjugaison. Ainsi, on trouve que $n = 0$ et $m = - 1$.

Si $b < 0$, on consid{\`e}re la base $(A \alpha, \alpha)$. Dans ce cas, $- b
> 0$ et on introduit une forme quadratique de la m{\^e}me fa{\c c}on (avec
pour coefficient dominant $- b$). On obtient ainsi que $A$ est conjugu{\'e}e
{\`a} $\left(\begin{array}{cc}
  0 & 1\\
  - 1 & 0
\end{array}\right)$ ou $\left(\begin{array}{cc}
  0 & - 1\\
  1 & 0
\end{array}\right)$.

Cela implique que pour toute matrice $B$ sans SL$_2 (\mathbb{Z})$ telle que
$\det B = 1$ et Tr$(B) = 0$, $A$ est conjugu{\'e}e {\`a} $B$ ou $- B$.

\tmtextbf{2.} Si a et b sont somme de carr{\'e}s $a = x^2 + y^2$ et $b = z^2 +
t^2$ alors $ab = (xz - yt)^2 + (xt + yz)^2$ On va se r{\'e}duire donc au cas
o{\`u} n est premier.

\

\tmtextbf{Lemme 2.}

Soit $p$ premier. Pour tout $a \in \mathbb{Z}$, $p \nmid a$. Il existe $x, y
\in \{1, 2, \ldots, \lfloor \sqrt{p} \rfloor\}$ tels que $ax \equiv y
\pmod{p}$ ou $ax \equiv - y \pmod{p}$.

\

\tmtextbf{Preuve du lemme 2.}

Consid{\'e}rons $S = \{1, 2, \ldots, \lfloor \sqrt{p} \rfloor\}^2$. On a $p <
(\lfloor \sqrt{p} \rfloor + 1)^2$, et donc card$(S) > p$. D'apr{\`e}s le
principe des tiroirs, il existe $(x_1, y_1) \neq (x_2, y_2)$ telles que $ax_1
- y_1 \equiv ax_2 - y_2 \pmod{p}$. On prouve facilement par l'absurde que $x$
et $y$ sont non nuls. D'o{\`u} le r{\'e}sultat.

\

\tmtextbf{Conclusion.}

Revenons {\`a} notre question, soit $x \in \mathbb{Z}$ tel que $x^2 \equiv - 1
\pmod{p}$. On a $p \nmid x$. Il existe $a, b$ d'apr{\`e}s le lemme tels que
$xa \equiv \pm b \pmod{p}$. On a $x^2 a^2 + a^2 = b^2 + a^2 \equiv 0
\pmod{p}$, et donc $a^2 + b^2 = kp$ avec $k \in \mathbb{Z}$. On a $x^2 + y^2
\geq 2$. Ainsi $k > 0$.

Nous montrons que $k = 1$. $a, b \leq \lfloor \sqrt{p} \rfloor$, d'o{\`u}
$a^2 + b^2 < 2 p$. Or, $p / a^2 + b^2$, d'o{\`u} $p = a^2 + b^2$.
\[ \maltese \maltese \maltese \maltese \maltese \maltese \maltese \]
