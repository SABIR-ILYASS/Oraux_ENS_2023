Cet exercice porte sur le th{\'e}or{\`e}me d'Hermite-Kakeya. Il demande de
prouver une caract{\'e}risation des paires de polyn{\^o}mes qui s'entrelacent,
c'est-{\`a}-dire dont les racines r{\'e}elles sont simples et altern{\'e}es.
L'exercice fait appel {\`a} des notions de th{\'e}orie des polyn{\^o}mes et
d'analyse r{\'e}elle.
\begin{exercise}[(Th{\'e}or{\`e}me d'Hermite--Kakeya)]
Soient $P$ et $Q \in \mathbb{R} [X]$ des polyn{\^o}mes non constants. On dit
que~$P$ et~$Q$ {\tmem{s'entrelacent}} si:

(1)~leurs racines sont r{\'e}elles et simples,

(2)~ils n'ont pas de racines r{\'e}elles communes,

(3)~entre deux racines cons{\'e}cutives de $Q$ (resp.~$P$), il y a une et une
seule racine de~$P$ (resp.~$Q$).

Montrer que si pour tout~$(\lambda, \mu) \in \mathbb{R}^2 \setminus \{ (0, 0)
\}$, les racines de $\lambda P + \mu Q$ sont toutes r{\'e}elles et simples,
alors $P$ et $Q$ s'entrelacent.

Montrer la r{\'e}ciproque.
\end{exercise}

\subsection*{Solution. (ETTOUSY Badr, ZINE Akram)}
\addcontentsline{toc}{subsection}{Solution. (ETTOUSY Badr, ZINE Akram)}


\tmtextbf{M{\'e}thode 1 : (ETTOUSY Badr)}

Soit $P$ et $Q$ dans $\mathbb{R} [X]$.

On suppose que $\forall (\lambda, \mu) \in \mathbb{R}^2$, le polyn{\^o}me
$\lambda P (X) + \mu Q (X)$ a toutes ses racines r{\'e}elles et simples.
Montrons que les racines de $P$ et $Q$ sont intercal{\'e}es (c'est-{\`a}-dire,
entre deux racines de $P$ il existe une racine de $Q$ et r{\'e}ciproquement).

En utilisant les couples $(\lambda, \mu) = (1, 0)$ et $(\lambda, \mu) = (0,
1)$, on voit d{\'e}j{\`a} que $P$ et $Q$ doivent avoir toutes leurs racines
r{\'e}elles. Les polyn{\^o}mes $P$ et $Q$ jouant des r{\^o}les
sym{\'e}triques, il suffit de montrer qu'entre deux racines de $P$, il existe
au moins une racine de $Q$.

Soit $a$ et $b$ deux racines cons{\'e}cutives de $P$, et supposons que $Q$ ne
s'annule pas sur $[a, b]$. La fraction rationnelle $R (x) = \frac{P (x)}{Q
(x)}$ est donc de classe $C^1$ sur $[a, b]$ avec $R (a) = R (b) = 0$.

D'apr{\`e}s le th{\'e}or{\`e}me de Rolle, il existe $c \in] a, b [$ tel que
$R' (c) = 0$. La fraction rationnelle $R (x) - R (c)$ admet donc $c$ comme
z{\'e}ro de multiplicit{\'e} au moins 2.

En vertu du lemme suivant, pour $r$ assez petit, il existe sur la
circonf{\'e}rence $|z - c| = r$ au moins quatre points tels que Im$(R (z) - R
(c)) = 0$, soit Im$\hspace{0.17em} R (z) = 0$. L'un au moins de ces points
$z_0$ n'est pas r{\'e}el. Alors, $R (z_0) = \mu \in \mathbb{R}$. Le
polyn{\^o}me $P (x) - \mu Q (x)$ s'annule en un point $z_0$ non r{\'e}el, ce
qui contredit notre hypoth{\`e}se.

\tmtextbf{Lemme 1.}

Soit $R$ une fraction rationnelle qui admet $z_0$ comme racine de
multiplicit{\'e} $k$. Alors, pour $r$ assez petit, il existe au moins $2 k$
points v{\'e}rifiant $|z - z_0 | = r$ et Im$\hspace{0.17em} R (z) = 0$.

\tmtextbf{Preuve du lemme 1.}

Posons $z - z_0 = re^{i \theta}$ et $\frac{1}{k!} R^{(k)} (z_0) = \rho e^{i
\alpha}$.

La formule de Taylor Young nous dit alors que
\[ R (z) = \rho r^k e^{i (\alpha + k \theta)}  (1 + \varepsilon (z - z_0)) \]


avec $\underset{u \to 0}{\lim}  \varepsilon (u) = 0$. Soit $\eta > 0$ tel que
$|u| < \eta$ et $| \varepsilon (u) | < 1$, et choisissons $r < \eta$.

Posons $f (\theta) = \text{Im} \hspace{0.17em} R (z_0 + re^{i \theta})$,
$\varepsilon (u) = \varepsilon_1 (u) + i \varepsilon_2 (u)$, avec
$\varepsilon_1 (u)$ et $\varepsilon_2 (u)$ r{\'e}els. On obtient :
\[ f (\theta) = \text{Im} \hspace{0.17em} R (z) = \rho r^k  (\cos (\alpha + k
   \theta) \varepsilon_2 (z - z_0) + \sin (\alpha + k \theta) (1 +
   \varepsilon_2 (z - z_0))) \]


D{\'e}finissons $\theta_m$ pour $m \in \{0, 1, \ldots, 2 k\}$ par $\alpha + k
\theta_m = m \pi + \frac{\pi}{2}$. On a alors :
\[ f (\theta_m) = \rho r^k  (- 1)^m  (1 + \beta_m) \]
\

avec $| \beta_m | < 1$. Donc $f (\theta_m)$ est du signe de $(- 1)^m$. La
fonction $f$ change donc $2 k + 1$ fois de signe sur un intervalle de longueur
$2 \pi$, et donc elle s'annule au moins $2 k$ fois, ce qui ach{\`e}ve la
d{\'e}monstration.

\tmtextbf{La r{\'e}ciproque }

Soit $(P, Q)$ un couple de polyn{\^o}mes r{\'e}els simplement scind{\'e}s tels
qu'entre deux racines de l'un il y ait toujours au moins une racine de
l'autre. Montrons que le polyn{\^o}me $\lambda P + \mu Q$ reste scind{\'e}
lorsque le couple $(\lambda, \mu)$ d{\'e}crit $\mathbb{R}^2$.

Il est loisible de se ramener au cas o{\`u} $\lambda \neq 0$ et o{\`u} les
deux polyn{\^o}mes sont de la forme :
\[ P = \prod_{i = 1}^n (X - a_i) \infixand Q = \prod_{j = 1}^m (X - b_j) \]


avec $m = n$ ou $m = n - 1$. Nous supposerons de plus que :
\[ a_1 < b_1 < a_2 < b_2 < \ldots < a_n < b_n \tmop{si} m = n \]


cas que nous traiterons en premier.

Le signe des valeurs de la fonction rationnelle $P / Q$ aux voisinages des
infinis et des r{\'e}els $b_j$, ainsi que son annulation en les r{\'e}els
$a_i$, montre que l'{\'e}quation $\lambda P (x) + \mu Q (x) = 0$ poss{\`e}de
au moins $n$ racines r{\'e}elles si $\lambda + \mu \neq 0$, {\`a} savoir une
dans chaque intervalle $] b_j, b_{j + 1} [$ et une autre dans l'un des deux
intervalles $] - \infty, b_1 [$ et $] b_n, + \infty [$, et au moins $n - 1$
racines r{\'e}elles dans le cas contraire -- la derni{\`e}re pouvant {\^e}tre
alors consid{\'e}r{\'e}e comme {\'e}tant devenue infinie.

Ce nombre de racines {\'e}tant toujours exactement {\'e}gal au degr{\'e} de
$\lambda P + \mu Q$, ce dernier polyn{\^o}me est donc (simplement) scind{\'e}.

Il en va tout de m{\^e}me si $m = n - 1$ (ici d'ailleurs le degr{\'e} de
$\lambda P + \mu Q$ est toujours {\'e}gal {\`a} $n$). Enfin $\lambda P + \mu
Q$ est scind{\'e} (mais alors non simplement) dans le cas $\lambda = \mu = 0$.

\tmtextbf{M{\'e}thode 2 : (ZINE Akram)}

Raisonnons par contrapos{\'e}e. Supposons que $P$ et $Q$ ne s'entrelacent pas.
Cela signifie qu'il existe deux racines $\lambda_1$ et $\lambda_2$ de $P$
telles qu'il n'y ait aucune racine de $Q$ entre elles. D{\'e}finissons alors
la fonction $F = \frac{P}{Q}$. Puisque $Q$ n'a pas de racine entre $\lambda_1$
et $\lambda_2$, $F$ est bien d{\'e}finie sur cet intervalle. De plus, nous
avons $F (\lambda_1) = F (\lambda_2) = 0$.

D'apr{\`e}s le th{\'e}or{\`e}me de Rolle, il existe un point $c$ situ{\'e}
entre $\lambda_1$ et $\lambda_2$ tel que $F' (c) = 0$.

Calculons alors $F' (x) = \left( \frac{P}{Q} \right)'$. En utilisant la
formule de d{\'e}rivation du quotient, on obtient :
\[ F' (x) = \frac{P' (x) Q (x) - P (x) Q' (x)}{Q (x)^2} . \]


Comme $F' (c) = 0$, cela implique que $P' (c) Q (c) - P (c) Q' (c) = 0$.
Puisque $Q (c) \neq 0$ (car $Q$ n'a pas de racine entre $\lambda_1$ et
$\lambda_2$), nous en d{\'e}duisons que $P (c) Q' (c) = P' (c) Q (c$).

Consid{\'e}rons alors le polyn{\^o}me $T = P + \alpha Q$, o{\`u} $\alpha = - F
(c)$. Puisque $F' (c) = 0$, $c$est une racine double de $T$. Ceci contredit
l'hypoth{\`e}se que pour tout $(\lambda, \mu) \in \mathbb{R}^2 \setminus \{(0,
0)\}$, les racines de $\lambda P + \mu Q$ sont toutes r{\'e}elles et simples.
Ainsi, nous avons montr{\'e} par contrapos{\'e}e que $P$ et $Q$ doivent
s'entrelacer.

\tmtextbf{Preuve du sens inverse}

Supposons maintenant que $P$ et $Q$ s'entrelacent. Nous voulons montrer que
pour tout $(\lambda, \mu) \in \mathbb{R}^2 \setminus \{(0, 0)\}$, les racines
de $R = \lambda P + \mu Q$ sont toutes r{\'e}elles et simples. Supposons que
$\lambda$ et $\mu$ sont non nuls ; sinon, la conclusion est triviale.

Soit $n = \deg P$ et $m = \deg Q$.

On suppose, sans perte de g{\'e}n{\'e}ralit{\'e}, que $n \geq m$ et que $p_1 <
q_1$.

Traitons le cas $n > m$ : \tmcolor{red}{}

On a alors $m = n - 1$. Pour tout $i \leq n - 1$, $R (p_i) = \mu Q (p_i)$ et
$R (p_{i + 1}) = \mu Q (p_{i + 1})$. Comme $Q$ change de signe en $q_i$, $R$
change {\'e}galement de signe sur $[p_i, p_{i + 1}]$ et admet donc une racine
sur cet intervalle. Ceci {\'e}tant vrai pour tout $i$, $R$ admet $n - 1$
racines r{\'e}elles distinctes.

Si $n$ est pair, soit $p$ et $q$ les coefficients dominants de $P$ et $Q$
respectivement. On a $\frac{R}{\mu q} (p_1) < 0$ et $\underset{x \to -
\infty}{\lim }  \frac{R}{\lambda p} (x) > 0$ ainsi que $\frac{R}{\mu q} (p_n)
> 0$ et $\underset{x \to \infty}{\lim }  \frac{R}{\lambda p} (x) > 0$. Par
cons{\'e}quent, l'un des deux couples $(\underset{x \to - \infty}{\lim } R
(x), R (p_1))$ ou $(R (p_n), \underset{x \to \infty}{\lim } R (x))$
poss{\`e}de deux {\'e}l{\'e}ments de signes oppos{\'e}s. Le th{\'e}or{\`e}me
des valeurs interm{\'e}diaires implique alors l'existence d'une
$n^{\text{i{\`e}me}}$ racine.

Si $n$ est impair, la conclusion reste la m{\^e}me.

Ainsi, $R$ poss{\`e}de $n$ racines r{\'e}elles distinctes et son degr{\'e} est
$n$, d'o{\`u} le r{\'e}sultat.

Le cas $n = m$ se traite de la m{\^e}me fa{\c c}on en prouvant que $n - 1$
racines de $R$ se trouvent dans les intervalles $[q_i, q_{i + 1}]$. Pour la
racine restante, on raisonne sur les couples $(\underset{x \to - \infty}{\lim}
R (x), R (q_1))$ et $(R (q_n), \underset{x \to \infty}{\lim}  R (x))$.
\[ \maltese \maltese \maltese \maltese \maltese \maltese \maltese \]
