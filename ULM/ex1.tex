L'exercice 1 porte sur les wronskiens et les syst{\`e}mes de Tchebychev. Il explore les propri{\'e}t{\'e}s des d{\'e}terminants wronskiens et leur application {\`a} l'{\'e}tude des z{\'e}ros de combinaisons lin{\'e}aires de fonctions. Cet exercice combine des aspects d'alg{\`e}bre lin{\'e}aire et d'analyse, mettant en {\'e}vidence des liens profonds entre ces domaines.

\begin{exercise}[(Wronskiens et syst{\`e}mes de Tchebychev)]

Soit~$I \subseteq \mathbb{R}$ un intervalle ouvert non vide.
Soit~$\mathcal{C}^r (I)$ le~$\mathbb{R}$-espace vectoriel des fonctions sur $I$ {\`a} valeurs r{\'e}elles continument d{\'e}rivables $r$ fois. Pour toutes fonctions $f_1, \ldots, f_r \in \mathcal{C}^{r - 1} (I)$ on d{\'e}finit une fonction $I \to \mathbb{R}$ par :

\[ \mathcal{W} [f_1, \ldots, f_r] (x) = \left| \begin{array}{ccc}
     f_1 (x) & \cdots & f_r (x)\\
     f_1' (x) & \cdots & f_r' (x)\\
     \vdots &  & \vdots\\
     f_1^{(r - 1)} (x) & \cdots & f_r^{(r - 1)} (x)
   \end{array} \right| \]


1. Montrer que pour toute fonctions~$g, f_1, \ldots, f_r \in \mathcal{C}^{r - 1} (I)$,
\[ \mathcal{W} [gf_1, \ldots, gf_r] (x) = g (x)^r \mathcal{W} [f_1, \ldots, f_r] (x) . \]

2. Soit~$f_1, \ldots, f_r \in \mathcal{C}^{r - 1} (I)$ telles que~$\mathcal{W}
[f_1, \ldots, f_k]$ est strictement positif sur~$I$, pour tout~$1 \leq k \leq r$. Montrer que pour tout~$a_1, \ldots, a_r \in \mathbb{R}$ non tous nuls, la fonction~$a_1 f_1 + \cdots + a_r f_r$ admet au plus~$r - 1$ z{\'e}ros sur $I$.
\end{exercise}

\subsection*{Solution. (SABIR Ilyass)}
\addcontentsline{toc}{subsection}{Solution. (SABIR Ilyass)}

1. Soient $g, f_1, \ldots, f_r \in \mathcal{C}^{r - 1} (I)$ et $x \in
\mathbb{R}$, on a :
\begin{eqnarray*}
  \mathcal{W} [gf_1, \ldots, gf_r] (x) & = & \left| \begin{array}{ccc}
    (g f_1) (x) & \cdots & (g f_r) (x)\\
    (g f_1)' (x) & \cdots & (g f_1)^{'} (x)\\
    \vdots &  & \vdots\\
    (g f_1)^{(r - 1)} (x) & \cdots & (g f_r)^{(r - 1)} (x)
  \end{array} \right|
\end{eqnarray*}

Or, pour tout $j, k \in \llbracket 1, r \rrbracket$, on a :
\[ (1) : (g f_j)^{(k)} (x) = \underset{i = 0}{\overset{k}{\sum}} \left(
   \begin{array}{c}
     k\\
     i
   \end{array} \right) g^{(i)} (x) f^{(k - i)}_j (x) \]

Donc
\begin{eqnarray*}
  {}[gf_1, \ldots, gf_r] (x) & = & g (x) \left| \begin{array}{ccc}
    f_1 (x) & \cdots & f_r (x)\\
    g (x) f_1' (x) + g' (x) f_1 (x) & \cdots & g (x) f_r' (x) + g' (x) f_r
    (x)\\
    \vdots &  & \vdots\\
    (g f_1)^{(r - 1)} (x) & \cdots & (g f_r)^{(r - 1)} (x)
  \end{array} \right|\\
  & = & g (x) \left| \begin{array}{ccc}
    f_1 (x) & \cdots & f_r (x)\\
    g (x) f_1' (x) & \cdots & g (x) f_r' (x)\\
    \vdots &  & \vdots\\
    (g f_1)^{(r - 1)} (x) & \cdots & (g f_r)^{(r - 1)} (x)
  \end{array} \right| L_2 \leftarrow L_2 - g' (x) L_1\\
  & = & g (x)^2 \left| \begin{array}{ccc}
    f_1 (x) & \cdots & f_r (x)\\
    f_1' (x) & \cdots & f_r' (x)\\
    \vdots &  & \vdots\\
    (g f_1)^{(r - 1)} (x) & \cdots & (g f_r)^{(r - 1)} (x)
  \end{array} \right|
\end{eqnarray*}

Soit $l \in \llbracket 1, r - 2 \rrbracket$, supposons que :
\[ \mathcal{W} [gf_1, \ldots, gf_r] (x) = g (x)^l \left| \begin{array}{ccc}
     f_1 (x) & \cdots & f_r (x)\\
     f_1' (x) & \cdots & f_r' (x)\\
     \vdots &  & \vdots\\
     f^{(l)}_1 (x) &  & f^{(l)}_r (x)\\
     (g f_1)^{(l + 1)} (x) &  & (g f_r)^{(l + 1)} (x)\\
     \vdots &  & \vdots\\
     (g f_1)^{(r - 1)} (x) & \cdots & (g f_r)^{(r - 1)} (x)
   \end{array} \right| \]

Et montrons que :
\[ \mathcal{W} [gf_1, \ldots, gf_r] (x) = g (x)^{l + 1} \left|
   \begin{array}{ccc}
     f_1 (x) & \cdots & f_r (x)\\
     f_1' (x) & \cdots & f_r' (x)\\
     \vdots &  & \vdots\\
     f^{(l + 1)}_1 (x) &  & f^{(l + 1)}_r (x)\\
     (g f_1)^{(l + 2)} (x) &  & (g f_r)^{(l + 2)} (x)\\
     \vdots &  & \vdots\\
     (g f_1)^{(r - 1)} (x) & \cdots & (g f_r)^{(r - 1)} (x)
   \end{array} \right| \]

Ce qui est {\'e}vident, en effectuant l'op{\'e}ration $L_{l + 1} \leftarrow L_{l + 1} - \underset{i = 1}{\overset{l}{\sum}} \left( \begin{array}{c} l\\ i \end{array} \right) g^{(i)} (x) L_i$, (Selon $(1)$).

D'o{\`u} par r{\'e}currence pour tout $l \in \llbracket 1, r - 1 \rrbracket$,
on a :
\[ \mathcal{W} [gf_1, \ldots, gf_r] (x) = g (x)^l \left| \begin{array}{ccc}
     f_1 (x) & \cdots & f_r (x)\\
     f_1' (x) & \cdots & f_r' (x)\\
     \vdots &  & \vdots\\
     f^{(l)}_1 (x) &  & f^{(l)}_r (x)\\
     (g f_1)^{(l + 1)} (x) &  & (g f_r)^{(l + 1)} (x)\\
     \vdots &  & \vdots\\
     (g f_1)^{(r - 1)} (x) & \cdots & (g f_r)^{(r - 1)} (x)
   \end{array} \right| \]

En particulier, pour $l = r - 1$, on a alors :
\[ \mathcal{W} [gf_1, \ldots, gf_r] (x) = g (x)^r \mathcal{W} [f_1, \ldots,
   f_r] (x) \]


2. Soient~$f_1, \ldots, f_r \in \mathcal{C}^{r - 1} (I)$ telles
que~$\mathcal{W} [f_1, \ldots, f_k]$ soit strictement positif sur~$I$, pour
tout~$1 \leq k \leq r$. Soient $a_1, \ldots, a_r \in \mathbb{R}$ non tous
nuls. Montrons que la fonction $a_1 f_1 + \cdots + a_r f_r$ admet au plus~$r -
1$ z{\'e}ros sur $I$.

Commen{\c c}ons par examiner les petites valeurs de $r$.

\textbf{Pour $r = 1$}, on a pour tout $x \in I$
\[ f_1 (x) =\mathcal{W} [f_1] (x) > 0 \]


Donc pour tout $a \not{=} 0$, on a $a f_1$ n'admet pas de z{\'e}ros sur $I$.

\textbf{Pour $r = 2$,} on a pour tout $x \in I$,
\[ \left\{\begin{array}{l}
     f_1 (x) =\mathcal{W} [f_1] (x) > 0\\
     f_2' (x) f_1 (x) - f_1' (x) f_2 (x) =\mathcal{W} [f_1, f_2] (x) > 0
   \end{array}\right. \]

Alors
\[ \left( \frac{f_2}{f_1} \right)' (x) = \frac{\mathcal{W} [f_1, f_2] (x)}{f_1
   (x)^2} > 0 \]


Soient $a, b \in \mathbb{R}$ non tous nuls, montrons que la fonction $x \in I
\longmapsto a f_1 (x) + b f_2 (x)$ admet au plus $1$ z{\'e}ro sur $I$.

Si $b = 0$, il est clair que la fonction $x \in I \longmapsto a f_1 (x) + b
f_2 (x)$ n'admet pas de z{\'e}ros sur $I$.

Sinon, les z{\'e}ros de \ $x \in I \longmapsto a f_1 (x) + b f_2 (x)$ sont
exactement les z{\'e}ros de la fonction \ $\gamma : x \in I \longmapsto a + b
\frac{f_2 (x)}{f_1 (x)}$. Or, pour tout $x \in I$, on a $\gamma' (x) = b
\left( \frac{f_2}{f_1} \right)'$, ce qui signifie que $\gamma$ est strictement
monotone sur $I$, par cons{\'e}quent $\gamma$ admet au plus un z{\'e}ro sur
$I$. D'o{\`u} le r{\'e}sultat.

Dans toute la suite, on suppose que $\tmmathbf{r \geqslant 3}$, Le
r{\'e}sultat n'est pas facile {\`a} d{\'e}montrer, donc nous allons prouver
dans un premier temps trois lemmes qui nous aideront {\`a} r{\'e}pondre {\`a}
la question.

\textbf{D{\'e}finition.}

Notons, pour tout $k \in \llbracket 1, r \rrbracket$
\[ \left\{\begin{array}{l}
     \varphi_1 \assign \mathcal{W} [f_1]\\
     \varphi_2 \assign \frac{\mathcal{W} [f_1, f_2]}{\mathcal{W} [f_1]^2}\\
     \varphi_k \assign \frac{\mathcal{W} [f_1, \ldots, f_{k - 2}] \mathcal{W}
     [f_1, \ldots, f_k]}{\mathcal{W} [f_1, \ldots, f_{k - 1}]^2} \tmop{si} 3
     \leqslant k \leqslant r
   \end{array}\right. \]

Et pour tout $k \in \llbracket 1, r - 1 \rrbracket$, on d{\'e}finit
l'op{\'e}rateur diff{\'e}rentiel $D_k$ par :
\[ D_k .f \assign \frac{d}{d x} \left( \frac{f}{\varphi_k} \right) \quad,
   \forall f \in \mathcal{C}^{r - 1} (I) \infixand D_0 .f = f \]


\textbf{Lemme 1.}

On a pour tout $k \in \llbracket 1, r \rrbracket$
\[ D_{k - 1} .D_{k - 2} \ldots D_1 .f_k = \varphi_k \]


\textbf{Preuve du lemme 1.}

Montrons le r{\'e}sultat par r{\'e}currence sur $k \in \llbracket 1, r
\rrbracket$.

Le r{\'e}sultat est trivial pour $k = 1, 2$ ($c.f$ le cas $r = 1, 2$).

Soit $k \in \llbracket 3, r \rrbracket$, supposons que pour tout $i \in
\llbracket 1, k - 1 \rrbracket$ $D_{i - 1} \ldots D_1 .f_i = \varphi_i$.
Montrons que $D_{k - 1} .D_{k - 2} \ldots D_1 .f_k = \varphi_k$.

On a d'apr{\`e}s la question 1 :
\begin{eqnarray*}
  \frac{1}{\varphi^k_1} \mathcal{W} [f_1, \ldots, f_k] & = & \left|
  \begin{array}{cccc}
    1 & \frac{f_2}{\varphi_1} & \cdots & \frac{f_k}{\varphi_1}\\
    D_1 .f_1 & D_1 .f_2 & \cdots & D_1 .f_k\\
    \vdots &  &  & \vdots\\
    D_1 .f_1 & D_1 .f_2 & \cdots & D_1 .f_k
  \end{array} \right|\\
  & = & \left| \begin{array}{cccc}
    1 & \frac{f_2}{\varphi_1} & \cdots & \frac{f_r}{\varphi_1}\\
    0 & D_1 .f_2 & \cdots & D_1 .f_r\\
    0 & \frac{d}{d x} D_1 .f_2 &  & \frac{d}{d x} D_1 .f_2\\
    \vdots &  &  & \vdots\\
    0 & \left( \frac{d}{d x} \right)^{r - 2} D_1 .f_2 & \cdots & \left(
    \frac{d}{d x} \right)^{r - 2} D_1 .f_r
  \end{array} \right|\\
  & = & \mathcal{W} [D_1 .f_2, \ldots, D_1 .f_k]
\end{eqnarray*}

Ainsi,
\begin{eqnarray*}
  \frac{1}{\varphi^{k - 1}_2} \mathcal{W} [D_1 .f_2, \ldots, D_1 .f_k] & = &
  \left| \begin{array}{ccc}
    \frac{1}{\varphi_2} D_1 .f_2 & \cdots & \frac{1}{\varphi_2} D_1 .f_r\\
    D_2 .D_1 .f_2 &  & D_2 .D_1 .f_2\\
    &  & \vdots\\
    \left( \frac{d}{d x} \right)^{r - 3} D_2 .D_1 .f_2 & \cdots & \left(
    \frac{d}{d x} \right)^{r - 3} D_2 .D_1 .f_2
  \end{array} \right|\\
  & = & \left| \begin{array}{cccc}
    1 & \frac{1}{\varphi_2} D_1 .f_3 & \cdots & \frac{1}{\varphi_2} D_1 .f_r\\
    0 & D_2 .D_1 .f_3 &  & D_2 .D_1 .f_2\\
    &  &  & \vdots\\
    0 & \left( \frac{d}{d x} \right)^{r - 3} D_2 .D_1 .f_3 & \cdots & \left(
    \frac{d}{d x} \right)^{r - 3} D_2 .D_1 .f_2
  \end{array} \right|\\
  & = & \mathcal{W} [D_2 .D_1 .f_3, \ldots, D_2 .D_1 .f_k]
\end{eqnarray*}

Ainsi, par r{\'e}currence, on a pour tout $j \in \llbracket 0, k - 1
\rrbracket$ :
\begin{eqnarray*}
  \frac{1}{\varphi^{k - j + 1}_j} \times \ldots \times \frac{1}{\varphi^{k -
  1}_2} \times \frac{1}{\varphi^k_1} \mathcal{W} [f_1, \ldots, f_k] & = &
  \mathcal{W} [D_j \ldots D_1 .f_{j + 1}, \ldots, D_j \ldots D_1 .f_k]
\end{eqnarray*}


En particulier, pour $j = k - 1$, on a :
\[ \frac{1}{\varphi^2_{k - 1}} \times \ldots \times \frac{1}{\varphi^{k -
   1}_2} \times \frac{1}{\varphi^k_1} \mathcal{W} [f_1, \ldots, f_k] = D_{k -
   1} \ldots D_1 .f_k \]

Ainsi, $D_{k - 1} \ldots D_1 .f_k $ vaut :
\begin{eqnarray*}
  & = & \mathcal{W} [f_1, \ldots, f_k] \underset{j = 1}{\overset{k -
  1}{\prod}} \frac{1}{\varphi^{k - j + 1}_j} \\
  & = & \mathcal{W} [f_1, \ldots, f_k] \times \frac{1}{\mathcal{W} [f_1]^k}
  \times \left( \frac{\mathcal{W}[f_1]^2}{\mathcal{W}[f_1, f_2]} \right)^{k -
  1} \times  \underset{j = 3}{ \overset{k - 1}{\prod}} \left(
  \frac{\mathcal{W}[f_1, \ldots, f_{j - 1}]^2}{\mathcal{W}[f_1, \ldots, f_{j -
  2}]\mathcal{W}[f_1, \ldots, f_j]} \right)^{k - j + 1}\\
  & = & \mathcal{W} [f_1, \ldots, f_k] \times \frac{\mathcal{W} [f_1]^{k -
  2}}{\mathcal{W} [f_1, f_2]^{k - 1}} \times \frac{\underset{j = 2}{
  \overset{k - 2}{\prod}} \mathcal{W} [f_1, \ldots, f_j]^{2 (k -
  j)}}{\underset{j = 1}{ \overset{k - 3}{\prod}} \mathcal{W} [f_1, \ldots,
  f_j]^{k - j - 1} \times \underset{j = 3}{ \overset{k - 1}{\prod}}
  \mathcal{W} [f_1, \ldots, f_j]^{k - j + 1}}\\
  & = & \frac{\mathcal{W} [f_1, \ldots, f_k] \mathcal{W} [f_1, \ldots, f_{k -
  2}]}{\mathcal{W} [f_1, \ldots, f_{k - 1}]^2}\\
  & = & \varphi_k
\end{eqnarray*}

D'o{\`u} le r{\'e}sultat par r{\'e}currence.

\textbf{lemme 2.}

Soit $f \in \mathcal{C}^1 (I)$ et $k \in \llbracket 1, r - 1 \rrbracket$, s'il
existe $a < b \in I$ tels que $f (a) = f (b) = 0$, alors il existe $c \in] a,
b [$ tel que $D_k .f (c) = 0$

\textbf{Preuve du lemme 2.}

Trivial, en appliquant le th{\'e}or{\`e}me de Rolle {\`a}
$\frac{f}{\varphi_k}$.

\

\textbf{Lemme 3.}

Soit $f \in \mathcal{C}^{r - 1} (I)$. On suppose que $f$ admet au moins $r$
z{\'e}ros sur $I$, alors pour tout $k \in \llbracket 1, r - 1 \rrbracket$, on
a $D_k .D_{k - 1} \ldots D_1 .f$ admet au moins $r - k$ racines sur $I$.

\textbf{Preuve du lemme 3.}

Soit $f \in \mathcal{C}^{r - 1} (I)$, on suppose que $f$ admet au moins $r$
z{\'e}ros $\zeta_1 < \ldots < \zeta_r $ sur $I$.

On va montrer le r{\'e}sultat par r{\'e}currence finie sur $k \in \llbracket
1, r - 1 \rrbracket$.

Pour $k = 1$, on a pour tout $j \in \llbracket 1, r - 1 \rrbracket$, on a $f
(\zeta_j) = f (\zeta_{j + 1}) = 0$. Donc, via le lemme 2, il existe $\tau_j
\in] \zeta_j, \zeta_{j + 1} [$ tel que $D_1 .f (\tau_j) = 0$

Ainsi $D_1 .f$ admet au moins $r - 1$ z{\'e}ros sur $I$.

Soit $k \in \llbracket 1, r - 2 \rrbracket$, supposons que $D_k .D_{k - 1}
\ldots D_1 .f$ admet au moins $r - k$ z{\'e}ros sur $I$ et montrons que $D_{k
+ 1} .D_k \ldots D_1 .f$ admet au moins $r - (k + 1)$ z{\'e}ros sur $I$.

Par le m{\^e}me raisonnement que pr{\'e}c{\'e}demment, entre deux z{\'e}ros de
\ $D_k .D_{k - 1} \ldots D_1 .f$, il existe un z{\'e}ro de $D_{k + 1} .D_k
\ldots D_1 .f$.

D'o{\`u} le r{\'e}sultat.

Revenons {\`a} notre question. Montrons que \ $a_1 f_1 + \cdots + a_r f_r$
admet au plus~$r - 1$ z{\'e}ros sur $I$.

Par l'absurde, supposons que $a_1 f_1 + \cdots + a_r f_r$ admet au moins $r$
z{\'e}ros sur $I$.

Notons $I = \left\{ k \in \llbracket 1, r \rrbracket, a_i \not{=} 0 \right\}$
et $i_0 = \max (I)$. On a d'apr{\`e}s le lemme 1,

\begin{eqnarray*}
  D_{i_0 - 1} \cdots D_1 \left( \sum_{k = 1}^{r} a_k f_k \right) 
  &=& D_{i_0 - 1} \cdots D_1 \left( \sum_{k \in I} a_k f_k \right) \\
  &=& \sum_{k \in I \setminus \{ i_0 \}} a_k \, D_{i_0 - 1} \cdots D_{k+1} D_{k - 1} \cdots D_1 f_k + a_{i_0} \, D_{i_0 - 1} \cdots D_1 f_{i_0} \\
  &=& a_{i_0} \, \varphi_{i_0}
\end{eqnarray*}



Or, via le lemme 3, $D_{i_0 - 1} \ldots D_1 . \left( \overset{r}{\underset{k =
1}{\sum}} a_k f_k \right)$ admet au moins $r - i_0 + 1 \geqslant 1$ z{\'e}ros
sur $I$.

Ainsi $a_{i_0} . \varphi_{i_0}$ s'annule sur $I$, absurde avec $\varphi_{i_0}
> 0$.

D'o{\`u} $a_1 f_1 + \cdots + a_r f_r$ admet au plus~$r - 1$ z{\'e}ros sur $I$.

\textbf{Commentaire.} Le r{\'e}sultat de la deuxi{\`e}me question semble
int{\'e}ressant pour caract{\'e}riser certains types de fonctions.

1- Notons pour tout $i \in \llbracket 1, r \rrbracket$ $f_i : x \longmapsto
x^{i - 1}$, on a alors pour tout $k \in \llbracket 1, r \rrbracket$ et pour
tout $x \in \mathbb{R}$
\begin{eqnarray*}
  \mathcal{W} [f_1, \ldots, f_k] (x) & = & \left| \begin{array}{ccccc}
    1 & x & \cdots &  & x^{k - 1}\\
    0 & 2 & \cdots &  & (k - 1) x^{k - 2}\\
    \vdots &  &  &  & \vdots\\
    0 &  &  & (k - 2) ! & (k - 2) !x\\
    0 &  & \cdots & 0 & (k - 1) !
  \end{array} \right|\\
  & = & \underset{i = 1}{\overset{k - 1}{\prod}} i!\\
  & > & 0
\end{eqnarray*}

Donc pour tous $a_1, \ldots, a_r$ non tous nuls, on a $x \longmapsto
\underset{k = 1}{\overset{r}{\sum}} a_k x^{k - 1}$ admet au plus~$r - 1$
z{\'e}ros sur $\mathbb{R}$.

Ainsi, pour tout polyn{\^o}me $P \in \mathbb{R} [X]$ non nul, $P$ admet au
plus $\deg (P)$ racines sur $\mathbb{R}$.

2- Soient $t_1 < \cdots < t_r \in \mathbb{R}$

Pour tout $i \in \llbracket 1, r \rrbracket$, et pour tout $f_k : x
\longmapsto e^{t_k x}$, on a alors pour tout $k \in \llbracket 1, r
\rrbracket$ et pour tout $x \in \mathbb{R}$
\begin{eqnarray*}
  \mathcal{W} [f_1, \ldots, f_k] (x) & = & e^{(t_1 + \cdots + t_k) x} \left|
  \begin{array}{ccccc}
    1 & 1 & \cdots &  & 1\\
    t_1 & t_2 & \cdots &  & t_k\\
    \vdots &  &  &  & \vdots\\
    t^{k - 1}_1 & t^{k - 1}_k & \cdots &  & t^{k - 1}_k
  \end{array} \right|\\
  & = & e^{(t_1 + \cdots + t_k) x} \underset{1 \leqslant i < j \leqslant
  k}{\prod} (t_j - t_i)\\
  & > & 0
\end{eqnarray*}

Ainsi pour tous $a_1, \ldots, a_r$ non tous nuls, on a $x \longmapsto
\underset{k = 1}{\overset{r}{\sum}} a_k e^{t_k x}$ admet au plus~$r - 1$
z{\'e}ros sur $\mathbb{R}$.
\[ \maltese \maltese \maltese \maltese \maltese \maltese \maltese \]
