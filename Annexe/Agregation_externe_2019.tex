\begin{center}
\subsection*{Agr{\'e}gation externe 2019}\label{agreg}
\textbf{Corrig{\'e} par: SABIR ILYASS.}
\end{center}
\[ \star \star \star \]
\addcontentsline{toc}{subsection}{Agr{\'e}gation externe 2019}


Le probl{\`e}me pr{\'e}sent{\'e} ici est issu du sujet d'alg{\`e}bre du
concours de l'Agr{\'e}gation externe 2019.

\

Ce probl{\`e}me est long : il est compos{\'e} de cinq grandes parties,
aboutissant finalement {\`a} une d{\'e}monstration du th{\'e}or{\`e}me de
Fermat-Wiles pour certains nombres premiers dits {\tmem{r{\'e}guliers}}.

\

\

Le probl{\`e}me est un excellent sujet formateur pour les futurs candidats de
l'agr{\'e}gation en maths, et peut-{\^e}tre aussi int{\'e}ressant pour les
{\'e}tudiants de CPGE scientifiques.

\

C'est un excellent sujet de formation pour les futurs candidats {\`a}
l'agr{\'e}gation de math{\'e}matiques et peut-{\^e}tre {\'e}galement
int{\'e}ressant pour les {\'e}tudiants en classes pr{\'e}paratoires
scientifiques.

\

La premi{\`e}re partie consiste {\`a} d{\'e}montrer quelques r{\'e}sultats
sur la division euclidienne des polyn{\^o}mes {\`a} coefficients entiers et
{\`a} montrer que l'anneau $\mathbb{Z}[\zeta]$ est euclidien pour $\zeta =
\exp (2 i \pi / 3)$. Ensuite, on introduit la notion de polyn{\^o}me
cyclotomique, un r{\'e}sultat important concernant ces polyn{\^o}mes {\'e}tant
qu'ils ont des coefficients entiers et sont irr{\'e}ductibles sur
$\mathbb{Q}[X] .$ Cela nous fournit un excellent exemple pour montrer que,
pour tout entier $n \in \mathbb{N}$, il existe un polyn{\^o}me de
\ensuremath{\mathbb{Q}}$[X]$ irr{\'e}ductible dans $\mathbb{Q}[X]$. Ce
r{\'e}sultat n'est pas valable pour \ensuremath{\mathbb{R}}$[X]$ ni pour
$\mathbb{C}[X]$. En particulier, \ensuremath{\mathbb{Q}} n'est pas
alg{\'e}briquement clos.

\

D'autre part, les questions 4.a, 4.b, et 4.c servent {\`a} introduire les
matrices compagnons, un outil souvent utilis{\'e} pour simplifier les
d{\'e}monstrations (comme celle du th{\'e}or{\`e}me de Cayley-Hamilton, par
exemple !). Ici, la pr{\'e}sence des matrices compagnons permet de
d{\'e}montrer le r{\'e}sultat {\'e}nonc{\'e} dans la question 4.e, qui sera
utilis{\'e} {\`a} plusieurs reprises dans la suite du probl{\`e}me.

\

La deuxi{\`e}me partie porte sur les nombres alg{\'e}briques. Un tr{\`e}s bon
r{\'e}sultat est pr{\'e}sent{\'e} dans la question 1.b concernant la finitude
des racines de l'unit{\'e} incluses dans une extension finie de $\mathbb{Q}$,
ce qui implique que le corps des nombres alg{\'e}briques est de degr{\'e}
infini sur $\mathbb{Q}$. Vers la fin de cette partie, on montre que l'ensemble
des nombres entiers alg{\'e}briques est un sous-anneau de $\mathbb{C}$.

\

Passons ensuite {\`a} la troisi{\`e}me partie, qui consiste {\`a} {\'e}tudier
et caract{\'e}riser quelques r{\'e}sultats sur$\mathbb{Z}[\zeta]$ (qui seront
utilis{\'e}s dans les parties 4 et 5), o{\`u} $\zeta = \exp ((2 i \pi) / p)$,
notamment en ce qui concerne les {\'e}l{\'e}ments inversibles de l'anneau
$\mathbb{Z}[\zeta$].

\

La partie 4 a pour but de d{\'e}montrer le th{\'e}or{\`e}me de Fermat pour $n
= 3$.

\

Enfin, la partie 5 traite du th{\'e}or{\`e}me de Fermat pour certains nombres
premiers et dans des cas particuliers.

\

En conclusion, le th{\'e}or{\`e}me de Fermat (en anglais, {\tmem{Fermat's
Last Theorem}}) a {\'e}t{\'e} {\'e}nonc{\'e} (conjectur{\'e}) par le
math{\'e}maticien fran{\c c}ais Pierre de Fermat en 1637, et il n'a
{\'e}t{\'e} d{\'e}montr{\'e} qu'en 1994. Cette d{\'e}monstration a {\'e}t{\'e}
pr{\'e}sent{\'e}e pour la premi{\`e}re fois par le math{\'e}maticien
britannique Andrew Wiles. Elle repose sur plusieurs th{\'e}ories, comme la
th{\'e}orie de Galois, ainsi que sur des r{\'e}sultats de math{\'e}matiques
modernes qui n'existaient pas au XVIIe si{\`e}cle. Pour ceux qui souhaitent
d{\'e}couvrir la d{\'e}monstration de ce th{\'e}or{\`e}me, n'h{\'e}sitez pas
{\`a} cliquer sur le lien {\`a} la fin de ce document (la beaut{\'e} du livre
vous donnera envie de le lire en entier).

\

Bon courage pour la suite{\ldots}

\

\subsubsection*{D{\'e}finitions et rappels.}

\

--- Soit $A$ un anneau commutatif unitaire int{\`e}gre dont on note $1_A$
l'{\'e}l{\'e}ment unit{\'e}.

\

--- On rappelle qu'un {\'e}l{\'e}ment $u \in A$ est inversible s'il existe
$u' \in A$tel que $u u' = 1_A .$ On note $A^{\times}$ l'ensemble des
inversibles de $A$, qui \tmtextbf{est un groupe multiplicatif}.

\

--- Un {\'e}l{\'e}ment $x$ de $A$ est dit \tmtextbf{irr{\'e}ductible} si $x$
n'est pas inversible et si pour tous $\alpha, \beta \in A$, $x = \alpha \beta$
implique $\alpha \in A^{\times}$ ou $\beta \in A^{\times}$.

\

--- Deux {\'e}l{\'e}ments $x, y \in A$ sont dits associ{\'e}s s'il existe $u
\in A^{\times}$ tel que $x = u y$. On note alors $x \sim y$.

\

--- Soit $I$ un id{\'e}al de $A$; on dit que deux {\'e}l{\'e}ments $\alpha,
\beta \in A$ sont congrus modulo $I$si $\alpha \nonconverted{minus} \beta \in
I$. On {\'e}crit alors $\alpha = \beta (\tmop{mod} I) .$

\

--- Pour$x \in A$, on note$ \langle x \rangle = x A$ l'id{\'e}al engendr{\'e}
par $x$. Un tel id{\'e}al est dit \tmtextbf{principal}.

\

--- Soient $I, J$ deux id{\'e}aux de $A$. On dit que $I$ divise $J$ si $J
\subseteq I$. Par ailleurs, on note$I J$ l'id{\'e}al produit de $I$ et $J$,
qui est l'ensemble des sommes finies $\underset{i}{\sum} x_i y_i$ avec $x_i
\in I$ et $y_i \in J$.

\

--- On rappelle qu'un nombre complexe $\alpha$ est dit alg{\'e}brique (sur
$\mathbb{Q}$) s'\tmtextbf{il existe un polyn{\^o}me non nul $P$ de
$\mathbb{Q}[X]$ tel que $P (\alpha) = 0$}.

Il existe alors un polyn{\^o}me unitaire de plus petit degr{\'e} annulant
$\alpha$, que l'on appelle \tmtextbf{polyn{\^o}me minimal de $\alpha$} et que
l'on note $\pi_{\alpha}$. Les racines complexes de ce polyn{\^o}me sont
appel{\'e}es les conjugu{\'e}s de $\alpha$.

\

--- On appelle \tmtextbf{entier alg{\'e}brique} tout nombre complexe qui est
racine d'un polyn{\^o}me unitaire {\`a} coefficients dans $\mathbb{Z}$.

\

--- On rappelle une version du \tmtextbf{lemme de Gauss}, que l'on pourra
utiliser librement : soit $P \in \mathbb{Z}[X]$ tel que $P = P_1 P_2$, avec
$P_1$ et $P_2$ des polyn{\^o}mes de $\mathbb{Q} [\nobracket X] .$ Alors, il
existe un rationnel $r \in \mathbb{Q}$, non nul, tel que $r P_1 \in
\mathbb{Z}[X]$et $\frac{1}{r} P_2 \in \mathbb{Z}[X]$.

\

--- On dit qu'un groupe ab{\'e}lien $G$ est de \tmtextbf{type fini} s'il
existe une famille g{\'e}n{\'e}ratrice finie de $G$, c'est-{\`a}-dire un
entier $r$ et une famille $(a_1, . . ., a_r)$ d'{\'e}l{\'e}ments de $G$ tels
que tout {\'e}l{\'e}ment de $G$ s'{\'e}crit comme une combinaison lin{\'e}aire
{\`a} coefficients entiers des $a_1, . . ., a_r$.

\

\subsubsection*{Notations.}

\

--- Pour un anneau $A$ commutatif et un entier naturel non nul $n$, on note
$\mathcal{M}_n (A)$ l'alg{\`e}bre des matrices carr{\'e}es $n \times n$ {\`a}
coefficients dans $A$ ; la matrice unit{\'e} est not{\'e}e $I_n$.

Si $M$ est une matrice de $\mathcal{M}_n (A)$, on note $\chi_M$ son
\tmtextbf{polyn{\^o}me caract{\'e}ristique}, qui est le polyn{\^o}me unitaire
d{\'e}fini par
\[ \chi_M = \det (X I_n \nonconverted{minus} M) \]
et on note $\pi_M$ son \tmtextbf{polyn{\^o}me minimal}.

\

--- Pour un nombre premier $p$, on note $\mathbb{F}_p$ le corps $\mathbb{Z}/
p\mathbb{Z}.$

\

--- Pour tout entier alg{\'e}brique $\alpha$, on note $\mathbb{Z}[\alpha]$
l'anneau des {\'e}l{\'e}ments de la forme $P (\alpha)$ o{\`u} $P$ parcourt
$\mathbb{Z}[X]$.

\

Dans le probl{\`e}me, les textes plac{\'e}s entre les symboles $\maltese
\maltese${\textdots}$\maltese \maltese$ pr{\'e}cisent des notations et
d{\'e}finitions qui sont utilis{\'e}es dans la suite de l'{\'e}nonc{\'e}.

\

\subsubsection*{I. Exercices pr{\'e}liminaires}

\

1. Soit $B \in \mathbb{Z}[X]$ un polyn{\^o}me unitaire et $A \in
\mathbb{Z}[X]$. Montrer qu'il existe $Q, R \in \mathbb{Z}[X]$ tels que $A =
\tmop{BQ} + R$ avec $\deg R < \deg B$ ou $R = 0$.

\

\tmtextbf{{\tmname{Indication}} :} On pourra faire une preuve par
r{\'e}currence sur le degr{\'e} de $A$.

\

\paragraph{\textbf{2. L'anneau $\mathbb{Z}[j]$.}} On note $j = e^{\frac{2i\pi}{3}}$.


\

\quad (a) D{\'e}montrer que $j$ est un {\'e}l{\'e}ment alg{\'e}brique sur
$\mathbb{Q}$ et pr{\'e}ciser son polyn{\^o}me minimal.

\

\quad (b) D{\'e}montrer que $\mathbb{Z}[j] =\{a + b j, (a, b) \in
\mathbb{Z}^2 \}$.

\

\quad Pour tout nombre complexe $z$, on pose $N (z) = z \bar{z} = |z|^2$ .

\

\quad (c) D{\'e}montrer que pour tout $z \in \mathbb{Z}[j]$, on a $N (z) \in
\mathbb{N}$. En d{\'e}duire que si $z \in \mathbb{Z}[j]$ est inversible, alors
$N (z) = 1$, puis que $\mathbb{Z}[j]^{\times}$ poss{\`e}de $6$
{\'e}l{\'e}ments que l'on pr{\'e}cisera.

\

\quad (d) Soient $x \in \mathbb{Z}[j]$ et $y \in
\mathbb{Z}[j]\backslash\{0\}$. D{\'e}terminer un {\'e}l{\'e}ment$q \in
\mathbb{Z}[j]$ tel que $N \left( \frac{x}{y} - q \right) < 1$.

\quad En d{\'e}duire que l'anneau $\mathbb{Z}[j]$ est euclidien.

\

\paragraph{\tmtextbf{3. Polyn{\^o}mes cyclotomiques. }}

Soit $n$ un entier naturel non nul. On note $\Phi_n$ le $n - i{\`e} \tmop{me}$
polyn{\^o}me cyclotomique. On rappelle que si $\mu_n^{\ast}$ d{\'e}signe
l'ensemble des racines primitives $n - i{\`e} \tmop{mes}$ de l'unit{\'e} dans
$\mathbb{C}$, ce polyn{\^o}me est d{\'e}fini par
\[ \Phi_n (X) = \underset{\mu \in \mu^{\ast}_n}{\prod} (X - \mu) \]


\

\quad (a) D{\'e}montrer que
\[ X^n - 1 = \underset{d \overline{} | n \nobracket}{\prod} \Phi_d (X) \]


\quad (b) En d{\'e}duire que $\Phi_n (X) \in \mathbb{Z}[X]$.

\

\quad (c) Soit $p$ un nombre premier.

On note $\pi : \mathbb{Z} \rightarrow \mathbb{F}_p $ la surjection canonique.
Le morphisme d'anneaux $\pi$ s'{\'e}tend, coefficient par coefficient, en un
morphisme d'anneaux de $\mathbb{Z}[X]$ sur $\mathbb{F}_p [X]$, not{\'e}
\^{$\pi$} (on ne demande pas de justifier ce point). Si $\Phi_p$ d{\'e}signe
le $p - i{\`e} \tmop{me}$ polyn{\^o}me cyclotomique, on rappelle que
\[ \Phi_p = \underset{k = 0}{\overset{p - 1}{\sum}} X^k  \tmmathbf{}
   \tmmathbf{} \]


\quad i. D{\'e}montrer que $\hat{\pi} (X^p \nonconverted{minus} 1) = (X
\nonconverted{minus} 1_{\mathbb{F}_p})^p$ .

\

\quad ii. Soient$P$ et $Q$ deux polyn{\^o}mes unitaires et non constants dans
$\mathbb{Z}[X]$ tels que $X^p - 1 = P Q$. D{\'e}montrer que $P (1)$ et $Q (1)$
sont des entiers multiples de $p$.

\

\quad iii. Retrouver ainsi que $\Phi_p$ est un polyn{\^o}me irr{\'e}ductible
de $\mathbb{Q}[X]$.

\

$\maltese \maltese$ De mani{\`e}re g{\'e}n{\'e}rale, $\Phi_n$ est
irr{\'e}ductible pour tout $n \in \mathbb{N}\backslash\{0\}$, r{\'e}sultat que
l'on admet ici et que l'on pourra utiliser librement dans la suite. $\maltese
\maltese$

\

\quad iv. Soit ${\zeta = e^{\frac{2 i \pi}{p}}} $. D{\'e}terminer le
polyn{\^o}me minimal de $\zeta$ sur $\mathbb{Q}$ et en d{\'e}duire le
degr{\'e} de l'extension de corps $\mathbb{Q} (\nobracket \zeta) /\mathbb{Q}$.

\paragraph{\tmtextbf{4. Matrices compagnons. }}

Soit $n$ un entier naturel non nul. Soit $P = X^n + a_{n - 1} X^{n - 1} +
\cdots + a_0$ un polyn{\^o}me unitaire de $\mathbb{C}[X]$. On lui associe sa
matrice compagnon $C_P$ d{\'e}finie dans$\mathcal{M}_n (\mathbb{C})$ par
\[ C_P = \left( \begin{array}{c}
     0\\
     1\\
     0\\
     .\\
     .\\
     .\\
     0
   \end{array} \begin{array}{c}
     \\
     
   \end{array} \quad \begin{array}{c}
     0\\
     0\\
     1\\
     0\\
     .\\
     .\\
     0
   \end{array} \quad  \begin{array}{c}
     .\\
     \\
     \\
     \\
     \\
     .\\
     0
   \end{array} \quad \begin{array}{c}
     .\\
     \\
     \\
     \\
     \\
     .\\
     0
   \end{array} \quad \begin{array}{l}
     0\\
     0\\
     .\\
     .\\
     .\\
     0\\
     1
   \end{array} \quad  \begin{array}{l}
     - a_0\\
     - a_1\\
     .\\
     .\\
     .\\
     - a_{n - 2}\\
     - a_{n - 1}
   \end{array} \right) \]




On note $\mathcal{E}= (e_1, . . ., e_n)$ la base canonique de $\mathbb{C}^n$.

\

\quad (a) Pour $k \in \{1, . . ., n \nonconverted{minus} 1\}$, exprimer
$C_P^k e_1$ dans la base $\mathcal{E}$. En d{\'e}duire que pour tout
polyn{\^o}me $Q \in \mathbb{C}[X]$ non nul et de degr{\'e} inf{\'e}rieur ou
{\'e}gal {\`a} $n \nonconverted{minus} 1$, la matrice $Q (C_P)$ est non nulle.

\quad En d{\'e}duire le degr{\'e} du polyn{\^o}me minimal de $C_P$.

\

\quad (b) Exprimer $C^n_p e_1$ dans la base $\mathcal{E}$. En d{\'e}duire que
$P$ est le polyn{\^o}me minimal de $C_P$ .

\

\quad (c) En d{\'e}duire le polyn{\^o}me $\chi_{C_P}$ .

\

Soit $M \in \mathcal{M}_n (\mathbb{C})$ de polyn{\^o}me caract{\'e}ristique
$\chi_M$ . Soient $\alpha_1, . . ., \alpha_n$ les racines complexes de
$\chi_M$ compt{\'e}es avec leur multiplicit{\'e}. Soit $Q$ un polyn{\^o}me de
$\mathbb{C}[X]$.

\

\quad (d) D{\'e}montrer que le polyn{\^o}me caract{\'e}ristique de la matrice
$Q (M)$ est
\[ \chi_{Q (M)} = \underset{k = 1}{\overset{n}{\prod}} (X - Q (\alpha_k)) \]


\

\tmtextbf{{\tmname{\tmtextbf{Indication} :}}} On pourra commencer par traiter
le cas o{\`u} $M$ est triangulaire.

\

\quad (e) Soit $A$ un sous-anneau de $\mathbb{C}$. On suppose que le
polyn{\^o}me $Q$ est dans $A [X]$. Soit $P \in A [X]$ un polyn{\^o}me unitaire
dont on note $\alpha_1, . . ., \alpha_n$ les racines complexes compt{\'e}es
avec leur multiplicit{\'e}.

D{\'e}montrer que $\underset{k = 1}{\overset{n}{\prod}} (X - Q (\alpha_k))$
est un polyn{\^o}me de $A [X]$.

\

\subsubsection*{II. Nombres alg{\'e}briques.}

\

1. (a) On d{\'e}signe par$\varphi$ l'indicatrice d'Euler, qui {\`a} tout
entier $n \in \mathbb{N}\backslash\{0\}$ associe le nombre d'entiers non nuls
inf{\'e}rieurs {\`a} $n$ et premiers avec$n$. Justifier que pour tout entier$d
\geqslant 1$, l'ensemble des entiers $n$ tels que $\varphi (n) \leqslant d$
est fini.

\

\quad (b) En d{\'e}duire que si $\tmmathbf{K} /\mathbb{Q}$ est une extension
finie de $\mathbb{Q}$, o{\`u} $\tmmathbf{K}$ est un sous-corps de
$\mathbb{C}$, alors $\tmmathbf{K}$ contient un nombre fini de racines de
l'unit{\'e}.

\

2. Soit $\alpha \in \mathbb{C}$ un nombre alg{\'e}brique dont on rappelle que
l'on a not{\'e} $\pi_{\alpha}$ son polyn{\^o}me minimal. On note $\tmmathbf{K}
=\mathbb{Q}(\alpha)$ le plus petit corps contenant $\alpha$ et $\mathbb{Q}$,
et $d = [\tmmathbf{K} : \mathbb{Q}]$, le degr{\'e} de l'extension de corps
$\mathbb{Q}(\alpha) /\mathbb{Q}$.

\

\quad (a) Montrer que $\pi_{\alpha}$ est un polyn{\^o}me irr{\'e}ductible de
$\mathbb{Q}[X]$ et que son degr{\'e} est {\'e}gal {\`a} $d$.

\

\quad (b) Montrer que si $\sigma$ est un morphisme de $\mathbb{Q}- \tmop{alg}
{\`e} \tmop{bre}$ de $\tmmathbf{K}$ dans $\mathbb{C}$, $\sigma (\alpha)$ est
une racine de $\pi_{\alpha}$, c'est-{\`a}-dire un conjugu{\'e} de $\alpha$. En
d{\'e}duire qu'il y a exactement $d$ tels morphismes de
$\mathbb{Q}$-$\tmop{alg} {\`e} \tmop{bre}$, que l'on notera $\sigma_k :
\tmmathbf{K} \rightarrow \mathbb{C}, k \in \{1, . . ., d\}$.

\

3. Soit $\alpha \in \mathbb{C}$ un nombre alg{\'e}brique et soit $\theta \in
\tmmathbf{K} =\mathbb{Q}(\alpha)$. Comme dans la question pr{\'e}c{\'e}dente,
les $\sigma_k$ avec $k \in \{1, . . ., d\}$ d{\'e}signent les morphismes de
$\mathbb{Q}- \tmop{alg} {\`e} \tmop{bre}$ de $\mathbb{Q}(\alpha)$.

\

\quad (a) Justifier que $\theta$ est un nombre alg{\'e}brique.

\

On pose
\[ P_{\theta} = \underset{k = 1}{\overset{d}{\prod}} (X \nonconverted{minus}
   \sigma_k (\theta)) \in \mathbb{C}[X] . \]


\

\quad (b) Montrer que $P_{\theta} \in \mathbb{Q}[X]$.

\

\quad (c) Justifier que $\pi_{\theta}$ divise $P_{\theta}$, puis montrer que
$P_{\theta}$ est une puissance de $\pi_{\theta}$.



4. Montrer qu'un nombre alg{\'e}brique $\alpha$ est un entier alg{\'e}brique
si et seulement si son polyn{\^o}me minimal est {\`a} coefficients entiers.

\

5. Soit $\alpha$ un nombre complexe.

\

\quad (a) Montrer que si $\alpha$ est un entier alg{\'e}brique, alors le
groupe additif $G$ engendr{\'e} par la partie $\{\alpha^n  | \nobracket n \in
N\}$ est de type fini.



\quad (b) R{\'e}ciproquement, montrer que si $G$ est de type fini alors
$\alpha$ est un entier alg{\'e}brique.

\tmtextbf{{\tmname{Indication :}}} En notant $(g_1, . . ., g_n)$ une famille
g{\'e}n{\'e}ratrice finie de $G$, on pourra consid{\'e}rer le d{\'e}terminant
du syst{\`e}me obtenu en {\'e}crivant les {\'e}l{\'e}ments $\alpha g_i$, $i
\in \{1, . . ., n\}$ comme combinaison lin{\'e}aire des $g_j$ .

\

6. En d{\'e}duire que l'ensemble $\mathfrak{D}_{\mathbb{C}}$ des entiers
alg{\'e}briques de $\mathbb{C}$ est un sous-anneau de $\mathbb{C}$.

\tmtextbf{{\tmname{Indication :}} }On pourra utiliser sans d{\'e}monstration
qu'un sous-groupe d'un groupe ab{\'e}lien de type fini est de type fini.

\

7. Montrer que $\mathfrak{D}_{\mathbb{C}} \cap \mathbb{Q}=\mathbb{Z}$.

\

$\maltese \maltese$\quad Dans la suite, on consid{\`e}re le corps
$\tmmathbf{K} =\mathbb{Q}(\zeta)$, o{\`u} $\zeta = e^{\frac{2 i \pi}{p}}$ avec
$p$ premier impair. On note $\mathfrak{D}_{\tmmathbf{K}}$ l'ensemble des
entiers alg{\'e}briques de $\tmmathbf{K}$. On pose $\lambda = 1
\nonconverted{minus} \zeta$.

\

On d{\'e}finit la norme et la trace de tout {\'e}l{\'e}ment $\theta \in
\tmmathbf{K} =\mathbb{Q}(\zeta)$ par
\[ N (\theta) = \underset{k = 1}{\overset{p - 1}{\prod}} \sigma_k (\theta) \]


et
\[ \tmop{Tr} (\theta) = \underset{k = 1}{\overset{p - 1}{\sum}} \sigma_k
   (\theta) \]


o{\`u} les $\sigma_k$ sont les morphismes de $\mathbb{Q}- \tmop{alg} {\`e}
\tmop{bre}$de $\mathbb{Q}(\zeta)$ d{\'e}finis dans la question 2 de cette
partie.$\maltese \maltese$

\

\subsubsection*{III. Le corps $\mathbb{Q} (\zeta)$ et son anneau d'entiers}

\

1. (a) Montrer que les morphismes de $\mathbb{Q}- \tmop{alg} {\`e}
\tmop{bre}$ de $\mathbb{Q}(\zeta)$ sont les $\sigma_k$ tels que $\sigma_k
(\zeta) = \zeta^k$ , avec $k \in \{1, ..., p \nonconverted{minus} 1\}$.

\

\quad (b) \ i. Montrer que $N (\zeta) = 1$et $\tmop{Tr} (\zeta) =
\nonconverted{minus} 1$.

\

{\hspace{3em}}ii. Montrer que $N (1 \nonconverted{minus} \zeta) = p$ et $N (1
+ \zeta) = 1$.

\

2. Montrer l'inclusion $\mathbb{Z}[\zeta] \subseteq
\mathfrak{D}_{\tmmathbf{K}}$.

\

3. Soit $z \in \mathbb{Z}[\zeta]$.

\

\quad (a) Montrer que $z \in Z [\zeta]^{\times}$ si et seulement si $N (z)
\in \{\nonconverted{minus} 1, + 1\}$.

\

\quad (b) Montrer que si $N (z)$ est un nombre premier, alors $z$ est
irr{\'e}ductible.

\

4. Le but de cette question est de montrer que l'ensemble $G$ des racines de
l'unit{\'e} contenues dans$ \tmmathbf{K}$ est form{\'e} exactement des
{\'e}l{\'e}ments de la forme $\pm \zeta^k$, $k \in \{0, . . ., p
\nonconverted{minus} 1\}$.

\

\quad (a) Justifier que $G$ est un groupe fini cyclique, dont on notera $n$
le cardinal.

\

\quad (b) Soit $\omega$ un g{\'e}n{\'e}rateur de $G$. Justifier que $2 p |
n$et que $\mathbb{Q}(\zeta) =\mathbb{Q}(\omega)$.

\quad

\quad (c) En d{\'e}duire que $n = 2 p$ et conclure.

\

5. On note $\langle \lambda \rangle = \lambda \mathbb{Z}[\zeta]$, l'id{\'e}al
de $\mathbb{Z}[\zeta]$ engendr{\'e} par$\lambda$.

\

\quad (a) Montrer que $\langle \lambda \rangle \cap \mathbb{Z}= p\mathbb{Z}$.

\

\quad (b) Montrer que pour tout $k \in \{1, . . ., p \nonconverted{minus}
1\}$, on a
\[ \frac{1 \nonconverted{minus} \zeta}{1 - \zeta^k} \in Z [\zeta]^{\times} \]


et en d{\'e}duire que


\[ \lambda^{p - 1} \mathbb{Z}[\zeta] = p\mathbb{Z}[\zeta] . \]


\quad (c) Soit $\psi$ le morphisme d'anneaux de $\mathbb{Z}[X]$ dans
$\mathbb{Z} [\zeta] / \langle \lambda \rangle$, qui {\`a} $P \in
\mathbb{Z}[X]$ associe $P (\zeta) (\tmop{mod} \langle \lambda \rangle)$.
D{\'e}terminer l'image de $\psi$ et montrer que $\ker \psi$ est l'ensemble des
polyn{\^o}mes $P \in \mathbb{Z}[X]$ tels que $P (1) = 0 (\tmop{mod}
p\mathbb{Z})$.

\

\quad (d) En d{\'e}duire que $Z [\zeta] / \langle \lambda \rangle$ est
isomorphe {\`a} $\mathbb{F}_p$.

\

\quad (e) Que peut-on en d{\'e}duire pour l'id{\'e}al $\lambda$ ?

\

6. On d{\'e}termine ici la structure de $\mathbb{Z}[\zeta]^{\times}$. Le but
est de d{\'e}montrer que les {\'e}l{\'e}ments de $\mathbb{Z}[\zeta]^{\times}$
sont les $\zeta^r \varepsilon$, o{\`u} $r \in \mathbb{Z}$ et $\varepsilon$ est
un r{\'e}el inversible de $\mathbb{Z}[\zeta]$.

\

Soit $u \in \mathbb{Z}[\zeta]^{\times}$ .

\

\quad (a) Soit $P = \underset{k = 0}{\overset{d}{\sum}} a_k X^k \in
\mathbb{Z}[X]$ un polyn{\^o}me unitaire de degr{\'e} $d$, dont on note
$\alpha_1, . . ., \alpha_d$ les racines complexes compt{\'e}es avec leur
multiplicit{\'e}. On suppose que, pour tout$k \in \{1, . . ., d\}$, $\alpha_k$
est de module $1$.

\

{\hspace{3em}}i. Montrer que pour tout $k \in \{0, . . ., d\}$, on a $|a_k |
\leqslant \left( \begin{array}{c}
  d\\
  k
\end{array} \right)$ .

{\hspace{3em}}En d{\'e}duire qu'il n'existe qu'un nombre fini d'entiers
alg{\'e}briques de degr{\'e} $d$ dont tous les conjugu{\'e}s sont de module
$1$.

\

{\hspace{3em}}ii. En d{\'e}duire {\'e}galement que les racines de $P$ sont
des racines de l'unit{\'e}.

\tmtextbf{{\tmname{Indication}} :} On pourra consid{\'e}rer les polyn{\^o}mes
$P_n = \underset{k = 1}{\overset{d}{\prod}} (X - \alpha_k^n)$, $n \in
\mathbb{N}$, dont on montrera qu'ils sont dans $\mathbb{Z}[X]$.

\

\quad (b) Soit $P \in \mathbb{Z}[X]$ tel que $u = P (\zeta)$. Montrer que,
pour tout $k \in \{1, . . ., p \nonconverted{minus} 1\}$, $u_k = P (\zeta^k)$
est un conjugu{\'e} de $u$, et que c'est un {\'e}l{\'e}ment de
$\mathbb{Z}[\zeta]^{\times}$.

\quad

\quad (c) Justifier que $\frac{u_p}{u_{p - 1}}$ est un entier alg{\'e}brique
dont tous les conjugu{\'e}s sont de module 1.

\

\quad (d) En d{\'e}duire qu'il existe$m \in \mathbb{Z}$ tel que
$\frac{u_1}{u_{p - 1}} = \pm \zeta^m$.

\

\quad (e) \ i. Soit $\theta \in \mathbb{Z}[\zeta]$. Justifier qu'il existe un
entier $a \in \mathbb{Z}$ tel que $\theta = a (\tmop{mod} < \lambda >)$. En
d{\'e}duire que deux {\'e}l{\'e}ments conjugu{\'e}s de $\mathbb{Z}[\zeta]$
sont {\'e}gaux modulo $< \lambda >$.

\

{\hspace{3em}}ii. D{\'e}montrer que $\frac{u_1}{u_{p - 1}} = \zeta^m$.

\

\quad (f) Justifier l'existence de $r \in \mathbb{Z}$ tel que $2 r = m
(\tmop{mod} p\mathbb{Z})$. On pose $\varepsilon = \zeta^{- r} u.$ Montrer que
$\varepsilon \in \mathbb{R}$ et conclure.

\

7. Le but de ce qui suit est de montrer que $\mathfrak{D}_{\tmmathbf{K}}
=\mathbb{Z}[\zeta]$.

\

\quad (a) Montrer que pour tout $\theta \in \mathfrak{D}_{\tmmathbf{K}}$, on
a $N (\theta) \in \mathbb{Z}$ et $\tmop{Tr} (\theta) \in \mathbb{Z}$.

\

\quad (b) Soit $\theta \in \tmmathbf{K} =\mathbb{Q}(\zeta)$ un entier
alg{\'e}brique. Il existe des rationnels $a_0, . . ., a_{p - 2}$ tels que
\[ \theta = \underset{k = 0}{\overset{p - 2}{\sum}} a_k \zeta^k \]


\

{\hspace{3em}}i. Pour $k \in \{0, . . ., p \nonconverted{minus} 2\}$,
calculer $b_k = \tmop{Tr} (\theta \zeta^{- k} \nonconverted{minus} \theta
\zeta)$ et justifier que $b_k \in \mathbb{Z}$.

\

{\hspace{3em}}ii. Montrer qu'il existe des entiers $c_0, c_1, . . ., c_{p -
2},$ que l'on exprimera en fonction des $b_k$, tels que
\[ p \theta = \underset{k = 0}{\overset{p - 2}{\sum}} c_k \lambda^k \]


Justifier ensuite que pour tout$k \in \{0, . . ., p \nonconverted{minus}
2\}$,
\[ b_k = \underset{l = k}{\overset{p - 2}{\sum}} (- 1)^l \left(
   \begin{array}{c}
     l\\
     k
   \end{array} \right) c_l \]


\

{\hspace{3em}}iii. Montrer qu'il existe $\beta \in \mathbb{Z}[\zeta]$ tel que
$p = \lambda^{p - 1} \beta$. En d{\'e}duire que$p | c_0$, puis que pour tout
$k \in \{0, . . ., p \nonconverted{minus} 2\}$, on a $p | c_k$. Conclure.

\

\subsubsection*{IV. Le th{\'e}or{\`e}me de Fermat pour $p = 3$}

\

On cherche {\`a} d{\'e}montrer dans cette partie que l'{\'e}quation
\begin{equation}
  x^3 + y^3 + z^3 = 0
\end{equation}


n'a pas de solution enti{\`e}res non triviales, $i. e$., telles que $x y z
\neq 0.$

\

Soient $x, y \tmop{et} z$ trois entiers relatifs tels que $x^3 + y^3 + z^3 =
0$.

\

1. On suppose que $3 \not{| \nobracket} x y z$. Montrer que $x^3$ vaut $+ 1$
ou $\nonconverted{minus} 1 (\tmop{mod} 9)$ et conclure {\`a} une
impossibilit{\'e}.

\

$\maltese \maltese$ On traite {\`a} pr{\'e}sent le cas $3 | x y z$. Dans la
suite de cette partie, on note $\lambda = 1 \nonconverted{minus} j$ avec
toujours $j = e^{\frac{2 i \pi}{3}}$, et on suppose que les entiers $x, y$ et
$z$ sont premiers entre eux dans $\mathbb{Z}[j]$ (et non seulement dans
$\mathbb{Z}$), cas auquel on peut se ramener en divisant par leur pgcd dans
$\mathbb{Z}[j]$. $\maltese \maltese$

\

2. Montrer que $3$ et $\lambda^2$ sont associ{\'e}s dans $\mathbb{Z}[j]$, ce
que l'on a not{\'e} $3 \sim \lambda^2$.

\

3. Soit $s \in \mathbb{Z}[j]$ tel que $s \neq 0 (\tmop{mod} \langle \lambda
\rangle)$. Montrer qu'il existe $\varepsilon \in \{\nonconverted{minus} 1, +
1\}$ tel que $s^3 = \varepsilon (\tmop{mod} \langle \lambda^4 \rangle)$.

{\tmname{\tmtextbf{Indication :}}} On pourra remarquer que tout
{\'e}l{\'e}ment $s \in \mathbb{Z}[j]$ est congru {\`a} $\nonconverted{minus}
1, 0$ou $1 (\tmop{mod} \langle \lambda \rangle)$.

\

$\maltese \maltese$ Par sym{\'e}trie des r{\^o}les de $x, y$ et $z$, on peut
supposer que $3 | z$ (et donc $3 \not{| \nobracket} x$, $3 \not{| \nobracket}
y$ puisqu'ils sont premiers entre eux). En particulier, on a $\lambda | z$,
$\lambda \not{| \nobracket} x$ et$\lambda \not{| \nobracket} y$ dans
$\mathbb{Z}[j]$.

On note $n$ la valuation en $\lambda$ de $z$ ; il existe donc $\mu \in
\mathbb{Z}[j]$ premier avec $\lambda$ tel que $z = \mu \lambda^n$, et par
hypoth{\`e}se $n \geqslant 1.$ On a donc $x^3 + y^3 + \mu^3 \lambda^{3 n} =
0.$

La propri{\'e}t{\'e} suivante (qui pourra {\^e}tre utilis{\'e}e sans plus de
justification) est donc v{\'e}rifi{\'e}e :
\[ (P_n) : \tmop{il} \tmop{existe} \alpha, \beta, \delta \in \mathbb{Z}[j]
   \tmop{et} \omega \in \mathbb{Z}[j]^{\times} \tmop{tels} \tmop{que}
   \left\{\begin{array}{l}
     \lambda \not{| \nobracket} \alpha \beta \delta\\
     \alpha \infixand \beta \tmop{premiers} \tmop{entre} \tmop{eux}\\
     \alpha^3 + \beta^3 + \omega \lambda^{3 n} \delta^3 = 0
   \end{array}\right. \]


Nous allons montrer que si $(P_n)$ est v{\'e}rifi{\'e}e, alors $n \geqslant
2$ et $(P_{n - 1})$ est {\'e}galement v{\'e}rifi{\'e}e. $\maltese \maltese$

\

4. Supposons $(P_n)$ v{\'e}rifi{\'e}e pour un quadruplet $(\alpha, \beta,
\delta, \omega)$. En consid{\'e}rant les valeurs de $\alpha^3, \beta^3$ et
$\omega \lambda^{3 n} \delta^3 (\tmop{mod} \langle \lambda^4 \rangle)$,
montrer que $n \geqslant 2$.

\

5. Supposons $(P_n)$v{\'e}rifi{\'e}e pour un quadruplet $(\alpha, \beta,
\delta, \omega)$. On montre dans cette question que $(P_{n - 1})$ est
{\'e}galement v{\'e}rifi{\'e}e.

\

\quad (a) Montrer que


\[ \nonconverted{minus} \omega \lambda^{3 n} \delta^3 = (\alpha + \beta)
   (\alpha + j \beta) (\alpha + j^2 \beta) . \]


\quad (b) En d{\'e}duire que $\lambda$ divise chacun des facteurs $\alpha +
\beta, \alpha + j \beta$ et $\alpha + j^2 \beta$.

\

\quad (c) D{\'e}montrer que $\lambda$ est un\tmtextbf{ pgcd} de $\alpha +
\beta$et $\alpha + j \beta$. En d{\'e}duire que $\lambda^2$ divise exactement
l'un des {\'e}l{\'e}ments $\alpha + \beta, \alpha + j \beta$ ou$\alpha + j^2
\beta$.

\

Quitte {\`a} remplacer$\beta$ par $j \beta$ ou$j^2 \beta$, on peut supposer
que $\lambda^2$ divise $\alpha + \beta$. Il existe donc des {\'e}l{\'e}ments
$\kappa_1$, $\kappa_2$ et $\kappa_3$ de $\mathbb{Z}[j]$ tels que $\lambda
\not{| \nobracket} \kappa_1 \kappa_2 \kappa_3$ et
\[ \left\{\begin{array}{l}
     \alpha + \beta = \lambda^{3 n - 2} \kappa_1\\
     \alpha + j \beta = \lambda \kappa_2\\
     \alpha + j^2 \beta = \lambda \kappa_3
   \end{array}\right. \]


\quad (d) Montrer que $\nonconverted{minus} \omega \delta^3 = \kappa_1
\kappa_2 \kappa_3$ et en d{\'e}duire qu'il existe des {\'e}l{\'e}ments
$\gamma_1, \gamma_2 \tmop{et} \gamma_3$ de $\mathbb{Z}[j]$ tels que pour tout
$l \in \{1, 2, 3\}$, $\kappa_l \sim \gamma^3_l$.

\

\quad (e) D{\'e}montrer qu'il existe deux inversibles $\tau$ et $\tau'$ de
$\mathbb{Z}[j]^{\times}$ tels que
\[ \gamma^3_2 + \tau \gamma^3_3 + \tau' \lambda^{3 (n - 1)} \gamma^3_1 = 0.
\]
\quad

\quad (f) Montrer que si $\tau = \pm 1$, alors $(P_{n - 1})$ est
v{\'e}rifi{\'e}e.



\

\quad (g) Montrer que $\tau = \pm 1 (\tmop{mod} \langle \lambda^3 \rangle)$,
puis que $\tau \nin \{j, \nonconverted{minus} j, j^2, \nonconverted{minus} j^2
\}$.

\

6. Conclure que l'{\'e}quation $(1)$ n'a pas de solution$(x, y, z)$ dans le
cas $3 | x y z$.

\

\subsubsection*{V. Le th{\'e}or{\`e}me de Fermat pour $p$ r{\'e}gulier et $p
\not{| \nobracket} x y z$}

\

$\maltese \maltese$\quad On admet dans la suite que pour tout corps
$\tmmathbf{K}$ de degr{\'e} fini sur $\mathbb{Q}$, son anneau des entiers
$\mathfrak{D}_{\tmmathbf{K}}$ v{\'e}rifie la propri{\'e}t{\'e} suivante :

Tout id{\'e}al non nul de $\mathfrak{D}_{\tmmathbf{K}}$ s'{\'e}crit comme
produit d'id{\'e}aux premiers, de mani{\`e}re unique {\`a} l'ordre pr{\`e}s
des facteurs.

\

Dans ce contexte, on dit que deux id{\'e}aux $I$ et $J$ sont premiers entre
eux s'ils n'ont pas d'id{\'e}al premier en commun dans leur d{\'e}composition
en produit d'id{\'e}aux premiers.

\

L'anneau $\mathbb{Z}[\zeta]$ qui est, d'apr{\`e}s les r{\'e}sultats de la
Partie 3, l'anneau des entiers de $\tmmathbf{K} =\mathbb{Q}(\zeta)$,
v{\'e}rifie donc cette propri{\'e}t{\'e} de factorisation de ses id{\'e}aux.

\

On suppose dans cette partie que $p > 3$ est un nombre premier r{\'e}gulier,
ce qui signifie que si $I$ est un id{\'e}al de $\mathbb{Z}[\zeta]$ tel que
$I^p$ est principal, alors $I$ est lui-m{\^e}me principal. On rappelle que
l'on a not{\'e}$\lambda = 1 \nonconverted{minus} \zeta$ et que certaines
propri{\'e}t{\'e}s de l'id{\'e}al $\langle \lambda \rangle$ ont {\'e}t{\'e}
{\'e}tudi{\'e}es en Partie $3$, question 5.

\

On d{\'e}montre dans cette partie que l'{\'e}quation
\begin{equation}
  x^p + y^p + z^p = 0
\end{equation}
n'admet pas de solutions enti{\`e}res non triviales dans le cas o{\`u} $p
\not{| \nobracket} x y z$.

\

Par l'absurde, on se donne trois entiers $x, y, z \in \mathbb{Z}$, deux {\`a}
deux premiers entre eux dans $\mathbb{Z}$, tels que $p \not{| \nobracket} x y
z$ et qui v{\'e}rifient l'{\'e}quation (2). $\maltese \maltese$

\

1. Montrer l'{\'e}galit{\'e} d'id{\'e}aux
\[ \underset{k = 0}{\overset{p - 1}{\prod}} \langle x + \zeta^k y \rangle =
   \langle z^p \rangle \]


2. Soient deux entiers $k$ et $l$ tels que $0 \leqslant k < l \leqslant p -
1$. On montre dans cette question que les id{\'e}aux $\langle x + \zeta^k y
\rangle$ et $\langle x + \zeta^l y \rangle$ de $\mathbb{Z} [\zeta]$ sont
premiers entre eux. Par l'absurde, soit $\mathfrak{B}$ un id{\'e}al premier
divisant $\langle x + \zeta^k y \rangle$ et $\langle x + \zeta^l y \rangle$.

\

\quad$(a)$ En consid{\'e}rant $\langle x + \zeta^l y \rangle - \langle x +
\zeta^k y \rangle$, montrer que $\lambda y \in \mathfrak{B}$.

\

\quad$(b)$ Montrer que $y \nin \mathfrak{B}$, en d{\'e}duire que $x + y \in
\langle \lambda \rangle \cap \mathbb{Z}$ et conclure {\`a} une absurdit{\'e}.

\

3. Justifier l'existence d'un id{\'e}al $I$ tel que $\langle x + \zeta y
\rangle = I^p$.

\

4. Montrer qu'il existe $r \in \mathbb{Z}$, $\varepsilon$ r{\'e}el inversible
de $\mathbb{Z}[\zeta]$ et $\alpha \in \mathbb{Z}[\zeta]$ tels que $x + \zeta y
= \zeta^r \varepsilon \alpha^p$.

\

5. Montrer qu'il existe $a \in \mathbb{Z}$ tel que $\alpha^p = a (\tmop{mod}
\langle p \rangle)$ (attention, ici $\langle p \rangle = p\mathbb{Z}[\zeta]$
et non $p\mathbb{Z}$) et en d{\'e}duire que
\[ x \zeta^{- r} + y \zeta^{1 - r} \nonconverted{minus} x \zeta^r
   \nonconverted{minus} y \zeta^{r \nonconverted{minus} 1} = 0 (\tmop{mod}
   \langle p \rangle) . \]


6. Supposons que$r = 0 (\tmop{mod} p\mathbb{Z})$. Montrer alors que $p | y$
dans $\mathbb{Z}$, ce qui est contraire {\`a} l'hypoth{\`e}se.

\

On montrerait de m{\^e}me que l'on ne peut avoir $r = 1 (\tmop{mod}
p\mathbb{Z})$, ce que l'on admet.

\

7. D'apr{\`e}s la question 5, il existe $\beta \in \mathbb{Z}[\zeta]$ tel que
\[ x \zeta^{\nonconverted{minus} r} + y \zeta^{1 \nonconverted{minus} r}
   \nonconverted{minus} x \zeta^r \nonconverted{minus} y \zeta^{r
   \nonconverted{minus} 1} = \beta p. \]


Montrer que deux des entiers $\pm r, \pm (1 \nonconverted{minus} r)$ sont
{\'e}gaux modulo $p$ ; en d{\'e}duire que $2 r = 1 (\tmop{mod} p\mathbb{Z})$.

\

8. Montrer que $\beta p \zeta^r = (x \nonconverted{minus} y) \lambda$, puis
que $x = y (\tmop{mod} p\mathbb{Z})$.

\

9. Conclure {\`a} une absurdit{\'e}.6. Supposons que$r = 0 (\tmop{mod}
p\mathbb{Z})$. Montrer alors que $p | y$ dans $\mathbb{Z}$, ce qui est
contraire {\`a} l'hypoth{\`e}se.

\

On montrerait de m{\^e}me que l'on ne peut avoir $r = 1 (\tmop{mod}
p\mathbb{Z})$, ce que l'on admet.

\

7. D'apr{\`e}s la question 5, il existe $\beta \in \mathbb{Z}[\zeta]$ tel que
\[ x \zeta^{\nonconverted{minus} r} + y \zeta^{1 \nonconverted{minus} r}
   \nonconverted{minus} x \zeta^r \nonconverted{minus} y \zeta^{r
   \nonconverted{minus} 1} = \beta p. \]


Montrer que deux des entiers $\pm r, \pm (1 \nonconverted{minus} r)$ sont
{\'e}gaux modulo $p$ ; en d{\'e}duire que $2 r = 1 (\tmop{mod} p\mathbb{Z})$.

\

8. Montrer que $\beta p \zeta^r = (x \nonconverted{minus} y) \lambda$, puis
que $x = y (\tmop{mod} p\mathbb{Z})$.

\

9. Conclure {\`a} une absurdit{\'e}.

\

\begin{center}
  {\tmname{\subsubsection*{\tmtextbf{Solution}}}}
\end{center}

\subsubsection*{1. Exercices pr{\'e}liminaires}

\

\tmtextbf{1.} Soit $B \in \mathbb{Z} [X]$ un polyn{\^o}me unitaire, et $A \in
\mathbb{Z} [X]$.

Montrons par r{\'e}currence sur $n = \deg (A)$ que :
\[ \exists Q, R \in \mathbb{Z} [X] \tmop{tels} \tmop{que}
   \left\{\begin{array}{l}
     A = B.Q + R\\
     \deg (R) < \deg (B) \tmop{ou} R = 0
   \end{array}\right. \]


Remarquons tout d'abord que si $\deg (B) = 0$, alors $B = 1$, donc le
r{\'e}sultat est trivial puisque pour tout $A \in \mathbb{Z} [X]$, on a : $A =
B \times A$.

On suppose dans la suite que $\deg (B) \geqslant 1$.

\tmtextbf{Pour $n = 0$,} on a pour tout $A \in \mathbb{Z} [X]$, tel que $\deg
(A) = 0$. On a
\[ A = 0 \times B + A \]


Avec,
\[ \deg (A) = 0 < \deg (B) \]


Donc, le r{\'e}sultat est vrai pour $n = 0$.

Soit $n \in \mathbb{N}$, supposons que le r{\'e}sultat est vrai pour $k = 0,
1, \ldots, n$, et montrons le pour $n + 1$.

Soit $A \in \mathbb{Z} [X],$ $\tmop{tel} \tmop{que}$ $\deg (A) = n + 1$. On
peut {\'e}crire $A$ comme :
\[ A = X A_1 + a_0 \]


Avec,
\[ \left\{\begin{array}{l}
     A_1 \in \mathbb{Z} [X]\\
     a_0 = A (0) \in \mathbb{Z} [X]
   \end{array}\right. \]


On tire $\deg (A_1) = n$, donc, d'apr{\`e}s l'hypoth{\`e}se de r{\'e}currence,
il existe $Q_1, R_1 \in \mathbb{Z} [X]$ tels que :
\[ \left\{\begin{array}{l}
     A_1 = B Q_1 + R_1\\
     \deg (R_1) < \deg (B)
   \end{array}\right. \]


On a alors :
\begin{equation}
  A = B (X Q_1) + X R_1 + a_0
\end{equation}


Or :
\begin{equation}
  \deg (X R_1) = 1 + \deg (R_1) \leqslant \deg (B)
\end{equation}


\tmtextbf{Si $\deg (B) \geqslant \deg (A)$ :}

\tmtextbf{$\rightarrow$Si $\deg (B) > \deg (A) $:}

On a
\[ A = 0 \times B + A \]


Donc le r{\'e}sultat est vrai.

\tmtextbf{$\rightarrow$ Si $\deg (B) = \deg (A)$:}

On note
\[ A = \underset{k = 0}{\overset{n + 1}{\sum}} a_k X^k \]


et
\[ B = X^{n + 1} + \underset{k = 0}{\overset{n}{\sum}} b_k X^k \]


avec $a_0, \ldots, a_{n + 1}, b_0, \ldots, b_n \in \mathbb{Z}$

On a
\begin{eqnarray*}
  A - a_{n + 1} B & = & \underset{k = 0}{\overset{n}{\sum}} (a_k - a_{n + 1}
  b_k) X^k
\end{eqnarray*}


Donc
\[ \begin{array}{lll}
     A & = & a_{n + 1} B + \underset{k = 0}{\overset{n}{\sum}} (a_k - a_{n +
     1} b_k) X^k
   \end{array} \]


avec
\[ \deg \left( \underset{k = 0}{\overset{n}{\sum}} (a_k - a_{n + 1} b_k) X^k
   \right) < n + 1 \]


Donc par l'unicit{\'e} de la division euclidienne, on a le r{\'e}sultat,
puisque :
\[ \left\{\begin{array}{l}
     a_{n + 1} \in \mathbb{Z} [X]\\
     \underset{k = 0}{\overset{n}{\sum}} (a_k - a_{n + 1} b_k) X^k \in
     \mathbb{Z} [X]
   \end{array}\right. \]


\tmtextbf{On suppose dans le suite que $\deg (B) < \deg (A)$}.

D'apr{\`e}s $(2)$, on a :
\begin{eqnarray*}
  \deg (X R_1) & \leqslant & \deg (B)\\
  & < & \deg (A)\\
  & = & n + 1
\end{eqnarray*}


Donc $\deg (X R_1) \leqslant n$.

Si $R_1 = 0$, alors d'apr{\`e}s $(1)$, on a :
\[ A = B (X Q_1) + a_0 \]


Avec $X Q_1 \in \mathbb{Z} [X]$, et $a_0 \in \mathbb{Z} [X]$. C'est fini !

\tmtextbf{Sinon} ( $R_1 \neq 0$), on a :
\[ \deg (X R_1) \in \llbracket 0, n \rrbracket \]


Par hypoth{\`e}se de r{\'e}currence, on a l'existence de $Q_2, R_2 \in
\mathbb{Z} [X]$, tels que :
\[ \left\{\begin{array}{l}
     X R_1 = B Q_2 + R_2\\
     \deg (R_2) < \deg (B)
   \end{array}\right. \]


Donc, via $(1)$, on a :
\begin{eqnarray*}
  A & = & B (X Q_1) + B Q_2 + R_2 + a_0\\
  & = & B (X Q_1 + Q_2) + R_2 + a_0
\end{eqnarray*}


Avec $X Q_1 + Q_2 \in \mathbb{Z} [X]$, $R_2 + a_0 \in \mathbb{Z} [X]$ et $\deg
(R_2 + a_0) < \deg (B)$.

D'o{\`u} le r{\'e}sultat par r{\'e}currence.

\

Il ne reste que le cas o{\`u} $\deg (A) \nin \mathbb{N}$, donc $\deg (A) = -
\infty$, c'est-{\`a}-dire $A = 0$.

On a :
\[ A = 0 \times B + 0 \]


Avec
\[ \left\{\begin{array}{l}
     0 \in \mathbb{Z} [X]\\
     \deg (0) = - \infty < \deg (B)
   \end{array}\right. \]


D'o{\`u} le r{\'e}sultat.

\

\paragraph{2. L'anneau $\mathbb{Z} [j] .$}

\tmtextbf{2.a.} On a
\[ 1 + j + j^2 = \frac{1 - j^3}{1 - j} = 0 \]


Donc $P (j) = 0$, o{\`u} $P = X^2 + X + 1 \in \mathbb{Q} [X]$, et donc $j$ est
alg{\'e}brique de $\mathbb{Q}$.

\

\tmtextbf{D{\'e}terminons le polyn{\^o}me minimal de $j$ :}

\tmtextbf{Lemme 1.}

Soient $\alpha \in \mathbb{C}$ et $P \in \mathbb{Q} [X]$ annulant $\alpha$,
alors $\pi_{\alpha} | P \nobracket$.

\

\tmtextbf{Preuve du lemme 1.}

Puisque $\pi_{\alpha} \neq 0$, et $\mathbb{Q} [X]$ est euclidien (car
$\mathbb{Q}$ est un corps), alors il existe $(Q, R) \in \mathbb{Q} [X]^2$ tels
que:
\[ \left\{\begin{array}{l}
     P = Q \pi_{\alpha} + R\\
     \deg (R) < \deg (\pi_{\alpha})
   \end{array}\right. \]


On a :
\[ R (\alpha) = P (\alpha) - Q (\alpha) \pi_{\alpha} (\alpha) = 0 \]


Donc $R$ annule $\alpha$. De plus, $\deg (R) < \deg (\pi_{\alpha})$, et
$\pi_{\alpha}$ est par d{\'e}finition le polyn{\^o}me non nul, unitaire,
annulant $\alpha$ et de \tmtextbf{degr{\'e} minimal}. Puisque $\deg (R) < \deg
(\pi_{\alpha})$, alors forc{\'e}ment $R = 0$.

D'o{\`u}
\[ \pi_{\alpha} | P \nobracket \]


Montrons maintenant que $\pi_j = X^2 + X + 1$.

Puisque $j^2 + j + 1 = 0$, alors $\pi_j | X^2 + X + 1 \nobracket$. Pour
conclure, il suffit de montrer que $X^2 + X + 1$ est irr{\'e}ductible sur
$\mathbb{Q} [X]$.

On a le discriminant du trin{\^o}me $X^2 + X + 1$ est $- 3 < 0$.

Donc $X^2 + X + 1$ est irr{\'e}ductible sur $\mathbb{R} [X]$.

Or, $\mathbb{Q}$ est un sous-corps de $\mathbb{R}$, alors $X^2 + X + 1$ est
{\'e}galement irr{\'e}ductible sur $\mathbb{Q} [X]$.

Via la relation \ $\pi_j | X^2 + X + 1 \nobracket$ et puisque $X^2 + X + 1$
est irr{\'e}ductible aussi sur $\mathbb{Q} [X]$, alors ou bien $\pi_j$ est
inversible, ou bien $\pi_j $est associ{\'e} {\`a} $X^2 + X + 1$.

Avec $\deg (\pi_j) \geqslant 1$, on en d{\'e}duit que $\pi_j $et $X^2 + X + 1$
sont associ{\'e}s. De plus, ils sont unitaires, donc ils sont {\'e}gaux.

D'o{\`u}
\[ \pi_j = X^2 + X + 1 \]


\tmtextbf{2.b.} \ Montrons que
\[ \mathbb{Z} [j] = \{ a + b j | \nobracket  (a, b) \in \mathbb{Z}^2 \}
\]

\

Soit $(a, b) \in \mathbb{Z}^2$. On a $a + b j = P_{a, b} (j)$, avec $P_{a, b}
= b X + a \in \mathbb{Z} [X]$, donc $a + b j \in \mathbb{Z} [j]$.

D'o{\`u}
\[ \{ a + b j | \nobracket  (a, b) \in \mathbb{Z}^2 \} \subseteq \mathbb{Z}
   [j] \]


Montrons maintenant que $\mathbb{Z} [j] \subseteq \{ a + b j | \nobracket  (a,
b) \in \mathbb{Z}^2 \}$.

Soit $\alpha \in \mathbb{Z} [j]$. Alors par d{\'e}finition, il existe $P \in
\mathbb{Z} [X] $tel que $\alpha = P (j)$.

En utilisant la question 1 de cette partie et puisque $X^2 + X + 1 \in
\mathbb{Z} [X]$est unitaire, on a l'existence de $Q, R \in \mathbb{Z} [X]$
tels que :
\[ \left\{\begin{array}{l}
     P = Q (X^2 + X + 1) + R\\
     \deg (R) < \deg (X^2 + X + 1) = 2
   \end{array}\right. \]


Donc
\[ \alpha = P (j) = Q (j) (j^2 + j + 1) + R (j) = R (j) \]


En {\'e}crivant $R = b_1 X + a_1 \in \mathbb{Z} [X]$, on a alors $\alpha = b_1
j + a_1 \in \{ a + b j | \nobracket  (a, b) \in \mathbb{Z}^2 \}$

D'o{\`u}
\[ \mathbb{Z} [j] \subseteq \{ a + b j | \nobracket  (a, b) \in \mathbb{Z}^2
   \} \]


Finalement
\[ \mathbb{Z} [j] = \{ a + b j | \nobracket  (a, b) \in \mathbb{Z}^2 \} \]
Si vous {\^e}tes int{\'e}ress{\'e}, je vous invite {\`a} d{\'e}montrer la
g{\'e}n{\'e}ralisation suivante :

\tmtextbf{G{\'e}n{\'e}ralisation 1.}

Soit $n \in \mathbb{N} $ tel que $n \geqslant 2$. Notons $u_n = e^{\frac{2 i
\pi}{n}}$. On a
\[ \mathbb{Z} [u_n] = \{ a_0 + a_1 u_n + \cdots + a_{n - 2} u^{n - 2}_n  |
   \nobracket  (a_0, a_1, \ldots, a_{n - 2}) \in \mathbb{Z}^{n - 1} \} \]


\tmtextbf{2.c.} Soit $z \in \mathbb{Z} [j]$, on a l'existence de $(a, b) \in
\mathbb{Z}^2 $tel que $z = a + j b$.

On a alors\footnote{ car $\bar{j} = j^2$ et $j^2 + j = - 1$.}
\begin{eqnarray*}
  N (z) & = & z \bar{z}\\
  & = & (a + j b) (a + \bar{j} b)\\
  & = & (a + j b) (a + j^2 b)\\
  & = & a^2 + a b (j^2 + j) + b^2\\
  & = & a^2 - a b + b^2
\end{eqnarray*}


Or, on sait que :
\[ a^2 + b^2 \geqslant 2 | a b | \]


Donc :
\[ N (z) \geqslant 2 | a b | - \tmop{ab} \geqslant 0 \]


D'o{\`u} $N (z) \in \mathbb{N}$.

Si $z \in \mathbb{Z} [j]$ est inversible dans $\mathbb{Z} [j]$, alors par
d{\'e}finition, il existe $z' \in \mathbb{Z} [j]$ tel que $z.z'$=1.

On a alors :
\[ N (z.z') = N (1) = 1. \bar{1} = 1 \]


Et en d{\'e}veloppant :
\[ N (z.z') = z.z' . \overline{z.z'} = z. \bar{z} .z' . \overline{z'} = N (z)
   N (z') \]


Donc, $N (z) N (z') = 1$ et $N (z), N (z') \in \mathbb{N}$, ce qui implique $N
(z) = 1$.

\

\tmtextbf{D{\'e}terminons les {\'e}l{\'e}ments de $\mathbb{Z} [j]^{\times}$.}

On a $z = a + b j$, o{\`u} $a, b \in \mathbb{Z}$, et $N (z) = 1$, donc :
\[ a^2 - a b + b^2 = 1 \]


On en d{\'e}duit que :
\[ a b + 1 = a^2 + b^2 \geqslant 2 | a b | \]


Ainsi,
\[ (| a b | - a b) + | a b | \leqslant 1 \]


Avec $| a b | - a b, | a b | \in \mathbb{N}$.

Alors,
\[ \left\{\begin{array}{l}
     | a b | - a b = 0\\
     | a b | = 0
   \end{array}\right. \tmop{ou} \left\{\begin{array}{l}
     | a b | - a b = 0\\
     | a b | = 1
   \end{array}\right. \infixor \left\{\begin{array}{l}
     | a b | - a b = 1\\
     | a b | = 0
   \end{array}\right. \]
\[ \  \]


Or, $\left\{\begin{array}{l}
  | a b | - a b = 0\\
  | a b | = 0
\end{array}\right.$ {\'e}quivaut {\`a} dire que ($a = 0 \text{ ou $b = 0$}$),

Si $a = 0$, on a dans ce cas $1 = N (z) = b^2$, donc $b = 1$ ou $b = - 1$.

D'o{\`u} $z = j$ ou $z = - j$.

Si $b = 0$, on a de m{\^e}me $a = 1$ ou $a = - 1$, donc $z = 1$ ou $z = - 1$.

\

Et le syst{\`e}me d'{\'e}quations $\left\{\begin{array}{l}
  | a b | - a b = 0\\
  | a b | = 1
\end{array}\right.$ {\'e}quivaut {\`a} $\left[ \left( a = 1 \infixand b = 1
\right) \infixor (a = - 1 \tmop{et} b = - 1) \right]$

Ainsi, $z = 1 + j$ ou $z = - 1 - j$

\

Et $\left\{\begin{array}{l}
  | a b | - a b = 1\\
  | a b | = 0
\end{array}\right.$ n'admet pas de solutions.

\

En conclusion, on a :
\[ \mathbb{Z} [j]^{\times} \subseteq \{ - 1, 1, - j, j, - 1 - j, 1 + j \} \]


R{\'e}ciproquement, v{\'e}rifions que ces {\'e}l{\'e}ments sont bien des
unit{\'e}s de $\mathbb{Z} [j]$ :
\begin{eqnarray*}
  (- 1)^2 & = & 1\\
  1^2 & = & 1\\
  - j (1 + j) & = & 1\\
  j (- 1 - j) & = & 1
\end{eqnarray*}


Par d{\'e}finition, on a donc :
\[ \{ - 1, 1, - j, j, - 1 - j, 1 + j \} \subseteq \mathbb{Z} [j]^{\times} \]


D'o{\`u} :
\[ \mathbb{Z} [j]^{\times} = \{ - 1, 1, - j, j, - 1 - j, 1 + j \} = \{ - 1, 1,
   - j, j, - j^2, j^2 \} \]


\tmtextbf{2.d.} \ Soient $x \in \mathbb{Z} [j]$ et $y \in \mathbb{Z} [j]
\backslash \{ 0 \}$. Cherchons $q \in \mathbb{Z} [X]$ tel que $N \left(
\frac{x}{y} - q \right) < 1$.

Notons
\[ x = a + j b \infixand y = c + j d \]


On a
\begin{eqnarray*}
  \frac{x}{y} & = & \frac{a + j b}{c + j d}\\
  & = & \frac{(a + j d) (c + j^2 d)}{c^2 - c d + d^2}\\
  & = & \frac{a c - a d + b d}{c^2 - c d + d^2} + j \frac{b c - a d}{c^2 - c
  d + d^2}
\end{eqnarray*}


Notons
\[ \alpha = \frac{a c - a d + b d}{c^2 - c d + d^2} \in \mathbb{Q} \infixand
   \beta = \frac{b c - a d}{c^2 - c d + d^2} \in \mathbb{Q} \]


\tmtextbf{}\tmtextbf{Lemme 2.}

Soit $a \in \mathbb{R},$ alors il existe $t_a \in \mathbb{Z}$, tel que $| a -
t_a | \leqslant \frac{1}{2}$.

\

\tmtextbf{Preuve du lemme 2.}

On a si $a - \lfloor a \rfloor > \frac{1}{2}, \tmop{alors}$
\[ (\lfloor a \rfloor + 1) - a = (\lfloor a \rfloor - a) + 1 < \frac{1}{2} \]


D'o{\`u} le r{\'e}sultat.

\

En particulier, il existe $t_{\alpha}, t_{\beta} \in \mathbb{Z}$ tels que
\[ \left\{\begin{array}{l}
     | \alpha - t_{\alpha} | \leqslant \frac{1}{2}\\
     | \beta - t_{\beta} | \leqslant \frac{1}{2}
   \end{array}\right. \]


Notons $q = t_{\alpha} + j t \beta \in \mathbb{Z} [j]$, on a alors
\begin{eqnarray*}
  N \left( \frac{x}{y} - q \right) & = & N ((\alpha - t_{\alpha}) + j (\beta -
  t_{\beta}))\\
  & = & (\alpha - t_{\alpha})^2 - (\alpha - t_{\alpha}) (\beta - t_{\beta}) +
  (\beta - t_{\beta})^2\\
  & \leqslant & \frac{3}{2} [(\alpha - t_{\alpha})^2 + (\beta -
  t_{\beta})^2]\\
  & \leqslant & \frac{3}{4}\\
  & < & 1
\end{eqnarray*}


Do{\`u} le r{\'e}sultat.

Montrons que $\mathbb{Z} [j]$ est un anneau euclidien.

Pour tout $x, y \in \mathbb{Z} [j] $tel que $y \neq 0$, on a l'existence de $q
\in \mathbb{Z} [j]$ tel que $N \left( \frac{x}{y} - q \right) < 1$.

On a alors $x = y q + (x - q y)$ avec
\begin{eqnarray*}
  N (x - q y) & = & N (y) N \left( \frac{x}{y} - q \right)\\
  & < & N (y)
\end{eqnarray*}


et $x - q y \in \mathbb{Z} [j]$.

D'o{\`u} l'application $N : \mathbb{Z} [j] \rightarrow \mathbb{R}$ v{\'e}rifie
pour tout $x, y \in \mathbb{Z} [j]$ tel que $y \neq 0$, l'existence de $(q, r)
\in \mathbb{Z} [j]^2$ tel que
\[ \left\{\begin{array}{l}
     x = q.y + r\\
     N (r) < N (y)
   \end{array}\right. \]


D'o{\`u} $\mathbb{Z} [j]$ est euclidien.

\

\paragraph{3. Polyn{\^o}mes cyclotomiques. }

\tmtextbf{3.a.} Montrons que :
\[ X^n - 1 = \underset{d | n \nobracket}{\prod} \Phi_d (X) \]


Notons $U_n$ l'ensemble des racines $n$-i{\`e}me de l'unit{\'e}. Montrons que
:
\[ U_n = \underset{d | n \nobracket}{\bigsqcup} \mu_d^{\star} \]


Soit $d \in \mathbb{N}$, tel que $d | n \nobracket$, et soit $z \in
\mu_d^{\star}$.

On a $z$ est une racine $d$-i{\`e}me de l'unit{\'e} donc il existe $k \in
\mathbb{N}$, tel que $z = \exp \left( \frac{2 i \pi k}{d} \right)$.

On a alors :
\begin{eqnarray*}
  z^n & = & \exp \left( \frac{2 i \pi k n}{d} \right)\\
  & = & \exp \left( 2 i \pi k \frac{n}{d} \right)\\
  & = & 1
\end{eqnarray*}


Donc $z \in U_n$, ensuite $\mu_d^{\star} \subset U_n$, et ceci pour tout $d
\in \mathbb{N}$, tel que $d | n \nobracket$.

Donc
\[ \underset{d | n \nobracket}{\bigcup} \mu_d^{\star} \subset U_n \]


R{\'e}ciproquement, soit $z \in U_n$, alors il existe $k \in \llbracket 0, n -
1 \rrbracket$, tel que $z = \exp \left( \frac{2 i k \pi}{n} \right)$

Notons :
\[ k' = \frac{k}{n \wedge k} \infixand n' = \frac{n}{n \wedge k} \]


On a $n' \wedge k' = 1$, et
\[ z = \exp \left( \frac{2 i k' \pi}{n'} \right) \]


Avec $n' | n \nobracket$, et $n' \wedge k' = 1$, alors $z \in \mu_{n'}^{\star}
\subset \underset{d | n \nobracket}{\bigcup} \mu_d^{\star}$.

\

Donc :
\[ U_n \subset \underset{d | n \nobracket}{\bigcup} \mu_d^{\star} \]


Par suite :
\[ U_n = \underset{d | n \nobracket}{\bigcup} \mu_d^{\star} \]


Il ne reste qu'{\`a} montrer que $\underset{d | n \nobracket}{\bigcup}
\mu_d^{\star}$ est disjoint.

Plus g{\'e}n{\'e}ralement, soit $k, l \in \mathbb{N}^{\star}$, tel que $k \neq
l$, montrons que $\mu_k^{\star} \cap \mu_l^{\star} = \emptyset$.

Par l'absurde, supposons que $\mu_k^{\star} \cap \mu_l^{\star} \not{=}
\emptyset$, alors il existe $z \in \mu_k^{\star} \cap \mu_l^{\star}$

Donc,
\[ \begin{array}{c}
     \exists k_1 \in \llbracket 0, k - 1 \rrbracket \tmop{tel} \tmop{que} z =
     \exp \left( \frac{2 i k_1 \pi}{k} \right) \infixand k_1 \wedge k = 1\\
     \tmop{et}\\
     \exists l_1 \in \llbracket 0, l - 1 \rrbracket \tmop{tel} \tmop{que} z =
     \exp \left( \frac{2 i l_1 \pi}{l} \right) \infixand l_1 \wedge l = 1
   \end{array} \]


Par sym{\'e}trie, on peut supposer que $k < l$. On a alors :
\[ \exp \left( \frac{2 i k_1 l \pi}{k} \right) = \exp \left( \frac{2 i l_1 l
   \pi}{l} \right) = \exp (2 \tmop{il}_1 \pi) = 1 \]


Donc $\frac{k_1 l}{k} \in \mathbb{Z}$, par suite $k | k_1 l \nobracket,$ avec
$k_1 \wedge k = 1$.D'apr{\`e}s le lemme de Gauss, on en d{\'e}duit que $k | l
\nobracket,$ absurde, puisque $k, l \in \mathbb{N}^{\star}$ et $k < l$.

D'o{\`u}
\[ \mu_k^{\star} \cap \mu_l^{\star} = \emptyset \]


Ainsi,
\[ \underset{d | n \nobracket}{\bigcup} \mu_d^{\star} = \underset{d | n
   \nobracket}{\bigsqcup} \mu_d^{\star} \]


Donc :
\[ U_n = \underset{d | n \nobracket}{\bigsqcup} \mu_d^{\star} \]


{\`A} partir de ce r{\'e}sultat, on peut d{\'e}duire que :
\begin{eqnarray*}
  X^n - 1 & = & \underset{z \in U_n}{\prod} (X - z)\\
  & = & \underset{z \in \underset{d | n \nobracket}{\bigsqcup}
  \mu_d^{\star}}{\prod} (X - z)\\
  & = & \underset{d | n \nobracket}{\prod} \left[ \underset{z \in
  \mu_d^{\star}}{\prod} (X - z) \right]\\
  & = & \underset{d | n \nobracket}{\prod} \Phi_d (X)
\end{eqnarray*}


\tmtextbf{3.b.} En d{\'e}duire que $\Phi_n (X) \in \mathbb{Z} [X]$,

Montrons le r{\'e}sultat par r{\'e}currence sur $n \in \mathbb{N}^{\ast}$

Pour$n = 1$, on a $\Phi_1 = X - 1 \in \mathbb{Z} [X]$.

Soit $n \in \mathbb{N}^{\ast}$, supposons que $\Phi_1, \Phi_2, \ldots, \Phi_n
\in \mathbb{Z} [X]$ et montrons que $\Phi_{n + 1} \in \mathbb{Z} [X]$.

D'apr{\`e}s la question pr{\'e}c{\'e}dente, on a :
\[ X^n - 1 = \Phi_{n + 1} \underset{d \leqslant n}{\underset{d | n + 1
   \nobracket}{\prod}} \Phi_d \]


Par hypoth{\`e}se de r{\'e}currence, on a $\underset{d \leqslant
n}{\underset{d | n + 1 \nobracket}{\prod}} \Phi_d \in \mathbb{Z} [X]$. De
plus, $\underset{d \leqslant n}{\underset{d | n + 1 \nobracket}{\prod}}
\Phi_d$ est unitaire. En utilisant la question 1 de cette partie, il vient :
\[ \exists (Q, R) \in \mathbb{Z} [X]^2, \quad X^{n + 1} - 1 = Q \underset{d
   \leqslant n}{\underset{d | n + 1 \nobracket}{\prod}} \Phi_d + R \]


avec
\[ \deg (R) < \deg \left( \underset{d \leqslant n}{\underset{d | n + 1
   \nobracket}{\prod}} \Phi_d \right) \]


Pour tout $\alpha \in \mathbb{C}$ racine de $\underset{d \leqslant
n}{\underset{d | n + 1 \nobracket}{\prod}} \Phi_d$, on a
\[ R (\alpha) = \alpha^{n + 1} - 1 - Q (\alpha) \underset{d \leqslant
   n}{\underset{d | n + 1 \nobracket}{\prod}} \Phi_d (\alpha) = 0 \]


Donc $\alpha$ est {\'e}galement racine de $R$.

Avec $\deg (R) < \deg \left( \underset{d \leqslant n}{\underset{d | n + 1
\nobracket}{\prod}} \Phi_d \right)$, donc forc{\'e}ment $R = 0$.

\

D'o{\`u}
\[ X^{n + 1} - 1 = Q \underset{d \leqslant n}{\underset{d | n + 1
   \nobracket}{\prod}} \Phi_d \]


Avec
\[ \Phi_{n + 1} = \frac{X^{n + 1}}{\underset{d \leqslant n}{\underset{d | n +
   1 \nobracket}{\prod}} \Phi_d} = Q \in \mathbb{Z} [X] \]


Ce qui conclut la d{\'e}monstration par r{\'e}currence.

\

\tmtextbf{3.c.} Soit $p$ un nombre premier

\tmtextbf{3.c.i.} On a
\begin{eqnarray*}
  X^p - 1 - (X - 1)^p & = & - \underset{k = 1}{\overset{p - 1}{\sum}} \left(
  \begin{array}{c}
    p\\
    k
  \end{array} \right) X^k (- 1)^{p - k}\\
  & = & \underset{k = 1}{\overset{p - 1}{\sum}} (- 1)^k \left(
  \begin{array}{c}
    p\\
    k
  \end{array} \right) X^k
\end{eqnarray*}


On a, pour tout $k \in \llbracket 1, p - 1 \rrbracket$,
\[ \left( \begin{array}{c}
     p\\
     k
   \end{array} \right) = \frac{p!}{k! (p - k) !} \]


donc
\[ p! = k! (p - k) ! \left( \begin{array}{c}
     p\\
     k
   \end{array} \right) \]


Avec, pour tout $i \in \llbracket 1, k \rrbracket$, on a $p \wedge i = 1$,
donc $p \wedge k! = 1$, et pour tout $i \in \llbracket 1, p - k \rrbracket$ on
a {\'e}galement $p \wedge i = 1$, donc $p \wedge (p - k) ! = 1$.

Ainsi,
\[ p \wedge k! (p - k) ! = 1 \]


avec $p | \nobracket k! (p - k) ! \left( \begin{array}{c}
  p\\
  k
\end{array} \right)$, donc d'apr{\`e}s le lemme de Gauss, on a
\[ p | \nobracket \left( \begin{array}{c}
     p\\
     k
   \end{array} \right) \]


Ainsi,
\[ \pi \left( (- 1)^k \left( \begin{array}{c}
     p\\
     k
   \end{array} \right) \right) = 0 \]


D'o{\`u} :
\begin{eqnarray*}
  \hat{\pi} (X^p - 1 - (X - 1)^p) & = & \underset{k = 1}{\overset{p -
  1}{\sum}} \pi \left( (- 1)^k \left( \begin{array}{c}
    p\\
    k
  \end{array} \right) \right) X^k\\
  & = & 0
\end{eqnarray*}


Donc :
\[ \hat{\pi} (X^p - 1) - \hat{\pi} ((X - 1)^p) = 0 \]


D'o{\`u} :
\[ \hat{\pi} (X^p - 1) = (X - 1_{\mathbb{F}_p})^p \]


Ce qui conclut la d{\'e}monstration.

\

\tmtextbf{3.c.ii.} Soient $P$ et $Q$ deux polyn{\^o}mes unitaires non
constants dans $\mathbb{Z} [X]$ tels que
\[ X^p - 1 = P Q \]


Montrons que $P$ divise $P (1)$ et $Q (1)$.

D'apr{\`e}s la question pr{\'e}c{\'e}dente, on a
\[ \hat{\pi} (X^p - 1) = (X - 1_{\mathbb{F}_p})^p \]


Donc
\[ \hat{\pi} (P) \hat{\pi} (Q) = (X - 1_{\mathbb{F}_p})^p \]


Puisque $P$ et $Q$ sont non constant, alors $\hat{\pi} (P)$ et $\hat{\pi} (Q)$
les sont aussi.

Par unicit{\'e} de la d{\'e}composition en irr{\'e}ductibles dans $\mathbb{Z}/
p\mathbb{Z} [X]$, on en d{\'e}duit l'existence de $k \in \llbracket 1, p - 1
\rrbracket$ tel que
\[ \left\{\begin{array}{l}
     \hat{\pi} (P) = (X - 1_{\mathbb{F}_p})^k\\
     \hat{\pi} (Q) = (X - 1_{\mathbb{F}_p})^{p - k}
   \end{array}\right. \]


Ainsi,
\[ \left\{\begin{array}{l}
     \hat{\pi} (P) (1_{F_p}) = 0_{\mathbb{F}_p}\\
     \hat{\pi} (Q) (1_{F_p}) = 0_{\mathbb{F}_p}
   \end{array}\right. \]


Ensuite
\[ \left\{\begin{array}{l}
     \pi (P (1)) = 0\\
     \pi (Q (1)) = 0
   \end{array}\right. \]


Cela implique que $p | P (1) \nobracket$ et $p | Q (1) \nobracket$.

\

\tmtextbf{3.c.iii.} Montrons que $\Phi_p$ est irr{\'e}ductible de $\mathbb{Q}
[X] .$

Par l'absurde, supposons que $\Phi_p$ n'est pas irr{\'e}ductible de
$\mathbb{Q} [X]$.

Alors, il existe $P, Q \in \mathbb{Q} [X]$ tel que $\Phi_p = P Q$, et $P, Q$
sont non-constants.

D'apr{\`e}s le r{\'e}sultat admis, il existe $r \in \mathbb{Q}$, tel que :
\[ \left\{\begin{array}{l}
     r P \in \mathbb{Z} [X]\\
     \frac{1}{r} Q \in \mathbb{Z} [X]
   \end{array}\right. \]


Notons $P_1 = r P$ et $Q_1 = \frac{1}{r} Q$. On a alors
\[ \Phi_p = P_1 Q_1 \]


avec $P_1, Q_1 \in \mathbb{Z} [X]$ sont non-constants.

Notons
\begin{eqnarray*}
  R_p (X) & = & \Phi_p (X + 1)\\
  & = & \underset{k = 0}{\overset{p - 1}{\sum}} (X + 1)^k\\
  & = & \frac{(X + 1)^p - 1}{X}\\
  & = & \underset{k = 1}{\overset{p}{\sum}} \left( \begin{array}{c}
    p\\
    k
  \end{array} \right) X^{k - 1}\\
  & = & \underset{k = 0}{\overset{p - 1}{\sum}} \left( \begin{array}{c}
    p\\
    k + 1
  \end{array} \right) X^k
\end{eqnarray*}


{\'E}crivons {\'e}galement $P_1 = \underset{k = 0}{\overset{l}{\sum}} a_k X^k$
et $Q_1 = \underset{k = 0}{\overset{m}{\sum}} b_k X^k$, o{\`u} $l, k \geqslant
1$ et $a_0, \ldots, a_l, b_0, \ldots, b_m \in \mathbb{Z}$.

Or, pour tout $k \in \llbracket 0, p - 2 \rrbracket$, on a $p | \nobracket
\left( \begin{array}{c}
  p\\
  k + 1
\end{array} \right)$, donc dans $\mathbb{Z}/ p\mathbb{Z} [X]$, on obtient :
\[ \hat{\pi} (R_p (X)) = X^{p - 1} \]


Ainsi,
\[ \hat{\pi} (P_1 Q_1) = X^{p - 1} \]


Par cons{\'e}quent,
\[ \hat{\pi} (P_1) \hat{\pi} (Q_1) = X^{p - 1} \]


Par unicit{\'e} de la d{\'e}composition en irr{\'e}ductibles dans $\mathbb{Z}/
p\mathbb{Z} [X]$, on a
\[ \hat{\pi} (P_1) = \overline{a_l} X^l \]


rt
\[ \hat{\pi} (Q_1) = \overline{b_m} X^m \]


Ainsi $\overline{a_0} = 0$ et $\overline{b_0} = 0$, donc $p | a_0 \nobracket$
et $p | b_0 \nobracket$, ainsi $p^2 | a_0 b_0 \nobracket$

Or,
\[ a_0 b_0 = \left( \begin{array}{c}
     p\\
     1
   \end{array} \right) = p \]


Alors $p^2 | p \nobracket$, absurde.

D'o{\`u} $\Phi_p$ est irr{\'e}ductible.

\

\tmtextbf{3.c.iv.} Soit $\zeta = e^{\frac{2 i \pi}{p}}$.

On a $\zeta$ est une racine primitive de l'unit{\'e}, donc $\Phi_p (\zeta) =
0$.

Par cons{\'e}quent,
\[ \pi_{\zeta} | \Phi_p \nobracket \]


Or, d'apr{\`e}s la question pr{\'e}c{\'e}dente, on a $\Phi_p$ est
irr{\'e}ductible ; ainsi, soit $\pi_{\zeta}$ est constant, soit $\pi_{\zeta}
$est associ{\'e} {\`a} $\Phi_p$.

Avec $\pi_{\zeta}$ n'est pas constant, alors $\pi_{\zeta}$ et $\Phi_p$ sont
associ{\'e}. De plus, ils sont unitaires, donc ils sont {\'e}gaux.

D'o{\`u},
\[ \pi_{\zeta} = \Phi_p \]


On a :
\[ \mathbb{Q} (\zeta) = \{ P (\zeta) / P \in \mathbb{Q} [X] \} \]


est un corps (facile {\`a} v{\'e}rifier).

De plus, si $P \in \mathbb{Q} [X]$, par division euclidienne de $P \tmop{par}
\Phi_p$, on a l'existence de $Q, R \in \mathbb{Q} [X]^2$ tels que
\[ P = Q \Phi_p + R \infixand \deg (R) < p \]


Notons $R = \underset{k = 0}{\overset{p - 1}{\sum}} a_k X^k$, on a alors
\begin{eqnarray*}
  P (\zeta) & = & R (\zeta)\\
  & = & \underset{k = 0}{\overset{p - 1}{\sum}} a_k \zeta^k\\
  & \in & \tmop{vect}_{\mathbb{Q}} (1, \zeta, \ldots, \zeta^{p - 1})
\end{eqnarray*}


Donc
\[ \mathbb{Q} (\zeta) \subset \tmop{vect}_{\mathbb{Q}} (1, \zeta, \ldots,
   \zeta^{p - 1}) \]


R{\'e}ciproquement, pour tout $k \in \llbracket 0, p - 1 \rrbracket$, on a
$\zeta^k \in \mathbb{Q} (\zeta)$, avec $\mathbb{Q} (\zeta)$ est un
$\mathbb{Q}$-espace vectoriel.

Ainsi,
\[ \tmop{vect}_{\mathbb{Q}} (1, \zeta, \ldots, \zeta^{p - 1}) \subset
   \mathbb{Q} (\zeta) \]


D'o{\`u}
\[ \mathbb{Q} (\zeta) = \tmop{vect}_{\mathbb{Q}} (1, \zeta, \ldots, \zeta^{p -
   1}) \]


Ainsi, la famille $(1, \zeta, \ldots, \zeta^{p - 1})$ est g{\'e}n{\'e}ratrice
de $\mathbb{Q} (\zeta)$. Montrons qu'elle est $\mathbb{Q}$-libre.

Pour cela, soient $a_0, \ldots, a_{p - 1} \in \mathbb{Q}$, tels que
\[ \underset{k = 0}{\overset{p - 1}{\sum}} a_k \zeta^k = 0 \]


On a donc
\[ \left. \underset{k = 0}{\overset{p - 1}{\sum}} a_k X^k \right|_{X = \zeta}
   = 0 \]


Avec $\deg \left( \underset{k = 0}{\overset{p - 1}{\sum}} a_k X^k \right)
\leqslant p - 1 < p$. Par minimalit{\'e} du degr{\'e} de $\pi_{\zeta} =
\Phi_p$, on a forc{\'e}ment
\[ \underset{k = 0}{\overset{p - 1}{\sum}} a_k X^k = 0 \]


Par suite $a_0 = \cdots = a_{p - 1} = 0$.

Ainsi, la famille $(1, \zeta, \ldots, \zeta^{p - 1})$ est $\mathbb{Q}-
\tmop{libre} .$

Donc
\[ [\mathbb{Q} (\zeta) : \mathbb{Q}] = \dim_{\mathbb{Q}} (\mathbb{Q} (\zeta))
   = p \]


On en conclut que l'extension de corps $\mathbb{Q} (\zeta) /\mathbb{Q}$ est de
degr{\'e} $p$.

\

\paragraph{4. Matrice compagnon}

\tmtextbf{4.a.} Soit $k \in \{ 1, \ldots, n - 1 \}$. Par r{\'e}currence finie
sur $k$, on peut montrer facilement que :
\[ C^k_p e_1 = e_{k + 1} \]


Donc, pour tout $Q \in \mathbb{C} [X] $non nul de degr{\'e} inf{\'e}rieur ou
{\'e}gal {\`a} $(n - 1)$, si l'on note :
\[ Q = \underset{k = 0}{\overset{n - 1}{\sum}} a_k X^k \]


On a :
\begin{eqnarray*}
  Q (C_p) e_1 & = & \underset{k = 0}{\overset{n - 1}{\sum}} a_k  C^k_p e_1\\
  & = & \underset{k = 0}{\overset{n - 1}{\sum}} a_k e_{k + 1}\\
  & \neq & 0
\end{eqnarray*}


Car $(e_1, \ldots, e_n)$ est $\mathbb{C}$-libre, et $(a_0, \ldots, a_{n - 1})
$ne sont pas tous nuls.

Ainsi, on obtient :
\[ Q (C_p) \neq 0 \]


En particulier, le degr{\'e} du polyn{\^o}me minimal $\pi_{C_p}$ est
sup{\'e}rieur ou {\'e}gal {\`a} $n$.

Or,
\[ \deg (\pi_{C_p}) \leqslant \deg (\chi_{C_p}) = n \]


O{\`u} $\chi_{C_p}$ d{\'e}signe le polyn{\^o}me caract{\'e}ristique de $C_p$.

Donc :
\[ \deg (\pi_{C_p}) = n \]


\tmtextbf{4.b.} D'apr{\`e}s la question pr{\'e}c{\'e}dente, on a :
\[ C^{n - 1}_p e_1 = e_n \]


Donc,
\begin{eqnarray*}
  C^n_p e_1 & = & C_p e_n\\
  & = & \left( \begin{array}{c}
    - a_0\\
    - a_1\\
    .\\
    .\\
    .\\
    - a_{n - 1}
  \end{array} \right)\\
  & = & - \underset{k = 0}{\overset{n - 1}{\sum}} a_k e_{k + 1}
\end{eqnarray*}


Ainsi,
\[ P (C_p) = C^n_p + \underset{k = 0}{\overset{n - 1}{\sum}} a_k C^k_p \]


Pour tout $j \in \llbracket 2, n \rrbracket$, on a :
\begin{eqnarray*}
  P (C_p) e_j & = & C^n_p e_j + \underset{k = 0}{\overset{n - 1}{\sum}} a_k
  C^k_p e_j\\
  & = & C_p^n (C_p^{j - 1} e_1) + \underset{k = 0}{\overset{n - 1}{\sum}} a_k
  C^k_p (C_p^{j - 1} e_1)
\end{eqnarray*}


L'{\'e}galit{\'e} reste vrai aussi pour $j = 1$, donc pour tout $j \in
\llbracket 1, n \rrbracket$, on a :
\[ \begin{array}{lll}
     P (C_p) e_j & = & C^{n + j - 1}_p e_1 + \underset{k = 0}{\overset{n -
     1}{\sum}} a_k C^{k + j - 1}_p e_1\\
     & = & C^{j - 1}_p \left( C^n_p e_1 + \underset{k = 0}{\overset{n -
     1}{\sum}} a_k C^k_p e_1 \right)\\
     & = & C^{j - 1}_p \left( - \underset{k = 0}{\overset{n - 1}{\sum}} a_k
     e_k + \underset{k = 0}{\overset{n - 1}{\sum}} a_k e_k \right)\\
     & = & 0
   \end{array} \]


Ainsi, $e_j \in \tmop{Ker} (P (C_p))$ pour tout $j \in \llbracket 1, n
\rrbracket$, donc $\tmop{Ker} (P (C_p)) =\mathbb{C}^n$.

Alors,
\[ P (C_p) = 0 \]


Ensuite,
\[ \pi_{C_p} | P \nobracket \]


Avec $\deg (\pi_{C_p}) = \deg (P)$, et $\pi_{C_p}$ et $P$ sont unitaires, donc
$\pi_{C_p}$ et $P$ sont {\'e}gaux.

\

\tmtextbf{4.c.} D'apr{\`e}s le th{\'e}or{\`e}me de Cayley-Hamilton,
$\pi_{C_p} | \chi_{C_p} \nobracket$, avec $\deg (\pi_{C_p}) = \deg
(\chi_{C_p}) = n$

Ainsi, $\pi_{C_p}$ et $\chi_{C_p}$ sont associ{\'e}s. De plus, {\'e}tant
unitaires, on obtient :
\[ \chi_{C_p} = \pi_{C_p} = P \]


\tmtextbf{4.d.} Montrons que
\[ \chi_{Q (M)} = \underset{k = 1}{\overset{n}{\prod}} (X - Q (\alpha_k)) \]


Puisque $\mathbb{C}$ est alg{\'e}briquement clos, alors $M$ est
trigonalisable. Donc, il existe $P \in \tmop{GL}_n (\mathbb{C})$ et $T \in
T_{n, s} (\mathbb{C})$ tels que :
\[ M = P T P^{- 1} \]


On a $\alpha_1, \ldots, \alpha_n$ sont les racines de $\chi_M$, donc il existe
une permutation $\sigma : \mathcal{S}_n \rightarrow \mathcal{S}_n$ telle que
\[ T = \left( \begin{array}{c}
     \alpha_{\sigma (1)}\\
     0\\
     .\\
     .\\
     .\\
     0
   \end{array} \begin{array}{c}
     \\
     
   \end{array} \begin{array}{c}
     \ast\\
     \alpha_{\sigma (2)}\\
     0\\
     .\\
     .\\
     0
   \end{array}  \begin{array}{c}
     .\\
     \\
     \\
     \\
     .\\
     0
   \end{array} \quad \begin{array}{c}
     .\\
     \\
     \\
     \\
     .\\
     0
   \end{array} \quad \begin{array}{l}
     \ast\\
     \star\\
     .\\
     .\\
     \ast\\
     \alpha_{\sigma (n)}
   \end{array} \right) \]


Pour tout $k \in \mathbb{N}$, on a :
\[ M^k = P \left( \begin{array}{c}
     \alpha^k_{\sigma (1)}\\
     0\\
     .\\
     .\\
     .\\
     0
   \end{array} \begin{array}{c}
     \\
     
   \end{array} \begin{array}{c}
     \ast\\
     \alpha^k_{\sigma (2)}\\
     0\\
     .\\
     .\\
     0
   \end{array}  \begin{array}{c}
     .\\
     \\
     \\
     \\
     .\\
     0
   \end{array} \quad \begin{array}{c}
     .\\
     \\
     \\
     \\
     .\\
     0
   \end{array} \quad \begin{array}{l}
     \ast\\
     \star\\
     .\\
     .\\
     \ast\\
     \alpha^k_{\sigma (n)}
   \end{array} \right) P^{- 1} \]


Notons $Q (X) = \underset{k = 0}{\overset{l}{\sum}} a_k X^k$. Alors,
\begin{eqnarray*}
  Q (M) & = & \underset{k = 0}{\overset{l}{\sum}} a_k P \left(
  \begin{array}{c}
    \alpha^k_{\sigma (1)}\\
    0\\
    .\\
    .\\
    .\\
    0
  \end{array} \begin{array}{c}
    \\
    
  \end{array} \begin{array}{c}
    \ast\\
    \alpha^k_{\sigma (2)}\\
    0\\
    .\\
    .\\
    0
  \end{array}  \begin{array}{c}
    .\\
    \\
    \\
    \\
    .\\
    0
  \end{array} \quad \begin{array}{c}
    .\\
    \\
    \\
    \\
    .\\
    0
  \end{array} \quad \begin{array}{l}
    \ast\\
    \star\\
    .\\
    .\\
    \ast\\
    \alpha^k_{\sigma (n)}
  \end{array} \right) P^{- 1}\\
  & = & P \left( \begin{array}{c}
    \underset{k = 0}{\overset{l}{\sum}} a_k \alpha^k_{\sigma (1)}\\
    0\\
    .\\
    .\\
    .\\
    0
  \end{array} \begin{array}{c}
    \\
    
  \end{array} \begin{array}{c}
    \ast\\
    \underset{k = 0}{\overset{l}{\sum}} a_k \alpha^k_{\sigma (2)}\\
    0\\
    .\\
    .\\
    0
  \end{array}  \begin{array}{c}
    .\\
    \\
    \\
    \\
    .\\
    0
  \end{array} \quad \begin{array}{c}
    .\\
    \\
    \\
    \\
    0\\
    0
  \end{array} \quad \begin{array}{l}
    \ast\\
    \star\\
    .\\
    .\\
    \ast\\
    \underset{k = 0}{\overset{l}{\sum}} a_k \alpha^k_{\sigma (n)}
  \end{array} \right) p^{- 1}\\
  & = & P \left( \begin{array}{c}
    Q (\alpha_{\sigma (1)})\\
    0\\
    .\\
    .\\
    .\\
    0
  \end{array} \begin{array}{c}
    \\
    
  \end{array} \begin{array}{c}
    \ast\\
    Q (\alpha _{\sigma (2)})\\
    0\\
    .\\
    .\\
    0
  \end{array}  \begin{array}{c}
    .\\
    \\
    \\
    \\
    .\\
    0
  \end{array} \quad \begin{array}{c}
    .\\
    \\
    \\
    \\
    .\\
    0
  \end{array} \quad \begin{array}{l}
    \ast\\
    \star\\
    .\\
    .\\
    \ast\\
    Q (\alpha _{\sigma (n)})
  \end{array} \right) P^{- 1}
\end{eqnarray*}


Ainsi,
\begin{eqnarray*}
  \chi_{Q (M)} & = & \det (X I_n - M)\\
  & = & \det \left( P \left( \begin{array}{c}
    X - Q (\alpha_{\sigma (1)})\\
    0\\
    .\\
    .\\
    .\\
    0
  \end{array} \begin{array}{c}
    \\
    
  \end{array} \begin{array}{c}
    \ast\\
    X - Q (\alpha _{\sigma (2)})\\
    0\\
    .\\
    .\\
    0
  \end{array}  \begin{array}{c}
    .\\
    \\
    \\
    \\
    .\\
    0
  \end{array} \quad \begin{array}{c}
    .\\
    \\
    \\
    \\
    .\\
    0
  \end{array} \quad \begin{array}{l}
    \ast\\
    \star\\
    .\\
    .\\
    \ast\\
    X - Q (\alpha _{\sigma (n)})
  \end{array} \right) P^{- 1} \right)\\
  & = & \det \left( P \times P^{- 1} \times \left( \begin{array}{c}
    X - Q (\alpha_{\sigma (1)})\\
    0\\
    .\\
    .\\
    .\\
    0
  \end{array} \begin{array}{c}
    \\
    
  \end{array} \begin{array}{c}
    \ast\\
    X - Q (\alpha _{\sigma (2)})\\
    0\\
    .\\
    .\\
    0
  \end{array}  \begin{array}{c}
    .\\
    \\
    \\
    \\
    .\\
    0
  \end{array} \quad \begin{array}{c}
    .\\
    \\
    \\
    \\
    .\\
    0
  \end{array} \quad \begin{array}{l}
    \ast\\
    \star\\
    .\\
    .\\
    \ast\\
    X - Q (\alpha _{\sigma (n)})
  \end{array} \right) \right)\\
  & = & \det \left( \begin{array}{c}
    X - Q (\alpha_{\sigma (1)})\\
    0\\
    .\\
    .\\
    .\\
    0
  \end{array} \begin{array}{c}
    \\
    
  \end{array} \begin{array}{c}
    \ast\\
    X - Q (\alpha _{\sigma (2)})\\
    0\\
    .\\
    .\\
    0
  \end{array}  \begin{array}{c}
    .\\
    \\
    \\
    \\
    .\\
    0
  \end{array} \quad \begin{array}{c}
    .\\
    \\
    \\
    \\
    .\\
    0
  \end{array} \quad \begin{array}{l}
    \ast\\
    \star\\
    .\\
    .\\
    \ast\\
    X - Q (\alpha _{\sigma (n)})
  \end{array} \right)\\
  & = & \underset{k = 1}{\overset{n}{\prod}} (X - Q (\alpha_{\sigma (k)}))\\
  & = & \underset{k = 1}{\overset{n}{\prod}} (X - Q (\alpha_k))
\end{eqnarray*}


\tmtextbf{4.e.} $A$ est un sous-anneau de $\mathbb{C}$, et $Q \in A [X]$.

Soit $P \in A [X]$ un polyn{\^o}me unitaire dont les racines complexes,
compt{\'e}es avec leur multiplicit{\'e}, sont $\alpha_1, \ldots, \alpha_n$.

Puisque $P \in A [X]$, alors $C_p \in \mathcal{M}_n (A)$. {\'E}tant donn{\'e}
que $\mathcal{M}_n (A)$ est un anneau, alors
\[ \forall k \in \mathbb{N}, \quad C^k_p \in \mathcal{M}_n (A) \]


Ensuite,
\[ \forall R \in A [X], \quad R (C_p) \in \mathcal{M}_n (A) \]


En particulier,
\[ Q (C_p) \in \mathcal{M}_n (A) \]


Ainsi,
\[ X I_n - Q (C_p) \in \mathcal{M}_n (A) \]


Par d{\'e}finition du d{\'e}terminant, et puisque $A$ est un anneau, on en
d{\'e}duit que
\[ \chi_{Q (C_p)} = \det (X I_n - Q (C_p)) \in A [X] \]


Or, le polyn{\^o}me caract{\'e}ristique de $C_p$ est $P$, dont les racines
sont $\alpha_1, \ldots, \alpha_n$. D'apr{\`e}s la question pr{\'e}c{\'e}dente,
on a donc
\[ \chi_{Q (C_p)} = \underset{k = 1}{\overset{n}{\prod}} (X - Q (\alpha_k)) \]


Enfin,
\[ \underset{k = 1}{\overset{n}{\prod}} (X - Q (\alpha_k)) \in A [X] \]


\subsubsection*{II Nombres alg{\'e}briques}

\tmtextbf{1.a. }On consid{\`e}re la $\varphi : \mathbb{N}^{\star} \rightarrow
\mathbb{N}$ d{\'e}finie pour tout $n \in \mathbb{N}^{\ast}$ par :
\[ \varphi (n) = \tmop{card} \{ k \in \llbracket 1, n \rrbracket  | \nobracket
   k \wedge n = 1 \} \]


Soit $d \geqslant 1$ un entier. Montrons que $\{ n \in \mathbb{N}^{\star},
\varphi (n) \leqslant d \}$ est fini.

\

Pour tout $n \geqslant 2$, {\'e}crivons $n$ sous la forme :
\[ n = \underset{j = 1}{\overset{r}{\prod}} p^{\alpha_{i_j}}_{i_j} \]


o{\`u} $i_1 < \ldots < i_r \in \mathbb{N}^{\ast}$, $(p_i)_{i \in
\mathbb{N}^{\ast}} $est la suite des nombres premiers, et $\alpha_{i_1},
\ldots, \alpha_{i_r} \in \mathbb{N}^{\ast}$.

On a alors :
\[ \varphi (n) = n \underset{j = 1}{\overset{r}{\prod}} \left( 1 -
   \frac{1}{p_{i_j}} \right) \]


Or, pour tout $i \in \mathbb{N}^{\ast}$, on a $p_i \geqslant (i + 1)$ (Car la
suite $(p_i)_{i \in \mathbb{N}^{\ast}}$ est une suite d'entiers strictement
croissante avec $p_1 = 2$).

Donc :
\begin{eqnarray*}
  \varphi (n) & \geqslant & n \underset{j = 1}{\overset{r}{\prod}} \left( 1 -
  \frac{1}{i_j + 1} \right)\\
  & \geqslant & n \underset{j = 1}{\overset{i_r}{\prod}} \left( 1 -
  \frac{1}{j + 1} \right)\\
  & = & \frac{n}{i_r + 1}
\end{eqnarray*}


Avec :
\begin{eqnarray*}
  n & = & \underset{j = 1}{\overset{r}{\prod}} p^{\alpha_{i_j}}_{i_j}\\
  & \geqslant & 2^{\alpha_{i_1} + \cdots + \alpha_{i_r}}\\
  & \geqslant & 2^{i_r}
\end{eqnarray*}


On en d{\'e}duit que :
\[ i_r \leqslant \frac{\log (n)}{\log (2)} \]


Ainsi :
\begin{eqnarray*}
  \varphi (n) & \geqslant & \frac{n}{\frac{\log (n)}{\log (2)} + 1}\\
  & \underset{n \rightarrow + \infty}{\sim} & \log (2) \frac{n}{\log (n)}\\
  & \underset{n \rightarrow + \infty}{\rightarrow} & + \infty
\end{eqnarray*}


Par cons{\'e}quent, il existe un entier $N_0 \in \mathbb{N}$ tel que, pour
tout $n \geqslant N_0$, on a $\varphi (n) > d$.

D'o{\`u} :
\[ \{ n \in \mathbb{N}^{\ast}  | \nobracket \varphi (n) \leqslant d \}
   \subseteq \llbracket 1, N_0 - 1 \rrbracket \]


On en conclut que l'ensemble $\{ n \in \mathbb{N}^{\star}, \varphi (n)
\leqslant d \}$ est fini.

\

\tmtextbf{1.b.} Soit $K$ un sous-corps de $\mathbb{C}$, tel que $K
/\mathbb{Q}$ soit une extension finie.

Montrons que $K$ contient un nombre fini de racines de l'unit{\'e}.

Soit $u$ une racine de l'unit{\'e}, donc par d{\'e}finition, il existe $n \in
\mathbb{N}^{\ast}$, tel que $u^n = 1$.

Soit $n_0$ le plus petit entier non nul tel que $u^{n_0} = 1$.

Alors, u est une racine $n_0$-i{\`e}me primitive de l'unit{\'e}.

On a alors :
\[ \Phi_{n_0} (u) = 0 \]


o{\`u} $\Phi_{n_0}$ est irr{\'e}ductible sur $\mathbb{Q} [X]$, donc
$\pi_{\alpha} = \Phi_{n_0}$.

Ainsi, $\mathbb{Q} (u) /\mathbb{Q}$ est une extension finie de degr{\'e} $\deg
(\Phi_{n_0}) = \varphi (n_0)$.

On obtient donc :
\[ \varphi (n_0) \leqslant [K : \mathbb{Q}] \]


Ainsi,
\[ u \in \underset{\varphi (n_0) \leqslant [K : \mathbb{Q}]}{\underset{n_0 \in
   \mathbb{N}^{\star}}{\bigcup}} \{ \alpha \in \mathbb{C} | \nobracket
   \alpha^{n_0} = 1 \} \]


Or, pour tout $n_0 \in \mathbb{N}$, $\{ \alpha \in \mathbb{C} | \nobracket
\alpha^{n_0} = 1 \}$ est fini, et d'apr{\`e}s la question pr{\'e}c{\'e}dente,
on a l'ensemble
\[ \{ n_0 \in \mathbb{N}^{\ast}  | \nobracket \varphi (n_0) \leqslant [K :
   \mathbb{Q}] \} \]


est fini.

D'o{\`u} le r{\'e}sultat.

\

\tmtextbf{2.a. }Montrons que $\pi_{\alpha}$ est irr{\'e}ductible de
$\mathbb{Q} [X]$.

Soient $P, Q \in \mathbb{Q} [X]$ tels que $\pi_{\alpha} = \tmop{PQ}$.

On a
\[ P (\alpha) Q (\alpha) = \pi_{\alpha} (\alpha) = 0 \]


Donc $P (\alpha) = 0$ ou $Q (\alpha) = 0$.

Par sym{\'e}trie, on suppose que $P (\alpha) = 0$. Alors $\pi_{\alpha} | P
\nobracket$, car $P$ est non nul (si c'est le cas, on aurait $\pi_{\alpha} =
0$, ce qui contredit le fait que $\deg (\pi_{\alpha}) \geqslant 1$).

Comme $\deg (P) \leqslant \deg (\pi_{\alpha})$, on a n{\'e}cessairement $\deg
(P) = \deg (\pi_{\alpha})$.

Ce qui implique que $P$et $\pi_{\alpha}$ sont associ{\'e}s.

D'o{\`u}, $\pi_{\alpha}$ est irr{\'e}ductible de $\mathbb{Q} [X]$.

Notons $n = \deg (\pi_{\alpha})$. Et soit $P \in \mathbb{Q} [X] .$ Par
division euclidienne de $P$ par $\pi_{\alpha}$, on a l'existence de $Q, R \in
\mathbb{Q} [X] $tels que
\[ P = Q \pi_{\alpha} + R \]


Donc
\begin{eqnarray*}
  P (\alpha) & = & Q (\alpha) \pi_{\alpha} (\alpha) + R (\alpha)\\
  & = & R (\alpha)\\
  & \in & \tmop{vect}_{\mathbb{Q}} (1, \alpha, \ldots, \alpha^{n - 1})
\end{eqnarray*}


Cela signifie que
\[ \mathbb{Q} (\alpha) = \{ P (\alpha)  | \nobracket P \in \mathbb{Q} [X] \}
   \subset \tmop{vect}_{\mathbb{Q}} (1, \alpha, \ldots, \alpha^{n - 1}) \]


En particulier, la famille $(1, \alpha, \ldots, \alpha^{n - 1})$ est
g{\'e}n{\'e}ratrice de $\mathbb{Q} (\alpha)$.

Si $a_0, \ldots, a_{n - 1} \in \mathbb{Q}$, tels que
\[ \underset{k = 0}{\overset{n - 1}{\sum}} a_k \alpha^k = 0 \]


Alors $\pi_{\alpha} \left| \underset{k = 0}{\overset{n - 1}{\sum}} a_k X^k
\right.$. {\'E}tant donn{\'e} que $\deg (\pi_{\alpha}) = n$, on a
n{\'e}cessairement $\underset{k = 0}{\overset{n - 1}{\sum}} a_k X^k = 0$,
d'o{\`u} $a_0 = \cdots = a_{n - 1} = 0$.

Cela montre que la famille $(1, \alpha, \ldots, \alpha^{n - 1})$ est
$\mathbb{Q}$-libre.

En conclusion, la famille $(1, \alpha, \ldots, \alpha^{n - 1})$ est une base
de $\mathbb{Q}$-espace vectoriel $K =\mathbb{Q} (\alpha)$

Donc
\[ d = [K : \mathbb{Q}] = \dim_{\mathbb{Q}} (K) = n \]


Ainsi,
\[ \deg \left( {\pi_{\alpha}}  \right) = d \]


\tmtextbf{2.b.} Soit $\sigma$ un morphisme de $\mathbb{Q}$-alg{\`e}bre de $K$
dans $\mathbb{C}$.

Montrons que $\sigma (\alpha)$ est une racine de $\pi_{\alpha}$.

Notons $\pi_{\alpha} = \underset{k = 0}{\overset{d}{\sum}} a_k X^k$. On a
$\pi_{\alpha} (\alpha) = 0$, donc
\[ \underset{k = 0}{\overset{d}{\sum}} a_k \alpha^k = 0 \]


Ainsi,
\[ \sigma \left( \underset{k = 0}{\overset{d}{\sum}} a_k \alpha^k \right) = 0
\]


Par suite,
\[ \underset{k = 0}{\overset{d}{\sum}} a_k \sigma (\alpha)^k = 0 \]


D'o{\`u}
\[ \pi_{\alpha} (\sigma (\alpha)) = 0 \]


Cela prouve le r{\'e}sultat. Montrons maintenant qu'il y a exactement $d$
morphismes de $\mathbb{Q}$-alg{\'e}bre.

\

\tmtextbf{Lemme 3.}

Soit $P \in \mathbb{Q} [X]$ irr{\'e}ductible ; alors les racines complexes de
$P$ sont deux {\`a} deux distinctes.

\

\tmtextbf{Preuve du lemme 3.}

Si $P$ admet une racine double $\alpha \in \mathbb{C}$, alors $\alpha$ est
aussi racine de $P'$.

Donc on a $P \wedge P'$ est non constant dans $\mathbb{C} [X]$.

Or, la division euclidienne est invariante par extension de corps ; donc,
d'apr{\`e}s l'algorithme d'Euclide, $P \wedge P'$ reste non constant dans
$\mathbb{Q} [X]$.

Or, comme $P$ est irr{\'e}ductible, donc $P \wedge P' = P$, en particulier $P'
| P \nobracket .$

Et puisque $\deg (P') = \deg (P) - 1 < \deg (P)$, cela implique $P' = 0$, ce
qui est absurde, car $P$ serait alors constant !

D'o{\`u} le r{\'e}sultat du lemme.

\

On en d{\'e}duit que $\pi_{\alpha}$ admet exactement $d$ recines complexes,
not{\'e}es $\alpha_1, \ldots, \alpha_d$ avec $\alpha_1 = \alpha$.

Soient $\sigma, \tau : K \rightarrow \mathbb{C}$ deux morphismes d'alg{\`e}bre
tels que $\sigma (\alpha) = \tau (\alpha)$. Montrons que $\sigma = \tau$.

Soit $x \in K =\mathbb{Q} (\alpha)$. Alors il existe $P = \underset{k =
0}{\overset{n}{\sum}} a_k X^k \in \mathbb{Q} [X]$ tel que $x = P (\alpha)$

On a alors
\begin{eqnarray*}
  \sigma (x) & = & \sigma \left( \underset{k = 0}{\overset{n}{\sum}} a_k
  \alpha^k \right)\\
  & = & \underset{k = 0}{\overset{n}{\sum}} a_k \sigma (\alpha)^k\\
  & = & \underset{k = 0}{\overset{n}{\sum}} a_k \tau (\alpha)^k\\
  & = & \tau \left( \underset{k = 0}{\overset{n}{\sum}} a_k \alpha^k
  \right)\\
  & = & \tau (x)
\end{eqnarray*}


Cela est vrai pour tout $x \in K$, donc $\sigma = \tau$. On en conclut qu'un
morphisme d'alg{\`e}bre $K \rightarrow \mathbb{C}$ est d{\'e}termin{\'e} par
l'image de $\alpha$, et cette image est une racine de $\pi_{\alpha}$.

On en d{\'e}duit qu'il existe exactement $d$ morphismes de
$\mathbb{Q}$-alg{\`e}bre, not{\'e}s $\sigma_k : K \rightarrow \mathbb{C}$,
o{\`u} $k \in \{ 1, 2, \ldots, d \}$.

\

\tmtextbf{3.} Soit $\alpha \in \mathbb{C}$ un nombre alg{\'e}brique, et
$\theta \in K =\mathbb{Q} (\alpha)$.

\tmtextbf{3.a.} Justifiant que $\theta$ est un nombre alg{\'e}brique.

On a $\mathbb{Q} (\theta)$ est une sous-alg{\`e}bre de $\mathbb{Q} (\alpha)$.

Or, $\alpha$ est alg{\'e}brique, donc l'extension de corps $\mathbb{Q}
(\alpha) /\mathbb{Q}$ est finie. Par cons{\'e}quent, $\mathbb{Q} (\theta)
/\mathbb{Q}$ est aussi finie (car $\mathbb{Q} (\theta)$ est une
sous-alg{\`e}bre de $\mathbb{Q} (\alpha)$).

D'o{\`u} $\theta$ est un nombre alg{\'e}brique.

\

\tmtextbf{3.b.} Montrons que $P_{\theta} = \underset{k =
1}{\overset{d}{\prod}} (X - \sigma_k (\theta)) \in \mathbb{Q} [X]$

Comme $\theta \in \mathbb{Q} (\alpha)$, alors il existe $R = \underset{k =
0}{\overset{n}{\sum}} a_k X^k \in \mathbb{Q} [X]$ tel que $\theta = R
(\alpha)$.

On a pour tout $k \in \llbracket 1, d \rrbracket$ :
\begin{eqnarray*}
  \sigma_k (\theta) & = & \sigma_k \left( \underset{j = 0}{\overset{n}{\sum}}
  a_j \alpha^j \right)\\
  & = & \underset{j = 0}{\overset{n}{\sum}} a_j \sigma_k (\alpha)^j\\
  & = & R (\sigma_k (\alpha))
\end{eqnarray*}


Donc :
\[ P_{\theta} = \underset{k = 1}{\overset{d}{\prod}} (X - R (\sigma_k
   (\alpha))) \]


Or, $\sigma_1 (\alpha), \ldots, \sigma_d (\alpha)$ sont les racines de
$\pi_{\alpha} \in \mathbb{Q} [X]$, et $R \in \mathbb{Q} [X]$.

D'apr{\`e}s la question \tmtextbf{4.e} de la partie \tmtextbf{I}, on a donc :
\[ P_{\theta} = \underset{k = 1}{\overset{d}{\prod}} (X - R (\sigma_k
   (\alpha))) \in \mathbb{Q} [X] \]


\tmtextbf{3.c.} Justifions que $\pi_{\theta}$ divise $P_{\theta}$, en
utilisant la m{\^e}me notation que dans la question pr{\'e}c{\'e}dente:
\[ \theta = \underset{j = 0}{\overset{n}{\sum}} a_j \alpha^j \]


o{\`u} $a_0, \ldots, a_n \in \mathbb{Q}$.

On a :
\begin{eqnarray*}
  P_{\theta} (\theta) & = & \underset{k = 1}{\overset{d}{\prod}} (\theta -
  \sigma_k (\theta))\\
  & = & \underset{k = 1}{\overset{d}{\prod}} \left( \underset{j =
  0}{\overset{n}{\sum}} a_j \alpha^j - \sigma_k \left( \underset{j =
  0}{\overset{n}{\sum}} a_j \alpha^j \right) \right)\\
  & = & \underset{k = 1}{\overset{d}{\prod}} \left( \underset{j =
  0}{\overset{n}{\sum}} a_j (\alpha^j - \sigma_k (\alpha)^j) \right)
\end{eqnarray*}


Or, les $\sigma_1 (\alpha), \ldots, \sigma_d (\alpha)$ sont exactement les
racines de $\pi_{\alpha}$.

Donc, il existe $k_0 \in \llbracket 1, d \rrbracket$ tel que
\[ \sigma_{k_0} (\alpha) = \alpha \]


On a alors :
\begin{eqnarray*}
  P_{\theta} (\theta) & = & \left( \underset{j = 0}{\overset{n}{\sum}} a_j
  (\alpha^j - \alpha^j) \right) \underset{k \neq k_0}{\underset{k =
  1}{\overset{d}{\prod}}} \left( \underset{j = 0}{\overset{n}{\sum}} a_j
  (\alpha^j - \sigma_k (\alpha)^j) \right)\\
  & = & 0
\end{eqnarray*}


Comme $P_{\theta} \in \mathbb{Q} [X]$ (d'apr{\`e}s la question
pr{\'e}c{\'e}dente), on en d{\'e}duit, via le lemme 1, que $\pi_{\theta} |
P_{\theta} \nobracket$.

\

Montrons maintenant que $P_{\theta}$ est une puissance de $\pi_{\theta}$.

Notons :
\[ \mathcal{A}_{\theta} = \{ k \in \mathbb{N} | \nobracket \pi_{\theta} |
   P_{\theta} \nobracket \} \]


D'apr{\`e}s la premi{\`e}re partie de cette question $1 \in
\mathcal{A}_{\theta}$, donc $\mathcal{A}_{\theta} \neq \varnothing$. Ainsi,
cette partie de $\mathbb{N}$ admet un plus grand {\'e}l{\'e}ment, not{\'e}
$k_0 = \max (\mathcal{A}_{\theta})$.

On a alors $\pi^{k_0}_{\theta} | P_{\theta} \nobracket$ et $\pi^{k_0 +
1}_{\theta} \not{| \nobracket} P_{\theta}$.

Donc, il existe $Q \in \mathbb{Q} [X]$ tel que $P_{\theta} = Q
\pi^{k_0}_{\theta}$.

Comme $\pi_{\theta} \not{| \nobracket} Q$ et $\pi _{\theta}$ est
irr{\'e}ductible, alors $Q \wedge \pi_{\theta} = 1$.

Par suite, via le th{\'e}or{\`e}me de B{\'e}zout\footnote{Le th{\'e}or{\`e}me
de B{\'e}zout est valable sur $\mathbb{Q} [X]$, car l'anneau $\mathbb{Q} [X]$
est euclidien (puisque $\mathbb{Q}$ est un corps), donc $\mathbb{Q} [X]$ est
un anneau de B{\'e}zout.}, on a l'existence de $R, S \in \mathbb{Q} [X]$ tels
que :
\[ R \pi_{\theta} + S Q = 1 \]


Par l'absurde, supposons que $Q$ est non constant.

Alors, d'apr{\`e}s le th{\'e}or{\`e}me fondamental de l'alg{\`e}bre, $Q$ est
scind{\'e} sur $\mathbb{C}$. De plus, toutes ses racines sont des racines de
$P_{\theta}$.

Notons
\[ \pi_{\theta} = \underset{j = 0}{\overset{l}{\sum}} \beta_j X^j \]


On a les racines de $P_{\theta}$ sont $\sigma_1 (\theta), \ldots, \sigma_d
(\theta)$. Et pour tout $k \in \llbracket 1, d \rrbracket$, on a :
\begin{eqnarray*}
  \pi_{\theta} (\sigma_k (\theta)) & = & \pi_{\theta} (\sigma_k (\theta))\\
  & = & \underset{j = 0}{\overset{l}{\sum}} \beta_j \sigma_k (\theta)^j\\
  & = & \sigma_k \left( \underset{j = 0}{\overset{l}{\sum}} \beta_j \theta^j
  \right)\\
  & = & \sigma_k (\pi_{\theta} (\theta))\\
  & = & \sigma_k (0)\\
  & = & 0
\end{eqnarray*}


Soit $\gamma$ un racine de $Q$. Puisque les racines de $Q$ sont des racines de
$P_{\theta}$, alors il existe $k_0 \in \llbracket 1, d \rrbracket$ tel que :
\[ \gamma = \sigma_{k_0} (\theta) \]


via la relation $(5)$, on a :
\[ 1 = R (\sigma_{k_0} (\theta)) \pi_{\theta} (\sigma_{k_0} (\theta)) + S
   (\sigma_{k_0} (\theta)) Q (\sigma_{k_0} (\theta)) = 0 \]


ce qui est absurde !

D'o{\`u} $Q$ est constant. Comme $P_{\theta}$ et $\pi_{\theta}$ sont
unitaires, alors $Q = 1$.

Par suite,
\[ P_{\theta} = \pi^{k_0}_{\theta} \]


D'o{\`u} le r{\'e}sultat.

\

\tmtextbf{4.} Soit $\alpha \in \mathbb{C}$

$\Rightarrow$) Si $\alpha$ est un entier alg{\'e}brique, alors par
d{\'e}finition, il existe un polyn{\^o}me unitaire {\`a} coefficients entiers
tel que $P (\alpha) = 0$.

On a alors $\pi_{\alpha} | P \nobracket$, donc il existe $Q \in \mathbb{Q}
[X]$ tel que $P = \pi_{\alpha} Q$.

D'apr{\`e}s le r{\'e}sultat admis, on a l'existence de $r \in \mathbb{Q}$ tel
que :
\[ \left\{\begin{array}{l}
     r \pi_{\alpha} \in \mathbb{Z} [X]\\
     \frac{1}{r} Q \in \mathbb{Z} [X]
   \end{array}\right. \]


Le coefficient dominant de $r \pi_{\alpha}$ est $r$, donc $r \in \mathbb{Z}$.

De m{\^e}me, le coefficient dominant de $\frac{1}{r} Q$ est $\frac{1}{r}$,
alors $\frac{1}{r} \in \mathbb{Z}$.

Ainsi, $r \in \{ - 1, 1 \}$, d'o{\`u} $\pm \pi_{\alpha} \in \mathbb{Z} [X]$,
donc $\pi_{\alpha} \in \mathbb{Z} [X]$.

$\Leftarrow$) Si $\pi_{\alpha} \in \mathbb{Z} [X]$,

puisque $\pi_{\alpha} (\alpha) = 0$, par d{\'e}finition, $\alpha$ est un
entier alg{\'e}brique.

\

\tmtextbf{5.a.} Si $\alpha$ est un entier alg{\'e}brique, notons $d = \deg
(\pi_{\alpha})$.

Soit $x \in \tmop{gr} \{ \alpha^n  | \nobracket n \in \mathbb{N} \}$, donc il
existe $n_1, \ldots, n_r \in \mathbb{N}$ et $a_1, \ldots, a_r \in \mathbb{Z}$
tels que
\[ x = a_1 \alpha^{n_1} + \cdots + a_r \alpha^{n_r} \]


Notons
\[ P = \underset{k = 1}{\overset{r}{\sum}} a_k X^{n_k} \in \mathbb{Z} [X] \]


D'apr{\`e}s \tmtextbf{la question 1 de la partie $I$}, on a l'existence de $Q,
R \in \mathbb{Z} [X]$ tels que
\[ P = \pi_{\alpha} Q + R \]


avec $\deg (R) \leqslant d - 1$.

On a donc
\[ x = P (\alpha) = \pi_{\alpha} (\alpha) Q (\alpha) + R (\alpha) \]


Ainsi, $x$ s'{\'e}crit comme une combinaison lin{\'e}aire {\`a} coefficients
entiers de $1, \alpha, \ldots, \alpha^{d - 1}$.

D'o{\`u} le groupe engendr{\'e} par $\{ \alpha^n  | \nobracket n \in
\mathbb{N} \}$ est de type fini.

\

\tmtextbf{5.b.} R{\'e}ciproquement, si $G$ est de type fini, montrons que
$\alpha$ est un entier alg{\'e}brique.

Soit $(g_1, \ldots, g_n)$ une famille g{\'e}n{\'e}ratrice finie de $G$.

Notons, pour tout $i \in \llbracket 1, n \rrbracket$,
\[ \alpha g_i = \underset{k = 1}{\overset{n}{\sum}} a_{i, k} g_k \]


O{\`u} $a_{i, k} \in \mathbb{Z}$, pour tous $i, k \in \llbracket 1, n
\rrbracket$

Notons $A = (a_{i, k})_{1 \leqslant i, k \leqslant n}$ et $X = \left(
\begin{array}{c}
  g_1 \\
  g_2\\
  .\\
  .\\
  .\\
  g_n
\end{array} \right)$

On a alors
\[ \alpha X = A X \]


Donc,
\[ (A - \alpha I_n) X = 0 \]


En particulier,
\[ A - \alpha I_n \not{\in} \tmop{GL}_n (\mathbb{C}) \]


D'o{\`u},
\[ \det (A - \alpha I_n) = 0 \]


Ainsi,
\[ \underset{\sigma \in \mathcal{S}_n}{\sum} (- 1)^{^{\varepsilon (\sigma)}}
   \underset{i = 1}{\overset{n}{\prod}} (a_{\sigma (i), i} - \alpha
   \delta_{\sigma (i), i}) = 0 \]


O{\`u} $\varepsilon (\sigma) \in \{ - 1, 1 \}$ est la signature de $\sigma$,
pour tout $\sigma \in \mathcal{S}_n .$

D'o{\`u},
\[ P (\alpha) = 0 \]


Avec,
\[ P = \underset{\sigma \in \mathcal{S}_n}{\sum} (- 1)^{^{\varepsilon
   (\sigma)}} \underset{i = 1}{\overset{n}{\prod}} (a_{\sigma (i), i} - X
   \delta_{\sigma (i), i}) \in \mathbb{Z} [X] \]


D'o{\`u} $\alpha$ est un entier alg{\'e}brique.

\

\tmtextbf{6.} Montrons que $\mathcal{D}_{\mathbb{C}}$, l'ensemble des entiers
alg{\'e}briques de $\mathbb{C}$, est un sous-anneau de $\mathbb{C}$.

On a $0 \in \mathcal{D}_{\mathbb{C}}$ (car $\pi_0 = X \in \mathbb{Z} [X]$),
donc \ensuremath{\mathcal{D}}\tmrsub{\ensuremath{\mathbb{C}}}$\neq
\varnothing$.

Soient $\alpha, \beta \in \mathcal{D}_{\mathbb{C}}$.

Les deuc groupes $G_{\alpha} : = \tmop{gr} \{ \alpha^n  | \nobracket n \in
\mathbb{N} \}$ et $G_{\beta} : = \tmop{gr} \{ \beta^n | \nobracket n \in
\mathbb{N} \}$ sont de type fini.

Donc, il existe une famille $(g_1, \ldots, g_n)$ (respectivement $(l_1,
\ldots, l_m)$) g{\'e}n{\'e}ratrice de $G_{\alpha}$ (respectivement de
$G_{\beta}$).

On a, pour tout $i \in \mathbb{N}$, il existe $a_{i, 1}, \ldots, a_{i, n} \in
\mathbb{Z}$, tel que
\[ \alpha^i = \underset{k = 1}{\overset{n}{\sum}} a_{i, k} g_k \]


Et pour tout $i \in \mathbb{N}$, il existe $b_{i, 1}, \ldots, b_{i, m} \in
\mathbb{Z}$, tel que
\[ \beta^i = \underset{k = 1}{\overset{m}{\sum}} b_{i, k} l_k \]


Pour tout $x \in G_{\alpha \beta} : = \tmop{gr} \{ (\alpha \beta)^n |
\nobracket n \in \mathbb{N} \}$, on a l'existence de $c_0, \ldots, c_r \in
\mathbb{Z}$ tel que
\[ x = \underset{j = 0}{\overset{r}{\sum}} c_j (\alpha \beta)^j \]


On a alors
\begin{eqnarray*}
  x & = & \underset{j = 0}{\overset{r}{\sum}} c_j \alpha^j \beta^j\\
  & = & \underset{j = 0}{\overset{r}{\sum}} c_j \left( \underset{k =
  1}{\overset{n}{\sum}} a_{j, k} g_k \right) \left( \underset{i =
  1}{\overset{m}{\sum}} b_{j, i} l_i \right)\\
  & = & \underset{j = 0}{\overset{r}{\sum}} \underset{k =
  1}{\overset{n}{\sum}} \underset{i = 1}{\overset{m}{\sum}} c_j a_{j, k} b_{j,
  i} g_k l_i\\
  & = & \underset{k = 1}{\overset{n}{\sum}} \underset{i =
  1}{\overset{m}{\sum}} \left( \underset{j = 0}{\overset{r}{\sum}} c_j a_{j,
  k} b_{j, i} \right) g_k l_i
\end{eqnarray*}


Avec, pour tout $k \in \llbracket 1, n \rrbracket$ et $i \in \llbracket 1, m
\rrbracket$, $\underset{}{\overset{}{\underset{j = 0}{\overset{r}{\sum}} c_j
a_{j, k} b_{j, i} \in \mathbb{Z}}}$.

Donc la famille finie $(g_k l_i)_{\underset{1 \leqslant i \leqslant m}{1
\leqslant k \leqslant n}}$ est une famille g{\'e}n{\'e}ratrice de $G_{\alpha
\beta}$.

D'apr{\`e}s \tmtextbf{les questions $5. a$ et 5.$b$ de cette partie}, on
d{\'e}duit que $\alpha \beta \in \mathcal{D}_{\mathbb{C}}$.

Et pour tout $y \in G_{\alpha - \beta} : = \tmop{gr} \{ (\alpha - \beta)^n |
\nobracket n \in \mathbb{N} \}$, on a l'existence de $c_0 ; \ldots ; c_r \in
\mathbb{Z}$ tel que
\[ y = \underset{j = 0}{\overset{r}{\sum}} c_j (\alpha - \beta)^j \]


On a alors
\begin{eqnarray*}
  y & = & \underset{j = 0}{\overset{r}{\sum}} c_j (\alpha - \beta)^j\\
  & = & \underset{j = 0}{\overset{r}{\sum}} \underset{k =
  0}{\overset{j}{\sum}} \left( \begin{array}{c}
    j\\
    k
  \end{array} \right) c_j (- 1)^j \alpha^{j - k} \beta^k\\
  & = & \underset{j = 0}{\overset{r}{\sum}} \underset{k =
  0}{\overset{j}{\sum}} \left( \begin{array}{c}
    j\\
    k
  \end{array} \right) c_j (- 1)^j \left( \underset{i = 1}{\overset{n}{\sum}}
  a_{j - k, i} g_i \right) \left( \underset{s = 1}{\overset{m}{\sum}} b_{k, s}
  l_s \right)\\
  & = & \underset{j = 0}{\overset{r}{\sum}} \underset{k =
  0}{\overset{j}{\sum}} \underset{i = 1}{\overset{n}{\sum}} \underset{s =
  1}{\overset{m}{\sum}} \left( \begin{array}{c}
    j\\
    k
  \end{array} \right) c_j (- 1)^j a_{j - k, i} b_{k, s} g_i l_s\\
  & = & \underset{i = 1}{\overset{n}{\sum}} \underset{s =
  1}{\overset{m}{\sum}} \left( \underset{j = 0}{\overset{r}{\sum}} \underset{k
  = 0}{\overset{j}{\sum}} \left( \begin{array}{c}
    j\\
    k
  \end{array} \right) c_j (- 1)^j a_{j - k, i} b_{k, s} \right) g_i l_s
\end{eqnarray*}


Avec, pour tout $(i, s) \in \llbracket 1, n \rrbracket \times \llbracket 1, m
\rrbracket$, on a $\underset{j = 0}{\overset{r}{\sum}} \underset{k =
0}{\overset{j}{\sum}} \left( \begin{array}{c}
  j\\
  k
\end{array} \right) c_j (- 1)^j a_{j - k, i} b_{k, s} \in \mathbb{Z}$.

Alors, la famille finie $(g_i l_s)_{\underset{1 \leqslant s \leqslant m}{1
\leqslant i \leqslant n}}$ est g{\'e}n{\'e}ratrice de $G_{\alpha - \beta}$.

D'o{\`u}, d'apr{\`e}s ce qui pr{\'e}c{\`e}de, $\alpha - \beta \in
\mathcal{D}_{\mathbb{C}}$.

Par suite, $\mathcal{D}_{\mathbb{C}}$ est un sous anneau de $\mathbb{C}$.

\

\tmtextbf{7.} Montrons que $\mathcal{D}_{\mathbb{C}} \cap
\mathbb{Q}=\mathbb{Z}$.

Tout d'abord pour tout $z \in \mathbb{Z}$, on a $X - z \in \mathbb{Z} [X]$
annule $z$, donc $z \in \mathcal{D}_{\mathbb{C}}$, et de plus $z \in
\mathbb{Q}$.

Ainsi, $z \in \mathcal{D}_{\mathbb{C}} \cap \mathbb{Q}$. Par suite,
$\mathbb{Z} \subseteqq \mathcal{D}_{\mathbb{C}} \cap \mathbb{Q}$.

R{\'e}ciproquement, pour tout $z \in \mathcal{D}_{\mathbb{C}} \cap
\mathbb{Q}$.

On a $X - z \in \mathbb{Q} [X]$ annule $z$. Donc $\pi_z | X - z \nobracket$,
avec $\deg (\pi_z) \geqslant 1$. Alors, $\pi_z = X - z$.

Or, $z$ est un entier alg{\'e}brique, donc
\[ \pi_z = X - z \in \mathbb{Z} [X] \]


On en d{\'e}duit que $z \in \mathbb{Z}$.

D'o{\`u} $\mathcal{D}_{\mathbb{C}} \cap \mathbb{Q} \subseteq \mathbb{Z}$.

Enfin, $\mathcal{D}_{\mathbb{C}} \cap \mathbb{Q}=\mathbb{Z}$.

\

\subsubsection*{III Le corps $\mathbb{Q}(\zeta)$ et son anneau d'entiers}

\

\tmtextbf{1.a. }Montrons que les morphismes de $\mathbb{Q}$-alg{\'e}bre de
$\mathbb{Q} (\zeta)$ sont les $\sigma_k$, tels que $\sigma_k (\zeta) =
\zeta^k$, pour tout $k \in \{ 1, \ldots, p - 1 \}$.

Puisque $p$ est premier, donc $\pi_{\zeta} = \Phi_p$, dont les racines sont
$\zeta, \zeta^2, \ldots, \zeta^{p - 1}$.

D'apr{\`e}s les questions $\tmmathbf{2. a}$ et \tmtextbf{$2. b$ de la partie
$\mathbb{I}$}, il existe exactement $(p - 1)$ morphismes  de
$\mathbb{Q}$-alg{\'e}bre $\sigma_k : K =\mathbb{Q} (\zeta) \rightarrow
\mathbb{C}$ tels que pour tout $k \in \{ 1, \ldots, p - 1 \}$, $\sigma_k
(\zeta)$ soit racine de $\Phi_p$ et $\sigma_1 (\zeta), \ldots, \sigma_{p - 1}
(\zeta)$ soient deux {\`a} deux distinctes.

Quitte {\`a} r{\'e}ordonner les $\sigma_1, \ldots, \sigma_{p - 1}$. On conclut
que les les morphismes de $\mathbb{Q}$-alg{\`e}bre de $\mathbb{Q} (\zeta)$
sont les $\sigma_k$ tels que $\sigma_k (\zeta) = \zeta^k$ pour tout $k \in \{
1, 2, \ldots, p - 1 \}$.

\

\tmtextbf{1.b.i.} On a
\begin{eqnarray*}
  N (\zeta) & = & \underset{k = 1}{\overset{p - 1}{\prod}} \sigma_k (\zeta)\\
  & = & \underset{k = 1}{\overset{p - 1}{\prod}} \zeta^k\\
  & = & \zeta^{p \frac{p - 1}{2}}\\
  & = & \exp \left( 2 i \pi \frac{p - 1}{2} \right)
\end{eqnarray*}


Avec $p$ est impair, alors $\frac{p - 1}{2} \in \mathbb{N}$, donc $N (\zeta) =
1$.

Et
\begin{eqnarray*}
  \tmop{Tr} (\zeta) & = & \underset{k = 1}{\overset{p - 1}{\sum}} \sigma_k
  (\zeta)\\
  & = & \underset{k = 1}{\overset{p - 1}{\sum}} \zeta^k\\
  & = & \zeta \frac{1 - \zeta^{p - 1}}{1 - \zeta}\\
  & = & \frac{\zeta - 1}{1 - \zeta}\\
  & = & - 1
\end{eqnarray*}


\tmtextbf{1.b.ii.} Montrons que $N (1 - \zeta) = p$ et $N (1 + \zeta) = 1$

Notons
\begin{eqnarray*}
  P & = & \Phi_p (X + 1)\\
  & = & \overset{p - 1}{\underset{k = 0}{\sum}} (X + 1)^k\\
  & = & \frac{(X + 1)^p - 1}{(X + 1) - 1}\\
  & = & \underset{k = 1}{\overset{p}{\sum}} \left( \begin{array}{c}
    p\\
    k
  \end{array} \right) X^{k - 1}\\
  & = & \underset{k = 0}{\overset{p - 1}{\sum}} \left( \begin{array}{c}
    p\\
    k + 1
  \end{array} \right) X^k
\end{eqnarray*}


Avec
\[ P = \underset{k = 1}{\overset{p - 1}{\prod}} (X - (\zeta^k - 1)) \]


D'apr{\`e}s les identit{\'e}s de Newton des polyn{\^o}mes sym{\'e}triques, on
a
\[ (- 1)^{p - 1} \prod^{p - 1}_{k = 1} (\zeta^k - 1) = \left( \begin{array}{c}
     p\\
     1
   \end{array} \right) = p \]


Donc,
\[ \prod^{p - 1}_{k = 1} (1 - \zeta^k) = p \]


Ainsi,
\begin{eqnarray*}
  N (1 - \zeta) & = & \prod^{p - 1}_{k = 1} (1 - \sigma_k (\zeta))\\
  & = & \prod^{p - 1}_{k = 1} (1 - \zeta^k)\\
  & = & p
\end{eqnarray*}


Notons
\[ Q = \Phi_p (X - 1) \]


D'autre part, on a
\begin{eqnarray*}
  Q & = & \underset{k = 0}{\overset{p - 1}{\sum}} (X - 1)^k\\
  & = & \underset{k = 0}{\overset{p - 1}{\sum}} \underset{i =
  0}{\overset{k}{\sum}} \left( \begin{array}{c}
    k\\
    i
  \end{array} \right) (- 1)^{k - i} X^i\\
  & = & \underset{i = 0}{\overset{p - 1}{\sum}} \left[ \underset{k =
  i}{\overset{p - 1}{\sum}} \left( \begin{array}{c}
    k\\
    i
  \end{array} \right) (- 1)^{k - i} \right] X^i
\end{eqnarray*}


Avec
\[ Q = \underset{k = 1}{\overset{p - 1}{\prod}} (X - (\zeta^k + 1)) \]


Donc, d'apr{\`e}s les identit{\'e}s de Newton, on a
\begin{eqnarray*}
  (- 1)^{p - 1} \underset{k = 1}{\overset{p - 1}{\prod}} (\zeta^k + 1) & = &
  \underset{k = 0}{\overset{p - 1}{\sum}} \left( \begin{array}{c}
    k\\
    0
  \end{array} \right) (- 1)^k\\
  & = & \underset{k = 0}{\overset{p - 1}{\sum}} (- 1)^k\\
  & = & 1
\end{eqnarray*}


Ainsi,
\begin{eqnarray*}
  N (1 + \zeta) & = & \underset{k = 1}{\overset{p - 1}{\prod}} (\zeta^k + 1)\\
  & = & 1
\end{eqnarray*}


\tmtextbf{2.} Montrons que $\mathbb{Z} [\zeta] \subseteq \mathcal{D}_K$

Soit $x \in \mathbb{Z} [\zeta]$,

Alors il existe $(a_0, \ldots, a_n) \in \mathbb{Z}^{n + 1}$, tel que $x =
\underset{k = 0}{\overset{n}{\sum}} a_k \zeta^k$.

Tout d'abord, remarquons que $x \in \mathbb{Q} (\zeta) = K$.

D'autre part, en utilisant la division euclidienne de $P = \underset{k =
0}{\overset{n}{\sum}} a_k X^k$ par $\pi_{\zeta}$, et la question $1$ de la
partie $I$, on a l'existence de $Q, R \in \mathbb{Z} [X]$ tels que
\[ \left\{\begin{array}{l}
     P = Q \pi_{\zeta} + R\\
     \deg (R) < \deg \left( {\pi_{\zeta}}  \right) = \deg (\Phi_p) = p - 1
   \end{array}\right. \]


En {\'e}crivant $R = \underset{k = 0}{\overset{p - 2}{\sum}} r_k X^k$ o{\`u}
$r_0, \ldots, r_{p - 2} \in \mathbb{Z}$

On a alors
\begin{eqnarray*}
  x & = & P (\zeta)\\
  & = & Q (\zeta) \pi_{\zeta} (\zeta) + R (\zeta)\\
  & = & \underset{k = 0}{\overset{p - 2}{\sum}} r_k \zeta^k
\end{eqnarray*}


Or, $\zeta \in \mathcal{D}_{\mathbb{C}}$ (car $\Phi_p \in \mathbb{Z} [X]$
annule $\zeta$), et $\mathcal{D}_{\mathbb{C}}$ est un sous-anneau de
$\mathbb{C}$.

Ainsi,
\[ x = \underset{k = 0}{\overset{p - 2}{\sum}} r_k \zeta^k \in
   \mathcal{D}_{\mathbb{C}} \]


D'o{\`u}
\[ \mathbb{Z} [\zeta] \subseteq \mathcal{D}_{\mathbb{C}} \]


Ainsi,
\[ \mathbb{Z} [\zeta] \subseteq \mathcal{D}_{\mathbb{C}} \cap K =\mathcal{D}_K
\]


D'o{\`u} le r{\'e}sultat.

\

\tmtextbf{3.} Soit $z \in \mathbb{Z} [\zeta]$

\tmtextbf{3.a.} Montrons que
\[ z \in \mathbb{Z} [\zeta]^{\times} \tmop{si} \infixand \tmop{seulement}
   \tmop{si} N (z) \in \{ - 1, 1 \} \]


$\Rightarrow$)Si $z \in \mathbb{Z} [\zeta]^{\times}$.

Alors, il existe $z' \in \mathbb{Z} [\zeta]$ tel que $z.z' = 1$.

\

\tmtextbf{Lemme 4.}

Soit $\theta \in \mathbb{Z} [\zeta]$, si $\theta$ est un entier
alg{\'e}brique, alors $N (\theta) \in \mathbb{Z}$.

\

\tmtextbf{Preuve du lemme 4. }

On a $\theta$ est un entier alg{\'e}brique ; en particulier, il est
alg{\'e}brique sur $\mathbb{Q}$.

D'apr{\`e}s \tmtextbf{la question 3.b de la partie $\mathbb{I}$}, on a
\[ P_{\theta} = \underset{k = 1}{\overset{p - 1}{\prod}} (X - \sigma_k
   (\theta)) \in \mathbb{Q} [X] \]


En particulier :
\[ N (\theta) = \underset{k = 1}{\overset{p - 1}{\prod}} \sigma_k (\theta) \in
   \mathbb{Q} \]


En {\'e}crivant $\theta = P (\zeta)$, avec $P \in \mathbb{Z} [X]$, on obtient
alors :
\begin{eqnarray*}
  N (\theta) & = & \underset{k = 1}{\overset{p - 1}{\prod}} \sigma_k
  (\theta)\\
  & = & \underset{k = 1}{\overset{p - 1}{\prod}} P (\zeta^k)
\end{eqnarray*}


Comme $\zeta$ est un entier alg{\'e}brique, donc pour tout $k \in \llbracket
1, p - 1 \rrbracket$, on a $P (\zeta^k)$ est un entier alg{\'e}brique.
D'o{\`u} :
\[ \underset{k = 1}{\overset{p - 1}{\prod}} P (\zeta^k) \in
   \mathcal{D}_{\mathbb{C}} \]


Ainsi :
\[ \underset{k = 1}{\overset{p - 1}{\prod}} P (\zeta^k) \in
   \mathcal{D}_{\mathbb{C}} \cap \mathbb{Q}=\mathbb{Z} \]


D'o{\`u} :
\[ N (\theta) = \underset{k = 1}{\overset{p - 1}{\prod}} P (\zeta^k) \in
   \mathbb{Z} \]


Puisque $z.z' = 1$, alors :
\[ N (z.z') = N (1) = \underset{k = 1}{\overset{p - 1}{\prod}} \sigma_k (1) =
   1 \]


Or,
\begin{eqnarray*}
  N (z.z') & = & \underset{k = 1}{\overset{p - 1}{\prod}} \sigma_k (z.z')\\
  & = & \underset{k = 1}{\overset{p - 1}{\prod}} \sigma_k (z) \sigma_k (z')\\
  & = & \left( \underset{k = 1}{\overset{p - 1}{\prod}} \sigma_k (z) \right)
  \left( \underset{k = 1}{\overset{p - 1}{\prod}} \sigma_k (z') \right)\\
  & = & N (z) N (z')
\end{eqnarray*}


Donc :
\[ N (z) N (z') = 1 \]


Avec $N (z), N (z') \in \mathbb{Z}$ (d'apr{\`e}s le lemme 4), donc $N (z) \in
\{ - 1, 1 \}$.

\

$\Leftarrow$)R{\'e}ciproquement, si $N (z) \in \{ - 1, 1 \}$, alors
\begin{eqnarray*}
  N (z) & = & \underset{k = 1}{\overset{p - 1}{\prod}} \sigma_k \left(
  \underset{j = 0}{\overset{n}{\sum}} a_j \zeta^j \right)\\
  & = & \underset{k = 1}{\overset{p - 1}{\prod}} \left( \underset{j =
  0}{\overset{n}{\sum}} a_j \sigma_k (\zeta)^j \right)\\
  & = & \underset{k = 1}{\overset{p - 1}{\prod}} \left( \underset{j =
  0}{\overset{n}{\sum}} a_j \zeta^{j k} \right)\\
  & = & \underset{k = 1}{\overset{p - 1}{\prod}} P (\zeta^k)
\end{eqnarray*}


D'o{\`u},
\[ \underset{k = 1}{\overset{p - 1}{\prod}} P (\zeta^k) \in \{ - 1, 1 \} \]


Ainsi,
\[ z \times \underset{k = 2}{\overset{p - 1}{\prod}} P (\zeta^k) \in \{ - 1, 1
   \} \]


Avec $\underset{k = 2}{\overset{p - 1}{\prod}} P (\zeta^k) \in \mathbb{Z}
[\zeta]$.

Donc $z \in \mathbb{Z} [\zeta]^{\times}$.

D'o{\`u} l'{\'e}quivalence.

\

\tmtextbf{3.b.} Si $N (z)$ est un nombre premier.

Montrons que $z$ est irr{\'e}ductible.

Soient $a, b \in \mathbb{Z} [\zeta]$ tels que $z = a.b$

On a alors :
\begin{eqnarray*}
  N (z) & = & N (a b)\\
  & = & N (a) N (b)
\end{eqnarray*}


D'apr{\`e}s le lemme 4, on a
\[ N (a), N (b) \in \mathbb{Z} \]


Or, $N (z)$ est un nombre premier.

Ainsi, $N (a) \in \{ - 1, 1 \}$ ou $N (b) \in \{ - 1, 1 \}$.

Via la question pr{\'e}c{\'e}dente, on en d{\'e}duit que $a \in \mathbb{Z}
[\zeta]^{\times}$ ou $b \in \mathbb{Z} [\zeta]^{\times}$.

Ainsi, $z$ est irr{\'e}ductible de $\mathbb{Z} [\zeta]$.

\

\tmtextbf{4.a.} Justifions que $G$ est un groupe fini cyclique.

Par d{\'e}finition, $G$ est l'ensemble des racines de l'unit{\'e} contenues
dans $K$, o{\`u} $K$est un corps. Alors $1 \in G$.

Soient $z, z'$ deux racines de l'unit{\'e} dans $K$, Puisque $K$ est un
corps, alors $z \times \frac{1}{z'}$ est {\'e}galement une racine de
l'unit{\'e} contenue dans $K$.

Ainsi
\[ z \times \frac{1}{z'} \in G \]


Cela prouve que $G$ est un groupe.

De plus, $K /\mathbb{Q}$ est une extension finie. D'apr{\`e}s la question
\tmtextbf{1.b de la partie 2}, on a $K$ contient un nombre fini de racines de
l'unit{\'e}.

Donc, $G$ est un groupe fini ; notons $n =\#G$.

On a alors, pour tout $z \in G$, $z^n = 1$.

Ainsi, $G$ est un sous-groupe de $(\mathbb{U}_n, \times)$, qui est
monog{\`e}ne. Par cons{\'e}quent, $G$ est {\'e}galement monog{\`e}ne.

On en d{\'e}duit que $G$ est cyclique.

D'o{\`u} le r{\'e}sultat.

\

\tmtextbf{4.b.} Soit $\omega$ un g{\'e}n{\'e}rateur de $G$. Justifions que $2
p | n$et que$\mathbb{Q}(\zeta) =\mathbb{Q}(\omega)$.

On a $\omega \in G$, donc $| w | = 1$, et il existe $a_0, \ldots, a_{p - 1}
\in \mathbb{Q}$ tels que
\[ \omega = \underset{k = 0}{\overset{p - 1}{\sum}} a_k \zeta^k \]


On a alors
\begin{eqnarray*}
  \omega^p & = & \left( \underset{k = 0}{\overset{p - 1}{\sum}} a_k \zeta^k
  \right)^p\\
  & = & \underset{i_0 + \cdots + i_{p - 1} = p}{\sum}  \underset{k =
  0}{\overset{p - 1}{\prod}} a_{i_k} \zeta^{i_k}\\
  & = & \underset{i_0 + \cdots + i_{p - 1} = p}{\sum}  \underset{k =
  0}{\overset{p - 1}{\prod}} a_{i_k} \zeta^{i_k}\\
  & = & \underset{i_0 + \cdots + i_{p - 1} = p}{\sum}  \left( \underset{k =
  0}{\overset{p - 1}{\prod}} a_{i_k} \right) {\zeta^{i_0 + \cdots + i_{p -
  1}}} \\
  & = & \underset{i_0 + \cdots + i_{p - 1} = p}{\sum}  \left( \underset{k =
  0}{\overset{p - 1}{\prod}} a_{i_k} \right)\\
  & \in & \mathbb{R}
\end{eqnarray*}


Ainsi, $\omega^p \in \mathbb{R}$ et $| \omega^p | = 1$, donc $\omega^p = \pm
1$, ce qui entra{\^i}ne que $\omega^{2 p} = 1$.

Par cons{\'e}quent, $2 p | n \nobracket$.

\

Montrons maintenant que $\mathbb{Q} (\omega) =\mathbb{Q} (\zeta)$.

On a $\omega \in \mathbb{Q} (\zeta)$, donc $\mathbb{Q} (\omega) \subset
\mathbb{Q} (\zeta)$.

Or, $\zeta \in G = < \omega >$, donc il existe $k \in \mathbb{N}$, tel que
$\zeta = \omega^k \in \mathbb{Q} (\omega)$.

Ainsi, $\mathbb{Q} (\zeta) \subset \mathbb{Q} (\omega)$

On en d{\'e}duit donc que
\[ \mathbb{Q} (\zeta) =\mathbb{Q} (\omega) \]


\tmtextbf{4.c.} Montrons que $2 p = n$.

On a forc{\'e}ment $G =\mathbb{U}_n$, ce qu'on va justifier dans un premier
temps.

On a $\tmop{ord} (G) = n$, donc pour tout $g \in G$, on a $g^n = 1$, ce qui
implique $G \subset \mathbb{U}_n$.

Or,
\[ \tmop{card} (G) = \tmop{card} (\mathbb{U}_n) = n < + \infty \]


D'o{\`u},
\[ G =\mathbb{U}_n \]


Avant de continuer, montrons un lemme :

\

\tmtextbf{Lemme 5.}

Soit $z \in \mathbb{C}$, on a alors
\[ [\mathbb{Q} (z) : \mathbb{Q}] = \deg (\pi_z) \]


\tmtextbf{Preuve du lemme 5.}

Soit $z \in \mathbb{C}$, rappelons que $\pi_z$ est irr{\'e}ductible de
$\mathbb{Q} [X]$.

Soit $x \in \mathbb{Q} (z)$, alors il existe $a_0, \ldots, a_r \in \mathbb{Q}$
tels que
\[ x = \underset{j = 0}{\overset{n}{\sum}} a_j z^j \]


Par division euclidienne de $P = \underset{j = 0}{\overset{n}{\sum}} a_j X^j$
par $\Phi_z$, on a l'existence de $(Q, R) \in \mathbb{Q} [X]^2$ tel que
\[ P = Q \pi_z + R \]


Alors
\begin{eqnarray*}
  x & = & P (z)\\
  & = & Q (z) \pi_z (z) + R (z)\\
  & = & R (z)\\
  & \in & \tmop{vect}_{\mathbb{Q}} (1, z, \ldots, z^{\deg (\pi_z) - 1})
\end{eqnarray*}


Ainsi, $(1, z, \ldots, z^{\deg (\pi_z) - 1})$ est g{\'e}n{\'e}ratrice de
$\mathbb{Q}- \tmop{espace}$ vectoriel $\mathbb{Q} (z)$.

Montrons que cette famille est $\mathbb{Q}-$libre.

Pour cela, consid{\'e}rons $b_0, \ldots, b_{\deg (\pi_z) - 1} \in \mathbb{Q}$,
tels que
\[ \underset{k = 0}{\overset{\deg (\pi_z) - 1}{\sum}} b_k z^k = 0 \]


Alors, $\pi_z$ divise $\underset{k = 0}{\overset{\deg (\pi_z) - 1}{\sum}} b_k
z^k$, donc
\begin{eqnarray*}
  \deg (\pi_z) & \leqslant & \deg \left( \underset{k = 0}{\overset{\deg
  (\pi_z) - 1}{\sum}} b_k z^k \right)\\
  & = & \deg (\pi_z) - 1
\end{eqnarray*}


absurde!

\

D'o{\`u}, $(1, z, \ldots, z^{\deg (\pi_z) - 1})$ est $\mathbb{Q}$-libre.

Ensuite, $(1, z, \ldots, z^{\deg (\pi_z) - 1})$ est une base de $\mathbb{Q}
(z)$ en tant que $\mathbb{Q}$-espace vectoriel.

Ainsi,
\begin{eqnarray*}
  {}[\mathbb{Q} (z) : \mathbb{Q}] & = & \# \{ 1, z, \ldots, z^{\deg (\pi_z) -
  1} \}\\
  & = & \deg (\pi_z)
\end{eqnarray*}


D'o{\`u} le r{\'e}sultat.

\

On a $\omega$ est un g{\'e}n{\'e}rateur de $G$, donc $\omega$ est une racine
primitive de $n$.

En utilisant le lemme pr{\'e}c{\'e}dent, on a
\begin{eqnarray*}
  {}[\mathbb{Q} (\omega) : \mathbb{Q}] & = & \deg (\pi_{\omega})\\
  & = & \deg (\Phi_n)\\
  & = & \varphi (n)
\end{eqnarray*}


Or, d'apr{\`e}s la question pr{\'e}c{\'e}dente, on a $2 p | n \nobracket$,
donc il existe $k \in \mathbb{N}^{\ast}$ tel que $n = 2 k p$.

Si $k$ s'{\'e}crit sous la forme $k = 2^a p^b$, o{\`u} $a, b \in \mathbb{N}$
non tous nuls. On a alors
\begin{eqnarray*}
  \varphi (n) & = & \varphi (2^{a + 1} p^{b + 1})\\
  & = & 2^{a + 1} p^{b + 1} \left( 1 - \frac{1}{2} \right) \left( 1 -
  \frac{1}{p} \right)\\
  & = & 2^a p^b (p - 1)\\
  & \geqslant & \min (2 (p - 1), p (p - 1))\\
  & > & p
\end{eqnarray*}


Donc
\begin{eqnarray*}
  p & = & [\mathbb{Q} (\zeta) : Q]\\
  & = & [\mathbb{Q} (\omega) : \mathbb{Q}]\\
  & = & \varphi (n)\\
  & > & p
\end{eqnarray*}


ce qui est absurde !

\

Sinon, si $k$ est de la forme
\[ k = 2^a p^b \underset{i = 1}{\overset{l}{\prod}} p^{y_i}_i \]


o{\`u} $a, b \in \mathbb{N}$ et $y_1, \ldots, y_l > 1$, et $p_1, \ldots, p_l
\geqslant 3$. On a alors
\begin{eqnarray*}
  p & = & [\mathbb{Q} (\zeta) : \mathbb{Q}]\\
  & = & [\mathbb{Q} (\omega) : \mathbb{Q}]\\
  & = & \varphi (n)\\
  & = & \varphi \left( 2^{a + 1} p^{b + 1} \underset{i =
  1}{\overset{l}{\prod}} p^{y_i}_i \right)\\
  & = & \tmmathbf{} 2^{a + 1} p^{b + 1} \underset{i = 1}{\overset{l}{\prod}}
  p^{y_i}_i \left( 1 - \frac{1}{2} \right) \left( 1 - \frac{1}{p} \right)
  \underset{i = 1}{\overset{l}{\prod}} \left( 1 - \frac{1}{p_i} \right)\\
  & = & 2^a p^b \underset{i = 1}{\overset{l}{\prod}} p^{y_i - 1}_i (p_i - 1)
  (p - 1)\\
  & \geqslant & 2 (p - 1)\\
  & > & p
\end{eqnarray*}


absurde!

D'o{\`u} $k = 1$, par suite $n = 2 p$.

Ainsi,
\begin{eqnarray*}
  G & = & \mathbb{U}_{2 p}\\
  & = & \left\{ \exp \left( \frac{2 i k \pi}{2 p} \right) / k \in \llbracket
  0, 2 p - 1 \rrbracket \right\}\\
  & = & \{ \pm \zeta^k / k \in \llbracket 0, p - 1 \rrbracket \}
\end{eqnarray*}


D'o{\`u} le r{\'e}sultat.

\

\tmtextbf{5.a.} Montrons que
\[ < \lambda > \cap \mathbb{Z}= p\mathbb{Z} \]


On a
\[ < \lambda > = \lambda \mathbb{Z} [\zeta] \]


Avec $0 = \lambda \times 0 \in \lambda \mathbb{Z} [\zeta] = < \lambda >$. Donc
$< \lambda > \neq \emptyset$.

De plus, pour tout $x, y \in < \lambda >$, on a l'existence de $P, Q \in
\mathbb{Z} [X]$ tels que $x = \lambda P (\zeta)$ et $y = \lambda Q (\zeta)$.

Donc
\begin{eqnarray*}
  x - y & = & \lambda (P - Q) (\zeta)\\
  & \in & \lambda \mathbb{Z} [\zeta]\\
  & = & < \lambda >
\end{eqnarray*}


Ainsi, $< \lambda >$ est un sous-groupe de $\mathbb{C}$.

Comme $\mathbb{Z}$ est un sous-groupe de $\mathbb{C}$, donc l'intersection $<
\lambda > \cap \mathbb{Z}$ est un sous-groupe de $\mathbb{C}$.

{\'E}tant donn{\'e} que $< \lambda > \cap \mathbb{Z} \subseteq \mathbb{Z}$,
alors $< \lambda > \cap \mathbb{Z}$ est un sous-groupe de $\mathbb{Z}$.

Il existe donc $n \in \mathbb{N}$ tel que
\[ < \lambda > \cap \mathbb{Z}= n\mathbb{Z} \]


Or, d'apr{\`e}s la question $b. \tmop{ii} .$ de cette partie, on a
\begin{eqnarray*}
  p & = & N (\lambda)\\
  & = & N (1 - \zeta)\\
  & = & \underset{k = 1}{\overset{p - 1}{\prod}} \sigma_k (1 - \zeta)\\
  & = & \underset{k = 1}{\overset{p - 1}{\prod}} (1 - \zeta^k)\\
  & = & \lambda \underset{k = 2}{\overset{p - 1}{\prod}} (1 - \zeta^k)
\end{eqnarray*}


Avec $\underset{k = 2}{\overset{p - 1}{\prod}} (1 - \zeta^k) \in \mathbb{Z}
[\zeta]$

Donc
\[ p \in < \lambda > \cap \mathbb{Z}= n\mathbb{Z} \]


En particulier, $n \neq 0$.

De plus $p \in n\mathbb{Z}$, il existe donc $r \in \mathbb{N}$ tel que $p = n
r$.

Puisque $p$ est premier, alors $n = 1$ ou $n = p$.

Si $n = 1$, donc $< \lambda > \cap \mathbb{Z}=\mathbb{Z}$.

En particulier $1 \in < \lambda >$.

Donc, il existe $x \in \mathbb{Z} [\zeta]$ tel que
\[ 1 = \lambda x \]


Cela impliquerait que $\lambda \in \mathbb{Z} [\zeta]^{\times}$, ce qui est
absurde car $N (\lambda) = p \nin \{ - 1, 1 \}$.

Donc $n = p$.

En conclusion,
\[ < \lambda > \cap \mathbb{Z}= p\mathbb{Z} \]


\tmtextbf{5.b.} Soit $K \in \{ 1, 2, \ldots, p - 1 \}$. Montrons que
\[ \frac{1 - \zeta}{1 - \zeta^k} \in \mathbb{Z} [\zeta]^{\times} \]


{\tmname{\tmtextbf{M{\'e}thode 1.}}}

On a
\[ \frac{1 - \zeta^k}{1 - \zeta} = \underset{j = 0}{\overset{k - 1}{\sum}}
   \zeta^j \in \mathbb{Z} [\zeta] \]


Et
\begin{eqnarray*}
  N \left( \frac{1 - \zeta^k}{1 - \zeta} \right) & = & N \left( \underset{j =
  0}{\overset{k - 1}{\sum}} \zeta^j \right)\\
  & = & \underset{l = 1}{\overset{p - 1}{\prod}} \sigma_l \left( \underset{j
  = 0}{\overset{k - 1}{\sum}} \zeta^j \right)\\
  & = & \underset{l = 1}{\overset{p - 1}{\prod}}  \left( \underset{j =
  0}{\overset{k - 1}{\sum}} \sigma_l (\zeta)^j \right)\\
  & = & \underset{l = 1}{\overset{p - 1}{\prod}}  \left( \underset{j =
  0}{\overset{k - 1}{\sum}} \zeta^{j l} \right)\\
  & = & \underset{l = 1}{\overset{p - 1}{\prod}}  \left( \frac{1 - \zeta^{k
  l}}{1 - \zeta^l} \right)
\end{eqnarray*}


Pour tous $k, l \in \llbracket 1, p - 1 \rrbracket$, notons $r_{k, l}$
l'unique entier dans $\llbracket 0, p - 1 \rrbracket$ qui repr{\'e}sente le
reste de la division euclidienne de $k l$ par $p$.

On a pour tout $k, l \in \llbracket 1, p - 1 \rrbracket$ :
\[ k \wedge p = 1 \infixand l \wedge p = 1 \]


Donc $k l \wedge p = 1$, ainsi $r_{k l} \in \llbracket 1, p - 1 \rrbracket$

Donc l'application
\[ \begin{array}{lll}
     l \in \llbracket 1, p - 1 \rrbracket & \longmapsto & r_{k, l} \in
     \llbracket 1, p - 1 \rrbracket
   \end{array} \]


est bien d{\'e}finie. De plus, pour tous $l, l' \in \llbracket 1, p - 1
\rrbracket$, si $r_{k, l} = r_{k, l'}$, alors
\[ k (l - l') \tmop{est} \tmop{divisible} \tmop{par} p \]


Avec $p \wedge k = 1$, on en d{\'e}duit via le lemme de Gauss que
\[ P | l - l' \nobracket \]


Donc, il existe $s \in \mathbb{Z}$ tel que $l - l' = s.p$.

Or,
\[ - (p - 1) \leqslant l - l' \leqslant p - 1 \]


Donc,
\[ - \frac{p - 1}{p} \leqslant s \leqslant \frac{p - 1}{p} \]


Ainsi, $s = 0$, par suite $l = l'$. D'o{\`u} l'application$\begin{array}{lll}
  l \in \llbracket 1, p - 1 \rrbracket & \longmapsto & r_{k, l} \in \llbracket
  1, p - 1 \rrbracket
\end{array}$est injective, donc elle est bijective.

D'o{\`u},
\begin{eqnarray*}
  \underset{l = 1}{\overset{p - 1}{\prod}}  (1 - \zeta^{k l}) & = &
  \underset{l = 1}{\overset{p - 1}{\prod}}  (1 - \zeta^{r_{k, l}})\\
  & = & \underset{l = 1}{\overset{p - 1}{\prod}}  (1 - \zeta^l)
\end{eqnarray*}


D'o{\`u},
\begin{eqnarray*}
  N \left( \frac{1 - \zeta^k}{1 - \zeta} \right) & = & \frac{\underset{l =
  1}{\overset{p - 1}{\prod}}  (1 - \zeta^l)}{\underset{l = 1}{\overset{p -
  1}{\prod}}  (1 - \zeta^l)}\\
  & = & 1
\end{eqnarray*}


Finalement,
\[ \frac{1 - \zeta^k}{1 - \zeta} \in \mathbb{Z} [\zeta]^{\times} \]


D'o{\`u}
\[ \frac{1 - \zeta }{1 - \zeta^k} = \frac{1}{\frac{1 - \zeta^k}{1 - \zeta}}
   \in \mathbb{Z} [\zeta]^{\times} \]


{\tmname{\tmtextbf{M{\'e}thode 2.}}}

On a $p \wedge k = 1$, donc d'apr{\`e}s le th{\'e}or{\`e}me de B{\'e}zout, il
existe $n_0, m_0 \in \mathbb{Z}$ tel que
\[ p n_0 + k m_0 = 1 \]


Donc, pour tout $\alpha \in \mathbb{Z}$
\[ p (n_0 - \alpha k) + k (m_0 + \alpha p) = 1 \]


Avec $\underset{\alpha \rightarrow + \infty}{\lim} m_0 + \alpha p = + \infty$,
alors il existe $\alpha_0 \in \mathbb{N}$ tel que
\[ m_0 + \alpha_0 p > 0 \]


Pour ce $\alpha_0 \in \mathbb{N}$, on a
\begin{eqnarray*}
  \frac{1 - \zeta}{1 - \zeta^k} & = & \frac{1 - \zeta^{1 - p (n_0 - \alpha_0
  k)}}{1 - \zeta^k}\\
  & = & \frac{1 - \left( {\zeta^k}  \right)^{m_0 + \alpha_0 p}}{1 -
  \zeta^k}\\
  & = & \underset{j = 0}{\overset{m_0 + \alpha_0 p - 1}{\sum}} \zeta^{k j}\\
  & \in & \mathbb{Z} [\zeta]
\end{eqnarray*}


D'autre part,
\begin{eqnarray*}
  \frac{1}{\frac{1 - \zeta }{1 - \zeta^k}} & = & \frac{1 - \zeta^k}{1 -
  \zeta}\\
  & = & \underset{j = 0}{\overset{k - 1}{\sum}} \zeta^j\\
  & \in & \mathbb{Z} [\zeta]
\end{eqnarray*}


Donc, par d{\'e}finition,
\[ \frac{1 - \zeta}{1 - \zeta^k} \in \mathbb{Z} [\zeta]^{\times} \]


Montrons que
\[ \lambda^{p - 1} \mathbb{Z} [\zeta] = p\mathbb{Z} [\zeta] \]


Soit $x \in p\mathbb{Z} [\zeta]$, donc il existe $P \in \mathbb{Z} [X]$ tel
que $x = p P (\zeta)$.

Donc
\begin{eqnarray*}
  x & = & N (1 - \zeta) P (\zeta)\\
  & = & \underset{k = 1}{\overset{p - 1}{\prod}} \sigma_k (1 - \zeta ) P
  (\zeta)\\
  & = & \underset{k = 1}{\overset{p - 1}{\prod}} (1 - \zeta^k) P (\zeta)\\
  & = & \underset{k = 1}{\overset{p - 1}{\prod}} \left[ (1 - \zeta ) \left(
  \underset{j = 0}{\overset{k - 1}{\sum}} \zeta^j \right) \right] P (\zeta)\\
  & = & (1 - \zeta)^{p - 1} \left[ \underset{k = 1}{\overset{p - 1}{\prod}}
  \left( \underset{j = 0}{\overset{k - 1}{\sum}} \zeta^j \right) \right] P
  (\zeta)\\
  & \in & \lambda^{p - 1} \mathbb{Z} [\zeta]
\end{eqnarray*}


D'o{\`u}
\[ p\mathbb{Z} [\zeta] \subset \lambda^{p - 1} \mathbb{Z} [\zeta] \]


R{\'e}ciproquement, on a
\begin{eqnarray*}
  \lambda^{p - 1} & = & (1 - \zeta)^{p - 1}\\
  & = & \underset{k = 1}{\overset{p - 1}{\prod}} \left( \frac{1 - \zeta}{1 -
  \zeta^k} \right) \underset{k = 1}{\overset{p - 1}{\prod}} (1 - \zeta^k)\\
  & = & N (1 - \zeta) \underset{k = 1}{\overset{p - 1}{\prod}} \left( \frac{1
  - \zeta}{1 - \zeta^k} \right)\\
  & = & p \underset{k = 1}{\overset{p - 1}{\prod}} \left( \frac{1 - \zeta}{1
  - \zeta^k} \right)
\end{eqnarray*}


Avec $\frac{1 - \zeta}{1 - \zeta^k} \in \mathbb{Z} [\zeta]^{\times}$ pour tout
$k \in \llbracket 1, p - 1 \rrbracket$.

Donc,
\[ \underset{k = 1}{\overset{p - 1}{\prod}} \left( \frac{1 - \zeta}{1 -
   \zeta^k} \right) \in \mathbb{Z} [\zeta]^{\times} \]


D'o{\`u},
\[ \lambda^{p - 1} \in p\mathbb{Z} [\zeta] \]


Ainsi,
\[ \lambda^{p - 1} \mathbb{Z} [\zeta] \subset p\mathbb{Z} [\zeta] \]


Enfin,
\[ \lambda^{p - 1} \mathbb{Z} [\zeta] = p\mathbb{Z} [\zeta] \]


\tmtextbf{5.c.} Soit $\Psi$: le morphisme d'anneaux de $\mathbb{Z} [X]
\rightarrow \mathbb{Z} [\zeta] / < \lambda >$.

Soit $P = \underset{k = 0}{\overset{n}{\sum}} a_k X^k \in \mathbb{Z} [X]$.

On a, pour tout $k \in \llbracket 0, n \rrbracket$ :
\begin{eqnarray*}
  \zeta^k & = & (\zeta - 1 + 1)^k\\
  & = & (1 - \lambda)^k\\
  & = & \underset{j = 0}{\overset{k}{\sum}} (- 1)^j \left( \begin{array}{c}
    k\\
    j
  \end{array} \right) \lambda^j\\
  & = & 1 + \underset{j = 1}{\overset{k}{\sum}} (- 1)^j \left(
  \begin{array}{c}
    k\\
    j
  \end{array} \right) \lambda^j
\end{eqnarray*}


Donc,
\begin{eqnarray*}
  P (\zeta) & = & \underset{k = 0}{\overset{n}{\sum}} a_k \zeta^k\\
  & = & \underset{k = 0}{\overset{n}{\sum}} a_k \left( 1 + \underset{j =
  1}{\overset{k}{\sum}} (- 1)^j \left( \begin{array}{c}
    k\\
    j
  \end{array} \right) \lambda^j \right)\\
  & = & \underset{k = 0}{\overset{n}{\sum}} a_k + \lambda \underset{k =
  0}{\overset{n}{\sum}} a_k \underset{j = 1}{\overset{k}{\sum}} (- 1)^j \left(
  \begin{array}{c}
    k\\
    j
  \end{array} \right) \lambda^{j - 1}
\end{eqnarray*}


Avec, $\underset{k = 0}{\overset{n}{\sum}} a_k \underset{j =
1}{\overset{k}{\sum}} (- 1)^j \left( \begin{array}{c}
  k\\
  j
\end{array} \right) \lambda^{j - 1} \in \mathbb{Z} [\zeta]$.

Donc,
\[ \underset{k = 0}{\overset{n}{\sum}} a_k \underset{j = 1}{\overset{k}{\sum}}
   (- 1)^j \left( \begin{array}{c}
     k\\
     j
   \end{array} \right) \lambda^{j - 1} \in < \lambda > \]


Et donc,
\begin{eqnarray*}
  P (\zeta) & = & \underset{k = 0}{\overset{n}{\sum}} a_k  (\tmop{mod} <
  \lambda >)\\
  & = & P (1)  (\tmop{mod} < \lambda >)
\end{eqnarray*}


Par division euclidienne de $P (1)$ par $p$, on a l'existence de $q, r \in
\mathbb{N}$ tels que
\[ \left\{\begin{array}{l}
     P (1) = p q + r\\
     r \in \llbracket 0, p - 1 \rrbracket
   \end{array}\right. \]


Or, $p \in p\mathbb{Z} [\zeta]$, donc $p \in \lambda^{p - 1} \mathbb{Z}
[\zeta]$.

Donc, il existe $Q \in \mathbb{Z} [X]$ tel que $p = \lambda^{p - 1} Q
(\zeta)$.

Ainsi,
\begin{eqnarray*}
  p & = & \lambda (1 - \zeta)^{p - 2} Q (\zeta)\\
  & = & \lambda \nobracket [(1 - X)^{p - 2} Q] |_{X = \zeta}\\
  & = & 0 (\tmop{mod} < \lambda >)
\end{eqnarray*}


Donc,
\[ P (1) = r (\tmop{mod} < \lambda >) \]


Ainsi,
\[ \Psi (P) = r (\tmop{mod} < \lambda >) \]


D'o{\`u},
\[ \Psi (P) = P (1)  (\tmop{mod} p\mathbb{Z}) \]


\

Donc l'image de $P$ par $\Psi$ est le reste de la division euclidienne de $P
(1) $par $p$.

\

Soit $P \in \tmop{Ker} (\Psi)$, alors d'apr{\`e}s ce qui pr{\'e}c{\`e}de, le
reste de la division euclidienne de $P (1) $par $p$ est nul.

Donc,
\[ P (1) = 0 (\tmop{mod} p\mathbb{Z}) \]


R{\'e}ciproquement, si $P \in \mathbb{Z} [X]$ tel que $P (1) = 0 (\tmop{mod}
p\mathbb{Z})$.

Alors, d'apr{\`e}s ce qui pr{\'e}c{\`e}de, on a
\begin{eqnarray*}
  \Psi (P) & = & 0 (\tmop{mod} p\mathbb{Z})\\
  & = & 0 (\tmop{mod} < \lambda >)
\end{eqnarray*}


D'o{\`u} $P \in \tmop{Ker} (\Psi)$

D'o{\`u} le r{\'e}sultat.

\

\tmtextbf{5.d.} D'apr{\`e}s ce qui pr{\'e}c{\`e}de, $\tmop{Im} (\Psi)$ est
isomorphe {\`a} $\mathbb{F}_p =\mathbb{Z}/ p\mathbb{Z}$.

Donc $\mathbb{Z} [\zeta] / < \lambda >$ est isomorphe {\`a} $\mathbb{F}_p
=\mathbb{Z}/ p\mathbb{Z}$.

\

\tmtextbf{5.e.} Puisque $\mathbb{Z} [\zeta] / < \lambda >$ est isomorphe
{\`a} $\mathbb{F}_p$, et que $p$ est un nombre premier.

Alors l'id{\'e}al $< \lambda >$ est premier, comme {\'e}tant un id{\'e}al de
l'anneau $\mathbb{Z} [\zeta]$.

\

\tmtextbf{6.a.} Soit $P = \underset{k = 0}{\overset{d}{\sum}} a_k X^k$ un
polyn{\^o}me unitaire de degr{\'e} $d$, dont on note $\alpha_1, . . .,
\alpha_d$ les racines complexes compt{\'e}es avec leur multiplicit{\'e}. On
suppose que pour tout $k \in \{1, . . ., d\}$, $\alpha_k$ est de module $1$.

\

\tmtextbf{6.a.i.} Soit $k \in \llbracket 0, d \rrbracket$, montrons que
\[ | a_k | \leqslant \left( \begin{array}{c}
     d\\
     k
   \end{array} \right) \]


On a, d'apr{\`e}s les identit{\'e}s de Newton
\[ a_k = \underset{1 \leqslant i_1 < \cdots < i_k \leqslant d}{\sum} 
   \underset{j = 1}{\overset{k}{\prod}} \alpha_{i_j} \]


Donc
\begin{eqnarray*}
  | a_k | & \leqslant & \underset{1 \leqslant i_1 < \cdots < i_k \leqslant
  d}{\sum}  \left| \underset{j = 1}{\overset{k}{\prod}} \alpha_{i_j} \right|\\
  & = & \underset{1 \leqslant i_1 < \cdots < i_k \leqslant d}{\sum} 1\\
  & = & \left( \begin{array}{c}
    d\\
    k
  \end{array} \right)
\end{eqnarray*}


Notons
\[ \mathcal{E}_d = \left\{ z \in \mathcal{D}_{\mathbb{C}}  | \nobracket \deg
   (\pi_z) = d, \infixand \tmop{les} \tmop{conjug} {\'e}s \tmop{de} z
   \tmop{sont} \tmop{tous} \tmop{de} \tmop{module} 1 \right\} \]


et
\[ \mathcal{O}_d = \{ P \in \mathbb{Z} [X]  | \nobracket P \tmop{de}
   \tmop{degr} {\'e} d \tmop{dont} \tmop{toutes} \tmop{les} \tmop{racines}
   \tmop{sont} \tmop{de} \tmop{module} 1 \} \]


On a
\[ \mathcal{E}_d \subset \underset{P \in \mathcal{O}_d}{\bigcup} P^{- 1} (\{ 0
   \}) \]


Or, un polyn{\^o}me de degr{\'e} $d$ admet au plus $d$ racines distincts dans
$\mathbb{C}$.

Donc, pour tout $P \in \mathcal{O}_d$, on a
\[ \#P^{- 1} (\{ 0 \}) \leqslant d \]


De plus
\[ \mathcal{O}_d \subset \left\{ \underset{k = 0}{\overset{d}{\sum}} a_k X^k
   \in \mathbb{Z} [X]  | \nobracket \forall k \in \llbracket 0, d \rrbracket,
   a_k \leqslant \left( \begin{array}{c}
     d\\
     k
   \end{array} \right) \right\} \]


Et la famille
\[ \left\{ \underset{k = 0}{\overset{d}{\sum}} a_k X^k \in \mathbb{Z} [X]  |
   \nobracket \forall k \in \llbracket 0, d \rrbracket, a_k \leqslant \left(
   \begin{array}{c}
     d\\
     k
   \end{array} \right) \right\} \]
\ est en bijection avec$\left\{ (a_0, \ldots, a_d) \in \mathbb{Z}^d  |
\nobracket \forall k \in \llbracket 0, d \rrbracket, a_k \leqslant \left(
\begin{array}{c}
  d\\
  k
\end{array} \right) \right\}$

\

Comme cet ensemble est fini, cela implique que
\[ \left\{ \underset{k = 0}{\overset{d}{\sum}} a_k X^k \in \mathbb{Z} [X]  |
   \nobracket \forall k \in \llbracket 0, d \rrbracket, a_k \leqslant \left(
   \begin{array}{c}
     d\\
     k
   \end{array} \right) \right\} \]


est fini aussi. Ainsi $\mathcal{O}_d$ est fini.

Par suite,
\begin{eqnarray*}
  \#\mathcal{E}_d & \leqslant & \# \underset{P \in \mathcal{O}_d}{\bigcup}
  P^{- 1} (\{ 0 \})\\
  & \leqslant & d \times \#\mathcal{O}_d\\
  & \leqslant & d \times \# \left\{ \underset{k = 0}{\overset{d}{\sum}} a_k
  X^k \in \mathbb{Z} [X]  | \nobracket \forall k \in \llbracket 0, d
  \rrbracket, a_k \leqslant \left( \begin{array}{c}
    d\\
    k
  \end{array} \right) \right\}\\
  & = & d \underset{k = 0}{\overset{d}{\prod}} \left( 2 \left(
  \begin{array}{c}
    d\\
    k
  \end{array} \right) + 1 \right)\\
  & < & + \infty
\end{eqnarray*}


D'o{\`u} le r{\'e}sultat.

\

\tmtextbf{6.a.ii.} On a $X^n \in \mathbb{Z} [X]$, et $P \in \mathbb{Z} [X]$
est un polyn{\^o}me unitaire de degr{\'e} $d$, dont on note $\alpha_1, . . .,
\alpha_d$ les racines complexes compt{\'e}es avec leur multiplicit{\'e}. Avec,
$\mathbb{Z}$ est un sous-anneau de $\mathbb{C}$.

D'apr{\`e}s \tmtextbf{la question 4.e de la partie $I$}, on a pour tout $n \in
\mathbb{N}$
\[ P_n = \underset{k = 1}{\overset{d}{\prod}} (X - \alpha_k^n) \in \mathbb{Z}
   [X] \]


Avec pour tout $(n, d) \in \mathbb{N} \times \llbracket 1, d \rrbracket$, on a
$| \alpha_k^n | = 1$, En utilisant les m{\^e}mes notations que
pr{\'e}c{\'e}demment, on a, pour tout $k \in \llbracket 1, d \rrbracket$, et
pour tout $n \in \mathbb{N}$
\[ \alpha_k^n \in \mathcal{E}_d \]


o{\`u} $\mathcal{E}_d$ est fini. Donc il existe $n_0, n_1 \in \mathbb{N}$,
tels que $n_0 < n_1$ et $\alpha_k^{n_0} = \alpha_k^{n_1}$.

Avec $\alpha_k  \neq 0$, donc
\[ \alpha_k^{n_1 - n_0} = 1 \]


On en d{\'e}duit que $\alpha_k $ est un racine de l'unit{\'e}

D'o{\`u} le r{\'e}sultat.

\

\tmtextbf{6.b.} Soit $P \in \mathbb{Z}[X]$ tel que $u = P (\zeta)$. Montrons
que, pour tout $k \in \{1, . . ., p \nonconverted{minus} 1\}$, $u_k = P
(\zeta^k)$ est un conjugu{\'e} de $u$ et que c'est un {\'e}l{\'e}ment de
$\mathbb{Z}[\zeta]^{\times}$.

On a, en utilisant les m{\^e}mes notations de la partie $\mathbb{I}$.
\begin{eqnarray*}
  \underset{k = 1}{\overset{p - 1}{\prod}} (X - P (\zeta^k)) & = & \underset{k
  = 1}{\overset{p - 1}{\prod}} (X - P (\sigma_k (\zeta)))\\
  & = & \underset{k = 1}{\overset{p - 1}{\prod}} (X - \sigma_k (P (\nobracket
  \zeta) \nobracket))\\
  & = & \underset{k = 1}{\overset{p - 1}{\prod}} (X - \sigma_k (u))\\
  & = & P_u
\end{eqnarray*}


D'apr{\`e}s la \tmtextbf{question 3.c de la partie $\mathbb{I}$}, on a $P_u$
est une puissance de ${\pi_u} $.

Donc, pour tout $k \in \{1, . . ., p \nonconverted{minus} 1\}$, $u_k = P
(\zeta^k)$ est un conjugu{\'e} de $u$.

De plus $u = P (\zeta) \in \mathbb{Z}[\zeta]^{\times}$.

Ainsi, d'apr{\`e}s la \tmtextbf{question 3.a de cette partie}, on a :
\[ N (u) \in \{ - 1, 1 \} \]


Donc, pour tout $k \in \{ 1, 2, \ldots, p - 1 \}$, on a :
\begin{eqnarray*}
  u_k \underset{j \neq k}{\overset{p - 1}{\underset{j = 1}{\prod}}} P
  (\zeta^j) & = & \underset{}{\overset{p - 1}{\underset{j = 1}{\prod}}} P
  (\zeta^j)\\
  & = & \underset{}{\overset{p - 1}{\underset{j = 1}{\prod}}} \sigma_j (P
  (\zeta ))\\
  & = & \underset{}{\overset{p - 1}{\underset{j = 1}{\prod}}} \sigma_j (u)\\
  & = & N (u)\\
  & \in & \{ - 1, 1 \}
\end{eqnarray*}


Avec $\underset{j \neq k}{\overset{p - 1}{\underset{j = 1}{\prod}}} P
(\zeta^j) \in Z [\zeta]$, donc par d{\'e}finition
\[ u_k \in \mathbb{Z} [\zeta]^{\times} \]


\tmtextbf{6.c.} Justifions que $\frac{u_p}{u_{p - 1}}$ est un entier
alg{\'e}brique dont tous les conjugu{\'e}s sont de module 1.

Soit $k \in \llbracket 1, p - 1 \rrbracket$,

On a :
\begin{eqnarray*}
  u_{p - k} & = & P (\zeta^{p - k})\\
  & = & P (\zeta^{- k})\\
  & = & \overline{P (\zeta)}\\
  & = & \overline{u_k}
\end{eqnarray*}


Donc $\left| \frac{u_k}{u_{p - k}} \right| = 1$.

Or, $\frac{u_1}{u_{p - 1}} \in \mathbb{Z} [\zeta]^{\times} \subset
\mathfrak{D}_K$, et ses conjugu{\'e}s sont donn{\'e}s par :
\begin{eqnarray*}
  \sigma_k \left( \frac{u_1}{u_{p - 1}} \right) & = & \frac{\sigma_k (u_{k
  1})}{\sigma_k (u_{p - 1})}\\
  & = & \frac{u_k}{P (\sigma_k (\zeta^{- 1}))}\\
  & = & \frac{u_k}{u_{p - k}}
\end{eqnarray*}


D'o{\`u} le r{\'e}sultat.

\

\tmtextbf{6.d.} En d{\'e}duire qu'il existe$m \in \mathbb{Z}$ tel que
$\frac{u_1}{u_{p - 1}} = \pm \zeta^m$.

En utilisant la question 6.a.ii, on a $\frac{u}{u_{p - 1}}$ est une racine de
l'unit{\'e} de $K$.

Ainsi, il existe $m \in \mathbb{Z}$ tel que $\frac{u}{u_{p - 1}} = \pm
\zeta^m$ (cf. question 4 de la partie 3).

\

\tmtextbf{6.e.i.} Soit $\theta \in \mathbb{Z}[\zeta]$. Justifions qu'il
existe un entier $a \in \mathbb{Z}$ tel que $\theta = a (\tmop{mod} < \lambda
>)$.

On a $\theta \in \mathbb{Z} [\zeta]$, donc il existe des entiers $a_0,
\ldots, a_{p - 2}$ tels que
\[ \theta = \underset{k = 0}{\overset{p - 2}{\sum}} a_k \zeta^k \]


Ainsi,
\begin{eqnarray*}
  \theta & = & \underset{k = 0}{\overset{p - 2}{\sum}} a_k (\zeta^k - 1) +
  \underset{k = 0}{\overset{p - 2}{\sum}} a_k\\
  & = & \lambda \underset{k = 0}{\overset{p - 2}{\sum}} a_k \underset{j =
  0}{\overset{k - 1}{\sum}} \zeta^j + \underset{k = 0}{\overset{p - 2}{\sum}}
  a_k\\
  & = & \underset{k = 0}{\overset{p - 2}{\sum}} a_k  (\tmop{mod} < \lambda >)
\end{eqnarray*}


On a donc $\theta = a (\tmop{mod} < \lambda >)$, o{\`u} $a = \underset{k =
0}{\overset{p - 2}{\sum}} a_k \in \mathbb{Z}$.

\

En d{\'e}duire que deux {\'e}l{\'e}ments conjugu{\'e}s de $\mathbb{Z}[\zeta]$
sont {\'e}gaux modulo $< \lambda > .$

Soit $\theta = \underset{k = 0}{\overset{p - 2}{\sum}} a_k \zeta^k \in
\mathbb{Z} [\zeta]$, et $\sigma_j (\theta) = \underset{k = 0}{\overset{p -
2}{\sum}} a_k \zeta^{j k}$ un des conjugu{\'e}s de $\theta$, o{\`u} $k \in
\llbracket 0, p - 1 \rrbracket$.

On a
\begin{eqnarray*}
  \theta - \sigma_j (\theta) & = & \underset{k = 0}{\overset{p - 2}{\sum}} a_k
  (\zeta^k - \zeta^{j k})\\
  & = & \lambda \underset{k = 0}{\overset{p - 2}{\sum}} a_k \zeta^k
  \underset{l = 0}{\overset{j k - k - 1}{\sum}} \zeta l\\
  & = & 0 (\tmop{mod} < \lambda >)
\end{eqnarray*}


D'o{\`u} le r{\'e}sultat.

\

\tmtextbf{6.e.ii.} Montrons que $\frac{u_1}{u_{p - 1}} = \zeta^m$.

On a  $\frac{u_1}{u_{p - 1}} = \pm \zeta^m$. Par l'absurde, supposons que \
$\frac{u_1}{u_{p - 1}} = - \zeta^m .$

Alors
\begin{eqnarray*}
  u & = & - \zeta^m u_{p - 1}\\
  & = & - u_{p - 1} (\tmop{mod} < \lambda >)
\end{eqnarray*}


De plus, $u$ et $u_{p - 1} $sont conjugu{\'e}s, donc d'apr{\`e}s la question
pr{\'e}c{\'e}dente, on a
\[ u = u_{p - 1} (\tmop{mod} < \lambda >) \]


Donc
\[ 2 u = 0 (\tmop{mod} < \lambda >) \]


Ainsi, $2 u \in < \lambda >$, avec $< \lambda >$ est premier.

Donc, $2 \in < \lambda >$ ou $u \in < \lambda >$.

Si $2 \in < \lambda >$, alors
\[ N (\lambda) | \nobracket N (2) = 2^{p - 1} \]


Ainsi, $p | 2^{p - 1} \nobracket,$ absurde!

Donc $u \in < \lambda >$

Ainsi,
\[ p = N (\lambda) | N \nobracket (u) = 1 \]


Absurde!

D'o{\`u} le r{\'e}sultat.

\

\tmtextbf{Remarque.}

On peut r{\'e}pondre directement {\`a} la question \tmtextbf{6.e.ii} sans
faire les questions \tmtextbf{6.c, 6.d} et \tmtextbf{6.e.i}.

En justifiant que $\frac{u_1}{u_{p - 1}}$ est un entier alg{\'e}brique dont
tous les conjugu{\'e}s sont de module 1.

\

D'apr{\`e}s la question pr{\'e}c{\'e}cente, on a
\[ u_1, u_{p - 1} \in \mathbb{Z} [\zeta]^{\times} \]


Donc
\[ u_1, \frac{1}{u_{p - 1}} \in \mathbb{Z} [\zeta]  \]


D'apr{\`e}s la question 2, on sait que $\mathbb{Z} [\zeta] \subseteq
\mathcal{D}_K \subseteq D_{\mathbb{C}}$, o{\`u} $\mathcal{D}_{\mathbb{C}}$ est
un anneau (d'apr{\`e}s la \tmtextbf{question 6 de la partie $\mathbb{I}$}).

Ainsi,
\[ \frac{u_1}{u_{p - 1}} = u_1 \times \frac{1}{u_{p - 1}} \in
   \mathcal{D}_{\mathbb{C}}  \]


Autrement dit, $\frac{u_1}{u_{p - 1}}$ est un entier alg{\'e}brique.

Il ne reste qu'{\`a} montrer que tous les conjugu{\'e}s de $\frac{u_1}{u_{p -
1}} $sont de module 1. En s'inspirant des questions \tmtextbf{6.a.i} et
\tmtextbf{6.a.ii} de cette partie, il est pr{\'e}f{\'e}rable ({\'e}tant
donn{\'e} que $\frac{u_1}{u_{p - 1}}$ est un entier alg{\'e}brique) de montrer
que \ $\frac{u_1}{u_{p - 1}}$ est une racine de l'unit{\'e}.(Ce faisant, on
r{\'e}pond {\'e}galement {\`a} la deuxi{\`e}me partie de cette question).

Notons,
\[ P = \overset{n}{\underset{k = 0}{\sum}} a_k X^k \in \mathbb{Z} [X] \]
D'apr{\`e}s la formule multin{\^o}me, on a :
\begin{eqnarray*}
  u_{p - 1}^p & = & P (\zeta^{p - 1})^p\\
  & = & P \left( \frac{1}{\zeta} \right)^p\\
  & = & \left( \overset{n}{\underset{k = 0}{\sum}} a_k \zeta^{- k}
  \right)^p\\
  & = & \underset{i_1 + \cdots + i_n = p}{\sum} \underset{k =
  0}{\overset{n}{\prod}} a_{i_k} \zeta^{- i_k}\\
  & = & \underset{i_1 + \cdots + i_n = p}{\sum} \left( \underset{k =
  0}{\overset{n}{\prod}} a_{i_k} \right) \zeta^{- (i_1 + \cdots + i_n)}\\
  & = & \underset{i_1 + \cdots + i_n = p}{\sum} \left( \underset{k =
  0}{\overset{n}{\prod}} a_{i_k} \right) \zeta^{- p}\\
  & = & \underset{i_1 + \cdots + i_n = p}{\sum} \left( \underset{k =
  0}{\overset{n}{\prod}} a_{i_k} \right)\\
  & = & \underset{i_1 + \cdots + i_n = p}{\sum} \left( \underset{k =
  0}{\overset{n}{\prod}} a_{i_k} \right) \zeta^p\\
  & = & \underset{i_1 + \cdots + i_n = p}{\sum} \left( \underset{k =
  0}{\overset{n}{\prod}} a_{i_k} \right) \zeta^{i_1 + \cdots + i_n}\\
  & = & \underset{i_1 + \cdots + i_n = p}{\sum} \underset{k =
  0}{\overset{n}{\prod}} a_{i_k} \zeta^{i_k}\\
  & = & \left( \overset{n}{\underset{k = 0}{\sum}} a_k \zeta^k \right)^p\\
  & = & P (\zeta)^p\\
  & = & u^p_1
\end{eqnarray*}


Donc :
\[ \left( \frac{u_1}{u_{p - 1}} \right)^p = 1 \]


Ainsi, il existe $m \in \mathbb{Z}$ tel que :
\[ \frac{u_1}{u_{p - 1}} = \zeta^m \]


D'o{\`u} le r{\'e}sultat.

\

\tmtextbf{6.f.} Justifions l'existence de $r \in \mathbb{Z}$ tel que $2 r = m
(\tmop{mod} p\mathbb{Z})$.

D'apr{\`e}s le th{\'e}or{\`e}me de Fermat :
\[ 2^{p - 1} = 1 (\tmop{mod} p\mathbb{Z}) \]


Ainsi, pour $r = m 2^{p - 1}$, on a $r = m (\tmop{mod} p\mathbb{Z})$.

On pose $\varepsilon = \zeta^{- r} u$. Montrons que $\varepsilon \in
\mathbb{R}$.

On a\footnote{Il est facile {\`a} v{\'e}rifier que $\overline{u_{p - 1}} =
u$.}
\begin{eqnarray*}
  \bar{\varepsilon} & = & \zeta^r \bar{u}\\
  & = & \zeta^r u_{p - 1}\\
  & = & \zeta^r (\zeta^{- m} u)\\
  & = & \zeta^r (\zeta^{- 2 r} u)\\
  & = & \zeta^{- r} u\\
  & = & \varepsilon
\end{eqnarray*}


D'o{\`u} $\varepsilon \in \mathbb{R}$.

Puisque $u, \zeta^{- r} \in \mathbb{Z} [\zeta]^{\times}$, alors $\varepsilon
\in \mathbb{Z} [\zeta]^{\times}$.

Ainsi, $u = \zeta^r \varepsilon$, o{\`u} $\varepsilon$ est un r{\'e}el dans
$\mathbb{Z} [\zeta]^{\times}$.

\tmtextbf{\tmcolor{blue}{{\tmname{}}}Conculsion.}

Pour tout $u \in \mathbb{Z} [\zeta]^{\times}$, on a l'existence de $r \in
\mathbb{Z}$ et $\varepsilon$ un r{\'e}el inversible de $\mathbb{Z} [\zeta] $,
tels que :
\[ u = \zeta^r \varepsilon \]


\tmtextbf{7.} Le but de ce qui suit est de montrer que $\mathcal{D}_K
=\mathbb{Z}[\zeta]$.

\tmtextbf{7.a.} Soit $\theta \in \mathcal{D}_K$. Montrons que
\[ N (\theta) \in \mathbb{Z} \infixand \tmop{Tr} (\theta) \in \mathbb{Z} \]


Puisque $\theta \in K =\mathbb{Q} (\zeta)$, alors il existe $P = \underset{k =
0}{\overset{n}{\sum}} a_k X^k \in \mathbb{Q} [X]$, tel que $\theta = P
(\zeta)$.

On a alors
\[ P_{\theta} = \underset{k = 1}{\overset{p - 1}{\prod}} (X - \sigma_k
   (\theta)) \in \mathbb{Q} [X] \]


En particulier :
\[ \begin{array}{lll}
     N (\theta) & = & \underset{k = 1}{\overset{p - 1}{\prod}} \sigma_k
     (\theta) \in \mathbb{Q}
   \end{array} \]


De plus,
\[ \begin{array}{lll}
     N (\theta) & = & \underset{k = 1}{\overset{p - 1}{\prod}} P (\zeta^k)
   \end{array} \]


Et $\zeta \in \mathcal{D}_K$, $P \in \mathbb{Q} [X]$, et $\mathcal{D}_K$ est
un sous-anneau de $\mathbb{C}$, alors
\[ \begin{array}{lll}
     N (\theta) & = & \underset{k = 1}{\overset{p - 1}{\prod}} P (\zeta^k)
   \end{array} \in \mathcal{D}_K \]


Par suite :
\[ N (\theta) \in \mathcal{D}_K \cap \mathbb{Q}=\mathbb{Z} \]


Ensuite, pour la trace, on a :
\begin{eqnarray*}
  \tmop{Tr} (\theta) & = & \underset{k = 1}{\overset{p - 1}{\sum}} \sigma_k
  (\theta)\\
  & = & \underset{k = 1}{\overset{p - 1}{\sum}} \sigma_k (P (\zeta))\\
  & = & \underset{k = 1}{\overset{p - 1}{\sum}}  P (\nobracket \sigma_k
  (\nobracket \zeta))\\
  & = & \underset{k = 1}{\overset{p - 1}{\sum}}  P (\zeta^k)\\
  & = & \underset{k = 1}{\overset{p - 1}{\sum}}  \underset{j =
  0}{\overset{n}{\sum}} a_j \zeta^{j k}\\
  & = &  \underset{j = 0}{\overset{n}{\sum}} a_j \left( \underset{k =
  1}{\overset{p - 1}{\sum}} \zeta^{j k} \right)\\
  & = & \underset{j = 0}{\overset{n}{\sum}} a_j \zeta^j \frac{1 - \zeta^{j (p
  - 1)}}{1 - \zeta^j}\\
  & = & \underset{j = 0}{\overset{n}{\sum}} a_j \frac{\zeta^j - 1}{1 -
  \zeta^j}\\
  & = & - \underset{j = 0}{\overset{n}{\sum}} a_j\\
  & \in & \mathbb{Q}
\end{eqnarray*}


Comme $\zeta \in \mathcal{D}_K$, $P \in \mathbb{Q} [X]$, et $\mathcal{D}_K$
est un sous-anneau de $\mathbb{C}$, alors
\[ \tmop{Tr} (\theta) = \underset{k = 1}{\overset{p - 1}{\sum}}  P (\zeta^k)
   \in \mathcal{D}_K \]


Par suite :
\[ \tmop{Tr} (\theta) \in \mathcal{D}_K \cap \mathbb{Q}=\mathbb{Z} \]


D'o{\`u} le r{\'e}sultat.

\

\tmtextbf{7.b.i.} Soit $k \in \llbracket 0, p - 2 \rrbracket$, on a
\begin{eqnarray*}
  b_k & = & \tmop{Tr} (\theta \zeta^{- k} - \theta \zeta)\\
  & = & \underset{j = 1}{\overset{p - 1}{\sum}} \sigma_j (\theta \zeta^{- k}
  - \theta \zeta)\\
  & = & \underset{j = 1}{\overset{p - 1}{\sum}} (\sigma_j (\theta) \sigma_j
  (\zeta)^{- k} - \sigma_j (\theta) \sigma_j (\zeta))\\
  & = & \underset{j = 1}{\overset{p - 1}{\sum}} \left( \sigma_j \left(
  \underset{l = 0}{\overset{p - 2}{\sum}} a_l \zeta^l \right) \zeta^{- j k} -
  \sigma_j \left( \underset{l = 0}{\overset{p - 2}{\sum}} a_l \zeta^l \right)
  \zeta^j \right)\\
  & = & \underset{j = 1}{\overset{p - 1}{\sum}} \underset{l = 0}{\overset{p -
  2}{\sum}} a_l (\zeta^{j l - j k} - \zeta^{j l - j})\\
  & = & \underset{l = 0}{\overset{p - 2}{\sum}} a_l \left( \underset{j =
  1}{\overset{p - 1}{\sum}} (\zeta^{l - k})^j \right) - \underset{l =
  0}{\overset{p - 2}{\sum}} a_l \left( \underset{j = 1}{\overset{p - 1}{\sum}}
  (\zeta^{l - 1})^j \right)\\
  & = & \underset{l \neq k}{\underset{l = 0}{\overset{p - 2}{\sum}}} a_l
  \left( \underset{j = 1}{\overset{p - 1}{\sum}} (\zeta^{l - k})^j -
  \underset{j = 1}{\overset{p - 1}{\sum}} (\zeta^{l - 1})^j \right) + a_{_k}
  \left( p - 1 - \underset{j = 1}{\overset{p - 1}{\sum}} (\zeta^{l - 1})^j
  \right)\\
  & = & \underset{l \neq k}{\underset{l = 0}{\overset{p - 2}{\sum}}} a_l
  \left( \underset{j = 1}{\overset{p - 1}{\sum}} (\zeta^{l - k})^j -
  \underset{j = 1}{\overset{p - 1}{\sum}} (\zeta^{l - 1})^j \right) + a_{_k}
  \left( p - 1 - \zeta^{l - 1} \frac{1 - \zeta^{(l - 1) (p - 1)}}{1 - \zeta^{l
  - 1}} \right)\\
  & = & \underset{l \neq k}{\underset{l = 0}{\overset{p - 2}{\sum}}} a_l
  \left[ \zeta^{l - k} \frac{1 - \zeta^{(l - k) (p - 1)}}{1 - \zeta^{l - k}} -
  \zeta^{l - 1} \frac{1 - \zeta^{(l - 1) (p - 1)}}{1 - \zeta^{l - 1}} \right]
  + p a_k\\
  & = & \underset{l \neq k}{\underset{l = 0}{\overset{p - 2}{\sum}}} a_l [- 1
  - (- 1)] + p a_k\\
  & = & p a_k
\end{eqnarray*}


De plus,
\[ \theta \zeta^{- k} - \theta \zeta \in \mathcal{D}_K \]


Donc d'apr{\`e}s la question pr{\'e}c{\'e}dente :
\[ b_k = \tmop{Tr} (\theta \zeta^{- k} - \theta \zeta) \in \mathbb{Z} \]


\tmtextbf{7.b.ii.} Montrons qu'il existe des entiers $c_0, \ldots, c_{p - 2}$.
que l'on exprimera en fonction des $b_k$ tels que :
\[ p \theta = \underset{k = 0}{\overset{p - 2}{\sum}} c_k \lambda^k \]


On a :
\begin{eqnarray*}
  p \theta & = & \underset{k = 0}{\overset{p - 2}{\sum}} p a_k (1 -
  \lambda)^k\\
  & = & \underset{k = 0}{\overset{p - 2}{\sum}} \underset{j =
  0}{\overset{k}{\sum}} b_k (- 1)^j \left( \begin{array}{c}
    k\\
    j
  \end{array} \right) \lambda^j\\
  & = & \underset{j = 0}{\overset{p - 2}{\sum}} \underset{k = j}{\overset{p -
  2}{\sum}} b_k (- 1)^j \left( \begin{array}{c}
    k\\
    j
  \end{array} \right) \lambda^j
\end{eqnarray*}


On pose pour tout $k \in \llbracket 0, p - 2 \rrbracket$,
\[ c_k = \underset{j = k}{\overset{p - 2}{\sum}} b_j (- 1)^k \left(
   \begin{array}{c}
     j\\
     k
   \end{array} \right) \in \mathbb{Z} \]


On a alors
\[ p \theta = \underset{k = 0}{\overset{p - 2}{\sum}} c_k \lambda^k \]
$\text{}$

CQFD.

\

Montrons que, pour tout $k \in \llbracket 0, p - 2 \rrbracket$,
\[ b_k = \underset{l = k}{\overset{p - 2}{\sum}} (- 1)^l \left(
   \begin{array}{c}
     l\\
     k
   \end{array} \right) c_l \]


On a :
\begin{eqnarray*}
  \underset{k = 0}{\overset{p - 2}{\sum}} b_k \zeta^k & = & p \underset{k =
  0}{\overset{p - 2}{\sum}} a_k \zeta^k\\
  & = & p \theta\\
  & = & \underset{k = 0}{\overset{p - 2}{\sum}} c_k \lambda^k\\
  & = & \underset{k = 0}{\overset{p - 2}{\sum}} c_k (1 - \zeta)^k\\
  & = & \underset{k = 0}{\overset{p - 2}{\sum}} \tmmathbf{} \underset{j =
  0}{\overset{k}{\sum}} c_k (- 1)^j \left( \begin{array}{c}
    k\\
    j
  \end{array} \right) \zeta^j\\
  & = & \underset{k = 0}{\overset{p - 2}{\sum}} \tmmathbf{} \underset{l =
  k}{\overset{p - 2}{\sum}} c_l (- 1)^l \left( \begin{array}{c}
    l\\
    k
  \end{array} \right) \zeta^k
\end{eqnarray*}


Avec $(1, \zeta, \ldots, \zeta^{p - 2})$ est $\mathbb{Q}$-libre, alors pour
tout $k \in \llbracket 0, p - 2 \rrbracket$,
\[ b_k = \underset{l = k}{\overset{p - 2}{\sum}} c_l (- 1)^l \left(
   \begin{array}{c}
     l\\
     k
   \end{array} \right) \]


D'o{\`u} le r{\'e}sultat.

\

\tmtextbf{7.b.iii.} D'apr{\`e}s la question 1.b.ii de cette partie, on a :
\begin{eqnarray*}
  p & = & N (1 - \zeta)\\
  & = & \underset{k = 1}{\overset{p - 1}{\prod}} \sigma_k (1 - \zeta)\\
  & = & \underset{k = 1}{\overset{p - 1}{\prod}} (1 - \zeta^k)\\
  & = & \underset{k = 1}{\overset{p - 1}{\prod}} \left[ (1 - \zeta)
  \underset{j = 0}{\overset{k - 1}{\sum}} \zeta^j \right]\\
  & = & (1 - \zeta)^{p - 1} \underset{k = 1}{\overset{p - 1}{\prod}} \left(
  \underset{j = 0}{\overset{k - 1}{\sum}} \zeta^j \right)\\
  & = & \lambda^{p - 1} \underset{k = 1}{\overset{p - 1}{\prod}} \left(
  \underset{j = 0}{\overset{k - 1}{\sum}} \zeta^j \right)
\end{eqnarray*}


Avec
\[ \beta = \underset{k = 1}{\overset{p - 1}{\prod}} \left( \underset{j =
   0}{\overset{k - 1}{\sum}} \zeta^j \right) \in \mathbb{Z} [\zeta] \]


Montrons maintenant que $p | c_0 \nobracket$. On a :
\begin{eqnarray*}
  c_0 & = & p \theta - \underset{k = 1}{\overset{p - 2}{\sum}} c_k \lambda^k\\
  & = & \beta \lambda^{p - 1} - \lambda \underset{k = 1}{\overset{p -
  2}{\sum}} c_k \lambda^{k - 1}
\end{eqnarray*}


Donc,
\begin{eqnarray*}
  c^{p - 1}_0 & = & N (c_0)\\
  & = & N \left( \lambda \left( \beta \lambda^{p - 2} - \underset{k =
  1}{\overset{p - 2}{\sum}} c_k \lambda^{k - 1} \right) \right)\\
  & = & N (\lambda) N \left( \beta \lambda^{p - 2} - \underset{k =
  1}{\overset{p - 2}{\sum}} c_k \lambda^{k - 1} \right)\\
  & = & p N \left( \beta \lambda^{p - 2} - \underset{k = 1}{\overset{p -
  2}{\sum}} c_k \lambda^{k - 1} \right)
\end{eqnarray*}


Avec $N \left( \beta \lambda^{p - 2} - \underset{k = 1}{\overset{p - 2}{\sum}}
c_k \lambda^{k - 1} \right) \in \mathbb{Z}$, donc $p | c^{p - 1}_0
\nobracket$, et comme $p$ est premier, alors $p | c_0 \nobracket$.

\

Passons maintenant {\`a} montrer que $p | c_k \nobracket $pour tout $k \in
\llbracket 0, p - 2 \rrbracket$.

Montrons ce r{\'e}sultat par r{\'e}currence fini sur $k \in \llbracket 0, p -
2 \rrbracket$.

\tmtextbf{Pour $k = 0$,} c'est d{\'e}j{\`a} fait !

Soit $\tmmathbf{k \in \llbracket 0, p - 3 \rrbracket}$, supposons que le
r{\'e}sultat est vrai pour $1, 2, \ldots, k$ et montrons-le pour $k + 1$.

On a :
\begin{eqnarray*}
  c_{k + 1} \lambda^{k + 1} & = & p \left( \theta - \underset{l =
  0}{\overset{k}{\sum}} \frac{c_l}{p} \lambda^l \right) - \lambda^{k + 2}
  \underset{l = k + 2}{\overset{p - 2}{\sum}} c_l \lambda^{l - k - 2}
\end{eqnarray*}


\

Avec pour tout $l \in \llbracket 1, p - 2 \rrbracket$,
\begin{eqnarray*}
  c_{k + 1} p^{k + 1} & = & c_{k + 1} N (\lambda^{k + 1})\\
  & = & N (c_{k + 1} \lambda^{k + 1})\\
  & = & N \left( \beta \lambda^{^{p + 1}} \left( \theta - \underset{l =
  0}{\overset{k}{\sum}} \frac{c_l}{p} \lambda^l \right) - \lambda^{k + 2}
  \underset{l = k + 2}{\overset{p - 2}{\sum}} c_l \lambda^{l - k - 2}
  \right)\\
  & = & N \left( \lambda^{k + 2} \left[ \beta \lambda^{^{p - k - 1}} \left(
  \theta - \underset{l = 0}{\overset{k}{\sum}} \frac{c_l}{p} \lambda^l \right)
  - \underset{l = k + 2}{\overset{p - 2}{\sum}} c_l \lambda^{l - k - 2}
  \right] \right)\\
  & = & N (\lambda^{k + 2}) N \left( \left[ \beta \lambda^{^{p - k - 1}}
  \left( \theta - \underset{l = 0}{\overset{k}{\sum}} \frac{c_l}{p} \lambda^l
  \right) - \underset{l = k + 2}{\overset{p - 2}{\sum}} c_l \lambda^{l - k -
  2} \right] \right)\\
  & = & p^{k + 2} N \left( \left[ \beta \lambda^{^{p - k - 1}} \left( \theta
  - \underset{l = 0}{\overset{k}{\sum}} \frac{c_l}{p} \lambda^l \right) -
  \underset{l = k + 2}{\overset{p - 2}{\sum}} c_l \lambda^{l - k - 2} \right]
  \right)
\end{eqnarray*}


Donc,
\[ c_{k + 1} = p  N \left( \left[ \beta \lambda^{^{p - k - 1}} \left( \theta -
   \underset{l = 0}{\overset{k}{\sum}} \frac{c_l}{p} \lambda^l \right) -
   \underset{l = k + 2}{\overset{p - 2}{\sum}} c_l \lambda^{l - k - 2} \right]
   \right) \]


Ainsi, $p | c_{k + 1} \nobracket$.

D'o{\`u} le r{\'e}sultat.

\

\subsubsection*{IV Le th{\'e}or{\`e}me de Fermat pour$p = 3$}

\

\tmtextbf{1.} On a $3 \not{| \nobracket} x y z$, en particulier \ $3 \not{|
\nobracket} x$, donc $x \equiv 1 [3]$ ou $x \equiv - 1 [3]$.

Si $x \equiv 1 [3]$, on a :
\[ (x - 1)^3 = x^3 + 3 (x - x^2) - 1 \]


Avec $9 | (x - 1)^3 \nobracket$ et $9 | 3 (x - x^2) \nobracket$, donc
\[ x^3 \equiv 1 [9] \]


De m{\^e}me, si $x \equiv - 1 [3]$, on a :
\[ (x + 1)^3 = x^3 + 3 (x^2 + x ) + 1 \]


Avec $9 | (x - 1)^3 \nobracket$ et $9 | 3 (x^2 + x ) \nobracket$, donc
\[ x^3 \equiv - 1 [9] \]


De m{\^e}me, on obtient :
\[ y^3 \equiv 1 [9] \infixor y^3 \equiv - 1 [9] \]


et
\[ z^3 \equiv 1 [9] \infixor z^3 \equiv - 1 [9] \]


Avec
\[ x^3 = - (y^3 + z^3) \]


Donc, modulo 9, on a les possibilit{\'e}s suivantes :
\[ 1 \equiv - 2 [9] \infixor 1 \equiv 0 [9] \infixor 1 \equiv 2 [9] \infixor -
   1 \equiv - 2 [9] \infixor - 1 \equiv 0 [9] \infixor 1 \equiv 2 [9] \]


Ce qui est absurde !

\

\tmtextbf{2.} On a :
\begin{eqnarray*}
  \lambda^2 & = & (1 - j)^2\\
  & = & 1 - 2 j + j^2\\
  & = & (1 + j + j^2) - 3 j\\
  & = & - 3 j
\end{eqnarray*}


D'apr{\`e}s \tmtextbf{la question 2.c de la partie $I$}, on sait que $- j \in
\mathbb{Z} [j]^{\times}$.

Ainsi,
\[ 3 \sim \lambda^2 \]


\tmtextbf{3.} Soit $s \in \mathbb{Z}[j]$ tel que $s \neq 0 (\tmop{mod} <
\lambda >)$. Montrons qu'il existe $\varepsilon \in \{\nonconverted{minus} 1,
+ 1\}$ tel que $s^3 = \varepsilon (\tmop{mod} < \lambda >)$.

\

Suivant l'indication, montrons qu'il existe $\varepsilon \in \{ - 1, 1 \}$
tel que $s = \varepsilon (\tmop{mod} < \lambda  >)$.

D'apr{\`e}s la question pr{\'e}c{\'e}dente, on a $\lambda^2 \sim 3$ dans
$\mathbb{Z} [j]$.

Il existe $a, b \in \mathbb{Z}$ tels que $s = a + j b $.

Donc :
\begin{eqnarray*}
  s & = & a - 2 b + 3 j b\\
  & = & a - 2 b (\tmop{mod} < \lambda >)\\
  & = & \varepsilon (\tmop{mod} < \lambda >)
\end{eqnarray*}


O{\`u} $\varepsilon \in \{ - 1, 0, 1 \}$ est obtenu {\`a} partir de la
division euclidienne de $a - 2 b$ par $3$.

Puisque $s \neq 0 (\tmop{mod} < \lambda >)$, on a alors $\varepsilon \in \{ -
1, 1 \}$.

Il existe donc $\chi \in \mathbb{Z} [j]$ tel que : $s - \varepsilon = \chi
\lambda$.

Ainsi,
\begin{eqnarray*}
  s^3 - \varepsilon & = & s^3 - \varepsilon^3\\
  & = & (s - \varepsilon)^3 + 3 \varepsilon s^2 - 3 s\\
  & = & \chi^3 \lambda^3 + 3 \varepsilon s (s - \chi)\\
  & = & \chi^3 \lambda^3 - j^2 \lambda^2 \varepsilon (\chi \lambda +
  \varepsilon) \chi \lambda\\
  & = & \chi^3 \lambda^3 - j^2 \lambda^3 \varepsilon (\chi \lambda +
  \varepsilon) \chi\\
  & = & \chi  \lambda^3 (\chi^2 - j^2 \varepsilon (\chi \lambda +
  \varepsilon))\\
  & = & \chi \lambda^3 (\chi^2 - j^2 \varepsilon \nobracket \chi \lambda -
  j^2) \nobracket
\end{eqnarray*}


D'apr{\`e}s ce qui pr{\'e}c{\`e}de, on a l'existence de $\varepsilon' \in \{ -
1, 1 \}$ tel que $\chi = \varepsilon'  (\tmop{mod} < \lambda >)$.

\

Ainsi, on a, modulo $< \lambda >$
\begin{eqnarray*}
  \chi^2 - j^2 \varepsilon \chi \lambda - j^2 & = & {\varepsilon'}^2 - j^2
  \varepsilon \chi \lambda - j^2  (\tmop{mod} < \lambda >)\\
  & = & \lambda [(1 + j) - j^2 \varepsilon \chi] (\tmop{mod} < \lambda >)\\
  & = & 0 (\tmop{mod} < \lambda >)
\end{eqnarray*}


Ainsi, il existe $\mu \in \mathbb{Z} [j]$ tel que
\[ \chi^2 - j^2 \varepsilon \chi \lambda - j^2 = \mu \lambda \]


Par suite,
\begin{eqnarray*}
  s^3 - \varepsilon & = & \chi \mu \lambda^4\\
  & = & 0 (\tmop{mod} < \lambda^4 >)
\end{eqnarray*}


D'o{\`u} le r{\'e}sultat.

\

\tmtextbf{4.} Supposons que $(P_n)$ est v{\'e}rifi{\'e}e pour un quadruplet
$(\alpha, \beta, \delta, \omega)$.

Montrons que $n \geqslant 2$.

On a
\[ z = \mu \lambda^n \]


Donc
\[ \alpha^3 + \beta^3 + \omega \delta^3 \lambda^{3 n} = 0 \]


Et on a
\[ \lambda \not{| \nobracket} \alpha \beta \delta \]


Donc  $\lambda \not{| \nobracket} \alpha$, $\lambda \not{| \nobracket} \beta$
et $\lambda \not{| \nobracket} \delta$.

Ainsi, $\alpha \neq 0 (\tmop{mod} < \lambda >)$, $\beta \neq 0 (\tmop{mod} <
\lambda >)$ et $\delta \neq 0 (\tmop{mod} < \lambda >)$.

En utilisant la question 3 de cette partie, on a l'existence de
$\varepsilon_1, \varepsilon_2, \varepsilon_3 \in \{ - 1, 1 \}$ tels que :
\[ \left\{\begin{array}{l}
     \begin{array}{c}
       \alpha^3 = \varepsilon_1 (\tmop{mod} < \lambda^4 >) \text{}\\
       \beta^3 = \varepsilon_1 (\tmop{mod} < \lambda^4 >)\\
       \delta^3 = \varepsilon_1 (\tmop{mod} < \lambda^4 >)
     \end{array}
   \end{array}\right. \]


Donc
\begin{eqnarray*}
  \omega \delta^3 \lambda^{3 n} & = & - (\alpha^3 + \beta^3)\\
  & = & - (\varepsilon_1 + \varepsilon_2) \quad (\tmop{mod} < \lambda^4 >)
\end{eqnarray*}


Avec $\alpha \wedge \beta = 1$, alors $\varepsilon_1 \neq \varepsilon_2$, avec
$\varepsilon_1, \varepsilon_2 \in \{ - 1, 1 \}$, on a alors $\varepsilon_1 +
\varepsilon_2 = 0$.

Par suite,
\[ \begin{array}{lll}
     \omega \delta^3 \lambda^{3 n} & = & 0 \quad (\tmop{mod} < \lambda^4 >)
   \end{array} \]


Autrement dit
\[ \lambda^4 | \nobracket \omega \delta^3 \lambda^{3 n} \]


Avec $\omega \in \mathbb{Z} [j]^{\times}$, donc il existe $u = a + j b \in
\mathbb{Z} [j]^{\times}$ tel que $\omega (a + j b) = 1$.

Ainsi,
\[ (a + b) . \omega - (b \omega) \lambda = 1 \]


D'apr{\`e}s le th{\'e}or{\`e}me de B{\'e}zout, on a $\lambda \wedge \omega =
1$, donc $\lambda^4 \wedge \omega = 1$.

Par cons{\'e}quent, \ via le lemme de Gauss, on a :
\[ \lambda^4 | \nobracket \delta^3 \lambda^{3 n} \]


D'autre part, on a
\[ \delta^3 = \varepsilon_1 (\tmop{mod} < \lambda^4 >) \]


Donc, il existe $t \in \mathbb{Z} [j]$ tel que $\delta^3 = \varepsilon_1 + t
\lambda^4$.

Ainsi,
\[ \varepsilon_1 \delta^3 - (\varepsilon_1 t) \lambda^4 = 1 \]


D'apr{\`e}s le th{\'e}or{\`e}me de B{\'e}zout, on a $\delta^3 \wedge \lambda^4
= 1$. Le lemme da Gauss assure que :
\[ \lambda^4 | \nobracket \lambda^{3 n} \]


{\'E}tant donn{\'e} que $\lambda$ est non inversible, on en d{\'e}duit $3 n
\geqslant 4$. Comme $n \in \mathbb{N}$, alors $n \geqslant 2$.

D'o{\`u} le r{\'e}sulltat.

\

\tmtextbf{5.} On a
\[ \begin{array}{lll}
     - \omega \delta^3 \lambda^{3 n} & = & \alpha^3 + \beta^3
   \end{array} \]


Avec $\lambda \not{| \nobracket} \alpha \beta \delta$, en particulier $\beta
\neq 0$.

On a alors
\begin{eqnarray*}
  \begin{array}{l}
    - \omega \delta^3 \lambda^{3 n}
  \end{array} & = & \alpha^3 + \beta^3\\
  & = & (- \beta)^3 \left[ \left( \frac{\alpha}{- \beta} \right)^3 - 1^3
  \right]
\end{eqnarray*}


En utilisant la factorisation :
\[ X^3 - 1 = (X - 1) (X - j) (X - j^2) \]


(Car les racines cubiques de l'unit{\'e} sont exactement $1, j$ et $j^2$).

Alors,
\begin{eqnarray*}
  \begin{array}{l}
    - \omega \delta^3 \lambda^{3 n}
  \end{array} & = & \alpha^3 + \beta^3\\
  & = & (- \beta)^3 \left[ \left( \frac{\alpha}{- \beta} \right)^3 - 1^3
  \right]\\
  & = & (- \beta)^3 \left( \frac{\alpha}{- \beta} - 1 \right) \left(
  \frac{\alpha}{- \beta} - j \right) \left( \frac{\alpha}{- \beta} - j^2
  \right)\\
  & = & (\alpha + \beta) (\alpha + j \beta) \left( \alpha + j^2 \beta \right)
\end{eqnarray*}


D'o{\`u} le r{\'e}sultat.

\

\tmtextbf{Remarque.}

\tmcolor{red}{} On peut commencer par d{\'e}velopper le terme {\`a} droite, et
on obtient :


\begin{eqnarray*}
  (\alpha + \beta) (\alpha + j \beta) \left( \alpha + j^2 \beta \right) & = &
  \alpha^3 + j^2 \alpha^2 \beta - j^2 \alpha^2 \beta - j  \alpha  \beta^2 + j 
  \alpha  \beta^2 + \beta^3\\
  & = & \alpha^3 + \beta^3\\
  & = & - \omega \delta^3 \lambda^{3 n}
\end{eqnarray*}


\tmtextbf{5.b.} D'apr{\`e}s la question pr{\'e}c{\'e}dente, on a $\lambda  |
\nobracket (\alpha + \beta) (\alpha + j \beta) \left( \alpha + j^2 \beta
\right)$.

Comme $N (\lambda) = 3$ qui est un nombre premier, alors, d'apr{\`e}s la
question 3.b de la partie 3, on a $\lambda$ est irr{\'e}ductible.

Donc $\lambda  | \nobracket \alpha + \beta$ ou $\lambda  | \nobracket \alpha +
j \beta$ ou $\lambda  | \nobracket \alpha + j^2 \beta$.

Il existe donc $i_0 \in \{ 0, 1, 2 \}$ tel que $\lambda  | \nobracket {\alpha
+ j^{i_0}}^{} \beta$.

Soit $k \in \{ 0, 1, 2 \}$, on a :
\begin{eqnarray*}
  \alpha + j^k \beta & = & \alpha {+ j^{i_0}}^{} \beta + \left( j^k {-
  j^{i_0}}^{} \right) \beta\\
  & = & \alpha {+ j^{i_0}}^{} \beta + \varepsilon_{k, i_0} j^{\min (k, i_0)}
  (j^{\max (k, i_0)} - 1) \beta
\end{eqnarray*}


avec
\[ \varepsilon_{k, i_0} = \tmop{sgn} (k - i_0) \in \{ - 1, 0, 1 \} \]


Ainsi,
\begin{eqnarray*}
  \alpha + j^k \beta & = & \alpha {+ j^{i_0}}^{} \beta + \varepsilon_{k, i_0}
  j^{\min (k, i_0)} (j  - 1) \beta \left( \underset{l = 0}{\overset{\max (k,
  i_0) - 1}{\sum}} j^l \right)\\
  & = & \alpha {+ j^{i_0}}^{} \beta - \lambda \varepsilon_{k, i_0} j^{\min
  (k, i_0)} \beta \left( \underset{l = 0}{\overset{\max (k, i_0) - 1}{\sum}}
  j^l \right)
\end{eqnarray*}


Puisque
\[ \lambda \left| \alpha {+ j^{i_0}}^{} \beta \right. \infixand \lambda |
   \nobracket \lambda \varepsilon_{k, i_0} j^{\min (k, i_0)} \beta \left(
   \underset{l = 0}{\overset{\max (k, i_0) - 1}{\sum}} j^l \right) \]


alors
\[ \lambda | \nobracket \alpha {+ j^{i_0}}^{} \beta - \lambda \varepsilon_{k,
   i_0} j^{\min (k, i_0)} \beta \left( \underset{l = 0}{\overset{\max (k, i_0)
   - 1}{\sum}} j^l \right) = \alpha + j^k \beta \]


Cela est vrai pour $k = 0, 1, 2$.

D'o{\`u}
\[ \lambda  | \nobracket \alpha + \beta \infixand \lambda  | \nobracket
   \alpha + j \beta \infixand \lambda  | \nobracket \alpha + j^2 \beta \]


\tmtextbf{5.c.} Montrons que $\lambda$ est un \tmtextbf{pgcd} de $\alpha +
\beta$ et $\alpha + j \beta$.

Notons $d$ un pgcd de $\alpha + \beta$ et $\alpha + j \beta$.

D'apr{\`e}s la question pr{\'e}c{\'e}dente, $\lambda  | \nobracket \alpha +
\beta \infixand \lambda  | \nobracket \alpha + j \beta$.

En particulier,
\[ \lambda | d \nobracket \]


Et
\[ d | \lambda \beta = (\alpha + \beta) - (\alpha + j \beta) \nobracket \]


Alors
\[ d | \lambda \nobracket \]


Ainsi, $d$ et $\lambda$ sont associ{\'e}s, d'o{\`u} $\lambda$ est un pgcd de
$\alpha + \beta$ et $\alpha + j \beta$.

\

De la m{\^e}me mani{\`e}re, on peut montrer facilemet que $\lambda$ est aussi
un pgcd de $\alpha + \beta$ et $\alpha + j^2 \beta$ (respectivement de $\alpha
+ j \beta$ et $\alpha + j^2 \beta$).

Notons
\[ \left\{\begin{array}{l}
     \begin{array}{c}
       \alpha + \beta = \lambda^{m_1} r_1\\
       \alpha + j  \beta = \lambda^{m_2} r_2\\
       \alpha + j^2 \beta = \lambda^{m_3} r_3
     \end{array}
   \end{array}\right. \]


Avec $\lambda \not{| \nobracket} r_1$, $\lambda \not{| \nobracket} r_2$ et
$\lambda \not{| \nobracket} r_3$. et $m_1, m_2, m_3 \geqslant 1$ des entiers
qui repr{\'e}sentent respectivement la valuation $\lambda$-adique de $\alpha +
\beta$, $\alpha + j \beta$ et $\alpha + j^2 \beta$.

D'apr{\`e}s ce qui pr{\'e}c{\`e}de, on a
\[ \left\{\begin{array}{l}
     \begin{array}{c}
       \min (m_1, m_2) = 1\\
       \min (m_2, m_3) = 1\\
       \min (m_3, m_1) = 1
     \end{array}
   \end{array}\right. \]


Donc, forc{\'e}ment, deux des entiers $m_1, m_2, m_3$ sont inf{\'e}rieurs ou
{\'e}gaux {\`a} 1.

Par sym{\'e}trie (il suffit de remplacer $\beta$ par $j \beta$ ou $j^2
\beta$), on peut supposer que $m_1, m_2 \leqslant 1$.

Or, comme $m_1, m_2 \geqslant 1$, alors $m_1 = m_2 = 1$.

Avec
\[ \lambda^4 | \nobracket - \omega \delta^3 \lambda^{3 n} = (\alpha + \beta)
   (\alpha + j \beta) \left( \alpha + j^2 \beta \right) = \lambda^{m_1 + m_2 +
   m_3} r_1 r_2 r_3 \]


Comme $\lambda \not{| \nobracket } r_1 $et $\lambda \not{| \nobracket } r_2$
et $\lambda \not{| \nobracket } r_3$ et $\lambda$ est irr{\'e}ductible, alors
$\lambda \wedge r_1 r_2 r_3 = 1$.

Par suite,
\[ m_1 + m_2 + m_3 \geqslant 4 \]


Donc $m_3 \geqslant 2$.

D'o{\`u} $\lambda^{^2}$ divise $\alpha + j^2 \beta$.

D'o{\`u} le r{\'e}sultat.

\

\tmtextbf{5.d.} On a


\[ \begin{array}{lll}
     \begin{array}{l}
       - \omega \delta^3 \lambda^{3 n}
     \end{array} & = & (\alpha + \beta) (\alpha + j \beta) \left( \alpha + j^2
     \beta \right)\\
     & = & \lambda^{3 n} \kappa_1 \kappa_2 \kappa_3
   \end{array} \]


Avec $\lambda \neq 0$, alors
\[ \begin{array}{lll}
     \begin{array}{l}
       - \omega \delta^3
     \end{array} & = & \kappa_1 \kappa_2 \kappa_3
   \end{array} \]


Montrons l'existence de $\gamma_l \in \mathbb{Z} [j]$ tel que $\kappa_l \sim
\gamma_l^3$ pour tout $l \in \{ 1, 2, 3 \}$.

Soit $l \in \{ 1, 2, 3 \}$. Comme $\mathbb{Z} [j]$ est un anneau principal,
alors il existe $p_{1, l}, \ldots, p_{m_l, l} \in \mathbb{Z} [j]$ des
irr{\'e}ductibles deux {\`a} deux distincts, et $\eta_{1, l}, \ldots, \eta_{l 
m_l, l} \in \mathbb{N}^{\ast}$ et $\omega_l \in \mathbb{Z} [j]^{\times}$ tels
que :
\[ \kappa_l = \omega_l \underset{s = 1}{\overset{m_l}{\prod}} p^{\eta_{s,
   l}}_{s, l}, \quad \tmop{pour} \tmop{tout} l \in \{ 1, 2, 3 \} \]


Via la question pr{\'e}c{\'e}dente, on a $\lambda$ est un pgcd de $\alpha +
\beta$ et $\alpha + j \beta$ (respectivement de $\alpha + j \beta$, $\alpha +
j^2 \beta$, $\alpha + j^2 \beta$ et $\alpha + \beta$).

Donc $\kappa_1 \infixand \kappa_2$ (respectivement $\kappa_2 \infixand
\kappa_3$, et $\kappa_1$) sont premiers entre eux.

Ainsi, $p_{1, 1}, \ldots, p_{m_1, 1}, p_{1, 2}, \ldots, p_{m_2, 2}, p_{1, 3},
\ldots, p_{m_3, 3}$ sont deux {\`a} deux distincts.

Et on a :
\begin{eqnarray*}
  - \omega \delta^3 & = & \kappa_1 \kappa_2 \kappa_3\\
  & = & \underset{l = 1}{\overset{3}{\prod}} \left( \omega_l \underset{s =
  1}{\overset{m_l}{\prod}} p^{\eta_{s, l}}_{s, l} \right)\\
  & = & \omega_1 \omega_2 \omega_3 \underset{l = 1}{\overset{3}{\prod}}
  \underset{s = 1}{\overset{m_l}{\prod}} p^{\eta_{s, l}}_{s, l}
\end{eqnarray*}


Notons $\delta = \vartheta \underset{s = 1}{\overset{m}{\prod}} p^{\tau_s}_s$
la d{\'e}composition en produit d'irr{\'e}ductibles de $\delta$.

Avec $\vartheta \in \mathbb{Z} [j]^{\times}$, et $p_1, \ldots, p_m$ des
irr{\'e}ductibles deux {\`a} deux distincts et $\tau_1, \ldots, \tau_m \in
\mathbb{N}^{\ast}$.

On a alors :
\[ - \omega \vartheta^3 \underset{s = 1}{\overset{m}{\prod}} p^{3 \tau_s}_s =
   \omega_1 \omega_2 \omega_3 \underset{l = 1}{\overset{3}{\prod}} \underset{s
   = 1}{\overset{m_l}{\prod}} p^{\eta_{s, l}}_{s, l} \]


Par l'unicit{\'e} de la d{\'e}composition en facteurs irr{\'e}ductibles, on a
forc{\'e}ment $3$ divise $\eta_{s, l}$ pour tout $l \in \{ 1, 2, 3 \}$ et \ $s
\in \llbracket 1, m_l \rrbracket$.

Notons p$\tmop{our} \tmop{tout} l \in \{ 1, 2, 3 \}  \infixand s \in
\llbracket 1, m_l \rrbracket$ :
\[ \varsigma_{s, l} = \frac{\eta_{s, l}}{3} \in \mathbb{N}^{\ast} \]


On a alors, pour $\tmop{tout} l \in \{ 1, 2, 3 \}$,
\[ \kappa_{l =} \omega_l \left( \underset{s = 1}{\overset{m_l}{\prod}}
   p^{\varsigma_{s, l}}_{s, l} \right)^3 \]


Donc, pour tout $l \in \{ 1, 2, 3 \}$, on a
\[ \kappa_l \sim \gamma_l^3 \]


Avec
\[ \gamma_l^3 = \underset{s = 1}{\overset{m_l}{\prod}} p^{\varsigma_{s,
   l}}_{s, l} \]


D'o{\`u} le r{\'e}sultat.

\

\tmtextbf{5.e.} Montrons qu'il existe deux inversibles $\tau$ et $\tau'$ de
$\mathbb{Z} [j]^{\times}$ tels que
\[ \gamma^3_2 + \tau \gamma^3_3 + \tau' \lambda^{3 (n - 1)} \gamma^3_1 = 0 \]


Essayons de d{\'e}tailler la question tout en profitant des r{\'e}sultats
obtenus pr{\'e}c{\'e}demment.

\

Puisque pour tout $l \in \{ 1, 2, 3 \}$, on a $\kappa_l \sim \gamma^3_l$,
donc il existe $a_l \in \mathbb{Z} [j]^{\times}$ tel que: $\kappa_l = a_l
\gamma^3_l$.

Ainsi,
\[ \left\{\begin{array}{l}
     \alpha + \beta = \lambda^{3 n - 2} a_1 \gamma^3_1\\
     \alpha + j \beta = \lambda a_2 \gamma^3_2\\
     \alpha + j^2 \beta = \lambda a_3 \gamma^3_3
   \end{array}\right. \]


Il s'agit de trouver un triplet $(a, b, c) \in \mathbb{Z} [j]^{\times} \times
\mathbb{Z} [j]^{\times} \times \mathbb{Z} [j]^{\times}$, tel que :
\[ a \gamma^3_2 + b \gamma^3_3 + c \lambda^{3 (n - 1)} \gamma^3_1 = 0. \]


Cela {\'e}quivaut {\`a} trouver $(a, b, c) \in \mathbb{Z} [j]^{\times} \times
\mathbb{Z} [j]^{\times} \times \mathbb{Z} [j]^{\times}$, tel que :
\[ a (\alpha + \beta) + b (\alpha + j \beta) + c (\alpha + j^2 \beta) = 0 \]


Il suffit alors de trouver  $(a, b, c) \in \mathbb{Z} [j]^{\times} \times
\mathbb{Z} [j]^{\times} \times \mathbb{Z} [j]^{\times}$, tel que :
\[ \left\{\begin{array}{l}
     a + b + c = 0\\
     a + j b + j^2 c = 0
   \end{array}\right. \]


Le triplet $(a, b, c) = (j^2, 1, j)$ convient.

Ainsi,
\[ \lambda a_2 \gamma^3_2 + \lambda j a_3 \gamma^3_3 + j^2 a_1 \lambda^{3 n -
   2} \gamma^3_1 = 0 \]


Par suite,
\[ \gamma^3_2 + j a^{- 1}_2 a_3 \gamma^3_3 + j^2 a^{- 1}_2 a_1 \lambda^{3 (n -
   1)} \gamma^3_1 = 0 \]


D'o{\`u} le r{\'e}sultat pour $\tau = j a^{- 1}_2 a_3$, et $\tau' = j^2 a^{-
1}_2 a_1$.

\

\tmtextbf{5.f.} Si $\tau = \pm 1$, montrons que $(P_{n - 1})$ est
v{\'e}rifi{\'e}e.

On a :
\[ \gamma^3_2 + (\tau \gamma _3)^3 + \lambda^{3 (n - 1)} \gamma^3_1 = 0 \]


Avec $\lambda \not{| \nobracket} \omega \delta^3 = \kappa_1 \kappa_2 \kappa_3$
et $\kappa_l \sim \gamma_l^3$ pour $l \in \{ 1, 2, 3 \}$, donc $\lambda \not{|
\nobracket} (\gamma _1 \gamma _2 \gamma _3)^3$.

Or, $\lambda$ est irr{\'e}ductible dans $\mathbb{Z} [j]$, car $N (\lambda) =
3$ est premier.

Donc $\lambda \not{| \nobracket} \gamma _1 \gamma _2 \gamma _3 $.

De plus, on a :
\[ \left\{\begin{array}{l}
     \begin{array}{c}
       \alpha + j  \beta = \lambda  \kappa_2\\
       \alpha + j^2 \beta = \lambda  \kappa_3
     \end{array}
   \end{array}\right. \]


Donc :
\begin{eqnarray*}
  \lambda \alpha & = & (\alpha + j^2 \beta) - j (\alpha + j \beta)\\
  & = & \lambda \kappa_3 - j \lambda \kappa_2
\end{eqnarray*}


Ainsi :
\[ \alpha = \kappa_3 - j \kappa_2 \]


De m{\^e}me, on trouve :
\[ \beta = j^2 (\kappa_2 - \kappa_3) \]


Soit $d$ un PGCD de $\kappa_2$ et $\kappa_3$, alors $d$ divise {\`a} la fois
$\kappa_3 - j \kappa_2 = \alpha$ et $j^2 (\kappa_2 - \kappa_3) = \beta$.

Donc $d$ est un PGCD de $\alpha$ et $\beta$.

Or, $\alpha$ et $\beta$ sont premier entre eux, alors $d$ est inversible.

Par suite, $\kappa_2$ et $\kappa_3$ sont premiers entre eux.

D'apr{\`e}s ce qui pr{\'e}c{\'e}de, $(\kappa_1, \kappa_2, \kappa_3)$
v{\'e}rifie $(P_{n - 1})$.

D'o{\`u} le r{\'e}sultat.

\

\tmtextbf{5.g.} Montrons que $\tau = \pm 1 (\tmop{mod} \langle \lambda^3
\rangle)$.

D'apr{\`e}s la question 5.e de cette partie, on a :
\[ \gamma^3_2 {{+ \tau \gamma_3^3}  }  + \lambda^{3 (n - 1)} \gamma^3_1 = 0 \]


Avec $n \geqslant 2$, alors modulo $\langle \lambda^3 \rangle$, on a :
\[ \gamma^3_2 {{+ \tau \gamma_3^3}  }  = 0 (\tmop{mod} \langle \lambda^3
   \rangle) \]


Avec $\lambda \not{| \nobracket} \gamma_2, \gamma_3$, d'apr{\`e}s la qustion 3
de cette partie, on a l'existence de $\varepsilon_2, \varepsilon_3 \in \{ - 1,
1 \}$ tels que :
\[ \left\{\begin{array}{l}
     \gamma^3_{ 2} = \varepsilon_2  (\tmop{mod} \langle \lambda^4 \rangle)\\
     \gamma^3_{ 3} = \varepsilon_3  (\tmop{mod} \langle \lambda^4 \rangle)
   \end{array}\right. \]


En particulier,
\[ \left\{\begin{array}{l}
     \gamma^3_{ 2} = \varepsilon_2  (\tmop{mod} \langle \lambda^4 \rangle)\\
     \gamma^3_{ 3} = \varepsilon_3  (\tmop{mod} \langle \lambda^4 \rangle)
   \end{array}\right. \]


Ainsi,
\[ \varepsilon_2 + \varepsilon_3 \tau = 0 (\tmop{mod} \langle \lambda^3
   \rangle) \]


D'o{\`u}
\[ \tau = \pm 1 (\tmop{mod} \langle \lambda^3 \rangle) \]


On peut facilement montrer que $- j, j, - j^2, j^2$ ne sont pas congrus {\`a}
$\pm 1$ $(\tmop{mod} \langle \lambda^3 \rangle)$.

Ainsi, $\tau \not{\in} \{ j, - j, j^2, - j^2 \}$.

\

\tmtextbf{6.} D'apr{\`e}s tout ce qu'on vu dans cette partie, on a
\[ j \in \mathbb{Z} [j]^{\times} = \{ 1, - 1, j, - j, j^2, - j^2 \} \]


Or, d'apr{\`e}s la question pr{\'e}c{\'e}dente, on a montr{\'e} que $\tau
\not{\in} \{ j, - j, j^2, - j^2 \}$.

D'o{\`u} $\tau = \pm 1$.

Et, via la question 5.f, on en d{\'e}duit que $(P_{n - 1})$ est
v{\'e}rifi{\'e}e.

Ainsi, si $(P_n)$ est v{\'e}rifi{\'e}e, alors $n \geqslant 2$ et $(P_{n - 1})$
est {\'e}galement v{\'e}rifi{\'e}e.

Par principe de r{\'e}currence, on a $(P_1)$ est v{\'e}rifi{\'e}e et $1
\geqslant 2$, ce qui est absurde !

D'o{\`u} l'{\'e}quation $x^3 + y^3 + z^3 = 0$ n'a pas de solution $(x, y, z)
\in \mathbb{Z}_{\ast}^3$ dans le cas o{\`u} $3 | x y z \nobracket$.

\

\subsubsection*{V. Le th{\'e}or{\`e}me de Fermat pour p r{\'e}gulier et $p
\not{| \nobracket} x y z$}

\

\tmtextbf{1.} Montrons que
\[ \underset{k = 0}{\overset{p - 1}{\prod}} \langle x + \zeta^k y \rangle =
   \langle z^p \rangle \]


Par d{\'e}finition, on a :
\begin{eqnarray*}
  &  & \underset{k = 0}{\overset{p - 1}{\prod}} \langle x + \zeta^k y
  \rangle\\
  & = & \left\{ \underset{i \in J}{\sum} \underset{k = 0}{\overset{p -
  1}{\prod}} x_{k, i}  | \nobracket J \tmop{est} \tmop{un} \tmop{ensemble}
  \tmop{fini}, \infixand \tmop{pour} \tmop{tout} (i, k) \in J \times
  \llbracket 0, p - 1 \rrbracket x_{k, i} \in \langle x + \zeta^k y \rangle
  \right\}
\end{eqnarray*}


Soit $t \in \underset{k = 0}{\overset{p - 1}{\prod}} \langle x + \zeta^k y
\rangle$, on a alors l'existence d'un ensemble fini $J$, et une famille
$(x_{i, k})_{(i, k) \in J \times \llbracket 0, p - 1 \rrbracket}$ tels que :
\[ t = \underset{i \in J}{\sum} \underset{k = 0}{\overset{p - 1}{\prod}} x_{k,
   i} \]


Et p$\tmop{our} \tmop{tout} (i, k) \in J \times \llbracket 0, p - 1
\rrbracket$, on a $x_{k, i} \in \langle x + \zeta^k y \rangle$.

Donc, pour tout $(i, k) \in J \times \llbracket 0, p - 1 \rrbracket$, il
existe $a_{k, i} \in \mathbb{Z} [\zeta]$ tel que $x_{k, i} = (x + \zeta^k y)
a_{k, i}$.

On a alors :
\begin{eqnarray*}
  t & = & \underset{i \in J}{\sum} \underset{k = 0}{\overset{p - 1}{\prod}}
  [(x + \zeta^k y) a_{k, i}]\\
  & = & \underset{i \in J}{\sum} \underset{k = 0}{\overset{p - 1}{\prod}} (x
  + \zeta^k y) \underset{k = 0}{\overset{p - 1}{\prod}} a_{k, i}\\
  & = & \underset{k = 0}{\overset{p - 1}{\prod}} (x + \zeta^k y) \underset{i
  \in J}{\sum} \underset{k = 0}{\overset{p - 1}{\prod}} a_{k, i}
\end{eqnarray*}


Avec $x, y \neq 0$, on a alors :
\begin{eqnarray*}
  \underset{k = 0}{\overset{p - 1}{\prod}} (x + \zeta^k y) & = & (- y)^p
  \underset{k = 0}{\overset{p - 1}{\prod}} \left( \frac{x}{- y} - \zeta^k
  \right)\\
  & = & (- y)^p \left( \left( \frac{x}{- y} \right)^p - 1 \right)\\
  & = & x^p + y^p\\
  & = & - z^p
\end{eqnarray*}


Donc :
\begin{eqnarray*}
  t & = & - z^p \underset{i \in J}{\sum} \underset{k = 0}{\overset{p -
  1}{\prod}} a_{k, i}\\
  & \in & \langle z^p \rangle
\end{eqnarray*}


D'o{\`u} :
\[ \underset{k = 0}{\overset{p - 1}{\prod}} \langle x + \zeta^k y \rangle
   \subset \langle z^p \rangle \]


R{\'e}ciproquement, soit $u \in \langle z^p \rangle$. Alors, il existe $u' \in
\mathbb{Z} [\zeta]$ tel que $u = z^p u'$. On a alors
\begin{eqnarray*}
  u & = & u' z^p\\
  & = & - u' (x^p + y^p)\\
  & = & - u' \underset{k = 0}{\overset{p - 1}{\prod}} (x + \zeta^k y)\\
  & \in & \underset{k = 0}{\overset{p - 1}{\prod}} \langle x + \zeta^k y
  \rangle
\end{eqnarray*}


D'o{\`u}
\[ \langle z^p \rangle \subset \underset{k = 0}{\overset{p - 1}{\prod}}
   \langle x + \zeta^k y \rangle \]


Par suite,
\[ \underset{k = 0}{\overset{p - 1}{\prod}} \langle x + \zeta^k y \rangle =
   \langle z^p \rangle \]


\tmtextbf{2.a.} Montrons que $\lambda y \in \mathfrak{B}$.

On a
\begin{eqnarray*}
  (x + \zeta^l y) - (x + \zeta^k y) & = & \zeta^k (\zeta^{l - k} - 1) y\\
  & = & \zeta^k (\zeta - 1) \frac{1 - \zeta^{l - k}}{1 - \zeta} y\\
  & = & - \lambda y \zeta^k \frac{1 - \zeta^{l - k}}{1 - \zeta}
\end{eqnarray*}


Donc,
\[ \lambda y = - \frac{1}{\zeta^k} \times \frac{1 - \zeta}{1 - \zeta^{l - k}}
   [(x + \zeta^l y) - (x + \zeta^k y)] \]


Avec $l - k \in \llbracket 1, p - 1 \rrbracket$, et en utilisant \tmtextbf{la
question 5.b de la partie 3}, on a :
\[ \frac{1 - \zeta}{1 - \zeta^{l - k}} \in \mathbb{Z} [\zeta]^{\times} \]


De plus, on a $N (\zeta^k) = 1$. Donc, via \tmtextbf{la question 3.b de la
partie 3}, on a $\zeta^k \in \mathbb{Z} [\zeta]^{\times}$, donc
$\frac{1}{\zeta^k} \in \mathbb{Z} [\zeta]^{\times}$. Par suite,
\[ \frac{1}{\zeta^k} \times \frac{1 - \zeta}{1 - \zeta^{l - k}} \in \mathbb{Z}
   [\zeta]^{\times} \]


Donc,
\[ \lambda y \in \langle x + \zeta^l y \rangle \cap \langle x + \zeta^k y
   \rangle \]


\tmtextbf{2.b.} Montrons que $y \nin \mathfrak{B}$.

On sait que $\mathfrak{B}$ est premier, donc on a $\lambda \in \mathfrak{B}$
ou $y \in \mathfrak{B}$.

Par l'absurde, supposons que $y \in \mathfrak{B}$. D'apr{\`e}s la question 1
de cette partie, on a $\mathfrak{B}$ divise{\tmem{}} $< z^p >$. En
particulier, $z \in \mathfrak{B}$.

Or, $y$ et $z$ sont premiers entre eux, donc d'apr{\`e}s le th{\'e}or{\`e}me
de B{\'e}zout, on a l'existence de $(u, v) \in \mathbb{Z}^2$ tel que
\[ u y + v z = 1 \]


Ainsi, $1 \in \mathfrak{B}$, ce qui est absurde !

\

Montrons que $x + y \in \langle \lambda \rangle \cap \mathbb{Z}$.

On a $y \nin \mathfrak{B}$, et $\mathfrak{B}$ est premier, donc $\lambda \in
\mathfrak{B}$, ce qui implique que $< \lambda > \subset \mathfrak{B}$.

Or, $< \lambda >$ est premier (d'apr{\`e}s \tmtextbf{la question 5 de la
partie 3}).

Or, pour tout $k \in \llbracket 0, p - 1 \rrbracket$, on a
\begin{eqnarray*}
  \zeta^k - 1 & = & \lambda \underset{j = 0}{\overset{k - 1}{\sum}} \zeta^j\\
  & = & 0 (\tmop{mod} < \lambda >)
\end{eqnarray*}


Ainsi,
\[ \zeta^k = 1 (\tmop{mod} < \lambda >) \]


Par suite,
\[ x + y = x + \zeta^k y (\tmop{mod} < \lambda >) \]


Par d{\'e}finition de $\mathfrak{B}$,
\[ x + \zeta^k y = 0 (\tmop{mod} < \lambda >) \]


Donc,
\[ x + y = 0 (\tmop{mod} < \lambda >) \]


Ainsi, $x + y \in \mathbb{Z} \cap < \lambda > = p\mathbb{Z}$.

Par suite ,
\[ p | z^p = - \nobracket (x^p + y^p) = - (x + y) \underset{k = 0}{\overset{p
   - 1}{\sum}} x^k y^{p - 1 - k} \]


Comme $p$ est premier, alors $p | z \nobracket$, ce qui est absurde avec $p |
x y z \nobracket .$

D'o{\`u} le r{\'e}sultat souhait{\'e}.

\

\tmtextbf{3.} Justifions qu'il existe un id{\'e}al $I$ tel que $< x + \zeta y
> = I^p$.

D'apr{\`e}s la question 1, on a :
\[ \underset{k = 0}{\overset{p - 1}{\prod}} \langle x + \zeta^k y \rangle =
   \langle z^p \rangle \]


Comme les id{\'e}aux $< x + \zeta^k y >$ sont deux {\`a} deux premiers entre
eux (d'apr{\`e}s la question 2.a), et en vertu du th{\'e}or{\`e}me de
d{\'e}composition unique des id{\'e}aux en produits d'id{\'e}aux premiers dans
$\mathbb{Z} [\zeta]$, on peut conclure que cette d{\'e}composition est unique.

Or, pour tout $k \in \llbracket 0, p - 1 \rrbracket$ l'id{\'e}al $\langle x +
\zeta^k y \rangle$ est une puissance $p$-i{\`e}me d'un id{\'e}al. En
particulier, cela est vrai pour $< x + \zeta y >$.

\

\tmtextbf{4.} Montrons qu'il existe $r \in \mathbb{Z}$, un r{\'e}el
$\varepsilon$ inversible de $\mathbb{Z}[\zeta]$ et $\alpha \in
\mathbb{Z}[\zeta]$ tels que
\[ x + \zeta y = \zeta^r \varepsilon \alpha^p \]


Puisque $I^p = < x + \zeta y >$ est principal, et $p$ est r{\'e}gulier. alors
$I$ est principal.

Ainsi, il existe $\alpha \in \mathbb{Z} [\zeta]$ tel que
\[ < x + \zeta y > = < \alpha^p > \]


En particulier, il existe $\omega \in \mathbb{Z} [\zeta]^{\times}$ tel que
\[ x + \zeta y = u \alpha^p \]


Or, d'apr{\`e}s \tmtextbf{la question 6 de la partie 3}, on peut {\'e}crire
$u$ sous la forme $u = \zeta^r \varepsilon$, avec $r \in \mathbb{Z}$ et
$\varepsilon \in \mathbb{Z} [\zeta]^{\times} \cap \mathbb{R}$.

D'o{\`u} le r{\'e}sultat.

\

\tmtextbf{5.} Montrons l'existence de $a \in \mathbb{Z}$ tel que $\alpha^p =
a (\tmop{mod} \langle p \rangle)$.

{\'E}crivons $\alpha = c + \zeta b$, o{\`u} $c, b \in \mathbb{Z}$.

On a
\begin{eqnarray*}
  \alpha^p & = & (c + \zeta b)^p\\
  & = & \underset{k = 0}{\overset{p}{\sum}} \left( \begin{array}{c}
    p\\
    k
  \end{array} \right) \zeta^k b^k c^{p - k}
\end{eqnarray*}


Avec $p \left| \left( \begin{array}{c}
  p\\
  k
\end{array} \right) \right.$, pour tout $k \in \llbracket 1, p - 1
\rrbracket$, alors
\[ \alpha^p = c^p + b^p  (\tmop{mod} \langle p \rangle) \]


D'o{\`u} le r{\'e}sultat. (on pose $a = c^p + b^p \in \mathbb{Z}$)

Montrons maintenant que
\[ x \zeta^{- r} + y \zeta^{1 - r} - x \zeta^r - y \zeta^{r - 1} = 0
   (\tmop{mod} \langle p \rangle) \]


On a
\begin{eqnarray*}
  x \zeta^{- r} + y \zeta^{1 - r} - x \zeta^r - y \zeta^{r - 1} & = &
  (\zeta^{- r} - \zeta^r) (x + \zeta y)\\
  & = & (\zeta^{- r} - \zeta^r) \zeta^r \varepsilon \alpha^p\\
  & = & (1 - \zeta^{2 r})  \varepsilon \alpha^p\\
  & = & (1 - \zeta^{2 r})  \varepsilon a (\tmop{mod} \langle p \rangle)
\end{eqnarray*}


Notons $r_0$ le reste de la division euclidienne de $2 r$ par $p$.

On a alors $r_0 \in \llbracket 0, p - 1 \rrbracket$, et
\begin{eqnarray*}
  x \zeta^{- r} + y \zeta^{1 - r} - x \zeta^r - y \zeta^{r - 1} & = & (1 -
  \zeta^{2 r_0})  \varepsilon a (\tmop{mod} \langle p \rangle)
\end{eqnarray*}


Or,
\begin{eqnarray*}
  (1 - \zeta^{2 r_0}) \underset{k \neq r_0}{\underset{k = 1}{\overset{p -
  1}{\prod}}} (1 - \zeta^k) & = & \underset{}{\underset{k = 1}{\overset{p -
  1}{\prod}}} (1 - \zeta^k)\\
  & = & N (1 - \zeta)\\
  & = & N (\lambda)\\
  & = & p
\end{eqnarray*}


Donc,
\[ (1 - \zeta^{2 r_0}) (1 - \zeta)^{p - 1} = p  \underset{k \neq
   r_0}{\underset{k = 1}{\overset{p - 1}{\prod}}} \frac{(1 - \zeta)}{(1 -
   \zeta^k)} \]


Ainsi,
\begin{eqnarray*}
  1 - \zeta^{2 r_0} & = & p  \underset{k \neq r_0}{\underset{k = 1}{\overset{p
  - 1}{\prod}}} \frac{(1 - \zeta)}{(1 - \zeta^k)} - (1 - \zeta^{2 r_0}) [(1 -
  \zeta)^{p - 1} - 1]\\
  & = & p  \underset{k \neq r_0}{\underset{k = 1}{\overset{p - 1}{\prod}}}
  \frac{(1 - \zeta)}{(1 - \zeta^k)} - (1 - \zeta^{2 r_0}) \underset{j =
  1}{\overset{p - 1}{\sum}} \left( \begin{array}{c}
    p - 1\\
    j
  \end{array} \right) \zeta^j\\
  & = & p  \underset{k \neq r_0}{\underset{k = 1}{\overset{p - 1}{\prod}}}
  \frac{(1 - \zeta)}{(1 - \zeta^k)} - (1 - \zeta^{2 r_0}) \underset{j =
  1}{\overset{\frac{p - 1}{2}}{\sum}} \left[ \left( \begin{array}{c}
    p - 1\\
    2 j - 1
  \end{array} \right) + \left( \begin{array}{c}
    p - 1\\
    2 j
  \end{array} \right) \zeta \right] \zeta^{2 j - 1}\\
  & = & p  \underset{k \neq r_0}{\underset{k = 1}{\overset{p - 1}{\prod}}}
  \frac{(1 - \zeta)}{(1 - \zeta^k)} - (1 - \zeta^{2 r_0}) \underset{j =
  1}{\overset{\frac{p - 1}{2}}{\sum}} \left[ \left( \begin{array}{c}
    p\\
    2 j
  \end{array} \right) - \lambda \left( \begin{array}{c}
    p - 1\\
    2 j
  \end{array} \right) \right] \zeta^j\\
  & = & p  \underset{k \neq r_0}{\underset{k = 1}{\overset{p - 1}{\prod}}}
  \frac{(1 - \zeta)}{(1 - \zeta^k)} - (1 - \zeta^{2 r_0}) \underset{j =
  1}{\overset{\frac{p - 1}{2}}{\sum}} \left( \begin{array}{c}
    p\\
    2 j
  \end{array} \right) + \lambda (1 - \zeta^{2 r_0}) \underset{j =
  1}{\overset{\frac{p - 1}{2}}{\sum}} \left( \begin{array}{c}
    p - 1\\
    2 j
  \end{array} \right) \zeta^j
\end{eqnarray*}


Avec $p | \nobracket \left( \begin{array}{c}
  p\\
  2 j
\end{array} \right)$, pour tout $j \in \left\llbracket 1, \frac{p - 1}{2}
\right\rrbracket$, donc
\[ p  \underset{k \neq r_0}{\underset{k = 1}{\overset{p - 1}{\prod}}} \frac{(1
   - \zeta)}{(1 - \zeta^k)} - (1 - \zeta^{2 r_0}) \underset{j =
   1}{\overset{\frac{p - 1}{2}}{\sum}} \left( \begin{array}{c}
     p\\
     2 j
   \end{array} \right) \in p\mathbb{Z} [\zeta] \]


De plus,
\begin{eqnarray*}
  \lambda^p & = & (1 - \zeta)^{^p}\\
  & = & \underset{k = 0}{\overset{p}{\sum}} \left( \begin{array}{c}
    p\\
    k
  \end{array} \right) \zeta^k\\
  & = & \underset{k = 1}{\overset{p - 1}{\sum}} \left( \begin{array}{c}
    p\\
    k
  \end{array} \right) \zeta^k
\end{eqnarray*}


Or, $p | \nobracket \left( \begin{array}{c}
  p\\
  k
\end{array} \right)$, pour tout $k \in \llbracket 1, p - 1 \rrbracket$, alors
$p | \lambda^p \nobracket$.

Avec, $p$ est irr{\'e}ductible dans $\mathbb{Z} \subset \mathbb{Z} [\zeta]$,
donc $p$ est irr{\'e}ductible dans $\mathbb{Z} [\zeta]$, ainsi $p | \lambda
\nobracket$.

Par suite,
\[ \lambda (1 - \zeta^{2 r_0}) \underset{j = 1}{\overset{\frac{p -
   1}{2}}{\sum}} \left( \begin{array}{c}
     p - 1\\
     2 j
   \end{array} \right) \zeta^j \in p\mathbb{Z} [\zeta] \]


D'o{\`u},
\[ (1 - \zeta^{2 r_0}) \in p\mathbb{Z} [\zeta] \]


Finalement,
\[ x \zeta^{- r} + y \zeta^{1 - r} - x \zeta^r - y \zeta^{r - 1} = 0
   (\tmop{mod} < p >) \]


\tmtextbf{6.} Supposons que $r = 0 (\tmop{mod} p\mathbb{Z})$, Montrons que $p
| y \nobracket $dans $\mathbb{Z}$.

D'apr{\`e}s la question pr{\'e}c{\'e}dente, on a :
\[ x \zeta^{- r} + y \zeta^{1 - r} - x \zeta^r - y \zeta^{r - 1} = 0
   (\tmop{mod} \langle p \rangle) \]


Puisque $r = 0 (\tmop{mod} p\mathbb{Z})$, donc $\zeta^r = 1$. Ainsi,
\[ y (\zeta^1 - \zeta^{- 1}) = 0 (\tmop{mod} \langle p \rangle) \]


Cela signifie qu'il existe un $t \in \mathbb{Z} [\zeta]$, tel que
\[ y (\zeta^1 - \zeta^{p - 1}) = p t \]


On a alors\footnote{Car $\zeta^{p - 1} \in \mathbb{Z} [\zeta]^{\times}$, alors
$N (\zeta^{p - 1}) = 1$ et d'apr{\`e}s la partie 3 on a $N (1 - \zeta) = p$ et
$N (1 + \zeta) = 1$.}
\begin{eqnarray*}
  N (y (\zeta^1 - \zeta^{p - 1})) & = & N (y \zeta^{p - 1} (\zeta  + 1) (\zeta
  - 1))\\
  & = & N (y) N (\zeta^{p - 1}) N (\zeta - 1) N (\zeta + 1)\\
  & = & - p y^{p - 1}
\end{eqnarray*}


Dautre part,
\begin{eqnarray*}
  N (y (\zeta^1 - \zeta^{p - 1})) & = & N (p t)\\
  & = & p^{p - 1} N (t)
\end{eqnarray*}


Ainsi,
\[ p^{p - 2} N (t) = - y^{p - 1} \]


Puisque $N (t) \in \mathbb{Z}$ (voir le lemme 4), cela implique $p | y^p
\nobracket$.

{\'E}tant donn{\'e} que $p$ est premier, alors $p | y \nobracket$.

{\guillemotleft} On montrerait de m{\^e}me que l'on ne peut avoir $r = 1
(\tmop{mod} p\mathbb{Z})$, ce que l'on admet. {\guillemotright}

\

\tmtextbf{7.} D'apr{\`e}s la question 5, il existe $\beta \in \mathbb{Z}
[\zeta]$ tel que
\[ x \zeta^{- r} + y \zeta^{1 - r} - x \zeta^r - y \zeta^{r - 1} = \beta p \]


Montrons que deux des entiers parmi $\pm r$ et $\pm (1 - r)$ sont {\'e}gaux
modulo $p$.

Par l'absurde, supposons qu'aucun des entiers $\pm r$, $\pm (1 - r)$ ne soit
{\'e}gal modulo $p$.

On a alors,
\[ \beta = \frac{x}{p} \zeta^{- r} + \frac{y}{p} \zeta^{1 - r} - \frac{x}{p}
   \zeta^r - \frac{y}{p} \zeta^{r - 1} \]


Or, $(1, \zeta, \ldots, \zeta^{p - 2})$ est une $\mathbb{Q}$-base de
$\mathbb{Q} (\zeta)$, et $\beta \in \mathbb{Z} [\zeta]$.

Ainsi, $\frac{x}{p} \in \mathbb{Z}$, ce qui implique que $p | x \nobracket$,
ce qui est absurde.

D'o{\`u} le r{\'e}sultat

Donc $\pm r = \pm (1 - r)  (\tmop{mod} p\mathbb{Z})$, et cela n'est possible
que pour $r = (1 - r) (\tmop{mod} p\mathbb{Z})$

Ainsi,
\[ 2 r = 1 (\tmop{mod} p\mathbb{Z}) \]


\tmtextbf{8.} Montrons que
\[ \beta p \zeta^r = (x - y) \lambda \]


D'apr{\`e}s la question pr{\'e}c{\'e}dente, on a $2 r = 1 (\tmop{mod}
p\mathbb{Z})$, alors :
\begin{eqnarray*}
  \beta p \zeta^r & = & (x \zeta^{- r} + y \zeta^{1 - r} - x \zeta^r - y
  \zeta^{r - 1}) \zeta^r\\
  & = & x + \zeta y - x \zeta^{2 r} - y \zeta^{2 r - 1}\\
  & = & x + \zeta y - x \zeta - y\\
  & = & (x - y) (1 - \zeta)\\
  & = & (x - y) \lambda
\end{eqnarray*}


Ainsi,
\begin{eqnarray*}
  N (\beta p \zeta^r) & = & N ((x - y) \lambda)\\
  & = & N (\lambda) N (x - y)\\
  & = & p (x - y)^{p - 1}
\end{eqnarray*}


Avec
\begin{eqnarray*}
  N (\beta p \zeta^r) & = & N (\beta) N (p) N (\zeta^r)\\
  & = & p^{p - 1} N (\beta)
\end{eqnarray*}


Donc
\[ (x - y)^{p - 1} = p^{p - 2} N (\beta) \]


Comme $N (\beta) \in \mathbb{Z}$, alors $p | (x - y)^{p - 2} \nobracket$.
Puisque $p$ est premier, cela implique que $p$ divise $x - y$.

D'o{\`u}
\[ x = y (\tmop{mod} p\mathbb{Z}) \]


\tmtextbf{9.} D'apr{\`e}s la question pr{\'e}c{\'e}dente, on a :
\[ x = y (\tmop{mod} p\mathbb{Z}) \]


Par sym{\'e}trie, on trouve {\'e}galement $z = y (\tmop{mod} p\mathbb{Z})$

\

\

Alors,
\[ \left\{\begin{array}{l}
     x^p = y^p  (\tmop{mod} p\mathbb{Z})\\
     z^p = y^p  (\tmop{mod} p\mathbb{Z})
   \end{array}\right. \]


Ainsi
\begin{eqnarray*}
  3 x^p & = & x^p + y^p + z^p (\tmop{mod} p\mathbb{Z})\\
  & = & 0 (\tmop{mod} p\mathbb{Z})
\end{eqnarray*}


Alors $p | 3 x^p \nobracket$, avec $p > 3$, donc $p | x^p \nobracket$,
d'o{\`u} $p | x \nobracket$, ce qui est absurde avec $p \not{| \nobracket} x y
z$.

\

\subsubsection*{Pour aller plus loin{\textdots}}

\

Pour ceux qui sont int{\'e}ress{\'e}s par la d{\'e}monstration du
th{\'e}or{\`e}me de Wiles-Fermat ({\'e}galement connu sous le nom de dernier
th{\'e}or{\`e}me de Fermat), je vous conseille de lire l'excellent livre
{\tmem{The Proof of Fermat's Last Theorem}} de Nigel Boston, en cliquant sur
le lien suivant :

\begin{center}
  https://people.math.wisc.edu/\~{}boston/869.pdf
\end{center}