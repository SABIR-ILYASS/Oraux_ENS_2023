Cet exercice explore les propri{\'e}t{\'e}s des suites complexes dont la puissance $p$-i{\`e}me tend vers $1$ et dont la moyenne arithm{\'e}tique converge. L'objectif est de caract{\'e}riser l'ensemble des valeurs possibles de cette moyenne limite.
\begin{exercise}[]
Soit $n \geqslant 1$ et soit $P$ est une matrice de $M_n (\mathbb{C})$. On a l'{\'e}quivalence :

\begin{center}
  Le rayon spectral de $P$ $< 1$ si et seulement si $\underset{k \to + \infty}{\lim} P^k = 0$.
\end{center}

\end{exercise}

\subsection*{Solution. (SABIR Ilyass)}
\addcontentsline{toc}{subsection}{Solution. (SABIR Ilyass)}


$\Leftarrow$) Supposons que $\underset{k \to + \infty}{\lim} P^k = 0$

Soit $n \geqslant 1$ et soit $P$ est une matrice de $M_n (\mathbb{C})$.

Pour toute norme $N$ sur $M_{n, 1} (\mathbb{C})$, on note :
\[ \mathcal{N}(N) : T \in M_n (\mathbb{C}) \longmapsto \underset{x \neq
   0}{\sup}  \frac{N (Tx)}{N (x)} \]


$\mathcal{N}(N)$ est une norme sur $M_n (\mathbb{C})$. De plus, pour tout $(x,
T) \in M_{n, 1} (\mathbb{C}) \times M_n (\mathbb{C})$ :
\[ N (Tx) \leq \mathcal{N}(N) (T) N (x) \quad (\kreuz) \]


Fixons maintenant une norme $N$ de $M_n (\mathbb{C})$. Soit $\lambda$ une
valeur propre de $P$ et $x_{\lambda}$ un vecteur propre de $P$ associ{\'e}
{\`a} $\lambda$.

D'apr{\`e}s $(\kreuz)$, pour tout $k \in \mathbb{N}$ :
\[ N (P^k x_{\lambda}) \leq \mathcal{N} (N) (P)^k N (x_{\lambda}) \]


Ainsi, pour tout $k \in \mathbb{N}$
\[ | \lambda |^k N (x_{\lambda}) \leq \mathcal{N}(P)^k N (x_{\lambda}) \]


Avec $N (x_{\lambda}) \neq 0$ (puisque $x_{\lambda} \neq 0$), on en d{\'e}duit
que pour tout $k \in \mathbb{N}$ :
\[ | \lambda |^k \leq \mathcal{N}(P^k) \]


Comme $\underset{k \rightarrow + \infty}{\lim} P^k = 0$ et $\dim M_n
(\mathbb{C}) < + \infty$, donc toutes les normes de $M_n (\mathbb{C})$ sont
{\'e}quivalentes.

On a donc $\underset{k \rightarrow + \infty}{\lim} P^k = 0$ dans l'espace
$(M_n (\mathbb{C}), \mathcal{N}(N))$.

\

Puisque $\underset{k \rightarrow + \infty}{\lim} \mathcal{N}(P^k) = 0$, alors
$\underset{k \rightarrow + \infty}{\lim} | \lambda |^k = 0$.

\

Ceci n'est possible que si $| \lambda | < 1$, et {\c c}a pour tout $\lambda
\in \mathrm{Sp} (P)$.

Ainsi,
\[ \rho (P) = \underset{\lambda \in \mathrm{Sp} (P)}{\max } | \lambda | < 1.
\]


$\Rightarrow$) Supposons que $\rho (P) = \underset{\lambda \in \mathrm{Sp}
(P)}{\max } | \lambda | < 1$.

\

Toute matrice carr{\'e}e $P$ peut {\^e}tre mise sous forme de Jordan. Il
existe une matrice inversible $S$ telle que :
\[ P = SJS^{- 1} \]


o{\`u} $J$ est la matrice de Jordan compos{\'e}e de blocs associ{\'e}s aux
valeurs propres de $P$.

Donc pour tout $k \in \mathbb{N}$ :
\[ P^k = (SJS^{- 1})^k = SJ^k S^{- 1} \]


Chaque bloc de Jordan $J_i$ associ{\'e} {\`a} une valeur propre $\lambda_i$ a
la forme :
\[ J_i = \lambda_i I + N \]


o{\`u} $N$ est une matrice nilpotente.

La $k$-i{\`e}me puissance d'un bloc de Jordan est donn{\'e}e par :
\begin{eqnarray*}
  J_i^k & = & \lambda_i^k \underset{j = 0}{\overset{k}{\sum}} \left(
  \begin{array}{c}
    k\\
    j
  \end{array} \right) \frac{1}{\lambda^j_i} N^j
\end{eqnarray*}


Comme $| \lambda_i | < 1$, les termes $\lambda_i^k$ tendent vers 0. M{\^e}me
en tenant compte des puissances de $k$ dues aux termes en $N$, l'ensemble tend
vers 0 :
\[ \underset{k \rightarrow + \infty}{\lim} J_i^k = 0 \]


Puisque chaque bloc $J_i^k$ tend vers la matrice nulle, il en est de m{\^e}me
pour $J^k$, et donc :
\begin{eqnarray*}
  \underset{k \rightarrow + \infty}{\lim} P^k & = & \underset{k \rightarrow +
  \infty}{\lim} SJ^k S^{- 1}\\
  & = & S \left( \underset{k \rightarrow + \infty}{\lim} J^k \right) S^{- 1}
  \\
  & = & 0
\end{eqnarray*}


D'o{\`u} l'{\'e}quivalence.
\[ \maltese \maltese \maltese \maltese \maltese \maltese \maltese \]
