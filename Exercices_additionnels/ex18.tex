La caract{\'e}risation des boules unit{\'e}s ferm{\'e}es en dimension finie
fait appel {\`a} des propri{\'e}t{\'e}s g{\'e}om{\'e}triques et topologiques
fondamentales. Cet exercice {\'e}tablit un pont {\'e}l{\'e}gant entre la
th{\'e}orie des espaces norm{\'e}s et les propri{\'e}t{\'e}s
g{\'e}om{\'e}triques des ensembles convexes.
\begin{exercise}[]
Soit $E$ un espace vectoriel de dimension finie muni de sa topologie d'espace
vectoriel, et $B$ une partie de $E$. Montrer l'{\'e}quivalence :

il existe une norme $N$ sur $E$ telle que $B$ est la boule unit{\'e}
ferm{\'e}e de $(E, N)$ si, et seulement si, $B$ est convexe, compacte,
sym{\'e}trique par rapport {\`a} l'origine, et d'int{\'e}rieur non vide.

\end{exercise}

\subsection*{Solution. (SABIR Ilyass)}
\addcontentsline{toc}{subsection}{Solution. (SABIR Ilyass)}

$\Longrightarrow)$ Supposons qu'il existe une norme $N$ sur $E$ telle que $B$
est la boule ferm{\'e}e unit{\'e} de ($E, N$).

Alors, $B$ est convexe. En effet, pour tout $\lambda \in [0, 1]$ et pour tout
$x, y \in B$, on a :
\[ N ((1 - \lambda) x + \lambda y) \leq (1 - \lambda) N (x) + \lambda N (y)
   \leq 1 \]
Ainsi, ($1 - \lambda) x + \lambda y \in B$, d'o{\`u} $B$ est convexe.

$B$ est compacte, car elle est ferm{\'e}e et born{\'e}e (en dimension finie).

De plus, pour tout $x \in B$, on a $N (- x) = N (x) \leq 1$, donc $- x \in B$.
D'o{\`u} $B$ est sym{\'e}trique par rapport {\`a} $0$.

Il reste {\`a} montrer que $B$ est d'int{\'e}rieur non vide. En effet, pour $x
\neq 0$, on a $\frac{1}{2 N (x)} x \in B^{\circ}$. Donc $B$ est
d'int{\'e}rieur non vide.

$\Longleftarrow$) Supposons maintenant que $B$ est convexe, compacte,
sym{\'e}trique par rapport {\`a} l'origine, et d'int{\'e}rieur non vide, et
montrons qu'il existe une norme $N$ sur $E$ telle que $B$ est la boule
unit{\'e} ferm{\'e}e de ($E, N$).

Pour cela, l'id{\'e}e est d'utiliser l'homog{\'e}n{\'e}it{\'e}. Pour $x$
vecteur non nul de $E$, posons :
\[ T_x =\{\lambda > 0 \mid \frac{x}{\lambda} \in B\} \]


Montrons d'abord que cet ensemble est non vide. En effet, l'origine est
forc{\'e}ment un point int{\'e}rieur {\`a} $B$, car comme $B$ est
d'int{\'e}rieur non vide, on peut trouver $y \in B$ et $r > 0$, tel que $B (y,
r) \subset B$.

Par sym{\'e}trie par rapport {\`a} l'origine, on a {\'e}galement $B (- y, r)
\subset B$. Par convexit{\'e}, il en d{\'e}coule que $B (0, r) \in B$.

Ainsi, comme $B$ est convexe, et contient l'origine, si $\lambda \in T_x$,
alors $[\lambda, + \infty [\subset T_x$. Donc $T_x$ est un intervalle non
major{\'e} de $\mathbb{R}^{\ast}_+$.

Comme $B$ est compacte, elle est born{\'e}e. Soit $M > 0$ tel que $\|a\| \leq
M$ pour tout $a \in B$. (Pour une norme quelconque $\| \cdot \|$, puisque
toutes les normes sont {\'e}quivalentes en dimension finie). Si $\lambda \in
T_x$, on a $\lambda \geq \frac{\|x\|}{M} > 0$. Posons alors :
\[ N (x) = \inf T_x \]


On pose aussi $N (0) = 0$

On vient de prouver qu'il s'agit d'un r{\'e}el strictement positif. Comme $B$
est ferm{\'e}e, l'intervalle $T_x$ est aussi ferm{\'e} et il est donc {\'e}gal
{\`a} $[N (x), + \infty [$.

Il reste {\`a} montrer que $N$ est une norme et que $B$ en est la boule
unit{\'e} ferm{\'e}e.

L'application $N$ est positive car pour tout $x \in E$ non nul, on a :
\[ T_x \subset] 0, + \infty [ \]


Donc :
\[ N (x) = \inf (T_x) \geq \inf (] 0, + \infty [) = 0 \]


Et :
\[ N (0) = 0 \]


De plus, d'apr{\`e}s ce qui pr{\'e}c{\`e}de, on a pour tout $x \in E$ non nul
:
\[ \forall \lambda \in T_x, \lambda \geq \frac{\|x\|}{M} \]


Donc, pour tout $x \in E$ non nul, on a :
\[ N (x) \geq \frac{\|x\|}{M} > 0 \]


Donc l'axiome de s{\'e}paration est v{\'e}rifi{\'e}.

Si $x \in E$ est non nul et si $\mu > 0$, on a pour tout $\lambda > 0$ :
\[ \lambda \in T_{\mu x} \Longleftrightarrow \frac{\mu x}{\lambda} \in B
   \Longleftrightarrow \frac{\mu}{\lambda} x \in B \Longleftrightarrow
   \frac{\lambda}{\mu} \in T_x \Longleftrightarrow \lambda \in \mu T_x \]


Donc :
\[ T_{\mu x} = \frac{1}{\mu} T_x \]


Ainsi :
\[ N (\mu x) = \inf T_{\mu x} = \inf \mu T_x = \mu \inf T_x = \mu N (x) \]


Par sym{\'e}trie de $B$, on a pour tout $x \in E$ non nul $T_{- x} = T_x$,
donc $N (- x) = N (x$). Cela reste vrai pour $x = 0$. Donc $N$ est
homog{\`e}ne.

Il ne reste qu'{\`a} montrer que $N$ v{\'e}rifie l'in{\'e}galit{\'e}
triangulaire. Pour cela, on va montrer d'abord le lemme suivant :

\tmtextbf{Lemme 1. }

Soit $\Phi : E \to \mathbb{R}_+$ v{\'e}rifiant pour tout $x \in E$ et pour
tout $\lambda \in \mathbb{R}$ :
\[ \Phi (x) = 0 \Longleftrightarrow x = 0 \text{et } \Phi (\lambda x) = |
   \lambda | \Phi (x) \]


On a : $\Phi$ est une norme si, et seulement si, l'ensemble $\{x \in E, \Phi
(x) \leq 1\}$ est convexe.

\

\tmtextbf{Preuve du lemme 1.}

Si $\Phi$ est une norme, il est clair que $B$ est convexe. En effet, si ($x,
y) \in B^2$ et $t \in [0, 1]$, on a :
\[ \Phi ((1 - t) x + ty) \leq (1 - t) \Phi (x) + t \Phi (y) \leq (1 - t) + t =
   1 \]


Donc :
\[ (1 - t) x + ty \in B \]


D'o{\`u} $B$ est convexe.

\

R{\'e}ciproquement, supposons que $B$ est convexe, consid{\'e}rons $x$ et $y$
dans $E$. On veut prouver que :
\[ \Phi (x + y) \leq \Phi (x) + \Phi (y) \]


On peut supposer que $x$ et $y$ non nuls, sans quoi l'in{\'e}galit{\'e} est
triviale.

Par homog{\'e}n{\'e}it{\'e}, les vecteurs $\frac{x}{\Phi (x)}$ et
$\frac{y}{\Phi (y)}$ sont dans $B$. Il en est donc de m{\^e}me de leur
barycentre $z$ affect{\'e} des masses positives $\Phi (x$) et $\Phi (y$). On a
:
\[ z = \frac{x + y}{\Phi (x) + \Phi (y)} \]


Le fait que $\Phi (z) \leq 1$ conduit {\`a} $\Phi (x + y) \leq \Phi (x) + \Phi
(y$).

D'o{\`u} l'{\'e}quivalence.

Pour tout $x \in E$, on a $N (x) \leq 1$ si, et seulement si, $1 \in T_x$,
donc si, et seulement si, $x \in B$. Ainsi :
\[ \{x \in E \mid N (x) \leq 1\}= B \]


Comme $B$ est convexe, on en d{\'e}duit d'apr{\`e}s le lemme que $N$
v{\'e}rifie l'in{\'e}galit{\'e} triangulaire.

D'o{\`u} le r{\'e}sultat.
\[ \maltese \maltese \maltese \maltese \maltese \maltese \maltese \]
