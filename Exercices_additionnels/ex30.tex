La caract{\'e}risation des polyn{\^o}mes positifs sur $\mathbb{R}^+$ fait
appel {\`a} des techniques de d{\'e}composition en sommes de carr{\'e}s,
distinctes du cas g{\'e}n{\'e}ral sur \ensuremath{\mathbb{R}} tout entier. Cet
exercice propose d'{\'e}tablir une {\'e}quivalence {\'e}l{\'e}gante
adapt{\'e}e au cas des r{\'e}els positifs.
\begin{exercise}[(Oral de l'X 2016)]
Soit $P \in \mathbb{R}[X]$. Montrer que $P$ est positif sur $\mathbb{R}^+$, si
et seulement s'il existe $A, B \in \mathbb{R}[X]$ tels que $P = A^2 + XB^2$.
\end{exercise}

\subsection*{Solution. (SABIR Ilyass)}
\addcontentsline{toc}{subsection}{Solution. (SABIR Ilyass)}

Notons
\[ \mathcal{H}=\{P \in \mathbb{R}[X] | \nobracket \exists A, B \in
   \mathbb{R}[X] \text{ tels que } P = A^2 + XB^2 \} \]


Montrons que $\mathcal{H}$ est stable par multiplication.

Soit $(P, Q) \in \mathcal{H}^2$. Il existe alors $A_1, B_1 ; A_2, B_2 \in
\mathbb{R}[X]$ tels que :
\[ P = A_1^2 + XB_1^2 \infixand Q = A_2^2 + XB_2^2 \]


On a donc :
\begin{eqnarray*}
  PQ & = & (A_1^2 + XB_1^2) (A_2^2 + XB_2^2)\\
  & = & (A_1 A_2)^2 + (XB_1 B_2)^2 + X (A_1^2 B_2^2 + A_2^2 B_1^2)\\
  & = & (A_1 A_2 - XB_1 B_2)^2 + X (A_1 B_2 + A_2 B_1)^2
\end{eqnarray*}


Ce qui montre que $PQ \in \mathcal{H}$.

Pour montrer que pour tout $P \in \mathbb{R}[X]$, $P$ est positif sur
$\mathbb{R}^+$ si et seulement s'il existe $A, B \in \mathbb{R}[X]$ tels que
\[ P = A^2 + XB^2 \]


Il suffit de montrer l'{\'e}quivalence :

Pour tout $P \in \mathbb{R}[X]$ irr{\'e}ductible dans $\mathbb{R}[X]$,

$P$ est positif sur $\mathbb{R}^+$ si, et seulement si, il existe $A, B \in
\mathbb{R}[X]$ tels que $P = A^2 + XB^2$.

\

$\Leftarrow$) Notons tout d'abord que la r{\'e}ciproque est {\'e}vidente.

$\Rightarrow$) Soit $P \in \mathbb{R}[X]$ irr{\'e}ductible dans
$\mathbb{R}[X]$.

- \tmtextbf{Si $\deg P = 1$} : {\'e}crivons $P = aX + b$ avec $(a, b) \in
\mathbb{R}^{\ast} \times \mathbb{R}$

Puisque pour tout $x \geq 0$, on a $P (x) \geq 0$

D'o{\`u} $P (0) = b \geq 0$ et $\underset{x \to + \infty}{\lim} P (x) \geq 0$
donc $a \geq 0$

On a donc
\[ P = (\sqrt{b})^2 + X (\sqrt{a})^2 \in \mathcal{H} \]


-\tmtextbf{ Si $\deg P = 2$} : {\'e}crivons $P = aX^2 + bX + c$ avec $(a, b,
c) \in \mathbb{R}^{\ast} \times \mathbb{R} \times \mathbb{R}$ et $b^2 < 4 ac$

Puisque pour tout $x \geq 0, \hspace{0.27em} P (x) \geq 0$

Alors $P (0) = c \geq 0$ et $\underset{x \to + \infty}{\lim} P (x) \geq 0$
donc $a \geq 0$

On a $| b | < 2 \sqrt{a c}$, alors
\[ P = (\sqrt{a} X - \sqrt{c})^2 + (b + 2 \sqrt{ac}) X \]


avec $b + 2 \sqrt{ac} \geq |b| + b \geqslant 0$

D'o{\`u}
\[ P = (\sqrt{a} X - \sqrt{c})^2 + X (\sqrt{b + 2 \sqrt{ac}})^2 \in
   \mathcal{H} \]


On obtient ainsi le r{\'e}sultat souhait{\'e} !
\[ \maltese \maltese \maltese \maltese \maltese \maltese \maltese \]


