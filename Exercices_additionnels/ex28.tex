Cet exercice explore l'existence d'un point fixe commun pour deux fonctions
continues $f$ et $g$ sur $[0, 1]$, en supposant que $f$ est monotone et que $f
\circ g = g \circ f$. Nous d{\'e}montrerons qu'il existe un point $c \in [0,
1]$ tel que $f (c) = g (c) = c$.
\begin{exercise}[(Oral ULM 2008)]
Soient $f$ et $g$ deux fonctions continues sur $[0, 1] \rightarrow [0, 1]$
v{\'e}rifiant $f \circ g = g \circ f$, on suppose que $f$ est monotone,
montrer qu'il existe $c \in [0, 1]$ tel que
\[ f (c) = g (c) = c \]
\end{exercise}

\subsection*{Solution. (SABIR Ilyass)}
\addcontentsline{toc}{subsection}{Solution. (SABIR Ilyass)}

Pour prouver qu'il existe $c \in [0, 1]$ tel que $f (c) = g (c) = c$, nous
utiliserons les propri{\'e}t{\'e}s des fonctions continues et monotones, ainsi
que la propri{\'e}t{\'e} de commutation $f \circ g = g \circ f$.

{\'E}tant donn{\'e} que $f$ est continue et monotone sur l'intervalle $[0,
1]$, l'ensemble de ses points fixes,
\[ \text{Fix} (f) =\{x \in [0, 1] \mid f (x) = x\}, \]
est un intervalle ferm{\'e} $[\alpha, \beta] \subseteq [0, 1]$.

En effet, si $x_1, x_2 \in \text{Fix} (f)$ avec $x_1 < x_2$, alors pour tout
$x$ entre $x_1$ et $x_2$, $f (x)$ sera compris entre $f (x_1)$ et $f (x_2)$,
qui sont respectivement {\'e}gaux {\`a} $x_1$ et $x_2$. Ainsi, $f (x) \geq x$
ou $f (x) \leq x$, selon que $f$ est croissante ou d{\'e}croissante. Cela
garantit qu'il n'y a pas de sauts, donc $f (x) = x$ pour tous les $x \in [x_1,
x_2]$.

{\'E}tant donn{\'e} que $f \circ g = g \circ f$, pour tout $c \in \text{Fix}
(f)$, on a
\[ f (g (c)) = g (f (c)) = g (c) . \]


Cela implique $f (g (c)) = g (c)$, donc $g (c) \in \text{Fix} (f)$. Par
cons{\'e}quent, $g$ applique Fix$(f)$ dans lui-m{\^e}me :
\[ g : \text{Fix} (f) \rightarrow \text{Fix} (f) . \]


Puisque Fix($f$) est un intervalle ferm{\'e} $[\alpha, \beta]$ et que $g$ est
continue, la restriction de $g$ {\`a} Fix($f$) est une fonction continue de
$[\alpha, \beta]$ dans lui-m{\^e}me. Par le th{\'e}or{\`e}me du point fixe de
Brouwer en dimension un (qui affirme que toute fonction continue d'un
intervalle ferm{\'e} dans lui-m{\^e}me a au moins un point fixe, [voir
l'exercice 7 ULRS, lemme 1]), il existe un $c \in [\alpha, \beta]$ tel que :
\[ g (c) = c. \]


Puisque $c \in \text{Fix} (f)$, on a d{\'e}j{\`a} $f (c) = c$. On a
{\'e}galement $g (c) = c$. Par cons{\'e}quent :
\[ f (c) = c \quad \text{et} \quad g (c) = c. \]

\[ \maltese \maltese \maltese \maltese \maltese \maltese \maltese \]
