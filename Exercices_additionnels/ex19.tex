Cet exercice {\'e}tudie le rayon de convergence d'une s{\'e}rie enti{\`e}re
dont le terme g{\'e}n{\'e}ral fait intervenir une fonction trigonom{\'e}trique
de la racine carr{\'e}e de l'indice. La singularit{\'e} de ce terme
g{\'e}n{\'e}ral n{\'e}cessite une analyse fine du comportement asymptotique.

\begin{exercise}[(G{\'e}n{\'e}ralisation d'un exercice pos{\'e} {\`a}
l'oral de l'ENS Cachan)]
Soit $n_0$ un entier strictement positif tel que $\sqrt{n_0} \nin \mathbb{Q}$.
D{\'e}terminer le rayon de convergence de
\[ \underset{n \geqslant 1}{\sum} \frac{z^n}{\sin (\sqrt{n_0 } \pi n)} \]

\end{exercise}

\subsection*{Solution. (SABIR Ilyass)}
\addcontentsline{toc}{subsection}{Solution. (SABIR Ilyass)}

Notons $\alpha = \sqrt{n_0 }$. L'irrationalit{\'e} de $\alpha$ assure que pour
tout $n \in \mathbb{N}^{\ast}$,
\[ \sin (\alpha \pi n) \neq 0 \]


Pour $z = 1$, la s{\'e}rie $\underset{n \geqslant 1}{\sum} \frac{1}{\sin
(\alpha \pi n)}$ diverge, puisque pour tout $n \in \mathbb{N}^{\ast}$
\[ | \sin (\alpha \pi n) | \leq 1 \]


Donc, le rayon de convergence $R$ v{\'e}rifie $R \leq 1$.

Pour obtenir une minoration de $R$, on va majorer $\frac{1}{\sin (\alpha \pi
n)}$, donc minorer $\sin (\alpha \pi n)$.

On note pour tout $n \in \mathbb{N}^{\ast}$
\[ A_n =\{r \in \mathbb{N}: r \leq nd + \frac{1}{2} \} \]


Pour tout $n \in \mathbb{N}^{\ast}$, $A_n$ est une partie non vide de
$\mathbb{N}$ (puisque $0 \in A_n$) et major{\'e}e, donc $A_n$ admet un plus
grand {\'e}l{\'e}ment. Notons $P_n = \max A_n$

On a donc, pour tout $n \in \mathbb{N}^{\ast}$,
\[ P_n \leqslant n \alpha + \frac{1}{2} \infixand P_n + 1 > n \alpha +
   \frac{1}{2} \]


Ainsi,
\[ P_n - \frac{1}{2} \leqslant n \alpha < P_n + \frac{1}{2} \]


Posons pour tout $n \in \mathbb{N}^{\ast}$,
\[ \varepsilon_n = n \alpha - P_n \in \left[ - \frac{1}{2}, \frac{1}{2} 
   \right[ \]


Par suite, pour tout $n \in \mathbb{N}^{\ast}$,
\begin{eqnarray*}
  \sin (\alpha \pi n) & = & \sin (\pi \varepsilon_n)
\end{eqnarray*}


Avec la fonction $\sin$ est concave sur $\left[ 0, \frac{\pi}{2} \right]$,
donc pour tout $x \in \left[ 0, \frac{\pi}{2} \right]$,
\[ \sin (x) \geq \frac{2}{\pi} x \]


Ainsi, pour tout $x \in \left[ - \frac{\pi}{2}, \frac{\pi}{2} \right]$,
\[ | \sin (x) | \geqslant \frac{2}{\pi} | x | \]


Par cons{\'e}quent, pour tout $n \in \mathbb{N}^{\ast}$,
\begin{eqnarray*}
  &  & 
\end{eqnarray*}


Ce dernier terme est en $O (n)$. Il en r{\'e}sulte que pour tout nombre
complexe $z$ v{\'e}rifiant $|z| < 1$, la s{\'e}rie $\underset{n \geqslant
1}{\sum} \frac{z^n}{\sin (d \pi n)}$

converge, d'o{\`u} $R = 1$.
\[ \maltese \maltese \maltese \maltese \maltese \maltese \maltese \]
