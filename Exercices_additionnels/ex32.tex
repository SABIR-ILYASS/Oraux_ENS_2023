L'{\'e}tude des propri{\'e}t{\'e}s arithm{\'e}tiques des sommes harmoniques
r{\'e}v{\`e}le des r{\'e}sultats surprenants. Cet exercice combine la
th{\'e}orie des nombres et l'analyse pour {\'e}tablir une propri{\'e}t{\'e}
remarquable concernant la divisibilit{\'e} de ces sommes par des carr{\'e}s de
nombres premiers.
\begin{exercise}[(le th{\'e}or{\`e}me de Wolstenholme)]
Pour tout $n \in \mathbb{N}^{\ast}$, on note :
\[ \overset{n - 1}{\underset{k = 1}{\sum}} \frac{1}{k} = \frac{a_n}{b_n},
   \tmop{avec} a_n, b_n \in \mathbb{N} \tmop{tels} \tmop{que} a_n \wedge b_n =
   1 \]
   
Montrer que pour tout nombre premier $p \geqslant 5$, on a $p^2$ divise $a_p$.

\end{exercise}

\subsection*{Solution. (SABIR Ilyass)}
\addcontentsline{toc}{subsection}{Solution. (SABIR Ilyass)}


Soit $p \geqslant 5$, un nombre premier. Dans $\mathbb{Z}/ p\mathbb{Z}$, on a
:
\begin{eqnarray*}
  \frac{2}{p} \overset{p - 1}{\underset{k = 1}{\sum}} \frac{1}{k} & = &
  \frac{2}{p} \left( \overset{(p - 1) / 2}{\underset{k = 1}{\sum}} \frac{1}{k}
  + \overset{p - 1}{\underset{k = 1 + (p - 1) / 2}{\sum}} \frac{1}{k}
  \right)\\
  & = & \frac{2}{p} \overset{(p - 1) / 2}{\underset{k = 1}{\sum}} \frac{1}{k}
  + \frac{1}{p - k}\\
  & = & 2 \overset{(p - 1) / 2}{\underset{k = 1}{\sum}} \frac{1}{k (p - k)}\\
  & = & - \overset{p - 1}{\underset{k = 1}{\sum}} \frac{1}{k^2}\\
  & = & - \overset{p - 1}{\underset{k = 1}{\sum}} k^2\\
  & = & - \frac{p (p - 1) (2 p - 1)}{6}
\end{eqnarray*}


Ainsi,
\[ \overset{p - 1}{\underset{k = 1}{\sum}} \frac{1}{k} = - \frac{(p - 1) (2 p
   - 1)}{12} p^2 \]


D'o{\`u} le r{\'e}sultat.
\[ \maltese \maltese \maltese \maltese \maltese \maltese \maltese \]