Cet exercice explore les propri{\'e}t{\'e}s des racines d'un polyn{\^o}me $P$ et de sa d{\'e}riv{\'e}e $P^{\prime}$, ainsi que les liens entre les racines multiples et le signe d'une expression associ{\'e}e {\`a} $P$ et ses d{\'e}riv{\'e}es.
\begin{exercise}[(Oral de l'X 2016)]
Soit $P \in \mathbb{R}[X]$ d{\'e}finie sur $\mathbb{R}$.

1. Montrer que toute racine multiple de $P'$ est racine de $P$

2. Pour $x \in \mathbb{R}$, quel est le signe de $P (x) P'' (x) - P' (x)^2$ ?
\end{exercise}

\subsection*{Solution. (SABIR Ilyass)}
\addcontentsline{toc}{subsection}{Solution. (SABIR Ilyass)}

1. La premi{\`e}re question est classique. {\'E}crivons
\[ P = a \prod_{i = 1}^r (X - \alpha_i)^{n_i}  \prod_{i = 1}^{\ell} (X -
   \beta_i) \]


o{\`u} $\alpha_1, \ldots, \alpha_r$ sont les racines multiples de $P$ et
$\beta_1, \ldots, \beta_{\ell}$ sont les racines simples de $P$. Les
multiplicit{\'e}s $n_1, \ldots, n_r \geq 2$ repr{\'e}sentent les ordres des
racines $\alpha_1, \ldots, \alpha_r$ comme {\'e}tant racines de $P$.
($\alpha_1, \ldots, \alpha_r, \beta_1, \ldots, \beta_{\ell}$ sont deux {\`a}
deux distincts)

On a, pour tout $i \in \llbracket 1, r \rrbracket$,
\[ P (\alpha_i) = P' (\alpha_i) = P'' (\alpha_i) = \cdots = P^{(n_i - 1)}
   (\alpha_i) = 0 \infixand P^{(n_i)} (\alpha_i) \neq 0 \]


\

En particulier, pour tout $i \in \llbracket 1, r \rrbracket$,
\[ P' (\alpha_i) = (P')^{(1)} (\alpha_i) = \cdots = (P')^{(n_i - 2)}
   (\alpha_i) = 0 \infixand (P')^{(n_i - 1)} (\alpha_i) \neq 0 \]


Cela montre que, pour tout $i \in \llbracket 1, r \rrbracket$, $\alpha_i$ est
une racine de $P'$ de multiplicit{\'e} $n_i - 1$.

\

En ordonnant $\alpha_1, \ldots, \alpha_r, \beta_1, \ldots, \beta_{\ell}$ par
ordre croissant, posons, pour tout $i \in \llbracket 1, r + \ell \rrbracket$
\[ \left\{\begin{array}{l}
     \lambda_1 = \min \{\alpha_1, \ldots, \alpha_r, \beta_1, \ldots,
     \beta_{\ell} \}\\
     \lambda_i = \min \{(\alpha_1, \ldots, \alpha_r, \beta_1, \ldots,
     \beta_{\ell}) \setminus \{\lambda_1, \ldots, \lambda_{i - 1} \}\},
     \tmop{si} i \geqslant 2
   \end{array}\right. \]
\[ \  \]


Ainsi, $\lambda_1, \ldots, \lambda_{\ell + r}$ sont exactement les racines de
$P$ deux {\`a} deux distincts et $\lambda_1 < \lambda_2 < \cdots <
\lambda_{\ell + r}$.

Pour tout $i \in \llbracket 1, \ell + r - 1 \rrbracket$, on a $P$ est continue
sur $[\lambda_i, \lambda_{i + 1}]$ et d{\'e}rivable sur $] \lambda_i,
\lambda_{i + 1} [$ avec
\[ P (\lambda_i) = P (\lambda_{i + 1}) \]


Alors d'apr{\`e}s le th{\'e}or{\`e}me de Rolle, il existe $c_i \in] \lambda_i,
\lambda_{i + 1} [$ tel que $P' (c_i) = 0$

Donc, on a pour tout $i \in \llbracket 1, r \rrbracket$ $\alpha_i$ est une
racine de $P$ de multiplicit{\'e} $n_i - 1$, et pour tout $i \in \llbracket 1,
r + \ell - 1 \rrbracket$ $c_i$ est une racine de $P'$.

Ainsi, il existe $Q \in \mathbb{R}[X]$ tel que
\[ P' = a Q \prod_{i = 1}^r (X - \alpha_i)^{n_i - 1}  \prod_{i = 1}^{\ell + r
   - 1} (X - c_i) \]


Donc $P'$ est scind{\'e} sur $\mathbb{R}$.

\

Concernant $P'$ : une racine $\gamma$ multiple de $P'$ ne peut pas {\^e}tre
un certain $c_i$ puisque $c_1, \ldots, c_{l + r - 1}$ sont des racines simples
de $P'$. alors $\gamma$ est une racine de $P$.

2- On reprend les notations de la question 1. La fonction $x \mapsto \frac{P'
(x)}{P (x)}$ est d{\'e}rivable sur $\mathbb{R} \setminus \{\alpha_1, \ldots,
\alpha_r, \beta_1, \ldots, \beta_r \}$, et on a
\[ \frac{P'}{P} = \sum_{i = 1}^r \frac{n_i}{x - \alpha_i} + \sum_{i = 1}^l
   \frac{1}{x - \beta_i} \]


Donc,
\[ \frac{P \cdot P'' - (P')^2}{P^2} = - \sum_{i = 1}^r \frac{n_i}{(x -
   \alpha_i)^2} - \sum_{i = 1}^l \frac{1}{(x - \beta_i)^2} \]


Par cons{\'e}quent, pour tout $x \in \mathbb{R} \setminus \{\alpha_1, \ldots,
\alpha_r, \beta_1, \ldots, \beta_r \}$, on a
\[ \frac{P (x) P'' (x) - (P' (x))^2}{P^2 (x)} = - \sum_{i = 1}^r \frac{n_i}{(x
   - \alpha_i)^2} - \sum_{i = 1}^l \frac{1}{(x - \beta_i)^2} < 0 \]


Donc, pour tout $x \in \mathbb{R} \setminus \{\alpha_1, \ldots, \alpha_r,
\beta_1, \ldots, \beta_r \}$,
\[ P (x) P'' (x) - (P' (x))^2 \leq 0 \]


Par continuit{\'e} de $x \mapsto P (x) P'' (x) - (P' (x))^2$, on en d{\'e}duit
que, pour tout $x \in \mathbb{R}$
\[ P (x) P'' (x) - (P' (x))^2 \leq 0 \]


\tmtextbf{Remarque.}

{\`A} partir de la question 1,
\[ P' \wedge P = \underset{i = 1}{\overset{r}{\prod}} (X - \alpha_i)^{n_i - 1}
\]


Donc,
\[ \deg (P' \wedge P) = \underset{i = 1}{\overset{r}{\sum}} (n_i - 1) \]


D'o{\`u}
\begin{eqnarray*}
  \deg P - \deg (P' \wedge P) & = & l + \underset{i = 1}{\overset{r}{\sum}}
  n_i - \underset{i = 1}{\overset{r}{\sum}} (n_i - 1)\\
  & = & l + r
\end{eqnarray*}


Par cons{\'e}quent, le nombre de racines distincts de $P$ est $\deg P - \deg
(P' \wedge P)$.
\[ \maltese \maltese \maltese \maltese \maltese \maltese \maltese \]