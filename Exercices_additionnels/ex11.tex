Les in{\'e}galit{\'e}s entre moyennes sont au c{\oe}ur de l'{\'e}tude des
in{\'e}galit{\'e}s en g{\'e}n{\'e}ral. Cet exercice nous propose d'{\'e}tudier
une in{\'e}galit{\'e} faisant intervenir des puissances et qui
g{\'e}n{\'e}ralise les in{\'e}galit{\'e}s entre moyennes.

\begin{exercise}[]
Soient $a, b, c > 0$. Montrer que pour tout $k \in \mathbb{N}$,
\[ \frac{a^{k + 1}}{b^k} + \frac{b^{k + 1}}{c^k} + \frac{c^{k + 1}}{a^k}
   \geqslant a + b + c \]
\end{exercise}

\subsection*{Solution. (SABIR Ilyass)}
\addcontentsline{toc}{subsection}{Solution. (SABIR Ilyass)}

\tmtextbf{M{\'e}thode 1.}

En utilisant l'in{\'e}galit{\'e} arithm{\'e}tico-g{\'e}om{\'e}trique (AM-GM),
on a :
\[ \frac{a^{k + 1}}{b^k} + kb = \frac{a^{k + 1}}{b^k} + \sum^k_{j = 1} b
   \geqslant (k + 1) a \]


Par sym{\'e}trie, on obtient :
\[ \  \]
\[ \frac{a^{k + 1}}{b^k} + \frac{b^{k + 1}}{c^k} + \frac{c^{k + 1}}{a^k} + k
   (b + c + a) \geqslant (k + 1) (a + b + c) \]


D'o{\`u} le r{\'e}sultat.

\tmtextbf{M{\'e}thode 2.}

Montrons le resultat par r{\'e}currence forte sur $k \in \mathbb{N}$.

Pour $k = 0, \tmop{ona}$ :
\begin{eqnarray*}
  \frac{a^{k + 1}}{b^k} + \frac{b^{k + 1}}{c^k} + \frac{c^{k + 1}}{a^k}
  \underset{}{} & = & a + b + c\\
  & \geqslant & a + b + c
\end{eqnarray*}


Soit $k \in \mathbb{N}$, supposons que pour tout $j \in \llbracket 0, k
\rrbracket$, on ait :
\[ \frac{a^{j + 1}}{b^j} + \frac{b^{j + 1}}{c^j} + \frac{c^{j + 1}}{a^j}
   \geqslant a + b + c \]


Et montrons que :
\[ \frac{a^{k + 2}}{b^{k + 1}} + \frac{b^{k + 2}}{c^{k + 1}} + \frac{c^{k +
   2}}{a^{k + 1}} \geqslant a + b + c \]


Si $k$ est pair, alors il existe un $l \in \mathbb{N}$ tel que $k = 2 l$.
Ainsi,
\begin{eqnarray*}
  \frac{a^{k + 2}}{b^{k + 1}} + b & = & \frac{a^{2 l + 2}}{b^{2 l + 1}} + b\\
  & \geqslant & 2 \frac{a^{l + 1}}{b^l}
\end{eqnarray*}


Par sym{\'e}trie, on a alors :
\[ \frac{a^{k + 2}}{b^{k + 1}} + b + \frac{b^{k + 2}}{c^{k + 1}} + c +
   \frac{c^{k + 2}}{a^{k + 1}} + a \geqslant 2 \left( \frac{a^{l + 1}}{b^l} +
   \frac{b^{l + 1}}{c^l} + \frac{c^{l + 1}}{a^l} \right) \]


Or,
\[ \frac{a^{l + 1}}{b^l} + \frac{b^{l + 1}}{c^l} + \frac{c^{l + 1}}{a^l}
   \geqslant a + b + c \]


Ainsi,
\[ \  \]
\[ \frac{a^{k + 2}}{b^{k + 1}} + b + \frac{b^{k + 2}}{c^{k + 1}} + c +
   \frac{c^{k + 2}}{a^{k + 1}} + a \geqslant 2 (a + b + c) \]


D'o{\`u}
\[ \frac{a^{k + 2}}{b^{k + 1}} + \frac{b^{k + 2}}{c^{k + 1}} + \frac{c^{k +
   2}}{a^{k + 1}} \geqslant a + b + c \]


De m{\^e}me, on montre le r{\'e}sultat dans le cas o{\`u} $k$ est impair.

\

\tmtextbf{M{\'e}thode 3. }(Hors programme)

Via l'in{\'e}galit{\'e} de H{\"o}lder, on a
\[ \left( \frac{a^{k + 1}}{b^k} + \frac{b^{k + 1}}{c^k} + \frac{c^{k +
   1}}{a^k} \right) (b + c + a)^k \geqslant (a + b + c)^{k + 1} \]


D'o{\`u} le r{\'e}sultat.
\[ \maltese \maltese \maltese \maltese \maltese \maltese \maltese \]
