L'alg{\`e}bre lin{\'e}aire nous r{\'e}serve parfois des surprises en liant
commutateurs de matrices et nilpotence. Cet exercice propose d'{\'e}tablir un
r{\'e}sultat remarquable sur la nilpotence d'une matrice v{\'e}rifiant une
relation particuli{\`e}re avec son commutateur.
\begin{exercise}[(Oral ULM Lyon Cachan Rennes 2016)]
Soit $A$ et $B$ dans $M_n (\mathbb{R})$ telles que :
\[ AB - BA = A \]
\begin{enumerate}
  \item Montrer que $A$ est nilpotente
  
  \item M{\^e}me question en rempla{\c c}ant $\mathbb{R}$ par $\mathbb{Z}/
  p\mathbb{Z}$, o{\`u} $p$ est un nombre premier, et en supposant $p > n$.
\end{enumerate}
\end{exercise}

\subsection*{Solution. (SABIR Ilyass)}
\addcontentsline{toc}{subsection}{Solution. (SABIR Ilyass)}

1. On remarque tout d'abord qu'on a :
\[ A^2 B - ABA = A^2  \text{ et } ABA - BA^2 = A^2 \]


Donc,
\[ A^2 B - BA^2 = 2 A^2 \]


et
\[ A^3 B - A^2 BA = 2 A^3  \text{et } ABA^2 - BA^3 = A^3 \]


Ainsi,
\[ A^3 B - BA^3 = 3 A^3 \]


Par r{\'e}currence, on {\'e}tablit que, pour tout $k \in \mathbb{N}$
\[ A^k B - BA^k = kA^k  \quad (1) \]


\tmtextbf{M{\'e}thode 1:}

Par (1), on a pour tout $k \in \mathbb{N}^{\ast}$:
\[ \text{tr} (kA^k) = \text{tr} (A^k B - BA^k) = \text{tr} (A^k B) - \text{tr}
   (BA^k) = 0 \]


Donc
\[ k \cdot \text{tr} (A^k) = 0 \]


Par suite, opour tout $k \in \mathbb{N}^{\ast}$
\[ \text{tr} (A^k) = 0 \]


C'est tr{\`e}s classique, les seules matrices qui v{\'e}rifient cette
propri{\'e}t{\'e} sont les matrices nilpotentes.

\

\tmtextbf{M{\'e}thode 2 :}

Pour tout $P = \sum^n_{k = 0} a_k X^k \in \mathbb{K}[X]$

Pour tout $k \in \llbracket 0, n \rrbracket$, on a
\[ \sum_{k = 0}^n a_k (A^k B - BA^k) = \sum_{k = 0}^n a_k kA^k \]


Donc
\[ \left( \sum_{k = 0}^n a_k A^k \right) B - B \left( \sum_{k = 0}^n a_k A^k
   \right) = A \sum_{k = 1}^n a_k kA^{k - 1} \]


Ainsi
\[ P (A) B - BP (A) = AP' (A) \]


Notons $\Pi_A$ le polyn{\^o}me minimal de $A$. On a
\[ X \Pi_A' (A) = \Pi_A (A) B - B \Pi_A (A) = 0 \]


Donc $Q = X \Pi_A'$ annule $A$, et par cons{\'e}quent $\Pi_A$ divise $Q$.

Or, $\deg (Q) = 1 + \deg (\Pi_A') = \deg (\Pi_A)$, donc Q et $\Pi_A$ sont
associ{\'e}s, ce qui implique qu'il existe $r \in \mathbb{R}$ tel que
\[ X \Pi'_A = r \Pi_A \]


Par suite, il existe $k \in \mathbb{N}$ tel que $\Pi_A = X^k$.

D'o{\`u} $A$ est niplotante.

\

\tmtextbf{Remarque.}

Voir aussi la partie II, CCP(MP 2012) - exercice 2.

Ce r{\'e}sultat de la question 1 reste valable pour tout anneau de
caract{\'e}ristique nulle, on va voir dans la question 2 que le r{\'e}sultat
est {\'e}galement vrai sur $\mathbb{Z}/ p\mathbb{Z}$, avec $p > n$.

\

2. M{\^e}me raisonnement.

Etant donn{\'e} que $\mathbb{Z}/ p\mathbb{Z}$ est int{\`e}gre, et que $p > n$,
Le raisonnement pr{\'e}c{\'e}dent s'applique ici.

Ce r{\'e}sultat reste vrai pour tout anneau int{\`e}gre de caract{\'e}ristique
nulle ou de caract{\'e}ristique $r$ avec $r > n$.
\[ \maltese \maltese \maltese \maltese \maltese \maltese \maltese \]
