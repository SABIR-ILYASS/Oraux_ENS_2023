Cet exercice montre que l'ensemble des valeurs d'adh{\'e}rence d'une suite
born{\'e}e dont les diff{\'e}rences entre termes successifs tendent vers
z{\'e}ro forme un segment dans $\mathbb{R}$.
\begin{exercise}[]
Soit $(u_n)_{n \in \mathbb{N}}$ une suite born{\'e}e v{\'e}rifiant $\lim (u_{n
+ 1} - u_n) = 0$.

Montrer que $X_u$ : l'ensemble des valeurs d'adh{\'e}rence de la suite
$(u_n)_{n \in \mathbb{N}}$ est un segment.
\end{exercise}

\subsection*{Solution. (SABIR Ilyass)}
\addcontentsline{toc}{subsection}{Solution. (SABIR Ilyass)}


Selon le th{\'e}or{\`e}me de Bolzano-Weiestrass, $(u_n)_{n \in \mathbb{N}}$
admet au moins une valeur d'adh{\'e}rence.

Si $(u_n)_{n \in \mathbb{N}}$ admet une et une seule valeur d'adh{\'e}rence
alors $(u_n)_{n \in \mathbb{N}}$ est convergente (puisqu'elle est born{\'e}e),
et dans ce cas $X_u$ est r{\'e}duit {\`a} un singleton.

Si $(u_n)_{n \in \mathbb{N}}$ admet au mois deux valeurs d'dh{\'e}rence, on va
essayer dans un premier temps de montrer que $X_u$ est un intervalle de
$\mathbb{R}$, puis qu'il est ferm{\'e}e born{\'e}e afin d'atteindre le
r{\'e}sultat souhait{\'e}.

Soient $\alpha < \beta \in X_u$. Nous allons montrer que pour tout $x \in]
\alpha, \beta [$, $x$ est aussi une valeur d'adh{\'e}rence de $(u_n)_{n \in
\mathbb{N}}$.

Soit $\varepsilon > 0$, et $N \in \mathbb{N}$. L'objectif est de trouver $M
\geqslant N$ tel que $| u_M - x | \leqslant \varepsilon$. Quitte {\`a}
diminuer $\varepsilon$, on peut supposer qu'on a $\alpha < x - \varepsilon < x
+ \varepsilon < \beta$

Puisque $\lim (u_{n + 1} - u_n) = 0$, il existe $n_1 \in \mathbb{N}$ tel que
pour tout $n \geqslant n_1$ on a $| u_{n + 1} - u_n | \leqslant \varepsilon$.

Puisque $\alpha \in X_u$, alors il existe $n_2 \geqslant \max (N, n_1)$ tel
que $u_{n_2} \in] - \infty, x - \varepsilon [$. De m{\^e}me, $\beta \in X_u$,
alors il existe $n_3 \geqslant n_2$ tel que $u_{n_3} \in] x + \varepsilon, +
\infty [$

Notons
\[ n_{\max} = \max \{ n \in \llbracket n_2, n_3 \rrbracket  | \nobracket u_n
   < x - \varepsilon \} \]


$\{ n \in \llbracket n_2, n_3 \rrbracket  | \nobracket u_n < x - \varepsilon
\}$ est non vide (il contient $n_2$), et major{\'e} par $n_3$, donc $n_{\max}
$est bien d{\'e}fini, de plus $n_{\max} \leqslant n_3$. On a donc $u_{n_{\max}
+ 1} \geqslant x - \varepsilon$, mais aussi
\[ u_{n_{\max} + 1} \leqslant u_{n_{\max}} + | u_{n_{\max} + 1} - u_{n_{\max}}
   | \leqslant u_{n_{\max}} + \varepsilon \leqslant x \]


Et donc on a trouv{\'e} $M \assign n_{\max} + 1 \geqslant N$ tel que
$u_{n_{\max} + 1} \in [x - \varepsilon, x]$. Ceci montre que $x \in X_u$.

Ainsi, $X_u$est un intervalle de $\mathbb{R}$, il est born{\'e} car $(u_n)_{n
\in \mathbb{N}}$ est born{\'e}e ($X_u$ est major{\'e} par $\underset{n \in
\mathbb{N}}{\max} (u_n)$, et minor{\'e} par $- \underset{n \in
\mathbb{N}}{\max} (u_n)$). De plus $X_u$ peut {\^e}tre {\'e}crit sous la forme
:
\[ X_u = \underset{N \in \mathbb{N}}{\bigcap} \tmop{Adh} \{ u_n | n \geqslant
   N \nobracket \} \]


O{\`u} $\tmop{Adh} (A)$ d{\'e}signe l'adh{\'e}rence de $A$ pour la norme
usuelle de $\mathbb{R}$, pour tout $A \subset \mathbb{R}$.

Ainsi, $X_u$ est f{\'e}rm{\'e}e. En conclusion $X_u$ est un intervalle
ferm{\'e} et born{\'e} de $\mathbb{R}$, donc il s'agit bien d'un segment de
$\mathbb{R}$
\[ \maltese \maltese \maltese \maltese \maltese \maltese \maltese \]
