Cet exercice explore les conditions de positivit{\'e} d'une suite r{\'e}elle
$(u_n)_{n \geq 0}$ {\`a} travers une forme lin{\'e}aire associ{\'e}e aux
polyn{\^o}mes. Il {\'e}tablit une {\'e}quivalence entre des propri{\'e}t{\'e}s
de positivit{\'e} des polyn{\^o}mes et une in{\'e}galit{\'e} bilin{\'e}aire
sur les coefficients de la suite.
\begin{exercise}[(Oral de l'X 2016)]
Soit $(u_n)_{n \geq 0}$ une suite r{\'e}elle et $\phi_u$ l'unique forme
lin{\'e}aire dans $\mathbb{R}[X]$ telle que, pour tout $k \in \mathbb{N}$,
\[ \phi_u (X^k) = u_k \]

Montrer l'{\'e}quivalence entre :
\begin{enumerate}
  \item $\forall P \in \mathbb{R}[X]$ $P (\mathbb{R}) \subset \mathbb{R}^+
  \Rightarrow \phi_u (P) \geq 0$
  
  \item $\forall n \in \mathbb{N}$ $\forall (x_0, \ldots, x_n) \in
  \mathbb{R}^{n + 1}  \underset{0 \leq i, j \leq n}{\sum} u_{i + j} x_i x_j
  \geq 0$
\end{enumerate}
\end{exercise}

\subsection*{Solution. (SABIR Ilyass)}
\addcontentsline{toc}{subsection}{Solution. (SABIR Ilyass)}


\underline{2) $\Rightarrow$ 1)} Supposons que pour tout $n \in \mathbb{N}$,
pour tout $(x_0, \ldots, x_n) \in \mathbb{R}^{n + 1}$
\[ \underset{0 \leq i, j \leq n}{\sum} u_{i + j} x_i x_j \geq 0 \]


Soit $P \in \mathbb{R}[X]$, en {\'e}crivant $P = \underset{i =
0}{\overset{n}{\sum}} a_i X^i$ o{\`u} $n \in \mathbb{N}$ et $a_0, \ldots, a_n
\in \mathbb{R}$

On a :
\begin{eqnarray*}
  \underset{0 \leq i, j \leq n}{\sum} u_{i + j} x_i x_j  & = & \underset{0
  \leq i, j \leq n}{\sum} \phi_u (X^{i + j}) x_i x_j\\
  & = & \phi_u (\sum_{0 \leq i, j \leq n} x_i x_j X^{i + j})\\
  & = & \phi_u (P^2)
\end{eqnarray*}


Donc la condition (2) est {\'e}quivalente {\`a}
\[ \forall P \in \mathbb{R}[X] \quad \phi_u (P^2) \geq 0 \quad (\ast) \]


Montrons maintenant que :
\[ \forall P \in \mathbb{R}[X], \quad P (\mathbb{R}) \subset \mathbb{R}^+
   \Rightarrow \phi_u (P) \geq 0 \]


Soit $P \in \mathbb{R}[X]$ tel que $P (\mathbb{R}) \subset \mathbb{R}^+$

- Si $P$ admet une racine r{\'e}elle, en {\'e}crivant
\[ P = a \prod_{i = 1}^r (X - \alpha_i)^{m_i}  \prod_{i = 1}^{\ell} (X^2 -
   \lambda_i X + \beta_i)^{n_i} \]


o{\`u} $(m_i, n_i, \alpha_i, \lambda_i, \beta_i) \in \mathbb{N} \times
\mathbb{N} \times \mathbb{R} \times \mathbb{R} \times \mathbb{R}$ pour tout
$i$, avec $a > 0$ et $\lambda_i^2 < 4 \beta_i$.

Les $\alpha_1, \ldots, \alpha_r$ sont les racines r{\'e}elles de $P$.

On a alors $\underset{x \to \infty}{\lim} P (x) \geq 0$, donc $P$ est de
degr{\'e} pair.

Soit $i_0 \in \llbracket 1, r \rrbracket$, on a
\[ P \underset{x \rightarrow \alpha_{i_0}}{\sim} a (X -
   \alpha_{i_0})^{m_{i_0}}  \underset{i \neq i_0}{\prod_{i = 1}^r}
   (\alpha_{i_0} - \alpha_i)^{m_i}  \prod_{i = 1}^{\ell} (\alpha_{i_0}^2 -
   \lambda_i \alpha_{i_0} + \beta_i)^{n_i} \]


Le signe de $P$au voisinage de $\alpha_{i_0}$ est celui de $(X -
\alpha_{i_0})^{m_{i_0}}$. En particulier, $(X - \alpha_{i_0})^{m_{i_0}}$ est
positive au voisinage de $\alpha_{i_0}$, donc $m_{i_0}$ est pair. Cela est
vrai pour tout $i_0 \in \llbracket 1, r \rrbracket$.

Notons pour tout $i \in \llbracket 1, r \rrbracket$
\[ m_i = 2 d_i \]


On a alors
\[ P = a \prod_{i = 1}^r (X - \alpha_i)^{2 d_i}  \prod_{i = 1}^{\ell} (X^2 -
   \lambda_i X + \beta_i)^{n_i} \]


Notons
\[ \mathcal{L}= \{ P \in \mathbb{R} [X]  | \nobracket \exists (A, B) \in
   \mathbb{R} [X]^2 \tmop{tel} \tmop{que} P = A^2 + B^2 \} \]


Pour tout $P, Q \in \mathcal{L}$, il existe $(A, B), (C, D) \in \mathbb{R}
[X]^2$ tels que
\[ P = A^2 + B^2 \infixand Q = C^2 + D^2 \]


Ainsi, $\mathcal{L}$ est stable par produit.

\

Montrons que pour tout $a, b \in \mathbb{R}$ tels que $a^2 < 4 b$, on a
\[ X^2 + a X + b \in \mathcal{L} \]


Soit $(a, b) \in \mathbb{R}^2$, tel que $a^2 < 4 b$, on a
\begin{eqnarray*}
  X^2 + a X + b & = & \left( X + \frac{a}{2} \right)^2 + b - \frac{a^2}{4}\\
  & = & \left( X + \frac{a}{2} \right)^2 + \left( \sqrt{b - \frac{a^2}{4}}
  \right)^2\\
  & \in & \mathcal{L}
\end{eqnarray*}


En particulier, pour tout $i \in \llbracket 1, l \rrbracket$, on a $X^2 -
\lambda_i X + \beta_i \in \mathcal{L}$. Par stabilit{\'e} par produit, on en
d{\'e}duit que
\[ \prod_{i = 1}^{\ell} (\alpha_{i_0}^2 - \lambda_i \alpha_{i_0} +
   \beta_i)^{n_i} \in \mathcal{L} \]


Donc, il existe $A, B \in \mathbb{R} [X]$ tels que
\[ \prod_{i = 1}^{\ell} (\alpha_{i_0}^2 - \lambda_i \alpha_{i_0} +
   \beta_i)^{n_i} = A^2 + B^2 \]


Par suite,
\begin{eqnarray*}
  P & = & a \prod_{i = 1}^r (X - \alpha_i)^{2 d_i} (A^2 + B^2)\\
  &  & \left( \sqrt{a} \prod_{i = 1}^r (X - \alpha_i)^{d_i} A \right)^2 +
  \left( \sqrt{a} \prod_{i = 1}^r (X - \alpha_i)^{d_i} B \right)^2
\end{eqnarray*}


Ainsi, par lin{\'e}arit{\'e} de $\phi_u$ :
\begin{eqnarray*}
  \phi_u (P) & = & a \phi_u ((\prod_{i = 1}^r (X - \alpha_i)^{\alpha_i} A)^2)
  + a \phi_u ((\prod_{i = 1}^r (X - \alpha_i)^{\alpha_i} B)^2)\\
  & \geqslant & 0
\end{eqnarray*}
\underline{1) $\Rightarrow$ 2)} Supposons que pour tout $P \in \mathbb{R} [X]$
\[ P (\mathbb{R}) \subset \mathbb{R}^+ \Rightarrow \phi_u (P) \geq 0 \]


Montrons que, pour tout $n \in \mathbb{N}$, pour tout $x_0, x_1, \ldots, x_n
\in \mathbb{R}^{n + 1}$
\[ \sum_{0 \leq i, j \leq n} u_{i + j} x_i x_j \geq 0 \]


Soient $n \in \mathbb{N}$ et $(x_0, \ldots, x_n) \in \mathbb{R}^{n + 1}$. On a
\begin{eqnarray*}
  &  & 
\end{eqnarray*}
\begin{eqnarray*}
  \sum_{0 \leq i, j \leq n} u_{i + j} x_i x_j & = & \phi_u ((\sum_{i = 0}^n
  x_i X^i)^2)
\end{eqnarray*}


Or, pour tout $x \in \mathbb{R}$,
\[ (\sum_{i = 0}^n x_i X^i)^2 \geq 0 \]


Donc
\[ \phi_u ((\sum_{i = 0}^n x_i X^i)^2) \geq 0 \]


Ainsi,
\[ \sum_{0 \leq i, j \leq n} u_{i + j} x_i x_j \geq 0 \]


D'o{\`u} l'{\'e}quivalence.

\tmtextbf{Remarque.} Un exercice similaire a {\'e}t{\'e} pos{\'e} {\`a} l'oral
de l'X 2007 :

\

Soit $P \in \mathbb{R} [X]$, Montrer l'{\'e}quivalence suivante :
\[ \forall x \in \mathbb{R}, \quad P (x) \geqslant 0 \quad \Leftrightarrow
   \quad \exists (A, B) \in \mathbb{R} [X]^2 \quad P = A^2 + B^2 \]
\[ \maltese \maltese \maltese \maltese \maltese \maltese \maltese \]
