Dans cet exercice, nous nous int{\'e}ressons {\`a} une propri{\'e}t{\'e}
remarquable concernant la r{\'e}union de sous-espaces vectoriels. Plus
pr{\'e}cis{\'e}ment, nous allons voir que si un nombre fini de sous-espaces
vectoriels recouvre enti{\`e}rement l'espace $\mathbb{R}^n$, alors l'un
d'entre eux doit n{\'e}cessairement {\^e}tre {\'e}gal {\`a} l'espace tout
entier. Cette propri{\'e}t{\'e}, bien que surprenante {\`a} premi{\`e}re vue,
s'av{\`e}re {\^e}tre une cons{\'e}quence des caract{\'e}ristiques
fondamentales des sous-espaces vectoriels.
\begin{exercise}[(Oral de l'X 2016)]
Soient $V_1, \ldots, V_p$ des sous-espaces de $\mathbb{R}^n$ dont la
r{\'e}union est {\'e}gale {\`a} $\mathbb{R}^n$. Montrons que l'un des $V_i$
est {\'e}gal {\`a} $\mathbb{R}^n$.

\end{exercise}

\subsection*{Solution. (SABIR Ilyass)}
\addcontentsline{toc}{subsection}{Solution. (SABIR Ilyass)}

C'est un exercice classique, d{\'e}j{\`a} pos{\'e} {\`a} l'oral de l'ENS dans
une version plus forte.

Le r{\'e}sultat est valable pour tout $K$-espace vectoriel de dimension finie,
o{\`u} $K$ est un corps infini.

\

Montrons le r{\'e}sultat suivant :

Soit $K$ un corps infini, $E$ un $K$-espace vectoriel de dimension finie.
Alors, il n'existe aucune famille finie $(V_i)_{1 \leqslant i \leqslant n}$ de
sous-espaces vectoriels stricts de $E$ telle que :
\[ E = \underset{i = 1}{\overset{n}{\bigcup}} V_i \]


Raisonnons par l'absurde et supposons qu'il existe une famille $(V_i)_{1
\leqslant i \leqslant n}$ de sous-espaces vectoriels stricts de $E$ telle que
\[ E = \underset{i = 1}{\overset{n}{\bigcup}} V_i \]


Quitte {\`a} retirer un certain nombre de sous-espaces vectoriels, nous
pouvons supposer que la famille $(V_i)_{1 \leqslant i \leqslant n} $est
minimale, c'est-{\`a}-dire qu'elle est telle que, pour tout $i \in \llbracket
1, n \rrbracket$,
\[ V_i \not{\subset} \underset{j \neq i}{\underset{j =
   1}{\overset{n}{\bigcup}}} V_j \]


Pour $V_n$ par exemple, il existe $y \in V_n$ tel que $y \nin \bigcup_{i =
1}^{n - 1} V_i$ ($\star$)

D'autre part, \ $E = \bigcup_{i = 1}^n V_i$, ce qui est {\'e}vident puisque
$V_n$ est un sous-espace strict de $E$. Donc, il existe $x \in \bigcup_{i =
1}^{n - 1} V_i$ et $x \nin V_n$.

\

Pour tout $\lambda \in K$, on a $x + \lambda y \in E$. Le vecteur $x +
\lambda y \nin V_n$, donc
\[ x + \lambda y \in \bigcup_{i = 1}^{n - 1} V_i \]


Ainsi, il existe $i_{\lambda} \in \llbracket 1, n - 1 \rrbracket$ tel que $x
+ \lambda y \in V_{i_{\lambda}}$.

\

Consid{\'e}rons l'application
\[ \varphi : \left\{\begin{array}{l}
     K \to \llbracket 1, n - 1 \rrbracket\\
     \lambda \mapsto i_{\lambda}
   \end{array}\right. \]


Soient $\lambda, \beta \in K$ tels que $\varphi (\lambda) = \varphi (\beta)$.
On a alors $x + \lambda y \in V_{\varphi (\lambda)}$ et $x + \beta y \in
V_{\varphi (\beta)} = V_{\varphi (\lambda)}$

Comme $V_{\varphi (\lambda)}$ est un espace vectoriel, alors
\[ (\lambda - \beta) y \in V_{\varphi (\lambda)} \]


si $\lambda \neq \beta$ alors $y \in V_{\varphi (\lambda)} \subset \bigcup_{i
= 1}^{n - 1} V_i$, ce qui est absurde en vertu de ($\star$).

Donc $\lambda = \beta$, et par suite $\varphi$ est injective. Cette conclusion
contredit le fait que $K$ est infini alors que $\llbracket 1, n - 1
\rrbracket$ est fini.
\[ \maltese \maltese \maltese \maltese \maltese \maltese \maltese \]
