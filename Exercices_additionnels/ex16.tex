L'{\'e}tude des fonctions infiniment d{\'e}rivables satisfaisant des
{\'e}quations fonctionnelles est un domaine fascinant de l'analyse. Cet
exercice propose d'{\'e}tudier une fonction v{\'e}rifiant une relation
lin{\'e}aire particuli{\`e}re et invite {\`a} utiliser les propri{\'e}t{\'e}s
de d{\'e}rivation pour conclure.
\begin{exercise}[(G{\'e}n{\'e}ralisation de l'oral ULM 2007)]
Soient $a_1, a_2, \ldots, a_n > 0$, deux {\`a} deux distincts, et $f \in
\mathcal{C}^{\infty} (\mathbb{R}, \mathbb{R})$ v{\'e}rifiant :
\[ \forall x \in \mathbb{R}, f (a_1 x) + f (a_2 x) + \cdots + f (a_n x) = 0 \]

Montrer que $f = 0$.
\end{exercise}

\subsection*{Solution. (SABIR Ilyass)}
\addcontentsline{toc}{subsection}{Solution. (SABIR Ilyass)}

On a, pour tout $x \in \mathbb{R}$ :
\[ f (a_1 x) + f (a_2 x) + \cdots + f (a_n x) = 0 \]


Donc, pour tout $r \in \mathbb{N}$ et pour tout $x \in \mathbb{R}$, on obtient
:
\[ \sum_{k = 1}^n a_k^r \times f^{(r)} (a_k x) = 0 \quad (1) \]


En particulier, pour tout $r \in \mathbb{N}$, on a :
\[ f^{(r)} (0) = 0 \quad (2) \]


On peut supposer sans perte de g{\'e}n{\'e}ralit{\'e} que $0 < a_1 < a_2 <
\cdots < a_n$

Soit $a > 0$, pour tout $x \in [- a, a]$, et pour tout $r \in \mathbb{N}$, on
a d'apr{\`e}s (1) :
\[ f^{(r)} (x) = - \sum_{k = 1}^{n - 1} \left( \frac{a_k}{a_n} \right)^r
   \times f^{(r)} \left( \frac{a_k x}{a_n} \right) \]


Notons $M_r = \underset{x \in [- a, a]}{\sup}  |f^{(r)} (x) |$. On a alors,
pour tout $r \in \mathbb{N}$ :
\[ |f^{(r)} (x) | \leqslant M_r \times \sum_{k = 1}^{n - 1} \left(
   \frac{a_k}{a_n} \right)^r \]


Donc :
\[ M_r \leqslant M_r \times \sum_{k = 1}^{n - 1} \left( \frac{a_k}{a_n}
   \right)^r \]


Or, pour tout $k \in \llbracket 1, n - 1 \rrbracket$, on a :
\[ 0 < \frac{a_k}{a_n} < 1 \]


Donc $\left( \frac{a_k}{a_n} \right)^r \xrightarrow[r \to + \infty]{} 0$.
Ainsi,
\[ \sum_{k = 1}^{n - 1} \left( \frac{a_k}{a_n} \right)^r \xrightarrow[r \to +
   \infty]{} 0 < 1 \]


Donc, il existe un $r_0 \in \mathbb{N}$, tel que pour tout $r \geq r_0$
\[ \sum_{k = 1}^{n - 1} \left( \frac{a_k}{a_n} \right)^r < 1 \]


En particulier, pour tout $r \geq r_0$
\[ M_r \leqslant M_r \times \sum_{k = 1}^{n - 1} \left( \frac{a_k}{a_n}
   \right)^r < M_r \]


Cela implique que, pour tout $r \geq r_0$, $M_r = 0$, d'o{\`u} pour tout $r
\geq r_0$ et pour tout $x \in [- a, a]$, $f^{(r)} (x) = 0$, et ceci pour tout
$a > 0$.

Donc, pour tout $r \geq r_0$ et pour tout $x \in \mathbb{R}$, $f^{(r)} (x) =
0$

Ainsi, $f$ est polyn{\^o}miale de degr{\'e} inf{\'e}rieur ou {\'e}gal {\`a}
$r_0 - 1$, et on a pour tout $x \in \mathbb{R}$
\[ f (x) = \sum_{k = 0}^{r_0 - 1} \frac{f^{(k)} (0)}{k!} x^k = 0 \]


Finalement, $f = 0$. (d'apr{\`e}s (2)).
\[ \maltese \maltese \maltese \maltese \maltese \maltese \maltese \]
