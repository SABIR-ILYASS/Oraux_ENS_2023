Cet exercice explore les propri{\'e}t{\'e}s des suites complexes dont la puissance $p$-i{\`e}me tend vers $1$ et dont la moyenne arithm{\'e}tique converge. L'objectif est de caract{\'e}riser l'ensemble des valeurs possibles de cette moyenne limite.
\begin{exercise}[]
Soit $p \in \mathbb{N}^{\ast}$, soit $E$ l'ensemble des suites $(u_n)$ de
nombres complexes v{\'e}rifiant $\underset{n \to + \infty}{\lim} u_n^p = 1$,
et telles que la suite de terme g{\'e}n{\'e}ral $S_n = \frac{1}{n} 
\underset{k = 1}{\overset{n}{\sum}} u_k$ converge. On consid{\`e}re
l'application $\varphi$ de $E$ dans $\mathbb{C}$ qui, {\`a} la suite $(u_n)$,
associe la limite $l$ de la suite $(S_n)$. D{\'e}terminer $\varphi (E)$.
\end{exercise}

\subsection*{Solution. (SABIR Ilyass)}
\addcontentsline{toc}{subsection}{Solution. (SABIR Ilyass)}


Soit $p \in \mathbb{N}^{\ast}$,

Notons $W_p$ l'ensemble des racines $p$-i{\`e}mes de l'unit{\'e}, donn{\'e}es
par $\omega_k = e^{\frac{2 i \pi k}{p}}$ pour tout $1 \leq k \leq p$.

\

Pour d{\'e}terminer l'image $\varphi (E)$ de la fonction $\varphi$, nous
devons trouver toutes les limites possibles $l \in \mathbb{C}$ telles qu'il
existe une suite $(u_n) \in E$ satisfaisant les conditions suivantes :

1. $\underset{n \to + \infty}{\lim}  u_n^p = 1$,

2. La suite $S_n = \frac{1}{n}  \underset{k = 1}{\overset{n}{\sum}} u_k$
converge vers $l$.

\

\

La condition $\underset{n \to + \infty}{\lim}  u_n^p = 1$ implique que les
$p$-i{\`e}mes puissances des $u_n$ tendent vers 1. Cela signifie que les $u_n$
s'approchent des racines $p$-i{\`e}mes de l'unit{\'e} dans le plan complexe.

Les racines $p$-i{\`e}mes de l'unit{\'e} sont donn{\'e}es par :
\[ C_p = \left\{ e^{2 i \pi \frac{m}{p}} \mid m = 0, 1, \ldots, p - 1 \right\}
   . \]


La convergence de $S_n$ signifie que la moyenne des $u_n$ se stabilise lorsque
$n$ cro{\^i}t.

\tmtextbf{Construction des Suites et Limites :}

- Si $u_n \to \zeta$ o{\`u} $\zeta \in C_p$, alors $u_n^p \to \zeta^p = 1$. La
limite de $S_n$ sera $l = \zeta$.

\

- Si $u_n$ prend des valeurs dans $C_p$ avec certaines fr{\'e}quences, nous
pouvons d{\'e}finir $\alpha_m$ comme la fr{\'e}quence asymptotique de $u_n =
\zeta_m$.

Alors, $S_n \to l = \underset{m = 0}{\overset{p - 1}{\sum}} \alpha_m \zeta_m$.
avec $\alpha_m \geq 0$ et $\underset{m = 0}{\overset{p - 1}{\sum}} \alpha_m =
1$.

\

- M{\^e}me si les $u_n$ ne sont pas exactement les racines de l'unit{\'e}
mais s'en approchent, tant que $u_n^p \to 1$, une logique similaire
s'applique.

L'{\'e}quation $u_n^p \to 1$ implique que chaque $u_n$ s'approche d'une
solution de $z^p = 1$, c'est-{\`a}-dire des racines $p$-i{\`e}mes de
l'unit{\'e}.

Plus pr{\'e}cis{\'e}ment, pour tout $\varepsilon > 0$, il existe $N$ tel que
pour tout $n \geq N$, $\mid u_n^p - 1 \mid < \varepsilon$.

Donc, pour $n$ suffisamment grand, il existe $m_n \in \{ 0, 1, \ldots, p - 1
\}$ tel que $u_n$ est proche de $\omega_{m_n}$.

On peut {\'e}crire :


\[ u_n = e^{2 i \pi m_n / p} e^{i \varepsilon_n}, \]


o{\`u} : $m_n \in \{0, 1, \ldots, p - 1\}$ (car $u_n$ est proche de la
$m_n$-{\`e}me racine de l'unit{\'e}), et $\varepsilon_n  \underset{n \to
\infty}{\to} 0$ (petite d{\'e}viation par rapport {\`a} la racine de
l'unit{\'e} exacte).

On peut {\'e}crire la moyenne $S_n$ comme suit :
\begin{eqnarray*}
  S_n & = & \frac{1}{n}  \sum_{k = 1}^n u_k\\
  & = & \frac{1}{n}  \sum_{k = 1}^n e^{2 i \pi m_k / p} e^{i \varepsilon_k}
\end{eqnarray*}


Comme $\varepsilon_k  \underset{k \to \infty}{\to} 0$, pour les grandes
valeurs de $n$, $e^{i \varepsilon_k} = 1 + i \varepsilon_k + o
(\varepsilon_k)$. Par cons{\'e}quent :
\[ u_k = e^{2 i \pi m_k / p} + e^{2 i \pi m_k / p} i \varepsilon_k + o
   (\varepsilon_k) \]


La moyenne $S_n$ devient :
\[ S_n = \frac{1}{n}  \sum_{k = 1}^n e^{2 i \pi m_k / p} + \frac{1}{n} 
   \sum_{k = 1}^n e^{2 i \pi m_k / p} i \varepsilon_k + o (\varepsilon_k) \]


Le second terme $\frac{1}{n}  \underset{k = 1}{\overset{n}{\sum}} e^{2 i \pi
m_k / p} i \varepsilon_k$ implique des $\varepsilon_k$ qui sont petits et
tendent vers z{\'e}ro.

Comme $\varepsilon_k  \underset{k \to \infty}{\to} 0$ et sont born{\'e}s, la
somme des $\varepsilon_k$ sur $n$ termes divis{\'e}e par $n$ tend vers
z{\'e}ro :
\[ \left| \frac{1}{n}  \sum_{k = 1}^n \varepsilon_k \right| \leq \frac{1}{n} 
   \sum_{k = 1}^n | \varepsilon_k | \underset{k \to \infty}{\to} 0. \]


Par cons{\'e}quent, le terme d'erreur dispara{\^i}t dans la limite :
\[ \underset{n \to \infty}{\lim}   \frac{1}{n}  \sum_{k = 1}^n e^{2 i \pi m_k
   / p} i \varepsilon_k = 0 \]


La limite de $S_n$ est d{\'e}termin{\'e}e par les fr{\'e}quences des racines
de l'unit{\'e} $e^{2 i \pi m / p}$ parmi les $u_k$ :
\[ l = \underset{n \to \infty}{\lim} S_n = \sum_{m = 0}^{p - 1} \alpha_m e^{2
   i \pi m / p}  \]


o{\`u} $\alpha_m$ est la fr{\'e}quence asymptotique de $u_k$ {\'e}tant proche
de $e^{2 i \pi m / p}$.

\

D{\'e}finissons $N_m (n)$ comme le nombre de $k \leq n$ tels que $u_k$ soit
proche de $e^{2 i \pi m / p}$.

La fr{\'e}quence asymptotique est donn{\'e}e par $\alpha_m = \underset{n \to
\infty}{\lim}  \frac{N_m (n)}{n}$.

Comme $\underset{m = 0}{\overset{p - 1}{\sum}} N_m (n) = n$, on a
$\underset{m = 0}{\overset{p - 1}{\sum}} \alpha_m = 1$.

La limite $l$ est une moyenne pond{\'e}r{\'e}e des racines de l'unit{\'e} :
\[ l = \sum_{m = 0}^{p - 1} \alpha_m e^{2 i \pi m / p} . \]


Chaque $\alpha_m$ repr{\'e}sente la proportion de termes $u_k$ proches d'une
racine de l'unit{\'e} donn{\'e}e.

\

Comme $\alpha_m \geq 0$ et $\underset{m = 0}{\overset{p - 1}{\sum}} \alpha_m
= 1$, $l$ est une combinaison convexe des racines $p$-i{\`e}mes de
l'unit{\'e}. \

L'ensemble de tous ces $l$ forme l'enveloppe convexe de $C_p$ :
\[ \varphi (E) = \text{Conv} \{ e^{2 \pi im / p} \mid m = 0, 1, \ldots, p - 1
   \} . \]
\[ \maltese \maltese \maltese \maltese \maltese \maltese \maltese \]
