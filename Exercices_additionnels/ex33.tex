Cet exercice porte sur l'in{\'e}galit{\'e} de H{\"o}lder, utilis{\'e}e pour comparer des sommes et produits de valeurs positives. Il s'agit de d{\'e}montrer l'in{\'e}galit{\'e}, d'{\'e}tudier le cas d'{\'e}galit{\'e}, puis d'appliquer cette in{\'e}galit{\'e} {\`a} une propri{\'e}t{\'e} des produits augment{\'e}s.
\begin{exercise}[(In{\'e}galit{\'e} de H{\"o}lder)]
Soient ${m, n \in \mathbb{N}^{\ast}} $, et soient $(a_{2, 1}, \ldots, a_{2,
n})$, $(a_{1, 1}, \ldots, a_{1, n}), \ldots$ et $(a_{m, 1}, \ldots, a_{m, n})$
des r{\'e}els positifs.

1. Montrer que :
\[ \underset{i = 1}{\overset{m}{\prod}} \left( \overset{n}{\underset{j =
   1}{\sum}} a_{i, j} \right) \geqslant \left( \underset{j =
   1}{\overset{n}{\sum}} \sqrt[m]{\underset{i = 1}{\overset{m}{\prod}} a_{i,
   j}} \right)^m \]


2. {\'E}tudier le cas d'{\'e}galit{\'e} dans l'in{\'e}galit{\'e} de
H{\"o}lder.

3. Soient $a_1, \ldots, a_n > 0$, Montrer que :
\[ (1 + a_1) (1 + a_2) \ldots (1 + a_n) \geqslant \left( 1 + \sqrt[n]{a_1 a_2
   \ldots a_n} \right)^n \]
\end{exercise}

\subsection*{Solution. (SABIR Ilyass)}
\addcontentsline{toc}{subsection}{Solution. (SABIR Ilyass)}


1- Soient $m, n \in \mathbb{N}^{\star}$, $(a_{2, 1}, \ldots, a_{2, n})$,
$(a_{1, 1}, \ldots, a_{1, n}), \ldots,$et $(a_{m, 1}, \ldots, a_{m, n})$ des
r{\'e}els positifs.

Montrons que :
\[ \underset{i = 1}{\overset{m}{\prod}} \left( \overset{n}{\underset{j =
   1}{\sum}} a_{i, j} \right) \geqslant \left( \underset{j =
   1}{\overset{n}{\sum}} \sqrt[m]{\underset{i = 1}{\overset{m}{\prod}} a_{i,
   j}} \right)^m \]


On sait que la fonction $\ln$ est concave sur $] 0, + \infty [$.

D'apr{\`e}s l'in{\'e}galit{\'e} de Jensen, pour tous $p_1, .., p_m > 0$ tels
que
\[ \frac{1}{p_1} + \cdots + \frac{1}{p_m} = 1 \]


et pour tous $x_1, \ldots, x_m > 0$, on a :
\[ \ln \left( \underset{i = 1}{\overset{}{} \overset{m}{\sum}}
   \frac{x^{p_i}_i}{p_i} \right) \geqslant \underset{i = 1}{\overset{}{}
   \overset{m}{\sum}} \ln (x_i) \]


Ainsi :
\[ \underset{i = 1}{\overset{}{} \overset{m}{\sum}} \frac{x^{p_i}_i}{p_i}
   \geqslant \underset{i = 1}{\overset{}{} \overset{m}{\prod}} x_i \quad
   (\tmop{AM} - \tmop{GM} g{\'e}n{\'e} \tmop{ralis} {\'e}e)  \]


En particulier, pour tout $j \in \llbracket 1, n \rrbracket$, posons\quad$x_i
= \frac{a_{i, j}}{\left( \overset{n}{\underset{k = 1}{\sum}} a_{i, k} \right)
}$ , et $p_i = m$ pour tout $i \in \llbracket 1, m \rrbracket$.

On obtient, pour tout $j \in \llbracket 1, n \rrbracket$ :
\begin{eqnarray*}
  \underset{i = 1}{\overset{}{} \overset{m}{\sum}} \frac{a_{i, j}}{m \left(
  \overset{n}{\underset{k = 1}{\sum}} a_{i, k} \right) } & \geqslant &
  \underset{i = 1}{\overset{}{} \overset{m}{\prod}} \left( \frac{a_{i,
  j}}{\left( \overset{n}{\underset{k = 1}{\sum}} a_{i, k} \right) }
  \right)^{\frac{1}{m}}\\
  & = & \frac{\sqrt[m]{\underset{i = 1}{\overset{m}{\prod}} a_{i,
  j}}}{\underset{i = 1}{\overset{}{} \overset{m}{\prod}} \left(
  \overset{n}{\underset{k = 1}{\sum}} a_{i, k} \right)^{^{\frac{1}{m}}}}
\end{eqnarray*}


En sommant sur $j$, on obtient :
\begin{eqnarray*}
  \overset{n}{\underset{j = 1}{\sum}}  \underset{i = 1}{\overset{}{}
  \overset{m}{\sum}} \frac{a_{i, j}}{m \left( \overset{n}{\underset{k =
  1}{\sum}} a_{i, k} \right) } & \geqslant & \underset{j =
  1}{\overset{n}{\sum}} \frac{\sqrt[m]{\underset{i = 1}{\overset{m}{\prod}}
  a_{i, j}}}{\underset{i = 1}{\overset{}{} \overset{m}{\prod}} \left(
  \overset{n}{\underset{k = 1}{\sum}} a_{i, k} \right)^{^{\frac{1}{m}}}}
\end{eqnarray*}
\[ \  \]


Avec :
\begin{eqnarray*}
  \overset{n}{\underset{j = 1}{\sum}}  \underset{i = 1}{\overset{}{}
  \overset{m}{\sum}} \frac{a_{i, j}}{m \left( \overset{n}{\underset{k =
  1}{\sum}} a_{i, k} \right) } & = & \underset{i = 1}{\overset{}{}
  \overset{m}{\sum}} \frac{1}{m} \underset{j = 1}{\overset{n}{\sum}}
  \frac{a_{i, j}}{\overset{n}{\underset{k = 1}{\sum}} a_{i, k} }\\
  & = & \underset{i = 1}{\overset{}{} \overset{m}{\sum}} \frac{1}{m}\\
  & = & 1
\end{eqnarray*}


On obtient finalement :
\[ 1 \geqslant \underset{j = 1}{\overset{n}{\sum}} \frac{\sqrt[m]{\underset{i
   = 1}{\overset{m}{\prod}} a_{i, j}}}{\underset{i = 1}{\overset{}{}
   \overset{m}{\prod}} \left( \overset{n}{\underset{k = 1}{\sum}} a_{i, k}
   \right)^{^{\frac{1}{m}}}} \]


Donc :
\[ \underset{i = 1}{\overset{}{} \overset{m}{\prod}} \left(
   \overset{n}{\underset{k = 1}{\sum}} a_{i, k} \right)^{^{\frac{1}{m}}}
   \geqslant \underset{j = 1}{\overset{n}{\sum}} \sqrt[m]{\underset{i =
   1}{\overset{m}{\prod}} a_{i, j}} \]


D'o{\`u} :
\[ \underset{i = 1}{\overset{m}{\prod}} \left( \overset{n}{\underset{j =
   1}{\sum}} a_{i, j} \right) \geqslant \left( \underset{j =
   1}{\overset{n}{\sum}} \sqrt[m]{\underset{i = 1}{\overset{m}{\prod}} a_{i,
   j}} \right)^m \]


\

2. L'{\'e}galit{\'e} dans l'in{\'e}galit{\'e} de H{\"o}lder est atteinte si,
pour tout $j \in \llbracket 1, n \rrbracket$, on a :
\[ \underset{i = 1}{\overset{}{} \overset{m}{\sum}} \frac{a_{i, j}}{m \left(
   \overset{n}{\underset{k = 1}{\sum}} a_{i, k} \right) } = \underset{i =
   1}{\overset{}{} \overset{m}{\prod}} \left( \frac{a_{i, j}}{\left(
   \overset{n}{\underset{k = 1}{\sum}} a_{i, k} \right) }
   \right)^{\frac{1}{m}} \]


Cela implique :
\[ \ln \left( \underset{i = 1}{\overset{}{} \overset{m}{\sum}} \frac{a_{i,
   j}}{m \left( \overset{n}{\underset{k = 1}{\sum}} a_{i, k} \right) } \right)
   = \underset{i = 1}{\overset{}{} \overset{m}{\sum}} \ln \left( \frac{a_{i,
   j}}{\left( \overset{n}{\underset{k = 1}{\sum}} a_{i, k} \right) } \right)
\]


{\'E}tant donn{\'e} que la fonction ln est strictement concave, cela signifie
que $\frac{a_{i, j}}{\left( \overset{n}{\underset{k = 1}{\sum}} a_{i, k}
\right) }$ est ind{\'e}pendant de $i$, d'o{\`u} l'existence d'une constante
$C_j > 0$ telle que, pour tout $i, l \in \llbracket 1, m \rrbracket$
\[ \frac{a_{i, j}}{a_{l, j}} = C_j \]


3. Soient $a_1, \ldots, a_n > 0$. Montrons que :
\[ (1 + a_1) (1 + a_2) \ldots (1 + a_n) \geqslant \left( 1 + \sqrt[n]{a_1 a_2
   \ldots a_n} \right)^n \]


\tmtextbf{M{\'e}thode 1.}

Par application directe de l'in{\'e}galit{\'e} de H{\"o}lder, le r{\'e}sultat
suit.

\

\tmtextbf{M{\'e}thode 2.}

En appliquant l'in{\'e}galit{\'e} arithm{\'e}tico-g{\'e}om{\'e}trique, on
obtient :


\[ \frac{1}{1 + a_1} + \cdots + \frac{1}{1 + a_n} \geqslant
   \frac{n}{\sqrt[n]{(1 + a_1) \ldots (1 + a_n)}} \]


Et
\[ \frac{a_1}{1 + a_1} + \cdots + \frac{a_n}{1 + a_n} \geqslant \frac{n
   \sqrt[n]{a_1 \ldots a_n}}{\sqrt[n]{(1 + a_1) \ldots (1 + a_n)}} \]


En sommant les deux, on obtient :
\[ n \geqslant \frac{n \left( 1 + \sqrt[n]{a_1 \ldots a_n}
   \right)}{\sqrt[n]{(1 + a_1) \ldots (1 + a_n)}} \]


ce qui entra{\^i}ne :
\[ (1 + a_1) (1 + a_2) \ldots (1 + a_n) \geqslant \left( 1 + \sqrt[n]{a_1 a_2
   \ldots a_n} \right)^n \]
\[ \maltese \maltese \maltese \maltese \maltese \maltese \maltese \]
