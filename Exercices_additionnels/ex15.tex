Les suites de nombres harmoniques sont riches en propri{\'e}t{\'e}s
int{\'e}ressantes et cet exercice nous propose d'{\'e}tudier la partie
enti{\`e}re de leur somme partielle. C'est une belle occasion d'explorer les
liens entre l'analyse et l'arithm{\'e}tique.
\begin{exercise}[]
Montrer que $f : \left\{\begin{array}{l}
  \mathbb{N}^{\ast} \rightarrow \mathbb{N}^{\ast}\\
  n \mapsto \left\lfloor 1 + \frac{1}{2} + \cdots + \frac{1}{n} \right\rfloor
\end{array}\right.$ est surjective.
\end{exercise}

\subsection*{Solution. (SABIR Ilyass)}
\addcontentsline{toc}{subsection}{Solution. (SABIR Ilyass)}


Montrons que $f : \left\{\begin{array}{l}
  \mathbb{N}^{\ast} \rightarrow \mathbb{N}^{\ast}\\
  n \mapsto \left\lfloor 1 + \frac{1}{2} + \cdots + \frac{1}{n} \right\rfloor
\end{array}\right.$ est surjective.

Notons :
\[ A =\{f (n) - 1 | \nobracket n \in \mathbb{N}^{\ast} \} \]


On a $f$ est croissante et $f (1) = 1$, donc $0 \in A$.

Montrons que $\underset{n \to + \infty}{\lim} f (n) = + \infty$

On a $t \mapsto \frac{1}{t}$ est d{\'e}croissante sur $] 0, + \infty [$, donc
pour tout $k \in \mathbb{N}^{\ast}$, on a :
\[ \frac{1}{k + 1} \leqslant \int_k^{k + 1} \frac{dt}{t} \leqslant
   \frac{1}{k} \]


Donc :
\[ \ln (n + 1) \leqslant 1 + \frac{1}{2} + \cdots + \frac{1}{n} \]


Ainsi, $\underset{n \to + \infty}{\lim} f (n) = + \infty$, et pour tout $n \in
\mathbb{N}^{\ast}$, on a :

\begin{align*}
  f (n + 1) - f (n) & = \left\lfloor 1 + \frac{1}{2} + \cdots + \frac{1}{n +
  1} \right\rfloor - \left\lfloor 1 + \frac{1}{2} + \cdots + \frac{1}{n}
  \right\rfloor\\
  & \leqslant \left( 1 + \frac{1}{2} + \cdots + \frac{1}{n + 1} \right) -
  \left[ \left( 1 + \frac{1}{2} + \cdots + \frac{1}{n} \right) - 1 \right]\\
  & = \frac{1}{n + 1} + 1
\end{align*}

\

Donc :
\[ 0 \leqslant f (n + 1) - f (n) \leqslant \left\lfloor \frac{1}{n + 1} + 1
   \right\rfloor = 1 \]


Soit $k \in A$, donc il existe un $n_0 \in \mathbb{N}^{\ast}$ tel que $k + 1 =
f (n_0)$.

Notons :
\[ A_k =\{j \in \mathbb{N} \mid j \geqslant k \text{et } f (j) = k + 1\} \]


Si $A_k$ est non born{\'e}e, on aurait par croissance de $f$, pour tout $j
\geqslant k$ $f (j) = k + 1$, ce qui est absurde avec $\underset{n \to +
\infty}{\lim} f (n) = + \infty$

Donc, $A_k$ est fini, et admet un plus grand {\'e}l{\'e}ment not{\'e} $j_0$.

On a alors $f (j_0) = k + 1$ et $f (j_0 + 1) \neq k + 1$, et par croissance de
$f$, on a $f (j_0 + 1) > k + 1$

Or :
\[ 0 \leqslant f (j_0 + 1) - f (j_0) \leqslant 1 \]


Donc, $f (j_0 + 1) = k + 2$, ainsi $k + 1 = f (j_0 + 1) - 1 \in A$

Conclusion
\[ \left\{\begin{array}{l}
     0 \in A\\
     \text{si } k \in A \text{alors } k + 1 \in A
   \end{array}\right. \]
\

Par l'axiome de Peano, $A =\mathbb{N}$.

D'o{\`u}
\[ f (\mathbb{N}^{\ast}) = A + 1 =\mathbb{N}^{\ast} \]
\[ \maltese \maltese \maltese \maltese \maltese \maltese \maltese \]
