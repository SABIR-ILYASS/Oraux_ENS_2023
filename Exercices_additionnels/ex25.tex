Cet exercice explore des propri{\'e}t{\'e}s g{\'e}om{\'e}triques des racines
d'un polyn{\^o}me et de sa d{\'e}riv{\'e}e dans le plan complexe. On
{\'e}tudie notamment la relation entre les racines d'un polyn{\^o}me et celles
de sa d{\'e}riv{\'e}e {\`a} travers le concept d'enveloppe convexe.

\begin{exercise}[(Oral l'X 2007)]
Soit $P \in \mathbb{C}[X]$

$a.$ Montrer que tout racine de $P'$ est dans l'enveloppe convexe des racines
de $P$.

$b.$ Soit $K$ un convexe ferm{\'e} de $\mathbb{C}$. On note $\Omega$
l'ensemble des complexes $w$ tels que $P^{- 1} (\{w\}) \subset K$. Montrer que
$\Omega$ est convexe.
\end{exercise}

\subsection*{Solution. (SABIR Ilyass)}
\addcontentsline{toc}{subsection}{Solution. (SABIR Ilyass)}

$a.$ Cette question est classique, et utilis{\'e}e pour montrer plusieurs
r{\'e}sultats importants. Ce r{\'e}sultat est connu sous le nom de
th{\'e}or{\`e}me de Lucas.

Soit $z \in \mathbb{C}$ une racine de $P'$.

Si $z$ est une racine de $P$, c'est termin{\'e} ! Sinon, la cl{\^o}ture de
$\mathbb{C}$ permet d'{\'e}crire :
\[ P = c \prod_{i = 1}^n (X - z_i) \]


avec $z_1, \ldots, z_n \in \mathbb{C}$, on a donc
\[ \frac{P'}{P} = \sum_{i = 1}^n \frac{1}{X - z_i} \]


En particulier,
\begin{eqnarray*}
  0 & = & \frac{P' (z)}{P (z)}\\
  & = & \sum_{i = 1}^n \frac{1}{z - z_i}
\end{eqnarray*}


En multipliant chaque terme de la somme par $\overline{z - z_i}$, on obtient :
\[ \sum_{i = 1}^n \frac{\overline{z - z_i}}{|z - z_i |^2} = 0 \]


En conjuguant cette {\'e}galit{\'e}, on obtient :


\[ z = \sum_{i = 1}^n \frac{z_i}{|z - z_i |^2} \cdot \frac{1}{\sum_{i = 1}^n
   \frac{1}{|z - z_i |^2}} \]


En posant pour tout $i \in \llbracket 1, n \rrbracket,$
\[ \lambda_i = \frac{1}{|z - z_i |^2} \times \frac{1}{\sum_{i = 1}^n
   \frac{1}{|z - z_i |^2}} \]


On obtient donc $\sum_{i = 1}^n \lambda_i = 1$, et
\[ z = \sum_{i = 1}^n \lambda_i z_i \]


Ce qui montre que $z$ est dans l'enveloppe convexe des $z_i$.

\

$b.$ Soient $u, v \in \Omega$. Montrons que pour tout $\lambda \in [0, 1]$
$\lambda \mu + (1 - \lambda) \nu \in \Omega$.

C'est-{\`a}-dire que pour tout $z$ tel que $P (z) = \lambda u + (1 - \lambda)
v$ est dans $K$.

\

Supposons d'abord $\lambda \in \mathbb{Q}$. On peut donc {\'e}crire $\lambda
= \frac{n}{n + m}$ avec $n, m \geq 0$. Il s'agit de montrer que si $f (z) =
\frac{mu + nv}{n + m}$ avec $z \in K$.

Consid{\'e}rons le polyn{\^o}me :
\[ R = (P - u)^n (P - v)^m \]


\

On note $\mathcal{Z}$ l'ensemble des racines de $R$.

On a donc, les racines de $R$ sont soit racines de $P - u$ ou $P - v$, donc
un {\'e}lement de $K$.

Ainsi $\mathcal{Z} \subset K$. Or
\[ R' = P' (P - u)^{n - 1} (P - v)^{m - 1} (n (P - v) + m (P - v)) \]


D'apr{\`e}s la question $a$, les racines de $R'$ sont dans l'enveloppe convexe
de $\mathcal{Z}$ not{\'e}e $\tmop{Conv} \mathcal{Z}$.

En particulier, pour tout $z$ tel que $P (z) = \frac{mu + nv}{n + m}$ est une
racine de $R'$, donc est dans Conv $\mathcal{Z}$, qui est inclus dans $K$ par
convexit{\'e} de $K$.

{\'E}tudions {\`a} pr{\'e}sent le cas g{\'e}n{\'e}ral, soit $\lambda \in [0,
1]$ et $(\lambda_k)_{k \in \mathbb{N}}$ une suite de rationnels de $[0, 1]$
convergeant vers $\lambda$. Posons :
\[ P (X) = (\lambda_k u + (1 - \lambda_k) v) = c \prod_{i = 1}^n (z - z_{i,
   k}) \]


o{\`u} $P (z) = \lambda u + (1 - \lambda) v$, on a donc :
\[ (\lambda - \lambda_k) (\mu - \nu) = c \prod_{i = 1}^n (z - z_{i, k}) \]


o{\`u} les $z_{i, k}$ sont dans $K$, d'apr{\`e}s ce qui pr{\'e}c{\`e}de.

Il en r{\'e}sulte que
\[ \prod_{i = 1}^n (z - z_{i, k}) \xrightarrow[k \to + \infty]{} 0 \]


pour tout $\varepsilon > 0$,cil existe $k \in \mathbb{N}$ tel que
\[ \prod_{i = 1}^n |z - z_{i, k} | \leq \varepsilon^{\text{deg } P} =
   \varepsilon^n \]


Par cons{\'e}quent, il existe un indice $j$ tel que $|z - z_{j, k} | \leq
\varepsilon$. Ainsi $z$ est adh{\'e}rent {\`a} $K$. Donc $z \in K$.

\tmtextbf{Remarques.}

1. Le th{\'e}or{\`e}me de Lucas (partie $a$) est un r{\'e}sultat fondamental
en analyse complexe. Il a de nombreuses applications, notamment dans
l'{\'e}tude de la distribution des z{\'e}ros des polyn{\^o}mes.

2. La partie b de cet exercice est une g{\'e}n{\'e}ralisation int{\'e}ressante
du th{\'e}or{\`e}me de Lucas. Elle montre que la convexit{\'e} est
pr{\'e}serv{\'e}e sous certaines transformations polynomiales.

3. Ces r{\'e}sultats peuvent {\^e}tre {\'e}tendus {\`a} des classes plus
larges de fonctions analytiques, au-del{\`a} des simples polyn{\^o}mes.

\tmtextbf{Exercices compl{\'e}mentaires.}

1. Montrer que si toutes les racines d'un polyn{\^o}me $P$ sont r{\'e}elles,
alors toutes les racines de $P'$ sont {\'e}galement r{\'e}elles.

2. G{\'e}n{\'e}raliser le th{\'e}or{\`e}me de Lucas aux d{\'e}riv{\'e}es
d'ordre sup{\'e}rieur : montrer que les racines de $P^{(k)}$ sont dans
l'enveloppe convexe des racines de $P$ pour tout $k \geq 1$.
\[ \maltese \maltese \maltese \maltese \maltese \maltese \maltese \]
