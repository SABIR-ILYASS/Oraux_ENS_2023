\section*{Avant-propos}
Ce recueil est bien plus qu'une simple collection d'exercices : il est conçu comme un compagnon de route pour tous les étudiants de prépa aspirant à intégrer les institutions les plus prestigieuses, telles que l’École normale supérieure (ENS) et l’École polytechnique (l’X), mais aussi pour les candidats des autres grandes écoles et concours de haut niveau. Les exercices rassemblés ici ne sont pas des problèmes standards : ils sont tirés d’épreuves orales réelles des concours des années précédentes, garantissant ainsi une immersion totale dans la réalité des défis à venir.

Le niveau de difficulté de chaque exercice peut varier selon l’étudiant, raison pour laquelle nous n’avons pas jugé utile de les classer selon ce critère. En effet, la difficulté perçue est une donnée subjective et dépend largement de votre maîtrise des sujets abordés. C’est pourquoi nous vous encourageons à aborder chaque exercice avec un esprit ouvert, prêt à explorer, à questionner et à approfondir vos connaissances. Nous vous recommandons également de travailler ces exercices dans des conditions aussi proches que possible de celles des oraux : une durée de 50 minutes pour les oraux d’ULM et de 45 minutes pour ceux du concours commun ULSR. Cela vous permettra non seulement de vous habituer à la gestion du temps, mais aussi de simuler l’environnement stressant des épreuves, où chaque minute compte et où la clarté de pensée est cruciale.

La variété des exercices dans ce recueil reflète les multiples facettes des mathématiques enseignées et évaluées lors des concours. Vous trouverez des problèmes d’algèbre, d’analyse, de géométrie, et même des défis en théorie des nombres. Certains exercices mettront à l’épreuve votre capacité à manipuler des objets classiques tels que les wronskiens, les séries alternées ou les matrices antisymétriques. D’autres vous amèneront à plonger plus profondément dans des problématiques actuelles, comme les espaces de translation ou les sous-groupes des isométries affines. Le but n’est pas simplement de résoudre ces problèmes, mais de comprendre les mécanismes sous-jacents qui permettent d’aborder des situations nouvelles avec confiance.

Pour vous soutenir dans cet effort, nous avons également inclus des solutions détaillées en annexe, qui couvrent non seulement les épreuves récentes de mathématiques A du concours X/ENS 2023 et 2024, mais aussi l’épreuve de mathématiques C du concours X/ENS 2018 et celle de l’agrégation externe de 2019. Ces solutions sont bien plus qu’une simple correction : elles vous guideront pas à pas dans le raisonnement et la méthodologie attendus au plus haut niveau.

En particulier, vous y trouverez une démonstration complète et rigoureuse du théorème de Dirichlet sur les progressions arithmétiques, un résultat fondamental qui revient régulièrement dans les concours. Cette démonstration, issue de l’épreuve ENS Paris-Lyon de 1993, est exposée avec minutie pour que chaque lecteur puisse en saisir tous les rouages, tant au niveau technique qu’intuitif.

En outre, nous vous invitons vivement à consulter les rapports des jurys des concours précédents. Ils vous permettront de mieux comprendre les attentes précises des examinateurs, les qualités qu’ils recherchent et les erreurs fréquentes à éviter. Enfin, en cas de difficulté, n’hésitez pas à recourir à des simplifications ou à des cas particuliers pour clarifier les concepts. Ce travail sur des versions plus abordables d’un problème peut souvent fournir des intuitions précieuses, vous permettant ensuite de revenir à l’énoncé général avec une meilleure compréhension.

La route vers l’excellence est exigeante, mais avec les bonnes méthodes et une persévérance inébranlable, elle est à votre portée. Nous espérons que ce recueil deviendra un compagnon de confiance dans cette aventure et vous aidera à aborder les oraux avec sérénité et confiance.

\bigskip

\noindent
Bonne préparation, et que vos efforts soient couronnés de succès.

\vspace{1cm}

\begin{flushright}
  \textbf{Les auteurs}
\end{flushright}
